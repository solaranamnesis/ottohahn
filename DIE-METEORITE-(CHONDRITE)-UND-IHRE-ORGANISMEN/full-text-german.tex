\documentclass[a4paper, 12pt, oneside]{article}
\usepackage[utf8]{inputenc}
\usepackage[T1]{fontenc}
\usepackage[ngerman]{babel}
\usepackage{fouriernc}
\usepackage{booktabs}
\usepackage{url}
\usepackage{graphicx}
\setlength{\emergencystretch}{15pt}
\graphicspath{ {./figures/} }
\usepackage[figurename=]{caption}
\usepackage{fancyhdr}
\usepackage{imakeidx}
\makeindex[columns=2, title=Alphabetischer Index, intoc]
\begin{document}
\begin{titlepage} % Suppresses headers and footers on the title page
	\centering % Centre everything on the title page
	\scshape % Use small caps for all text on the title page

	%------------------------------------------------
	%	Title
	%------------------------------------------------
	
	\rule{\textwidth}{1.6pt}\vspace*{-\baselineskip}\vspace*{2pt} % Thick horizontal rule
	\rule{\textwidth}{0.4pt} % Thin horizontal rule
	
	\vspace{0.75\baselineskip} % Whitespace above the title
	
	{\LARGE DIE METEORITE (CHONDRITE)\\ UND\\ IHRE ORGANISMEN\\} % Title
	
	\vspace{0.75\baselineskip} % Whitespace below the title
	
	\rule{\textwidth}{0.4pt}\vspace*{-\baselineskip}\vspace{3.2pt} % Thin horizontal rule
	\rule{\textwidth}{1.6pt} % Thick horizontal rule
	
	\vspace{1\baselineskip} % Whitespace after the title block
	
	%------------------------------------------------
	%	Subtitle
	%------------------------------------------------
	
	{Dargestellt und Beschrieben\\ von\\ \scshape\Large Dr. Otto Hahn\\} % Subtitle or further description
	
	\vspace*{1\baselineskip} % Whitespace under the subtitle
	
    {\small 32 Tafeln mit 142 Abbildungen} % Subtitle or further description
    
	%------------------------------------------------
	%	Editor(s)
	%------------------------------------------------
	
	\vspace{1\baselineskip} % Whitespace below the editor list
	
    %------------------------------------------------
	%	Cover photo
	%------------------------------------------------
	
	\includegraphics[scale=0.95]{cover}
	
	%------------------------------------------------
	%	Publisher
	%------------------------------------------------
		
	\vspace{1\baselineskip} % Whitespace under the publisher logo
	
	1$^{st}$ Edition, Tübingen 1880 % Publication year
	
	{\small Verlag der H. Laupp'schen Buchhandlung } % Publisher

	\vspace{1\baselineskip} % Whitespace under the publisher logo

    2$^{nd}$ Edition, Internet Archive 2020 % Publication year
	
	{\small Namensnennung Nicht-kommerziell Weitergabe unter gleichen Bedingungen 4.0 International } % Publisher
\end{titlepage}
\setlength{\parskip}{1mm plus1mm minus1mm}
\setcounter{tocdepth}{2}
\setcounter{secnumdepth}{3}
\tableofcontents
\clearpage
\listoffigures
\clearpage
\section{Einleitung}
\subsection{Einleitung}
\paragraph{}
Nicht die zum Teil wenig sachlichen Angriffe auf meine \emph{Urzelle} waren es, welche mich in meinen Anstrengungen, gewisse neue geologische Tatsachen festzustellen — nicht ermüden ließen: es war die durch Beobachtungen gewonnene Überzeugung von der Unhaltbarkeit der bisherigen Anschauung in dem unstreitig wichtigsten Teile der geologischen Wissenschaft, in dem Teile, durch welchen er gerade mit dem Kosmos zusammenhängt — in der Lehre von den sogenannten plutonischen Gesteinen.

Hatte ich es im ersten Teile meiner \emph{Urzelle} noch mit Ergebung hingenommen, dass der Erdkern und damit auch die Erkenntnis der wirklichen Entstehungs-Geschichte unserer Erde uns stets verborgen bleiben werde: so bot sich doch am Schluss dieses Buchs schon ein Ausblick: der Meteorstein zeigte die ferne Durchfahrt, welche noch von keinem Forscher gewagt worden war.

Mit diesem Führer nun entschloss ich mich vorwärts zu schreiten.

Ich tat es, begleitet auf der einen Seite von dem bald leiser bald schärfer ausgesprochenen Spotte der Fachmänner, auf der andern Seite aufgemuntert durch die früher und nun täglich neu gewonnenen Ergebnisse und unterstützt von dem Rat weniger Freunde, welche zu überzeugen mir gelungen war.

Das was mir meine bei einem anstrengenden Beruf fast über Menschenkraft gehenden Arbeiten des letzten Jahres an Ergebnissen geliefert haben, ist in den folgenden Blättern niedergelegt.

Es ist die Tierwelt in einem Gesteine, welches auf unsere Erde herabfiel und uns Kunde brachte von kleinsten Wesen aus fernsten Räumen — eine Tierwelt, welche zu erblicken ein sterbliches Auge kaum hoffen konnte: eine Welt von Wesen, welche uns zeigt, dass dieselbe Schöpferkraft, welche unsere Erde aus einem Dunstnebel hat werden lassen, überall und gleichmäßig im Weltraum gewirkt und geschafft hat.

Freilich weist das Gestein der Meteorite und zwar der Chondrite — denn diese sind's, welche ich vorzugsweise zum Gegenstand meiner Untersuchung machte, — keine Tiere höheren Baus auf; Alles sind niedere Tiere — dieselben, welche in unseren Silurschichten vorherrschen — Schwämme, Korallen und Crinoiden, und auch in ihren Spezies-Merkmalen stimmen sie mit dieser Schöpfung.

Das Gestein der Chondrite, welche ich untersucht habe, ist ein Olivin-Enstatit-Gestein. Es hat von der Zeit seiner Entstehung, vom Tierknochen, bis es fiel, Verwandlungen durchgemacht, aber keine erhebliche: es ist nur von einer Silicatlösung durchtränkt worden, wie alle unsere Jurameer-Ablagerungen von einer Lösung von Kalk. Wahrscheinlich machte es, so lange es noch ein Teil eines Planeten war, noch mehr Planeten-Perioden durch, wie auch den tieferen Schichten unserer Erde andere gefolgt sind, unter deren Einfluss dann die früheren eine gewisse, freilich nicht so erhebliche Umwandlung als man gewöhnlich annimmt, erfahren haben.

Wesentlich geändert hat sich nur die Oberfläche des Gesteins und zwar im letzten Augenblick seines planetarischen Lebens durch den Einfluss der Reibungswärme, entstanden im Falle durch die Erdatmosphäre. Doch das Bild des ursprünglichen Gesteins ist im Wesentlichen geblieben. Wir sehen nun vor uns ein Stück Planeten wie er im Werden war, und damit ist uns die Geschichte unseres Erdkörpers aufgeschlossen, sofern wir ein Recht haben von der Bildung eines seiner Bewegung nach gleichartigen, in seiner chemischen Zusammensetzung gleichen Weltkörpers auf die gleiche Bildung der Erde zu schließen und umgekehrt.

Gleichzeitig war mir durch die Zusendung des "`Meteorite von Ovifak"' (ich verdanke ihn der Güte des Herrn Professors Dr. von Nordenskjöld) Gelegenheit geboten, dieses Gestein in die Untersuchung hereinzuziehen.

Ich halte es für irdisch — halte es für die tiefste Schichte unserer Erde, der Olivinschichte, die unter dem Granit lagert, angehörend. Ich nenne die ursprüngliche Schichte Olivin-Formation. Da das Gestein einem Meteorite sehr ähnlich ist, lag es nahe, dasselbe für einen solchen zu erklären. Die Gründe, warum ich es nicht für meteoritisch, sondern für den wahren Erdkern halte, sind in diesem Buche niedergelegt.

So haben wir zwei feste Punkte gewonnen, von welchen aus ein Hebel angesetzt werden kann.

Die Chondrite, ein Olivin-Feldspat-(Enstatit-)Gestein besteht aus einer Tierwelt, sie sind nicht ein Lager, nicht ein Konglomerat, sondern ein Filz von Tieren, ein Gewebe, dessen Maschen alle lebendige Wesen waren, und zwar Tiere der niedersten Art, Anfänge einer Schöpfung.

Ich konnte nun allerdings von dieser Tierwelt, welche uns in den Meteoriten erhalten ist, keine systematische Aufzählung machen: ich wollte nur nachweisen, dass sie ist — da ist. Ich bildete daher nur ganz unzweifelhaft organische Wesen ab, wobei ich mich damit begnügen musste, einerseits die Gattungen festzustellen, welche mit unseren terrestrischen Formen übereinstimmen, andererseits die spezifisch meteoritischen Formen auszusondern und beides künftiger Untersuchung in die Hand zu geben.

Es ist zu erwarten, dass meine Aufzählung durch weitere Forschung mit Hilfe eines reicheren Materials, als mir zu Gebot stand, bald sich vermehren und ergänzen werde. Es mussten daher insbesondere Untereinteilungen unterbleiben: jedes neu gefundene Wesen würde die Einteilung umgestoßen und damit die mühevolle, voreilige Arbeit auch zur vergeblichen gemacht haben.

Dies war der Grund warum ich nur die großen Abteilungen und diese nur insoweit gemacht habe, als dies zum Verständnis der Formen beiträgt: erschöpfend und abgeschlossen soll, das wiederhole ich, die Arbeit in dieser Richtung nicht sein.

Auch in anderer Richtung muss ich Nachsicht in Anspruch nehmen: in der Abgrenzung der Hauptabteilungen selbst.

Wer meine Formen nur oberflächlich überblickt, wird bald finden, dass sie eine wirkliche Entwicklungsgeschichte an die Hand geben. Alle die Übergänge vom Schwamm zur Koralle, von der Koralle zum Crinoiden sind da, so dass es wirklich zweifelhaft werden kann, will man nicht eine neue Tiergattung machen, wohin man diese Übergänge stellen soll.

In solchen Anfängen sind Irrtümer unvermeidlich, es ist daher nur eine Forderung der Billigkeit, sie zu verzeihen. Auch wollte ich die Veröffentlichung des Werkes nicht zu lange verzögern, und habe es daher eben so wie es jetzt vorliegt, abgeschlossen.
\clearpage
\subsection{Geschichte und Überblick}
$\Delta$o$\sigma$ $\mu$o$\iota$ $\chi\epsilon\nu\tau\rho$o$\nu$%Δός μοι χέντρον
\paragraph{}
Als ich im vorigen Jahre mein Tagebuch enthaltend gewisse neue Beobachtungen über die Zusammensetzung der Gesteine unserer Erde und schließlich auch der Meteorite, niederschrieb, war mir die Wichtigkeit der letzteren für unsere Erdkunde noch nicht völlig klar.

Erst als ich durch die Angriffe der Gegner gezwungen war, die Untersuchung aufs Neue in die Hand zu nehmen, trat es mir klar vor die Augen, welch' hohe Bedeutung eine sorgfältige Erforschung der Meteorite für die Geschichte unserer Erde haben müsse. Zuletzt kam ich zu der Überzeugung, dass bei dem jetzigen Stand unserer Erdkunde die Meteorite und nur die Meteorite den Punkt abgäben, von welchem aus unsere Erdgeschichte wenigstens mit ziemlicher Sicherheit erforscht werden könne.

Wenn ich also in meiner \emph{Urzelle} mit dem Granit die mögliche Grenze der Forschung erreicht zu haben glaubte, so wurde ich bald eines Bessern belehrt. Ich erwog, dass unser Erdkern vermöge seines spezifischen Gewichts ebenfalls mindestens aus gediegenem Eisen bestehen müsse, erwog ferner die sehr wahrscheinliche Reihenfolge in den Meteoriten, welche vom reinen Eisen bis zu den Feldspatgesteinen unserer Erde geht. Ich glaubte ferner, dass ein Rückschluss von unserer Erde auf die Meteorite gewagt werden dürfe, der Schluss, dass auch in den übrigen Planeten und in denjenigen oder demjenigen, deren (oder dessen) Trümmer wir wohl in den hunderttausend von kreisenden Meteoriten vor uns haben, eine Reihenfolge der Schichtung vom Schweren zum Leichten bestanden habe, eine Schicht-Folge, welche wir wahrscheinlich in der Reihe vom reinen Eisen durch die Halbeisen (Pallasite, Hainholz) hindurch zu den Chondriten und Eukriten, dann zu den Ton-(Kohle-)Meteoriten (Bokkefeld) vor uns haben.

Nachdem diese Wahrscheinlichkeit einmal gewonnen war, lag es nahe, die Meteorite einer genauen Prüfung hinsichtlich ihrer morphologischen Eigenschaften zu unterwerfen. Dies war auch in hohem Grade geboten, denn dass bisher in dieser Richtung so gut wie nichts geschehen ist, davon kann man sich durch Vergleichung meiner Abbildungen mit den etwa zwanzig dürftigen Bildern überzeugen, welche zusammen das heute vorliegende Material unserer Wissenschaft bilden. Die akademischen Schriften von Berlin, Wien, München haben je nur einige Tafeln aufzuweisen, die Zeichnungen sind klein, und wie sich sofort zeigt, von den am wenigsten für diese Richtung der Untersuchung geeigneten Meteoriten und ferner wahrscheinlich auch nicht von dem besten Teile, dem Innern, genommen.

Sollte also auch, dachte ich, meine frühere Behauptung: der Meteorstein von Knyahinya bestehe durchaus aus Pflanzen, durch meine neuen Untersuchungen sich nicht bestätigen, so wäre der Wissenschaft doch ein Dienst getan, wenn ich nur die wahre Form dieses Gesteins zur Darstellung bringen würde. Doch dieser Rückzug blieb mir glücklicherweise erspart, im Gegenteil: das Ergebnis der neuen Forschung war ein alle Erwartung weit übersteigendes — eine neue Welt tat sich auf.

Aber freilich — unsere Wissenschaft ist ungläubig — sie fordert mit Recht strengere Beweise, als ich in meiner \emph{Urzelle} geboten habe; ein Buch, das fast mehr im Stadium, ich möchte sagen, der Intuition geschrieben ist. — Heute lege ich Beweise vor.

Man überblicke die Tafeln dieses Werks und es wird sofort zur Gewissheit, dass es sich hier nicht um Mineral-, sondern um organische Formen handelt, dass wir die Bilder von Tieren vor uns haben, Bilder von Tieren der niedersten Stufe, einer Schöpfung, welche zum größeren Teile wenigstens ihre nächsten Verwandten auf unserer Erde findet; — hinsichtlich der Korallen und Crinoiden ist dies mit unbedingter Sicherheit festgestellt: die Schwämme aber haben wenigstens eine solche Ähnlichkeit mit den Formen der Erde aufzuweisen, wie sie eben innerhalb verwandter irdischer Gattungen besteht.

So war die Entstehung hinsichtlich der Teile festgestellt. Nun bestätigte sich aber auch bei meiner Untersuchung von 20 Chondriten (und 360 Dünnschliffen davon) die in meiner \emph{Urzelle} aufgestellte Behauptung, dass das Gestein der Chondrite nicht etwa nach Art der Sedimentgesteine unserer Erde nur ein Schlamm sei, in welchen die Versteinerungen eingelagert sind, dass es nicht eine Konglomeratbildung sei; ihre ganze Masse ist vielmehr völlig aus organischen Wesen gebildet, wie unsere Korallenfelsen. Also keine Pflanze, wie ich früher annahm, aber Pflanzentiere! Und der ganze Stein ein Leben: — ich denke, die Wissenschaft darf mir den ersten Irrtum gerne verzeihen.

Selbstredend war nun auch das Meteoreisen nochmals einer Prüfung zu unterwerfen. Hier blieb es bei meiner ersten Beobachtung.

Allerdings gestatteten mir Zeit und Umstände, insbesondere der Mangel an verfügbarem Material nicht, die Untersuchung darüber vor dieser Veröffentlichung abzuschließen. Wenn ich aber heute die erste Behauptung, dass das Meteoreisen nichts als ein Pflanzenfilz sei, in der Hauptsache wiederhole, so darf ich mich doch jetzt zu der Behauptung eher legitimiert ansehen, als zur Zeit, als ich die \emph{Urzelle} schrieb. Beizufügen habe ich, dass ich auch im Eisen Tierformen fand. Die Forscher, denen die Formen der Chondrite entgingen, welche ich abbilde, können auch übersehen haben, dass die sogenannten Widmannstätten'schen Figuren in der Tat größtenteils Pflanzenzellen und keine Kristalle sind.

Die bisherigen Untersuchungen auf dem ganzen Gebiete mit Ausnahme der Arbeit [Karl Wilhelm von] Gümbel's in den Schriften der Münchener Akademie sind, sowohl was Genauigkeit der Beobachtung, noch mehr aber was die auf solcher Beobachtung, auf unbewiesenen Hypothesen und leeren Voraussetzungen ruhende Deutung betrifft — wenig geeignet, als eine wissenschaftliche Feststellung angesehen zu werden. So war mir in der Tat das Feld noch völlig offen, wobei ich nur bedaure, dass ich bezüglich der Eisen vorerst noch keine Vorlage machen kann.

Ich komme nun zur Schlussfolgerung für unsere Erdkunde. Sind nämlich die Chondrite — also ein Olivin- und Enstatit-Gestein wirklich, was ich zur Gewissheit bringe, nur Stücke von Schwamm-Korallen-Crinoiden-Felsen, so ist für die Wissenschaft unserer Erde eine Tatsache von unermesslicher Tragweite gewonnen.

Ein Feldspatmineral ist reines Wasserprodukt, ist Versteinerungsmittel für Millionen von Organismen! Damit fallen alle Hypothesen über die metamorphischen und plutonischen Gesteine unserer Erde, damit fällt die Theorie von dem feuerflüssigen Erdinnern, — wenigstens kann aus dem Gestein kein Schluss mehr darauf gezogen werden.

Ich muss dies noch näher begründen. Die Vergleichung der Gesteine der Erde und der Meteorite zeigt, dass der Chondrit, wenigstens nach seiner chemischen Beschaffenheit, seine allernächsten Verwandten auf der Erde hat.

Das Olivingestein unserer Erde ist als Llerzolith ein Lagergestein, als Basalt sehen wir es den Granit durchbrechen; ich traf hier mit den Ergebnissen, welche [Gabriel Auguste] Daubrée gewonnen hatte, zusammen.

Der tieferliegende Granit ist also jedenfalls jünger als der Olivin. Ist aber das Olivingestein der Meteorite vermöge seiner Zusammensetzung ein Wassergestein, so wird es wohl der Granit unserer Erde auch sein; besteht das Olivingestein der Meteore aus niederen Tieren, so wird dasselbe bei dem Olivingestein der Erde der Fall: es wird wohl der Schluss uns erlaubt sein, dass auch dieses Gestein unserer Erde auf seiner ursprünglichen Lagerstätte aus denselben Tieren zusammengesetzt ist, wie der Chondrit. — Und aus demselben Grunde wird auch der Granit, als jüngeres Gestein, wohl denselben Ursprung haben. Haben wir in unserem (schwäbischen) Basalt nur Auslaugungen aus dem ursprünglichen Olivingestein zu erblicken, so ist doch die Lagerung des Llerzoliths unter dem Granit festgestellt. Und erscheint auch dieses Gestein als eine Wasserablagerung ohne unterscheidbare Formen, so hat doch das Eisen von Ovifak solche; dieses aber ist so sehr mit dem Basalt, so innig und nicht bloß mechanisch verbunden, dass beide als ein Gestein angesehen werden müssen. Dieses ist also das ursprüngliche Olivin-Lagergestein. Damit aber ist der Annahme einer Entstehung der Erde auf feurigem Wege der wissenschaftliche Grund entzogen.

Bestand die Oberfläche der Planeten oder des Planeten in den Schichten des Olivins aus Tieren, so ist dieselbe Schichte unserer Erde wohl auch nicht durch Feuer entstanden: wenigstens ist nicht der mindeste Grund zu dieser Vermutung mehr vorhanden, im Gegenteil, es ist anzunehmen, dass auch dieselbe Schichte der Erde eine Wasserbildung gewesen sei. — Hier traf ich nun auf die Kant-Laplace'sche Theorie.

Ich kann mir die Stoffe der Planeten (einschließlich des Wassers, welches gewöhnlich vergessen wird!) zur Zeit der ersten Massenbildung, wie [Immanuel] Kant und [Pierre-Simon] Laplace nur in Dunstform, aber freilich nicht als einen glühenden Dunst denken, sondern nur als Dunst- und Gasmasse im kalten Weltraum. Hier hat man aber den großen logischen Fehler in der genannten Theorie übersehen.

Die Massenanziehung sollte die Masse bilden! Die Wirkung sollte zugleich Ursache sein! Die Masse nämlich sollte sich durch Masseanziehung bilden, also dadurch entstehen, dass sie schon da war! Es ist zu bedauern, dass man diesen Denkfehler nicht früher entdeckt hat. Die Masse kann, wenn sie da ist, sich durch Anziehung vergrößern, aber nicht dadurch werden: es ist als ob Jemand sein eigne Vater sein sollte!

Also eine andere Kraft musste die Masse bilden: diese aber konnte nur entweder die Kristallisations-Kraft oder die organische Bildungskraft sein.

Erstere reicht zur Erklärung der Planetenbildung nicht hin, und es finden sich keine Kristalle: folglich bleibt bloß die zweite Kraft übrig — die organische. Hier erinnere ich an meine Beobachtungen der Struktur des Meteoreisens und so steht heute, für mich wenigstens, die Tatsache fest, dass der erste Anfang unserer Erde, wie der übrigen Planeten, eine organische Ursache hatte.

Erscheint der Satz auch etwas betäubend, so braucht man nur zu ganz Bekanntem zu greifen.

Erstens: Die Masse der Baustoffe, welche im Anfang der Planetenbildung zu Gebot stand, reicht vollständig hin, um die Bildung auch einer Planeten-Masse auf organischem Wege zu erklären.

Zweitens lehrt die Erfahrung von heute, in welch' kurzer Zeit sich die niedersten Pflanzen und Tiere vermehren, dass ihre Zahl, also auch ihre Masse, lediglich durch die Masse der Baustoffe bedingt ist, während ihre Organisation selbst eine Ausdehnung ins Unendliche (so lange nämlich Baustoffe da sind) möglich macht.

Was dieser Erklärung entgegen zu stehen scheint, ist nur die Erdwärme und die damit in Verbindung gebrachte Erscheinung der heute noch tätigen Vulkane. Allein bezüglich dieser beiden Tatsachen ist man längst auf eine andere Erklärung, als auf ein feuerflüssiges Erdinneres, zurückgeführt. Das Wasser wirkt auf Feldspat zersetzend ein. Bei diesem Zersetzungsprozess wird Wärme frei. Die Vulkane folgen dem Meere, weil das Wasser die Gase bilden hilft, welche, von oben entzündet, das anstehende Gestein und auch nur dieses schmelzen.

Wie sollte endlich ein feuriger Erdkern ohne Sauerstoff bestehen können! Und führt nicht eben auch das Dasein brennbarer Gase (denn solche sind die Ursachen der vulkanischen Erscheinungen,) insbesondere das der Schwefelgase auf organische im Erdinnern vorhandene Stoffe zurück? Hier bedarf es wahrhaft keiner neuen Beweise, sondern nur des Aufgebens gewisser Vorstellungen, welche sich der aus einigen augenfälligen Erscheinungen erregten Phantasie bemächtigt haben.

Dies sind die Schlussfolgerungen aus der Untersuchung über die Meteorite für unsere Erdbildung. Ungleich bedeutender aber sind die Tatsachen, welche die Astronomie daraus ableiten kann.

Die Dünnschliffe von 20 von mir untersuchten Meteoriten (Chondriten), von Fällen, welche über ein Jahrhundert auseinander liegen, zeigen dieselben Formen, ähnlich wie eine Leitmuschel überall in derselben Formation vorkommt; dies hat schon Gümbel, wenn er die Formen der Chondrite auch nicht richtig gedeutet hat, trefflich ausgesprochen.

Diese Chondrite stammen also wahrscheinlich von einem und demselben Weltkörper, einem Planeten. Oder ist gar bei verschiedenen Planeten die Entwicklung eine so sehr übereinstimmende gewesen?

Dieser Weltkörper trägt Wassertiere, ist also im Wasser und durch Wasser entstanden, auch nicht durch Feuer vergangen, denn Spuren des Feuers zeigen diese Gesteine nicht: der Meteorit ist zersprungen, seine Trümmer haben nur in ihrem kurzen Weg durch unsere Atmosphäre eine 1 mm dicke Schmelzrinde, in Folge der Reibungswärme, erhalten.

Die Tier-Schöpfung der Chondrite ist beinahe durchaus eine mikroskopische, Tiere sind es von 0,20 bis höchstens 3 mm Durchmesser, oft bedarf es einer Vergrößerung von 1000, um ihre zarte Struktur klar zu sehen, während bei solcher Vergrößerung unsere Petrefacten in eine gestaltlose Fläche sich auflösen.

So war mir durch die erste in meiner \emph{Urzelle} niedergelegte Beobachtung ein Weg geöffnet, auf welchem weite, weite Strecken unserer Wissenschaft gewonnen werden müssen.

Es bedurfte aber wahrlich gerade keiner Titanenkraft mehr, um das alte Gebäude umzustürzen, es war schon viel vorgearbeitet, nur nicht beachtet: es bedarf nur eines einzigen durchschlagenden Beweises und die Arbeit ist getan. Überlieferungen, auf ungenügende Beobachtungen gestützt, lösen sich in das auf, was sie sind, und nun hat die Wissenschaft wieder freie Bahn.
\clearpage
\subsection{Die Bisherigen Ansichten über die Meteorite}
\paragraph{}
Es folgt nun zunächst eine kurze Darstellung der bisherigen Ansichten über die Entstehung und Natur der Meteorite.

Nur die morphologischen Arbeiten über einzelne Meteorite, von der Zeit an, als man das Mikroskop in der Geologie anzuwenden begann, sollen aufgezählt werden.

Was das Mikroskop bis jetzt zur Deutung der Meteorite geliefert hat, das ist, abgesehen von den vergrößerten Olivinkristallen in [Nikolai Ivanovich] Kokscharow's \emph{Mineralien Russlands VI} Band S. 4, in folgenden Schriften enthalten:

    [Gustav] Tschermak: die Trümmerstruktur der Meteoriten von Orvinio und Chantonnay, vorgelegt in der Sitzung der K. Akademie der Wissenschaften (Wien) am 12. November 1874. (XX. Band der Sitzungsberichte der K. Akademie der Wissenschaften, I. Abteilung, November-Heft 1874. Mit 2 Tafeln.)

    [Alexander] Makowsky und G. Tschermak: Bericht über den Meteoritenfall bei Tieschitz in Mähren. Mit 5 Tafeln und 2 Holzschnitten, vorgelegt in der Sitzung der mathematisch-naturwissenschaftlichen Klasse (der Kgl. Akademie der Wissenschaften in Wien) am 21. November 1878. XXIX. Band der Denkschriften der genannten Klasse.

    [Johann Gottfried] Galle und [Arnold Constantin Peter Franz] von Lasaulx, vorgelegt von [Christian Friedrich Martin] Websky: Bericht über den Meteorsteinfall bei Gnadenfrei am 17. Mai 1879. Sitzung vom 31. Juli 1879. Monatsberichte der K. preußischen Akademie zu Berlin.

Die früheren Beschreibungen beschränken sich auf die Untersuchung mit bloßem Auge und der Lupe, sowie die chemische Analyse.

Sie stimmen alle dahin überein: Die Chondrite bestehen aus einer Grundmasse mit Kugeln von Enstatit (Bronzit), Olivin und Eisen, eingesprengtem Nickel- und Chromeisen.

Eine andere Stellung nimmt ein: Gümbel: Über die in Bayern gefundenen Steinmeteoriten; Sitzungsberichte der mathematisch-physikalischen Klasse der K. b. Akademie der Wissenschaften zu München 1878. Heft 1, S. 14 ff. In der Beschreibung der Meteorite von Eichstädt und Schöneberg erwähnte er "`Maschenstruktur"' (S. 27. 46.) Allerdings spricht er auch von "`Abkömmlingen zerbrochener größerer Chondren"' (S. 28). Das Bedeutende seiner Beobachtungen ist auf S. 58, welche ich hier folgen lasse:

"`Überblickt man die Resultate der Untersuchung dieser wenn auch beschränkten Gruppe von Steinmeteoriten, so drängt sich die Wahrnehmung in den Vordergrund, dass sie, trotz einiger Verschiedenheit in der Natur ihrer Gemengteile, doch von vollständig gleichen Strukturverhältnissen beherrscht sind. Alle sind unzweifelhafte Trümmergesteine, zusammengesetzt aus kleinen und größeren Mineralsplitterchen, aus den bekannten rundlichen Chondren, welche meist vollständig erhalten, aber oft auch in Stücke zersprungen vorkommen und aus Gräupchen von metallischen Substanzen Meteoreisen, Schwefeleisen, Chromeisen. Alle diese Fragmente sind aneinander geklebt, nicht durch eine Zwischensubstanz oder durch ein Bindemittel verkittet, wie sich überhaupt keine amorphen, glas- oder lavaartigen Beimengungen vorfinden. Nur die Schmelzrinde und die oft auf Klüften auftretenden, der Schmelzrinde ähnlich entstandenen schwarzen Überrindungen bestehen aus amorpher Glasmasse, die aber erst beim Niederfallen innerhalb unserer Atmosphäre nachträglich entstanden ist. In dieser Schmelzrinde sind die schwerer schmelzbaren und größeren Mineralkörnchen meist noch ungeschmolzen eingebettet. Die Mineralsplitterchen tragen durchaus keine Spuren einer Abrundung oder Abrollung an sich, sie sind scharfkantig und spitzeckig. Was die Chondren anbelangt, so ist ihre Oberfläche nie geglättet, wie sie sein müsste, wenn die Kügelchen das Produkt einer Abrollung wären, sie ist vielmehr stets höckerig uneben, maulbeerartig rau und warzig oder facettenartig mit einem Ansatz von Kristallflächen versehen. Viele derselben sind länglich, mit einer deutlichen Verjüngung oder Zuspitzung nach einer Richtung, wie es bei Hagelkörnern vorkommt. Oft begegnet man Stückchen, welche offenbar als Teile zertrümmerter oder zersprungener Chondren gelten müssen. Als Ausnahme kommen zwillingsartig verbundene Kügelchen vor, häufiger solche, in welchen Meteoreisenstückchen ein- oder angewachsen sind. Nach zahlreichen Dünnschliffen sind sie verschiedenartig zusammengesetzt. Am häufigsten findet sich eine exzentrisch strahlig faserige Struktur in der Art, dass von einer weit aus der Mitte nach dem sich verjüngenden oder etwas zugespitzten Teil hin verrückten Punkte aus ein Strahlenbüschel gegen Außen sich verbreitet. Da die in den verschiedensten Richtungen geführten Schnitte immer säulen- oder nadelförmige, nie blätter- oder lamellenartige Anordnung in der diesen Büschel bildenden Substanz erkennen lassen, so scheinen es in der Tat säulenförmige Fasern zu sein, aus welchen sich solche Chondren aufbauen. Bei gewissen Schnitten gewahrt man, dieser Annahme entsprechend, in den senkrecht zur Längenrichtung gehenden Querschnitten der Fasern nur unregelmäßig eckige, kleinste Feldchen, als ob das Ganze aus lauter kleinen polyedrischen Körnchen zusammengesetzt sei. Zuweilen sieht es aus, als ob in einem Kügelchen gleichsam mehrere nach verschiedener Richtung hin strahlende Systeme vorhanden wären oder als ob gleichsam der Ausstrahlungspunkt sich während ihrer Bildung geändert habe, wodurch bei Durchschnitten nach gewissen Richtungen eine scheinbar wirre, stängliche Struktur zum Vorschein kommt. Gegen die Außenseite hin, gegen welche der Viereinigungspunkt des Strahlenbüschels einseitig verschoben ist, zeigt sich die Faserstruktur meist undeutlich oder durch eine mehr körnige Aggregatbildung ersetzt. Bei keinem der zahlreichen angeschliffenen Chondren konnte ich beobachten, dass die Büschel so unmittelbar bis zum Rande verlaufen, als ob der Ausstrahlungspunkt gleichsam außerhalb des Kügelchens läge, sofern nur dasselbe vollständig erhalten und nicht etwa ein bloßes zersprungenes Stück vorhanden war. Die zierlich quergegliederten Fäserchen verlaufen meist nicht nach der ganzen Länge des Büschels in gleicher Weise, sondern sie spitzen sich allmählich zu, verästeln sich oder endigen, um andere an ihre Stelle treten zu lassen, so dass in dem Querschnitte eine mannichfache, maschenartige oder netzförmige Zeichnung entsteht. Diese Fäserchen bestehen, wie dies schon vielfach im Vorausgehenden geschildert wurde, aus einem meist helleren Kern und einer dunkleren Umhüllung, jener durch Säuren mehr oder weniger zerlegbar, letztere dagegen dieser Einwirkung widerstehend. Höchst merkwürdig sind die schalenförmigen Überrindungen, welche aus Meteoreisen zu bestehen scheinen und in der Regel nur über einen kleineren Teil der Kügelchen sich ausbreiten. Die gleichen einseitigen, im Durchschnitt mithin als bogenförmig gekrümmte Streifchen sichtbaren Überrindungen kommen auch im Innern der Chondren vor und liefern einen starken Gegenbeweis gegen die Annahme, dass die Chondren durch Abrollung irgend eines Materials entstanden seien, wie denn überhaupt die ganze Anordnung der büscheligen Struktur mit Entschiedenheit gegen ihre Entstehung durch Abrollung spricht. Doch nicht alle Chondren sind exzentrisch faserig; viele, namentlich die kleineren besitzen eine feinkörnige Zusammensetzung, als beständen sie aus einer zusammengeballten Staubmasse. Auch hierbei macht sich zuweilen die einseitige Ausbildung der Kügelchen durch eine exzentrisch größere Verdichtung der Staubteile bemerkbar"'.

Und ferner S. 61:

"`Der gewöhnliche Typus der Meteorite von steiniger Beschaffenheit ist soweit überwiegend derjenige der sog. Chondrite und die Zusammensetzung sowie die Struktur aller dieser Steine so sehr übereinstimmend, dass wir den gemeinsamen Ursprung und die uranfängliche Zusammengehörigkeit aller dieser Art Meteorite — wenn nicht aller — wohl nicht weiter in Zweifel ziehen können.

"`Der Umstand, dass sie sämtlich in höchst unregelmäßig geformten Stückchen in unsere Atmosphäre gelangen — abgesehen von dem Zerspringen innerhalb der letzteren in mehrere Fragmente, was zwar häufig vorkommt, aber doch nicht in allen Fällen angenommen werden kann, namentlich nicht, wenn durch direkte Beobachtung das Fallen nur eines Stückes konstatiert ist, — lässt weiter schließen, dass sie bereits in regellos zertrümmerten Stücken als Abkömmlinge von einem einzigen größeren Himmelskörper ihre Bahnen im Himmelsraume ziehen und in ihrer Zerstreutheit einzeln zuweilen in das Attraktionsbereich der Erde geraten zur Erde niederfallen. Der Mangel ursprünglicher, lavaartiger, amorpher Bestandteile in Verbindung mit der äußern unregelmäßigen Form dürfte von geo- oder kosmologischen Standpunkte aus die Annahme ausschließen, dass diese Meteorite Auswürfe von Mondvulkanen, wie vielfach behauptet wird, sein können."'

Gümbel fasst, nachdem er die Meteorite in die Olivingesteine unserer Erde eingestellt hat, seine Ansicht hinsichtlich der Entstehung (S. 64) in den Satz zusammen:

"`Es scheinen daher die Meteorite aus einer Art erstem Verschlackungsprozess der Himmelskörper, aber da sie metallisches Eisen enthalten — bei Mangel von Sauerstoff und Wasser hervorgegangen zu sein."'

"`So geistreich, fährt er (S. 68) fort, diese Hypothesen Daubrée's und Tschermak's sind (Entstehung aus zertrümmertem Vulkangestein), so kann ich mich doch in Bezug auf die Entstehung der Kügelchen (Chondren) ihrer Ansicht auf Grund meiner neuesten Untersuchungen nicht anschließen. Ich habe im Gegensatze zu Tschermak's Annahme nachzuweisen gesucht, dass das innere Gefüge der Chondren nicht außer Zusammenhang mit ihrer kugeligen Gestalt stehe, und dass man diese Kügelchen weder als Stücke eines Mineralkristalls, noch eines festen Gesteins ansehen könne. Spricht schon ihre nicht geglättete, nicht polierte Oberfläche, welche, wenn durch Abreibung oder Abrollung gebildet, bei solcher Härte des Materials spiegelglatt sein müsste, während sie rauh, höckerig, oft strichweise kristallinisch facettirt erscheint, gegen die Abreibungstheorie, so ist auch gar kein Grund einzusehen, weshalb nicht alle anderen Mineral splitterchen wie Sandkörner abgerundet seien und weshalb namentlich das Meteoreisen, das Schwefeleisen und das sehr harte Chromeisen, wie ich in dem Meteorit von L'Aigle mich überzeugt habe, stets nichtgerundete, oft äußerst fein zerschlitzte Formen besitzen. Wie wäre es zudem denkbar, dass, wie häufig beobachtet wird, innerhalb der Kügelchen konzentrische Anhäufung von Meteoreisen vorkommen? Auch erscheint die exzentrisch faserige Struktur der meisten Kügelchen in ihrem einseitig gelegenen Ausstrahlungspunkte in Bezug auf die Oberfläche nicht als zufällig, sondern der Art der Struktur der Hagelkörner nachgebildet. Dieses innere Gefüge steht im engsten Zusammenhang mit dem Akt ihrer Entstehung, welche nur als eine Verdichtung Mineral bildender Stoffe unter gleichzeitiger drehender Bewegung in Dämpfen, welche das Material zur Fortbildung lieferten, sich erklären lässt, wobei in der Richtung der Bewegung einseitig mehr Material sich ansetzte."'

Weiter freilich spricht Gümbel sich dahin aus, dass das Material, aus welchem die Chondrite bestehen, durch eine gestörte Kristallisation und Zertrümmerung in Folge von explosiven Vorgängen innerhalb eines Raumes sich gebildet habe, welcher von den die Mineralien bildenden Stoffe liefernden Dampf- und Wasserstoffgasen erfüllt war. Er schließt S. 72 bei Besprechung des Meteorites von Kaba:

"`Vielleicht gelingt es dennoch, die Anwesenheit organischer Wesen auf außerirdischen Körpern nachzuweisen."' Ich hoffe dies sei gelungen. — Aus seinen Abbildungen ersieht man, dass bei der Untersuchung ein schlechtes Material zu Gebot stand. Auch hätten immerhin mehr Dünnschliffe gefertigt werden müssen, zudem reicht die Vergrößerung bei Weitem nicht. Ich verweise hier auf das Folgende und die Beschreibung meiner Tafeln.

Was ich in dem Berichte Gümbels so hoch schätze, ist die gewissenhafte vorurteilsfreie, ich möchte sagen unparteiische Beobachtung. Ich habe mir erlaubt, die Schrift Gümbels wörtlich anzuführen, weil es mir in der Tat schwer wird, solche Darstellungen zusammenzufassen und Tatsachen und Deutung zu trennen.

Richtige Beobachtungen und unrichtige Erklärungen stehen so nahe beisammen, dass es unmöglich ist beides zu sondern. Ich glaubte, als ich die Gümbel'sche Abhandlung (nach dem Abschluss meiner Untersuchungen und meines Manuskripts) durchlas, in jedem Augenblick auf meine Resultate zu treten. Aber wie die Woge der Brandung den, welcher das Land gewinnen will, jedes mal dann wieder ergreift und zurückwirft, wenn er schon das Land gefasst zu haben glaubt, so auch hier: allemal reißt das alte Dogma den geehrten Forscher von der rettenden Klippe hinweg in den bodenlosen Strudel der Traditionen zurück.

Daubrée's verdienstvolles Werk \emph{Experimentalgeologie} erhielt ich erst in der Übersetzung zur Hand und ebenfalls nach Abschluss meiner Arbeit. Dass es diese widerlegte, wird wohl Niemand finden. Daubrée hat selbst Knyahinya abgebildet. M. hat gepresst, geschmolzen, aufgelöst, berechnet, nur nicht — gesehen.
\clearpage
\subsection{Die Meteorite und ihre Mineralogischen Eigenschaften}
\paragraph{}
Die Literatur der Meteorite ist eine sehr umfangreiche. Sie ist jedoch, was die Art und Zahl, chemische Zusammensetzung betrifft, so bekannt, dass ich auf diesen Teil derselben, also insbesondere die früheren Arbeiten, nicht einzugehen brauche.

Die Meteorite werden eingeteilt in Eisen und Steine, zwischen beiden steht jedoch noch eine Klasse: Halbeisen, d. h. eine Verbindung von gediegenem Eisen und Stein — die Pallasite. Während die Eisen eine ziemliche Übereinstimmung, sowohl in ihrer chemischen Zusammensetzung, als in der Form ihrer Struktur zeigen, sind die Pallasite (je nach dem Vorwiegend des Eisens) sehr verschieden. Aber es finden sich noch weitere Verschiedenheiten darunter. Hainholz z. B. hat neben Eisen und Olivin ein blaues Mineral (Enstatit) und in diesem einen großen Reichtum von Tierformen. — Die Steine werden eingeteilt in Chondrite, Stannerite [eukriten], Luotolaxer [howarditen], Bokkefelder [karbonatisch], Bishopvillit [aubriten], (Quenstedt, Klar und Wahr S. 280 folg.)

Ich habe mich vorzugsweise mit den Chondriten beschäftigt und, wo ich von Meteoriten rede, rede ich von dieser allerdings auch am zahlreichsten vertretenen Klasse von Stein-Meteoriten.
\paragraph{}
Ich habe untersucht:
\begin{center}
\begin{tabular}{ l r }
 Tabor, Böhmen [Czech Republic] & July 3, 1753\index{meteorite!Tabor} \\
 Siena, Toskana [Italy] & June 16, 1794\index{meteorite!Siena} \\
 L'Aigle, Normandy [France] & April 26, 1803\index{meteorite!L'Aigle} \\
 Weston, Connecticut [USA] & December 14, 1807\index{meteorite!Weston} \\
 Tipperary, Ireland & November 23, 1810\index{meteorite!Tipperary} \\
 Blansko, Brünn [Czech Republic] & November 25, 1833\index{meteorite!Blansko} \\
 Château-Renard, Loiret [France] & July 12, 1841\index{meteorite!Château-Renard} \\
 Linn [Marion] County, Iowa [USA] & February 25, 1847\index{meteorite!Marion County}\index{meteorite!Linn} \\
 Cabarras [Monroe] County, North Carolina [USA] & October 31, 1849\index{meteorite!Monroe County}\index{meteorite!Cabarras} \\
 Mezö-Madaras [Romania] & September 4, 1852\index{meteorite!Mezö-Madaras} \\
 Borkut, Hungary & October 13, 1852\index{meteorite!Borkut} \\
 Bremervörde, Hanover [Germany] & May 13, 1855\index{meteorite!Bremervörde} \\
 Parnallee, East India [Tamil Nadu] & February 28, 1857\index{meteorite!Parnallee} \\
 Heredia, Costa Rica & April 1, 1857\index{meteorite!Heredia} \\
 New Concord, Ohio [USA] & May 1, 1860\index{meteorite!New Concord} \\
 Knyahinya, Hungary & June 9, 1866\index{meteorite!Knyahinya} \\
 Pultusk, Warsaw [Poland] & January 30, 1868\index{meteorite!Pultusk} \\
 Orvinio [Italy] & August 31, 1872\index{meteorite!Olvinio} \\
 Simbirsk [Russia] & [1838]\index{meteorite!Simbirsk} \\
\end{tabular}
\end{center}
\clearpage
\paragraph{}
Alle Gesteine sind durchaus beglaubigt. Ich habe hier vor Allem der Liberalität meines verehrten Lehrers, Herrn Professor Dr. [Friedrich August] von Quenstedt, mit welcher er mir die vorzügliche Tübinger Universitäts-Sammlung (welche bekanntlich zum größten Teil vom Freiherrn [Karl Ludwig] von Reichenbach in Wien stammt) dankend zu gedenken.

Von Knyahinya besitze ich 360 Dünnschliffe, von L'Aigle 6, von Pultusk 6, von den übrigen 1-3. Ich werde sämtliche Steine kurz nach dem Fallort benennen. Bei Herstellung der Dünnschliffe habe ich die Schnitte in 2 Richtungen genommen. Es ergab sich nämlich nach mehreren Versuchen an Knyahinya, dass derselbe nach einer bestimmten Richtung bricht.

Es konnte dies aus den Einschlüssen entnommen werden, welche, nachdem einmal die Stellung gefunden war, regelmäßig bestimmte Formen-Durchschnitte ergaben, welchen dann die Formen in einem senkrecht auf diese Stellung gefertigten Schnitte entsprachen.

Waren die Formen an diesem Steine gestellt, so wäre wohl dieselbe Stellung in den übrigen Steinen zu erhalten gewesen, vorausgesetzt natürlich, dass das Material zu Gebot gestanden hätte. Bei einzelnen ergab sich dieselbe zufällig — bei anderen nicht, es musste aber aus den angeführten Gründen auf weitere Feststellung in dieser Richtung verzichtet werden.

Ich fertigte ferner die Schliffe absichtlich in dreierlei Dicke: schwer durchsichtig, um die ganzen Einschlüsse möglichst vollständig zu bekommen: sehr dünn, um die Strukturverhältnisse klar zu stellen; den größten Teil aber so, dass beides noch zur Anschauung kam.

Ich reihe hier eine Bemerkung an, welche mir Jeder bestätigen wird, welcher sich mit Dünnschliffen von Petrefacten beschäftigt hat.

Nur in seltenen Fällen ist in völlig durchsichtigen, also ganz dünnen Schliffen, noch die Struktur sichtbar. Wer seinen Schliff, wenn er halbdurchsichtig, im Mikroskop betrachtet, ist im höchsten Grad erfreut über die schönen Formen und Linien. In der Freude darüber will er die Sache noch besser machen und erwartet bei fortgesetztem Schleifen ein vollendetes Bild. Aber wenn er den Schliff zum zweiten Mal unter das Mikroskop legt — ist nichts mehr da als eine fast strukturlose Fläche, kaum angedeutete, sogar in den Umrissen verschwommene Formen, aus welchen nun das, was man vorher schon mit der Lupe wahrnahm, nicht einmal mehr mit dem Mikroskop zu ersehen ist. Diese Erscheinung hängt aber mit der Art der Metamorphose des Gesteins und der darin eingeschlossenen Formen zusammen. Die Sache ist jedoch bekannt und bedarf deshalb keiner weiteren Ausführung. Ich musste der Tatsache nur deshalb erwähnen, damit solche, welche Beobachtungen erst anstellen wollen, ohne dass sie dieselbe kennen, nicht überrascht werden und ihre Beobachtungsweise verbessern können.

Dass die Chondrite zum größten Teile aus Bronzit-Enstatit (Augit) und Olivin sowie Magnetkies bestehen, ist eine in der Wissenschaft angenommen Tatsache. Quenstedt, \emph{Handbuch der Mineralogie} S. 722.

Insbesondere aber sind die Einschlüsse, welche ich für Korallen erkläre, für Enstatit angesprochen worden. Damit glaubte man die Struktur derselben erklären zu können. Andere gingen noch weiter und erklärten die Einschlüsse zum Teil für Gläser: (Tschermak).

Ehe ich also an die Begründung meiner Ansicht komme, muss die mikroskopische Erscheinung des hauptsächlich vorkommenden Minerals, des Enstatits, genau festgestellt werden.

Ich erlaube mir hier Kürze halber dasjenige anzuführen, was [Karl Heinrich Ferdinand] Rosenbusch in seinem Buch: \emph{Mikroskopische Physiographie der petrographisch wichtigen Mineralien} Stuttgart 1873 S. 252, über Enstatit (und Bronzit) sagt:

"`Bekanntlich hat man seit den optischen Untersuchungen von [Alfred] Des Cloizeaux den Enstatit, Bronzit und Hypersthen als rhombisch kristallisierend vom Pyroxen getrennt und sie in eine eigene Gruppe zusammengestellt. Dieselben zeigen neben der Spaltung nach dem Prisma von 87° noch weitere Spaltungen nach den vertikalen Pinakoiden, über deren relative Vollkommenheit die Angaben der verschiedenen Forscher nicht genau übereinstimmen. Chemisch bilden diese 3 Mineralien eine ununterbrochene Reihe, an deren Anfange der fast eisenfreie Enstatit und an deren Ende der sehr eisenreiche Hypersthen steht. Enstatit und Bronzit sind sich überdies auch in allen physikalischen Eigenschaften so ähnlich, dass eine Trennung derselben in zwei Spezies kaum durchzuführen sein dürfte. Der Hypersthen dagegen zeigt eine verschiedene optische Orientierung und mag daher immerhin eine eigene Spezies bilden. Interessant ist die von Tschermak gegebene Zusammenstellung der negativen Winkel der optischen Achsen und des Eisengehaltes der drei genannten Mineralien, wobei es sich ergibt, dass mit zunehmendem Gehalte an FeO der Winkel der optischen Achsen stetig abnimmt. Die Mikrostruktur aller Mineralien der Enstatit-Gruppe ist im Allgemeinen eine so ähnliche, dass im speziellen Falle eine sichere Entscheidung unter ihnen nur durch chemische und genaue optische Analyse gegeben werden kann."'

"`Enstatit und Bronzit finden sich in den Gesteinen nicht als Kristalle, sondern fast nur in unregelmäßig begrenzten Kristallkörnern, welche meistens eine sehr dichte Streifung erkennen lassen, die bei dem Enstatit mehr geradlinig, bei dem Bronzit mehr sanft wellig gewunden verläuft. Doch ist dieser Unterschied kein durchgreifender. Die gleiche Streifung zeigt auch der monokline Diallag und der rhombische Bastit, der sich aber durch andere, später zu besprechende, optische Erscheinungen nicht unschwer vom Bronzit trennen lässt. Traf der Schliff den Enstatit oder Bronzit stark geneigt zu seiner Hauptspaltungsfläche, so ist die Oberfläche nicht in gleicher Weise feinfaserig, sondern treppenförmig rauh. Querliegeende Absonderungsflächen und Zierbrechungen sind nicht selten."'

"`An fremdartigen Einlagerungen sind beide verhältnismäßig arm; ja sie fehlen z. B. im Enstatit aus dem Pseudophit des Aloysthals in Mähren und in manchen Enstatiten oder Bronziten der Lherzolithe und Olivinfelsen ganz. Ersterer ist nur von häufigen Adern des Pseudophit durchzogen, von welchen aus in senkrechter Richtung feinfaserige Zersetzungsprodukte in den Enstatit eindringen. Andere Vorkommnisse und selbst andere Individuen desselben Handstücks enthalten dagegen oft massenhafte Einschlüsse von grünen oder braunen Lamellen, Leistchen und Körnern (je nach der Lage der Schliffebene), welche ausnahmslos der vollkommensten Spaltungsrichtung parallel gelagert sind. Der Gedanke liegt nahe, dass die verschiedenen Angaben über die relative Vollkommenheit der pinakoidalen ($\infty$P$\infty$) Spaltung gegenüber der prismatischen vielleicht auf die mehr oder weniger massenhafte Anwesenseit dieser Interpositionen zurückzuführen seien, die zweifellos auch den Metalloiden Schiller auf dem Brachypinakoid bedingen. Dann wäre aber die Leichtigkeit der Trennung in der genannten Richtung mehr eine Absonderung, als eine eigentliche Spaltbarkeit."'

"`Der Enstatit ohne und der Bronzit mit metallischem Schimmer auf der brachypinakoidalen Spaltungsfläche finden sich in Serpentinen von Aloysthal in Mähren (Enstatit) und Mont Brésouars in den Vogesen, in den Lherzolithen und Olivinfelsen, in manchen Olivingabbros, in Streng's Enstatitfels vom Radauthal bei Harzburg und in den Olivinbomben des Dreiser Weihers, sowie in manchen Meteoriten; also stets in Gesellschaft des Olivin und in veränderten Olivingesteinen."'

Für diejenigen, welchen das Buch nicht zu Gebote steht, gebe ich 2 Abbildungen, die eine von Bronzit vom Kupferberg Tafel 1. 1, die andere von Enstatit von Texas Tafel 1. 2, welche mit den Rosenbusch'schen ziemlich übereinstimmen.

Was den Olivin betrifft, so bedarf es keiner Abbildung, da die Formen dieses Gesteins durch Zirkel vollständig erschöpft sind. Es genügt zu sagen, dass reiner frischer Olivin keine Struktur zeigt. Struktur zeigt der Olivin bloß, wenn man seine Einschlüsse oder Anwachsstellen des Kristalls oder Zersetzungserscheinungen (Serpentinbildung) Struktur nennen wollte. Aber sicher findet sich in keinem Kristall etwas, was meinen Formen auch nur ähnlich sieht. Was die Behauptung betrifft, die Kugeln seien Gläser, so wird nicht einmal unterschieden, welche chemische Zusammensetzung diese Gläser gegenüber Enstatit, Bronzit und Olivin haben sollen. Offenbar werden alle Formen zusammen geworfen und für Gläser erklärt, obgleich Enstatit nach Quenstedt (Mineralogie S. 318) unschmelzbar, nach Naumann-Zirkel S. 585 wenigstens schwer schmelzbar ist. Es wird sogar behauptet, dass diese Gläser erst im Fallen entstanden seien. Allein Feuereinwirkungen finden sich bloß in der Rinde. Die Schmelzrinde der meisten Meteorite hat kaum 2 mm Durchmesser.

Die Behauptung, es seien Gläser, wurde der Mitteilung meiner ersten Dünnschliffe entgegengehalten und dabei auf die Ähnlichkeit der meteoritischen Form mit solchen Gläsern in dem Gesteine unserer Erde hingewiesen. So wurde ich von [Ferdinand] Zirkel auf einen Sphaerulit-Liparit verwiesen, dessen Abbildung ich Tafel 1. Figur 3 gebe. Diese Form sollte dartun, dass meine Urania eine Täuschung sei. Ich halte die Form im Liparit für eine Kristallit-Bildung (wahrscheinlich Zeolith). Nun betrachte man die Strukturbilder daneben Tafel 1, Figur 4, 5, 6!

Unsere Forscher, mit Ausnahme Gümbels, sprechen von den Meteoriten als vulkanischen Bomben, erklären das Gestein als identisch mit dem Vulkangesteine der Erde, zählen also den Meteorstein ohne Bedenken zu den vulkanischen. Der Gegenbeweis ist der Gegenstand dieses Buchs.

Richtig allein hat Quenstedt die Frage für eine offene erklärt und gesagt: es sei dem Mikroskop vorbehalten, das Rätsel der Zusammensetzung der Meteorite zu lösen! \emph{Handbuch der Mineralogie} S. 722.
\clearpage
\section{Die Organische Natur der Chondrite}
\subsection{Organisch oder Unorganisch?}
\paragraph{}
Um den Beweis zu führen, dass ein pflanzlicher oder tierischer Organismus vorliege, halte ich für notwendig darzutun:
\begin{enumerate}
    \item eine geschlossene Form,
    \item eine wiederkehrende Form,
    \item wiederkehrend in Entwicklungsstufen,
    \item Struktur und zwar entweder Zellen oder Gefäße,
    \item Ähnlichkeit mit bekannten Formen.
\end{enumerate}
\paragraph{}
Sind diese Erfordernisse da, so bleibt nur noch zu entscheiden, ob Pflanze oder Tier? Nun fragt sich, erfüllen meine Formen diese Forderungen?

Ich glaube, ehe ich an den positiven Beweis gehe, den negativen Beweis führen zu sollen.

Der Beweis nämlich, den ich für das Dasein organischer Wesen antrete, ist ein doppelter: ein negativer, indem ich dartue, dass die meteoritischen Formen nicht dem Mineralreich angehören: ein positiver, indem ich die Übereinstimmung derselben mit den Formen unserer Erde, sei es lebender oder ausgestorbener, begründe: das erste also, was zu beweisen, ist der Satz:

Die Einschlüsse der Meteoriten sind keine Mineralbildungen.

1. Unsere Mineralogen erklären die Einschlüsse der Chondrite für Enstatit, Bronzit, Olivin.

Olivin hat keinen sichtbaren Blätterbruch, Enstatit und Bronzit einen deutlichen. Ich bilde einen Bronzit von Kupferberg, Tafel 1. 1. einen Enstatit von Texas, Tafel 1. 2. (Dünnschliff bei 75 facher Vergrößerung) ab. Figur 2. zeigt einen der besten Blätterbrüche. Man vergleiche nun damit zuerst Tafel 1. Figur 4, einen Teil eines Favositen des Meteorsteins von Knyahinya (etwa 250 mal vergrößert) und man wird wohl nicht mehr davon reden, dass der Blätterbruch die Ursache der Strukturerscheinungen der Chondrite sei. Nun betrachte man aber noch sämtliche Tafeln und es wird diese Erklärung ein für allemal abgetan sein.

2. Wenn die Einschlüsse der Chondrite nach der bisherigen Deutung aus Enstatit oder Olivin bestehen, oder wenn es Gläser wären: wie wäre es, frage ich, möglich, dass dasselbe Mineral oder Glas im Ganzen in so verschiedenen Formen (Umrissen und Strukturen), und verschiedene Minerale in so scharf übereinstimmenden Formen auftreten? Man betrachte einmal einen Hypersthen, eine Hornblende, einen Augit! Abgesehen von einigen sichtbaren, leicht zu erklärenden Einschlüssen — (und um diese handelt es sich ja hier nicht) immer dasselbe Bild! Von höchstens 3 Mineralen hundert verschiedene Bilder!

Das Mineral ist einfach, muss seinem Begriff nach einfach sein und daher stets das Bild einer homogenen Masse (Fläche) geben, höchstens mit einigen Einschlüssen. Wie sollte nun dasselbe Mineral in so verschiedenen Strukturen, dabei in so übereinstimmenden von den Kristallformen abweichenden Umrissen vorkommen?

3. Die Minerale sind entweder kristallisiert oder nicht kristallisiert. — In dem ersten Zustand haben sie bestimmte gesetzmäßige also wiederkehrende Formen: sie rühren von Flächen, welche im Durchschnitt sich als gerade Linien projizieren. Diese Formen (Linien und Winkel) sind wiederkehrend, wechseln bloß der Größe, nicht dem Verhältnis nach. Solche Formen finden sich unter den von mir als organisch angesprochenen Formen nicht. Hier ist keine Form mit einer Fläche oder mit einem Winkel; Alle sind Kugeln, Ellipsen mit Abweichungen von der mathematischen Form, Abweichungen, welche aber doch konstante sind. Also ganz abgesehen von der übereinstimmenden Struktur, zeigt sich eine Konstanz der Umrisse, aber andere Formen als die Kristallformen des Enstatits, des Olivins sie geben müssten.

Allerdings kommen seltene, kleine Stellen mit wirklichen Kristallen vor, aber in einer Weise, welche durchaus auf den Beweiswert dieser Tatsachen nicht einwirkt. Hierüber siehe unten und Tafel 32. Figur 2.

4. Waren die Minerale ursprünglich kristallisiert, haben aber durch mechanische Gewalt ihre Kristall-Form verloren, so ist die einzige Form, welche hier sich wiederholen könnte, die Kugel oder eine dieser sich nähernde Form, etwa die Ellipse. Hier wäre eine Wiederholung möglich, ohne dass aus der Form ein Schluss gezogen werden könnte. In den Rollsteinen schneidet die Oberfläche den Körper in einer Weise, dass sofort die Einwirkung der mechanischen Gewalt hervortritt, — insbesondere werden Einschlüsse ganz willkürlich getroffen.

In den Meteoreinschlüssen aber ist die Struktur im Stein stets, ich möchte sagen: symmetrisch, im Einklang mit den Umrissen.

5. Bei Verwitterung von Kristallen ändern sich die Schichten von außen nach innen — konzentrisch: — von Verwitterung aber ist keine Spur in den Einschlüssen der Chondrite zu sehen und die Strukturen sind stets exzentrisch.

6. Was die Einschlüsse der Mineralien betrifft, so können diese je nach ihrer Beschaffenheit verschiedene Bilder geben. Es kommen ganz willkürliche Formen der Einlagerung vor, wie Glas-Flüssigkeits-Einschlüsse, Kristalliten.

Wo aber ein Formengesetz in der Einlagerung auftritt, richtet sich dieses stets nach der Kristallform. Beides trifft bei den Meteoritformen nicht zu. Keine Spur von Einlagerung nach einer Kristallform!

7. Ein Blätterbruch wird nur sichtbar, wenn durch mechanische Gewalt Spalten und nun Lichtbrechungserscheinungen auf den Spaltungsflächen entstehen. Ohne diese ist er nicht wahrnehmbar. Spaltungsflächen sind nicht da, Lichtbrechungserscheinungen zeigen die Meteorit-Einschlüsse auch nicht, bloß "`Einstäubungen"'.

Es finden sich in den terrestrischen Mineralien Interpositionen parallel mit dem Blätterbruch eingelagert: diese zeigen die Meteoriten nicht.

Ich glaube, der Anblick meiner Formen wird eine weitere Auseinandersetzung über ihre Verschiedenheit von Mineral- und insbesondere von Kristallbildern nicht notwendig machen.

8. Es ist aber soviel von Kristalliten, von Kristallkonkretionen gesprochen worden.

Für solche wurden die Enstatit-Bronzit-Olivin-Kugeln bisher gehalten. Gümbel wies dementgegen darauf hin, dass es keine Kugel gebe, wo der Mittelpunkt nicht exzentrisch liege!

Hier gerade tritt der wesentliche Unterschied zwischen den Meteorit-Formen und den Kristalliten recht deutlich hervor.

Die Kristalliten legen sich stets um einen Punkt (konzentrisch) an. Die Formen in den Meteoriten sind alle elliptisch und birnenförmig: wenn die äußere Form aber auch kugelig ist, sind die angeblichen Einschlüsse exzentrisch geordnet und zwar liegt der Mittelpunkt an der Peripherie, (sogar jenseits derselben, nämlich dann, wenn er weggeschliffen ist, was Gümbel übersah) — eine Erscheinung, welche nie im Mineralreich vorkommt. Es ist eben die Bedingung der Kristalliten- d. h. Kugelbildung, dass die Kristalle um Einen Kristall gleichmassig sich anlegen, wodurch dann notwendig die konzentrische Form entsteht.

Wären also die Kugeln in den Meteoriten Kristalliten, so müssten sie, wenigstens nach dem Gesetz der Erde, konzentrische Bildungen aufweisen.

9. Schließlich muss ich einen Widerspruch aufzeigen, in welchen die Wissenschaft mit sich geriet, wenn sie die Struktur der Chondriten aus der Mineral-Eigenschaft erklären wollte. Dies ist das optische Verhalten dieser Einschlüsse.

Wären sie Kristalle und wäre der Blätterbruch (freilich Olivin hat keinen, und doch finden sich auch in den angeblichen Olivin-Kugeln Strukturen, also Blätterbruch!) die Ursache der Struktur, so müsste das Mineral notwendig das Licht brechen. Bei den meisten der Einschlüsse zeigt sich aber keine Lichtbrechung, nicht einmal Aggregat-Polarisation! — So können sie also weder einfache Mineralien noch Kristalle sein, am allerwenigsten ließe sich die Struktur aus Blätterbrüchen erklären. Diese Tatsache, das optische Verhalten, sollte allein schon zur richtigen Deutung geführt haben.

All diese Beweise sind freilich dem Botaniker und Zoologen fremd, während sie jeder Mineraloge kennt: daher muss ich diesen bitten dem Kollegen Botaniker und Zoologen das eben Vorgetragene zu bestätigen, zu bestätigen was meine Lichtbilder zeigen: Diese Formen sind keine Mineralformen. Damit hat der Mineraloge seinen Anteil an der Arbeit getan und nunmehr geht sie in die Hand des Paläontologen, oder richtiger des Zoologen über und es beginnt die positive Beweisführung.
\clearpage
\subsection{Die Einzelnen Formen: Schw"amme --- \emph{Urania}\index{Urania}}
\paragraph{}
Rundlappige Körper mit deutlicher Anwachsstelle. Tafel 2. gibt ein größeres Normalbild einer Urania (vergleiche Tafel 5. Figur 1, dasselbe Bild). Man sieht hier: die Gesamtform scharf, den äußersten Lappenrand angeschnitten (weis links), die Falten, welche beim Zusammenziehen entstehen, die Anwachsstelle. Noch deutlicher ist letztere mit Kelch, Tafel 4. Figur 3.

Urania spiralförmig zusammengelegt Tafel 3. Figur 5, 6.

In der Windung begriffen Tafel 4. Figur 1: die Struktur besteht in einer Außenhaut über lamellaren Schichten Tafel 3. Figur 4. Tafel 4. Figur 6 (letztere mit der Lupe zu betrachten). Mittlerer Durchmesser der Urania 1 mm, Farbe smalteblau.

Diese Struktur wurde für den Blätterbruch des Bronzits gehalten! Ob Tafel 4. Figur 4 zu den Uranien gehört, ist zweifelhaft. Äußern Form und Farbe sprechen dafür. Die Anschnitte an beiden Seiten zeigen deutliche Struktur.

Tafel 5. Figur 5 zeigt vollständig gewundene Lappen. Entweder ist es ein Körper spiralförmig aufgewunden oder sind es mehrere Lappen, von welchen der äußere die inneren mantelartig umgibt.

Tafel 4. Figur 6 ist ein Querschnitt, welcher allerdings wenig zeigt. Im Objekt selbst sieht man den Durchschnitt der Außenhaut weis.

Tafel 5. Figur 2 zeigt so deutliche Schichtung, dass wenn die äußere Form nicht wäre, man versucht sein könnte, die Form zu den Korallen zu stellen.

Tafel 4. Figur 5 zeigt Querschnitte durch beide Flügel der Lappen.

Tafel 6. Figur 3 Lamellen-Struktur. Figur 5 und 6 können auch die einfachsten Crinoiden sein, deren Arme sich an einander angelegt haben. Hinsichtlich des Übergangs der Formen in andere muss ich auf das betreffende Kapitel verweisen.

Am rätselhaften ist Tafel 6. Figur 1 und 2. Bei Figur 1 ist die matte Stelle im Präparat gelb, die gestreifte blau. Ich habe sie neben Figur 2 gestellt, diese zeigt deutlich zwei Lappen, welche wie zwei Muschelschalen an einer Stelle verbunden sind und beim ersten Anblick auch vollkommen den Eindruck eines Zweischaligen machen. (Es ist nicht ein bloßer Anschnitt.) Denkt man an Muscheln, so könnte die matte Stelle von Figur 1 der Steinkern sein. Allein die Struktur ist eben Uranienartig.

Tafel 5. Figur 3. 2 Individuen zeigen die Struktur überaus deutlich, ebenso die Anwachsstellen. In Figur 4 (welche ein schlechtes Bild gibt) legen sich mehrere Individuen fächerartig aneinander.

Bei Tafel 3. Figur 3, IV. 1, 2 glaubt man oben eine runde Mundöffnung angedeutet zu sehen.

Hiernach halte ich die Urania für einen festgewachsenen Schwamm, welcher sich spiralförmig zusammenzieht, hiebe Wasser einsaugt und austreibt, wie unsere lebenden Schwämme.

Urania nimmt etwa 3/20 der Gesteins-Masse ein.
\clearpage
\subsection{Die Einzelnen Formen: Schw"amme --- Nadel-Schw"amme}
\paragraph{}
Tafel 7. Die Formen Figur 1, 2, 3, 5, 6 zeigen ein Nadelgerüste. Figur 1 stelle ich zu Astrospongia. Die Nadeln liegen regelmäßig gekreuzt. Figur 6 ist ein unregelmäßiges Nadelgerüste mit einem Hohlraum, welchen das Bild allerdings sehr schwach andeutet. Diese beiden Formen scheinen mir unzweifelhaft zu sein.

Annähernd sicher sind Figur 2 und 5 (in Figur 2 ist der weiße Strich ein Gesteinsriss).

Die Form Figur 4 habe ich bei der Zusammenstellung der Tafeln für einen Schwamm gehalten. Nachdem eine Änderung der Anordnung nicht mehr möglich war, erkannte ich in dieser Form den schiefen Durchschnitt eines Crinoiden und was ich Anfangs für Nadeln hielt — als feine Crinoidenarme. Ich bemerke, dass die Bestimmung sehr schwierig ist wegen der außerordentlich einfachen meteoritischen Crinoidenformen, weshalb eine Entscheidung weiterer Untersuchung aufgespart bleiben muss. Es lässt sich der Hohlraum der Schwammnadeln mit dem Nahrungskanal der Crinoidenarme verwechseln, wenn letztere gerade gestreckt liegen und die Glieder nicht mehr deutlich erhalten sind. Diese Tatsache, so wenig angenehm sie für den Untersucher der einzelnen Formen ist, ist um so lohnender für denjenigen, welcher dem Zusammenhang der Formen nachgeht — für den Nachweis der Entwickelung einer Form aus der andern. Es reicht immer eine an die andere hin. In günstigere Lage versetzen uns:
\clearpage
\subsection{Die Einzelnen Formen: Die Korallen\index{koralle}}
\paragraph{}
Hier haben wir so wohl erhaltene terrestrische Formen, dass ein Zweifel nicht übrig bleibt.

Tafel 8. zeigt ein Musterbild, Tafel 9. dessen Kanalstruktur: deutliche Knospen-Kanäle, welche die Röhren (denn solche sind es) verbinden. Dazu kommt die mit einem Blätterbruch absolut nicht zu verwechselnde Kurvenrichtung der Kanäle, dazu kommen die ganz deutlichen Röhrenöffnungen und endlich die ebenso deutliche Anwachsstelle. (Tafel 1. Figur 4 zeigt ein noch schärferes Bild desselben Objekts.) Leider geben Färbungen des Präparats dem Struktur-Bild Tafel 9. widerwärtigen Schatten. Die Knospen-Kanäle stehen 0,003 mm von einander ab. Gewiss alles, was man von einer Struktur eines Favositen verlangen kann.

Tafel 10. Figur 3, 4 zeigen uns das Bild des Favosites multiformis aus dem Silur so, dass man hier auch nicht einmal Spezies zu trennen vermöchte.

Auf Tafel 11. in Figur 1, 2, 3 (wo 2 auch die Anwachsstellen zeigt) wird jeder Forscher das Bild lebender Korallenformen leicht erkennen, umso mehr als in Figur 1 oben noch die Becherform (Hohlraum) angedeutet ist. Dasselbe Objekt zeigt ferner in den Röhren Querscheidewände, die klar hervortreten. Leider ist ein Teil des Bildes in Folge der gelben Färbung des Präparats in der Photographie durch Schwarz verdeckt.

Tafel 10. Figur 1 und 2 zeigen weniger gut erhaltene Quer- und Längsschnitte, doch hebt die ganz gleiche Wiederholung beider in mehreren Schliffen den Zweifel daran, dass es organische Formen sind, und sind es solche, so können es bloß Korallen sein. Figur 3 scheint eine Becher-Koralle zu sein, Figur 4 ist an dieselbe angewachsen. Dass Figur 6 Korallenstruktur hat, bedarf wohl keines Nachweises. Diese Form kehrt mehrfach wieder.

Tafel 11. Figur 4. Diese Form kehrt ebenfalls mehrfach wieder. Eigentümliche Korallenformen zeigen Figur 5 und 6. — Figur 5 ist gebildet aus Röhrenringen und höchstwahrscheinlich auch Figur 6. Ich bemerke, dass diese Form hundertmal wiederkehrt.

Bei höherer Vergrößerung zeigen Zwischenwände Tafel 11. Figur 1, 2, 3, 6.

Tafel 12. Figur 1, 2, 3 zeigen deutliche Lammellarstruktur. Die Querfurche in Figur 4 erinnert an Fungia. Wahrscheinlich gehören auch hierher Tafel 30. Figur 1, 2 und Tafel 20.

Die Übereinstimmung der Struktur in Tafel 20. mit Tafel 30. Figur 1 (in zwei verschiedenen Schliffen) würde allein hinreichen jeden Gedanken an eine unorganische Bildung auszuschließen. Überdies kehrt die Form in 350 Schliffen etwa zwanzigmal wieder.

Tafel 12. Figur 5 habe ich nur einmal gefunden. Im Original sind deutliche Lamellen, welche im Bilde bloß am unteren Teil hervortreten. Figur 6 ist ein milchweißes Objekt, daher undeutlich. Ich glaube Sternform zu erkennen und habe die Form deshalb als Sternkoralle hierher gestellt.

Tafel 13. Figur 1, 2, 3, 4 sind Korallen, welche ganz unzweifelhaft den Röhrenkorallen angehören. Es sind im Original deutlich zu unterscheiden: Glasartige Zwischenmasse, schwarze Röhrenwand, gelbe Füllmasse der Röhren, zuweilen sind beide letztere schwarz. Diese Form kommt hundertfältig vor und zwar in allen Chondriten. Figur 5 aus Lamellen zusammengesetzt zeigt deutliche Hohlräume und Figur 6 Röhren mit Zwischenwänden. Die Formen gehören zu den größten Formen: sie haben bis zu 3 mm. Durchmesser.

Tafel 25. 1 und 2. Die Form ist hier so ausgezeichnet erhalten, dass an dem Vorhandensein eines Organismus nicht gezweifelt werden kann, um so weniger, als sie in zwei Schliffen übereinstimmend vorkommt und auch sonst häufig wiederkehrt. Vergl. Tafel 2. links unten, Tafel 5. Figur 6. Ich habe die Formen Tafel 1. Figur 6 und Tafel 25. Figur 1, 2 in der Folge zu den Crinoiden gestellt; die Kanäle sind unzweifelhaft, die Querlinien lassen sich auch als Crinoiden-Glieder deuten. Man sieht Einschnitte, ferner sind die Arme geknickt, was sich bloß bei Crinoiden denken lässt.

Geknickte Arme zeigt auch Tafel 25. Figur 4. Von dieser Form sind mehrere Exemplare da, welche genau dasselbe Bild geben.

Während die Korallenformen etwa 1/20 des Volumens der Gesamtmasse des Chondrit-Gesteins einnehmen, bilden den Rest mit 16/20 — also den bei weitem größten Teil der ganzen Masse:
\clearpage
\subsection{Die Einzelnen Formen: Crinoiden\index{crinoid}}
\paragraph{}
Sie finden sich von der einfachsten Form eines gegliederten Armes bis zum ausgebildeten Crinoiden mit Stiel, Krone, Haupt- und Hilfsarmen. Ihre Erhaltung ist größtenteils sehr gut. Die Schwierigkeit liegt bloß in den tausenderlei Richtungen der Schnitte, welche immer verschiedene Bilder desselben Objekts geben. Die birnenförmigen Körper, welche man als Gläser ansah, sind Crinoiden-Kronen.

Ich stelle 4 Crinoiden in aufrechter Stellung und in großem Format in Tafel 16, XVII, XVIII, 19 dar und einen im Querschnitt Tafel 20.

Tafel 21. Figur 1, 2, 3, 4, 5 zeigt senkrechte Durchschnitte eines schon höher entwickelten Crinoiden. Es sind Hauptarme mit Hilfsarmen und deutlichen Gelenkflächen.

Tafel 21. Figur 3 zeigt Stiel und Krone. (2 und 4 doppelte Vergrößerung von 1 und 3.) Figur 5, aus einem andern Dünnschliffe, ist da, um die Übereinstimmung der Formen zu zeigen. In Figur 6 glaube ich die Mundöffnung in dem Höcker zwischen den Armen erhalten zu sehen.

Tafel 22. Figur 1, 3, 4, 5 und Tafel 23. Figur 1, 2 zeigen die Zahl 5 der Arme, sowie die Hilfsarme.

In Tafel 23. Figur 2 und 3 sieht man die Knickung der Arme durch Druck von oben.

Tafel 22. Figur 2 und 4 erinnern an Comatula.

Eine besondere Art sind die Crinoiden, welche bloß aus einer beliebigen Anzahl von Armen bestehen. Zu diesen rechne ich Tafel 23. Figur 4, 5, Tafel 24. 4, 5, 6, Tafel 26. (Es ist auf dem Bilde Tafel 24. Figur 6 in kleinerem Maßstab die Koralle aus Cabarras, Tafel 13. Figur 6.)

Tafel 29. Figur 1, 2, 3, 4, 5, 6 und Tafel 27. Figur 3 geben Bilder von Crinoiden von oben gesehen.

Tafel 27. Figur 2 und Tafel 29. Figur 4 zeigen Crinoiden von unten: hier tritt der Stielansatz als heller Punkt hervor. Diese Querschnitte kehren in dutzend Fällen in übereinstimmender Form wieder. (Man vergleiche auch Tafel 3. Figur 2 links oben. Bessere Durchschnitte kann man wohl nicht fordern: die Muskelschichten sind hier deutlich sichtbar.)

Eigentümliche Verschlingungen zeigen Tafel 26. Figur 1, 2, 3, 4.

Die deutlichsten Querschnitte geben Tafel 25. Figur 5 und 6. Ein Längsschnitt ist Tafel 27. Figur 3 mit geknickten Armen.

Tafel 24 Figur 1 und 2 sind Formen, welche ich anfangs für Korallen ansah.

Tafel 28. Figur 1 könnte doch diesen letzteren zuzuzählen sein (die Struktur sollte deutlicher erhalten sein, um endgültig zu entscheiden).

Etwas deutlicher ist Tafel 27. Figur 1: eine scheinbare Außenwand, welche aber nichts als der Durchschnitt des regelmäßig gelagerten Hauptarms ist.

Ein sehr schönes Bild gibt Tafel 30. Figur 3; ob Crinoid? ist zweifelhaft. Nur bemerke ich, dass die beiden Teile symmetrisch und die Arme nicht aneinander gelegt sind, sondern sich kreuzen.

Tafel 30. Figur 5 mit einem Anschnitt hatte ich anfangs zu den Uranien gestellt. Sie wird den Crinoiden zuzuzählen sein.

Tafel 31. Figur 1, 2, 3 sind offenbar dieselben Formen. In Figur 1 und 3 ist eine deutliche Furche wahrzunehmen, vielleicht die Stelle wo zwei Crinoiden-Arme sich aneinander legen. Im Polarisationsapparat tritt diese Furche noch deutlicher hervor. Figur 4, zwei Individuen zusammengelegt, ließe die Deutung auf Schwamm oder Koralle offen. Figur 5 mit Maschenstruktur in dem mittleren Teil, ein Gewebe von Gliedern, zeigt oben Arme mit deutlicher Struktur. Gehören diese Dinge zu stammen? Da die Form nur einmal vorkommt, wage ich keine Entscheidung. Auffallend ist nur die Ähnlichkeit des Mittelbildes mit der Struktur des Schreibersits im Meteoreisen. Figur 6 findet sich zweimal, weshalb ich beide Teile als zusammenhängend angesehen habe.

Dieselbe Maschenstruktur zeigt Tafel 30. Figur 6 bei Lupenvergrößerung. Die Struktur beider stimmt, wie erwähnt, mit der Struktur des Schreibersits in dem Meteoreisen und kehrt mehrmals wieder.

Wie ich schon im Eingang bemerkte, halte ich es nicht für meine Aufgabe Spezies zu machen. Meine Aufgabe war nur das Dasein von Organismen mit dem Nachweise geschlossener wiederkehrender Formen von organischer Struktur unzweifelhaft festzustellen. Dies glaube ich getan zu haben und ich denke, es sollte Niemand auch nur den mindesten Zweifel mehr hegen, (insbesondere nach dem Anblick eines Dünnschliffes im Original), dass es sich hier nicht um Mineralformen handle. Sind aber nur 5 organische Formen unzweifelhaft nachgewiesen, so sind auch die übrigen weniger gut erhaltenen Formen organisch.

Um endgültig Genera und gar Spezies festzustellen, gehört mehr Material und jahrelange Untersuchung dazu. (Für ersteres werde ich dankbar sein.) Vor Allem müsste ich mehr Zeit haben, als die Nachtstunden und mehr Kraft, als mir mein anstrengender Beruf übrig lässt, um die Arbeit zu vollenden. Doch meine ich den geforderten Punkt gegeben zu haben, auf welchem man stehen kann.

Zum Schluss verweise ich auf die Tafelerklärung.

Damit sind die Formen vorgeführt. Ich habe eine Zeitlang den Plan verfolgt, eine förmliche Statistik über das Vorkommen der Formen in meiner Dünnschliffsammlung zu machen, aufzuzählen, wie oft ein und dieselbe Form in den 500 Dünnschliffen sich findet. Ich stand davon ab, weil ich mir sagen musste, dass es doch keinen großen Wert haben werde. Jede Vermehrung meiner Sammlung um nur 12 Nummern würde die Verhältniszahl ändern. Ich zog daher vor, bei einzelnen Formen das Zahlenverhältnis annähernd anzugeben.
\clearpage
\subsection{Alles Leben}
\paragraph{}
Es sind im Vorstehenden die einzelnen Formen zur Anschauung gebracht. Alle diese Formen sind nicht tot eingebettet, sondern die eine aus der anderen gewachsen und in Wahrheit lebend vom Leben begraben. Hier kann freilich nur die Anschauung Überzeugung geben. — Zu diesem Zweck betrachte man in sämtlichen Bildern die einzelnen Formen mit ihrer Umgebung!

Was auf den ersten Blick auch nur als ein heller Fleck erscheint, bei genauerer Untersuchung zeigt es den Durchschnitt eines Schwamms, einer Koralle, oder eines Crinoidengliedes. Nirgends sind, wie Gümbel ganz richtig beobachtet hat, Zierstück, zerbrochene, abgerollte Formen, Splitter — auch ist kein Bindemittel zwischen denselben. Nur die Weichteile fehlen, alles Andere ist erhalten, wie es sich im Leben im Wasser bewegte. Die Crinoidenformen zeigen dies am deutlichsten. Denn auch diese sind höchstens auf die Seite gebogen, gewunden, selten geknickt; man sieht auch den nur schwachen mechanischen Wiederstand gegen den über dem Haupt entstandenen Nachbar. — Aber Alles aneinander, auseinander gewachsen, Nichts niedergelegt, Nichts tot eingebettet. Da ist auch keine Masse, welche ein Grab hätte bilden können.

Die Tatsache, dass nichts Unorganisches in dem Chondrit-Gestein und kein Raum ohne Leben darin ist, halte ich für ebenso bedeutend, als das Dasein der Organismen selbst. Diese Tatsache erst wirft auf die Entstehung des Planeten das volle Licht. Nimmt man hinzu, dass das Gestein, welche diese Bildungen einschließt, aus Mineralen besteht, welche dem sogenannten Urgebirge, ja "`vulkanischem"' Gebirge angehören: so muss unsere Geologie notwendig einen andern Weg in der Erklärung der Tatsachen einschlagen. Ich glaube nun freilich keineswegs, dass es Schwämme, Korallen, Crinoiden aus den Mineralen gegeben habe, welche heute die Formen bilden. Die Organismen müssen ursprünglich anders zusammengesetzt gewesen sein, müssen also eine Umwandlung erlitten haben.

So viel ist, denke ich, über allen Zweifel erhaben, dass das, was jetzt Hornblende, Augit, Olivin ist und die genannten Formen ausfüllt, früher in einem andern Zustand gewesen sein muss, nämlich eine flüssige, und zwar wasserflüssige Lösung.

Nun finden wir aber diese Minerale in unserem Urgebirge in Formen, welche nicht Kristalle, wohl aber den meteoritischen ähnlich sind. Wir finden Gebirgsmassen aus solchen Formen zusammengesetzt. Also waren es auch hier höchst wahrscheinlich organische Formen, nachher verwandelt in das, was wir jetzt Gestein nennen. Dieses Gestein weist aber auf eine Schichte, welche ganz unzweifelhaft mit der meteoritischen (den Chondriten) näher, ja nächst verwandt ist — den Olivin. Und unter diesem muss Eisen liegen: das bezeugt das spezifische Gewicht der Erde. — Wieder eine gleiche Tatsache sehen wir in den gefallenen Eisen-Meteoriten: hier, wie im Ovifak-Gestein finden wir Übergänge, Zusammensetzung von Eisen und Olivin.

Damit sind uns die größten Grundlinien der Geologie gegeben — wir haben die zeitliche Entwicklung des Erdkörpers. Die Formentwicklung — die Ursache der Entwicklung der Formen selbst ist damit zugleich aufgeschlossen. Ist der Organismus in den untersten Schichten, die wir kennen, die Ursache der Massenbildung, so wird er auch die Ursache des Anfangs des Planeten selbst gewesen sein.

Die Annahme einer bloßen Massenanziehung, der mechanische Anfang der Erde und der Weltkörper überhaupt, wäre damit widerlegt.

Allerdings müssten auch noch Organismen im Eisen, im Erdkern, in dem Meteoreisen nachgewiesen werden. Diese Aufgabe habe ich mir als nächste gestellt; die bisherigen Resultate lassen ihre Lösung hoffen.
\clearpage
\subsection{Stein im Stein}
\paragraph{}
Wenn ich gesagt habe: die Chondrite sind nichts als ein Tiergewebe, ein Tierfilz, so leidet dies eine Einschränkung.

Es kommen allerdings in diesem Tierknochengesteine ganz kleine, scharf umschriebene Stellen vor, welche von Anfang an wahrscheinlich (aber nicht notwendig) Gestein sind. Das sind blaugraue, seltene Einschlüsse von 3-5 mm. Durchmesser ohne bestimmt wiederkehrende Form, welche in der grauen Masse deutliche Kristalle eines gelbgrünlichen Minerals, dessen Durchschnitte das einmal Quadrate oder Rhomben, das andermal Sechsecke sind, einschließen. Dieses Mineral kann Augit oder Olivin sein. Hier spricht die Kristallform für ein Mineral. Allein das Vorhandensein solcher Teile spricht auch für meine Ansicht. Warum hätten sich die Kristalle nicht überall gleich gebildet? Und warum sollten nicht auch Hohlräume neben Organismen übrig bleiben? Sodann ist bekannt, dass auch bloße Füllmassen in organischen Formen nachträglich kristallisieren. Endlich finden sich aber auch in organischen Formen Ausfüllungen von Höhlen, welche sich in ihren Umrissen dem Aussehen von Flächen und Winkeln nähern.

Der Grund, warum ich diese Einschlüsse aber doch als unorganische Teile der Chondrite zugebe, als eigentlichen Meteorstein (Stein im Stein), ist, weil die Umrisse einen Anhaltspunkt nicht geben, um die Form als organische anzusprechen. Diese Einschlüsse können Einlagerungen einer schon gebildeten Gesteinsmasse sein oder könnten sie sich in den Hohlräumen erst gebildet haben.

Dass eine Schlammablagerung möglich, dass ein Hineinfallen von Teilen eines schon abgelagerten, also fertig gebildeten Gesteins möglich, sogar wahrscheinlich sei, braucht nicht geleugnet zu werden: es stößt die Tatsache nicht um, dass in den Olivinschichten organische Bildungen vorhanden und dass diese den Aufbau des Planetenkörpers bewirkt, den Bau selbst gebildet und zusammengesetzt haben.

Unter allen Umständen aber ist im Chondrit-Gestein das Verhältnis das umgekehrte wie bei den Sedimentschichten unserer Erde. In diesen sind die Organismen eingelagert, das Gestein umschließt sie; jenes ist eben nichts als Organismen und das Gestein ist eine Masse solcher. Ich füge ein Bild eines wirklichen Gesteinteils aus Borkut bei. Tafel 32. Figur 2. Daneben (Figur 1) habe ich eine Form abgebildet, graublau wie Urania, aber ohne bestimmte Struktur, auch in ihren Umrissen unbeständig, weshalb sie eine bloße Füllmasse sein könnte. Wäre sie eine organische Form, so wäre sie die eines niedersten Wesens. Zur Vergleichung bilde ich in Tafel 32. Figur 4 einen Dünnschliff von Lias $\gamma\delta$ (Zwischenkalk) ab. Hier liegen die Schalen zum Teil ganz im Kalke, größtenteils aber sind es bloß Stücke von Schalen; die Teile sind in alle Größen zerschlagen, und, was ihre Herkunft betrifft, gerollt bis zur Unkenntlichkeit. Im Chondrite bleibt fast keine Stelle, welche Zweifel über ihre Zusammensetzung übrig ließe.
\clearpage
\subsection{Fortpflanzung}
\paragraph{}
In den Steinen findet sich eine Unzahl runder und birnenförmiger Formen von 0,10 mm. — 0,50 mm. Durchmesser, mit kaum angedeuteter Struktur. Ich halte diese für die ersten Entwicklungsformen. Unter diesen hebt sich am meisten hervor eine Kugelform aus durchsichtigem Gestein, in der Mitte die Anfänge von Kanälen. Da finden sich Kugeln mit einem Kanal, mit zwei weiteren unterhalb und oberhalb des größeren, und so fort bis zu den Formen Tafel 13. Figur 1, 2, 3, 4. Die Sache ist hier, glaube ich, sicher. Diese Form lässt sich nicht nur in allen Chondriten nachweisen; in allen finden sich auch alle Entwicklungsstufen von einem bis zu 20 und mehr Kanälen: sie ist die häufigste und zugleich, wegen der deutlichen Struktur der Kanäle, sicherste. Sie hat sich deshalb auch in denjenigen Chondriten erhalten, welche die übrigen Formen kaum mehr zeigen. Die Entwicklung besteht also darin, dass sich die Kanäle vermehren.

Nun finden sich aber eine Menge von Kugel- und Birnenformen mit schwach angedeuteter Struktur. Sie scheinen aus Sarcode bestanden zu haben, als sie einst begraben wurden. Ich würde es nicht wagen, diese Formen hereinzuziehen, wenn sie nicht doch eine bestimmte Gliederung zeigten. Sie bestehen aus zwei, drei, vier, fünf lappenförmigen Armen und sind wahrscheinlich die Anfänge von Crinoiden.

Dass die Feststellung von Entwicklungsformen am schwierigsten ist, ist bekannt. Ich erlaube mir daher hier auch nicht zu weit vorzugreifen. Was ich hier sage, kann nur ein Fingerzeig für künftige Forschung sein.

Die gute Erhaltung ist eine Unmöglichkeit. Die meteoritischen Formen werden daher auch zum mindesten das Schicksal der lebenden teilen: es ist immer die letzte Arbeit, die ersten Anfänge der Entwicklung, die Embryonen festzustellen.

Nur einer Tatsache will ich hier noch erwähnen, welche zugleich ein erhebliches Beweismoment für die organische Natur der Formen ist: die immer auftretende Vergesellschaftung der einzelnen Formen. Die meisten Formen finden sich mit gleichen zusammen: wenige stehen einzeln und zugleich als Unica da. Ich halte dies für sehr wichtig. Wenn mehrere Individuen der gleichen Spezies sich zusammenfinden, so geht daraus hervor, dass sie im Mutter- oder Geschwisterverhältnisse stehen. Dieselbe Erscheinung tritt auch bei den terrestrischen Arten auf. Dies wird um so bedeutender, als oft das Mineral, aus welchem eine Form besteht, unzweifelhaft das gleiche ist mit dem eine andere Spezies ausfüllenden Mineral, also ein mineralogischer Grund nicht da ist, aus welchem die Verschiedenheit der Struktur abgeleitet werden könnte.
\clearpage
\subsection{Entwickelung}
\paragraph{}
Nachdem ich die einzelnen Formen dargestellt habe, habe ich auch ihr Verhältnis zu einander, die Entwickelung der Formen aus einander, zu besprechen.

Dass Urania die einfachste Form ist, ist sicher. Diese Form bildet aber auch den Anfang zu den folgenden.

Der halbrunde Lappen teilt sich in Schichten, diese Schichten in Röhren, die Röhren teilen sich quer — jetzt bilden sich Arme, welche ein Kanal verbindet. Es entwickelt sich eine Krone zwischen Armen und Anwachsstelle und der einfachste Crinoid ist da. — Mag diese Kette allzukühn geschlungen erscheinen, die Formen fordern unwillkürlich dazu auf. — Aber muss denn, wenn wir nur irgendwo in unseren lebenden Formen eine Entwicklungsreihe feststellen wollen, nicht auch hier dieselbe Wandlung vor sich gegangen sein? — Sicher. Nur, glaube ich, finden sich in den meteoritischen Formen mehr und viel sichtbarere Übergänge. Man kann den Stammvater des Pentacrinus Briareus auf unserer Erde nirgends anders suchen, als in den Korallen und gewiss darf man den Anfang der Korallen selbst in der Schwammform erblicken: sie ist entschieden eine niederer Form als die der Korallen.

Was der Meteorschöpfung die größte Wichtigkeit für die Entwicklungslehre gibt, ist nicht nur das Vorkommen von Tierformen in den tiefsten Schichten, sondern der einheitliche Typus aller meteoritischen Organismen. Dieses wird klar, wenn man hunderte von Dünn-Schliffen nach einander betrachtet. Die Größe der Organismen ist eine gleichartige, verhältnismäßig mindestens 1000 mal kleinere als die der Erde: die Entwickelung der einzelnen Formen erreicht annähernd einen gleichen Höhepunkt. Der Aufbau der Formen entspricht vollkommen den Umständen, unter welchen sie entstanden, nämlich der überaus kurzen Lebenszeit, welche sie gehabt haben können: es ist eine hastige, relativ unvollkommene Schöpfung. Der Crinoid ist der höchste Repräsentant dieser Tierwelt. Ich halte für den höchstentwickelten die Form Tafel 22. Figur 1, 3, 5, 6, weil er schon die Fünfzahl enthält.

Will man aber nicht so weit gehen, die Crinoiden nicht durch die Korallen hindurch ableiten, so bietet die Form der Urania selbst Anhaltspunkte. Ich habe noch einige Formen abgebildet, welche lose Glieder zeigen. Sie sind in der Beschreibung bezeichnet. Insbesondere fand ich bei höherer Vergrößerung übereinanderliegende Arme.

Auch hier reicht die Beobachtung im Einzelnen noch nicht hin, um abschließen zu können.
\clearpage
\section{Das Meteoreisen\index{meteorite!eisen}}
\paragraph{}
Ich habe schon in meiner \emph{Urzelle} darauf hingewiesen, dass die Struktur des Meteoreisens nichts anders sei, als die eines Filzes von einzelligen Pflanzen. Die sogenannten Widmannstätten'schen Figuren sind größtenteils nichts anderes als einzellige Pflanzen.

Ein Stück Meteoreisen von Toluca liegt mir vor, in welchen die zylindrischen Zellen eine aus der andern hervorgehen, häufig sind zwei kopuliert. Die einzelnen Zellen zeigen doppelte Zellwände (Bandeisen), zeigen Querscheidewände, zeigen deutliche runde Ansatzstellen; in manchen hat die Marksubstanz (wie man sie gar nannte), wirklich im Innern der Zellen noch Struktur. Die ganzen Zellen selbst liegen in einer matten Füllmasse (Fülleisen).

Man vergleiche mit diesen Figuren die Formen aus dem Liasschiefer, insbesondere Algacites granulatus und frage sich, welche von beiden Formen die Pflanzen-Struktur deutlicher zeigt, Toluca-Eisen oder die Alge aus Lias-Epsilon.

Diese Formen sind zylindrisch, mitunter sieht man (im Durchschnitt) annähernd polyedrische Flächen: sie haben Wandungen. Was sie aber ganz besonders von Kristallen unterscheidet (abgesehen von ihrer runden Form), sind die Anwachsstellen.

Kristalle, welche aneinander wachsen, setzen sich stets mit einer bestimmten Kristallfläche an eine andere ebenso bestimmte Fläche an, (Dendriten von Silber, Kupfer). Sie legen sich an die Fläche des andern an, ohne in sie einzudringen, Im Meteoreisen aber findet ein Eindringen statt. Der Querschnitt ist nicht eine gerade Linie (Kristallfläche), sondern eine Kurve.

Damit hört alle Ähnlichkeit mit Kristallen auf, außer man nähme an, dass es auf andern Planeten Zylinder-Kristalle gäbe, welche auseinander hervorwachsen. Die Behauptung, dass die Figuren bestimmte mathematische Lagen haben, mag stellenweise zufällig zutreffen; allein alle Forscher geben zu, dass die Winkel nirgends konstante sind, was bei den Dendriten stets der Fall ist. Findet man auch eine Stelle, woraus man ein Oktaeder, einen Würfel, oder eine andere reguläre Kristallform, oder auch ein Rhomboeder abzuleiten im Stande wäre: sofort ist die Ordnung daneben eine ganz andere. Und wie wollte man noch von Kristallgesetzen sprechen, wenn von demselben Mineral nicht einmal ein bestimmtes Kristall-System eingehalten wäre? Denn es finden sich, wie gesagt, rhomboedrische Schnitte neben regulären.

Ich finde nur zwei Einwürfe scheinbar begründet:

1. den Einwurf, dass die Figuren zuweilen Platten sind. —

Hiergegen möchte ieh einwenden, dass, wenn einmal Zylinderform nachgewiesen ist, die Formen eben keine Kristalle sind, und dass nun die Folge nicht ist, dass jene Zylinder Kristalle, sondern umgekehrt, dass die Platten, welche dieselbe Struktur tragen, keine Kristalle sind.

2. Der zweite Einwurf ist der: Wie sollen sich Pflanzen in Eisen verwandeln?

Dieser Einwurf ist nicht schwer zu widerlegen. Man denke nur an die meisten unserer verkieselten Versteinerungen, insbesondere die verkieselten Stämme im Lias; man erinnere sich der sogenannten Mansfelder Ähren im Zechstein (Cupressites Ulmanni), wo Cypressen in silberhaltiges Kupfer verwandelt sind. Man sollte meinen, ein solcher Einwand könne nicht gemacht werden.

Nun bin ich aber durch einen verehrten Freund, Professor Dr. H. Karsten in Schaffhausen, in der Lage, für die Verwandlung von Pflanzen in Eisen einen schlagenden Beweis aus der Jetztzeit beizubringen. Karsten hat schon im Jahre 1869 nachgewiesen, dass unsere niedersten Pflanzen in ganz hervorragender Weise Eisen aufnehmen; seiner Güte verdanke ich Eisenpflanzen von heute. Mit seiner Erlaubnis lasse ich einen Auszug aus seiner ausgezeichneten Schrift: \emph{Der Chemismus der Pflanzenzelle}, Wien 1869, S. 53 hier folgen:

"`Bringt man Oidium lactis oder Hefe, welche einige Zeit in mäßig feuchter Luft (nicht unter Flüssigkeit) mit Milchzucker in Berührung war, mit metallischem Eisen zusammen, indem man über die auf dem Objektträger vegetierende Milchhefe Eisenfeilspähne streut, so nehmen zuerst manche dieser das Eisen berührenden Zellen, später auch viele von demselben entferntliegenden, mehr oder minder rasch eine intensiv rote Farbe und bald auch eine erstaunliche Größe an."'

"`Man würde sich gezwungen glauben, die Ursache der merkwürdigen und außerordentlichen, oft sehr beschleunigten Vergrößerung allein nur in einem mechanischen Aufquellen der Zellhäute zu suchen, sähe man nicht zugleich die im Innern der hiebe zum Teil schichtig verdickten Mutterzelle unter den oben angedeuteten Kulturverhältnissen vorhandenen Tochterzellchen verhältnismäßig mit heranwachsen und sich so vermehren, dass sie die Mutterzelle gänzlich ausfüllen."'

"`Auch die Haut der Tochterzellchen produziert Säure, wie die Eisenreaktion erkennen lässt; ihre Gestalt ist nach der Verbindung ihrer Haut mit dem Eisen derjenigen der oben beschriebenen Protein-Kristalloide sehr ähnlich; wie diese sind sie flache, 3-4-5seitige, wenn auch weniger scharfkantige und eckige Täfelchen; unregelmäßig neben einanderliegend, füllen sie die große Zellhöhlung völlig aus, fallen aber, wenn die Haut der Mutterzelle zerbrochen wird, mehr oder minder mit einander vereinigt aus derselben hervor."'

"`Ähnliche Metamorphosen erfahren auch die Oidiummycellen, besonders die in die Luft hineinragenden zergliedernden Äste, wenn sie in ähnliche Verhältnisse gebracht werden, und zwar der Art, dass die verschiedenen Gliedzellen sich oft ungleich ausdehnen, meistens die oberen zuerst und mehr als die unteren, gewöhnlich stielrund bleibenden, sich etwas streckenden, wodurch diese Zweige mit ihren knopfförmig angeschwollenen Endzellen Mucor- oder später frucht- oder blumenähnlich werden, wenn sie die oberste vergrößerte Zelle am Scheidel deckelartig, oder von oben nach unten klappig anreißend zu öffnen beginnt. Die Häute der primären und sekundären Zellen zerreißen, jede in ihrer eigentümlichen Weise."'

"`Auch in Rücksicht auf die Organisation der Pflanzenzelle im Allgemeinen sind manche dieser Vegetationen der Milchsäurezellen von großem Interesse."'

"`Diejenigen nämlich, welche die oben beschriebenen Kristalloid-Zellchen enthalten, sind auch an der inneren Oberfläche jeder der beiden in einander geschachtelten Zellhäute, welche die Wandung bilden, mit einer Schichte kleiner Zellchen belegt, die, entweder eng beisammen liegend und an einander abgeplattet, oder etwas von einander entfernt, dem ganzen Zellsysteme das Ansehen und die Struktur einer kleinnetzig, warzig oder porös verdickten Parenchym-Zelle geben. De Cella vitali 1843. Ges. Beilage pag. 37 und 437. Diese Zellchen, morphologisch den Sekretion-Zellchen der zusammengesetzten Pflanze gleichwertig, wachsen gleichzeitig mit ihrer Mutterzelle zu der Größe heran, dass die zwischen der primären und sekundären Zelle liegenden eine Epidermis bilden. Das ganze Zellsystem ist oft höchst ähnlich, mit der Außenhaut vieler Pollen- und Diatomaceen- (Gallionella, Biddulphia, Coscinodiscus, Triceratium, Amphitetras etc.) Zellen."'

"`Wird ein solches von aufgenommenem Eisen rotgefärbtes Zellsystem in eine neue Mischung der oben bezeichneten Nährstofflösung ohne Eisen gelegt, so zerfällt es bald in seine Elemente. Die Zellchen, welche dasselbe zusammensetzen, sowohl die kristalloidischen Inhaltszellchen als auch die der Oberhaut beginnen sich abzurunden und sich etwas zu Vergrößerern; es entstehen neue Generationen in ihnen, die endlich frei werden, indem ihre Spezialmutterzelle verflüssigt wird, und so sieht man sie bei Monate hindurch fortgesetzter Beobachtung sich in der Weise der Unterhefe mikrosporonartig, d. h. durch Entwicklung freier Tochterzelien vermehren."'

"`Diese mit milchsaurem Eisen durchdrungenen, warzig verdickten Oidiumszellen waren es auch, an welchen ein Hervorwachsen von sehr langgestielten Inhaltszellchen, aus oder neben den Zellchen, welche die netzig-warzige Oberhaut darstellen, beobachtet wurde, nach Art des Micrococcus, der Vibrionenkeime."'

"`Auch Hyphomyzeten, besonders Penicillium und Botrytis, sowie Rhizopus gaben, nachdem sie einige Zeit mit Milchzucker ernährt vegetierten und darauf mit metallischem Eisen in Berührung gebracht wurden, sehr interessante Präparate, zum Teil ähnlich denen des Oidium mit angeschwollenen Gonidienketten oder Hyphengliedzellen. An den Gonidienketten von Penicillium schwellen in der Regel die obersten ältesten Gonidien zuerst etwas an, dann folgen nach und nach die unteren. Die in Milchzuckerlösung mit Nährstoffsalzen gesättigten und bald darauf mit Eisen in Berührung gebrachten Penicillium-Gonidien schwellen langsam an und entwickeln an der inneren Oberfläche ihrer nach und nach außerordentlich vergrößerten und verdickten Außenhaut zahlreiche Zellchen, die derselben ein netzigoder porös verdicktes Ansehen geben, so dass dadurch Formen entstehen, die den oben von Oidium beschriebenen, porös dickwandigen ähnlich sind. In andern Fällen füllen die Tochterzellen mehr die Höhlung an und werden einem mit Gonidien gefüllten Mucorköpfchen ähnlich."'

"`Sehr häufig finden sich auch hier wie bei Oidium, wenn es mager kultiviert war, inhaltsleere Zellen mit ganz glatten Wandungen. Nicht selten durchbricht die innere, mit milchsaurem Eisen durchtränkte Zelle die äußere einfache oder auch zellig-warzig-etc. verdickte Haut, welche abblättert oder zerspaltet, während jene hervorwächst."'

"`Die für diesen Zweck angestellten Kulturen dürfen nicht feucht gehalten, nur in feuchter Luft unternommen werden, da diese mit saurem Eisensalze durchdrungenen Vegetationen dem Zerfließen sehr ausgesetzt sind. Auch ohne solche vorgängige Kultur habe ich die Gliedzellchen und Gonidien genannter Schimmel, sowie im Staube enthaltene Micrococcus-Zellen und Vibrionenkeime in beschriebener Weise anschwellen sehen, wenn sie mit poliertem metallischem Eisen in Berührung gebracht wurden, ohne Zweifel, weil diese Zellchen Säuren oder saure Salze enthielten."'

"`Wird es aus den eben mitgeteilten Erscheinungen des Wachstums dieser Pilzzellen ersichtlich, dass es deren assimilierende Membranen sind, welche die zerfließende Säure bilden, so ist die Ursache der abnormen Vergrößerung dieser Zellen in der nachträglichen Verbindung dieser Säure mit dem neutralen milchsauren Eisen zu einem sauren Salze zu suchen, so dass also die ganze Erscheinung der merkwürdigen Missbildung auf einem rein chemischen Prozesse beruht, der denjenigen, welcher in den unter normalen Bedingungen vegetierenden Zellen stattfindet, in der Weise ändert, dass die normale Entwicklung eine krankhafte wird, welche die endliche Zerstörung des Organismus herbeiführt."'

"`Gegen die Idee, dass die Säure hier bei den Pilzen ebenso wie das Harz, Wachs etc. durch die Assimilitations-Tätigkeit der Zellmenbran entstehe, könnte noch das Bedenken erhoben werden, dass es vielleicht die Sekretionszellchen (Microgonidien, Vibrionenkeime) allein seien, welche zwischen diesen Membranen des Zellensystems (der in einander geschachtelten Zellen 1., 2., 3. etc. Grads) wie oben bemerkt eingeschlossen, diese organischen Säuren durch ihre vegetative Tätigkeit erzeugen, um so mehr, da ohne Zweifel die Vibrionen, die sich aus ihnen entwickeln, auch bei völliger Abwesenheit von entwickelteren Zellenformen sehr energische Erzeuger von Säuren, z. B. von Milch-, Butter-, Essigsäure sind. Dagegen sprechen jedoch diejenigen durch Aufnahme von Eisen in gleicher Weise vergrößerten Zellen, deren Wandung durchaus strukturlos ist, d. h. ohne erkennbar zellige Organisationen zwischen den beiden sie zusammensetzenden Membranen der in einander geschachtelten Zellen und ohne eingeschlossene freie Zellchen in ihrer Höhlung; ferner die Tatsache, dass von dem Oidium-Mycelium und deren Hefezellen, wenn dieselben untergetaucht sich entwickeln, zuerst die Membranen, dann erst der flüssige Inhalt, der sich außerhalb der Kernzelle befindet, durch Eisen- und Schwefel-Ammonium geschwärzt werden. Gegen andere Metalle, gegen Aluminium, Magnesium, Zink, Kobalt, Nickel, selbst gegen Kupfer verhalten sich diese Milchsäurezellen ähnlich wie gegen Eisen, bilden mit demselben jedoch farblose oder nur schwach gefärbte, zum Teil (besonders mit Kupfer) sehr leicht zerfließliche Organisationen. Zu Versuchen mit dieser Säurehefe sind daher diese Metalle weniger günstig."'

Ich denke, wenn vor unserem Auge Eisenpflanzen entstehen, sollte man ein Bedenken gegen die Annahme desselben Vorgangs zu einer früheren Zeit, zu einer Zeit, als sämtliche Stoffe der organischen Bildung zur Verfügung waren, nicht erheben. Haben wir heute noch Massenbildungen vor uns in den Atollen des stillen Meeres, haben wir in den Chondriten die Zusammensetzung aus ähnlichen Tieren, wie dort nachgewiesen: was steht im Wege, vorhergehende Pflanzenmassenbildungen anzunehmen?

Endlich haben wir in der Hefebildung einen Vorgang, welcher vollständig analog ist, sobald nur die Gluthitze weggedacht wird.

Ich komme hier auf die Kant-Laplace'sche Hypothese von der Massenbildung zurück. Oben schon habe ich ihren logischen Fehler erwiesen. Wie will man aus der Dunstmasse, welche sicher auch das Wasser einschloss, einen glühenden Ball herausbringen? Oder soll die Erde erst, nachdem sie gebildet war, in Glut gekommen sein? Nun wodurch? Die Erfahrung spricht bloß für Massenbildung auf organischem Wege. Offenbar hat nur der Anblick der Vulkane dazu geführt, ein feuerflüssiges Erdinneres anzunehmen, und diese Vorstellung führte zu der Annahme, dass die ganze Erde einmal in diesem Zustande gewesen und dass die plutonischen Gesteine die Produkte jener Periode seien. Auch ist es ja keineswegs gewiss, dass der Wärmestrahl der Sonne von einem feuerflüssigen Körper herrühre. Wenn aber auch, so spricht eben die Tatsache der Loslösung unserer Erde mit dem Wasser und insbesondere des Mondes (ohne Atmosphäre!) dafür, dass die Masse von Anfang an eine feuerflüssige feste Masse nicht gewesen und eine solche auch nicht geworden sein kann.

Soviel ist jedenfalls gewiss, dass das Meteoreisen nicht ein Schmelzprodukt ist, und was sollte das Meteoreisen in Glut versetzt haben? Ich habe auch im Meteoreisen Crinoiden- und Schwammformen gefunden. Ganz unzweifelhaft zeigt Hainholz solche.

Zeigen aber schon die Pallasite organische und sogar tierische Formen, Gesteine, welche den Übergang von reinem Eisen zum Chondrit bilden, so ist auch kein Grund vorhanden, das reine Eisen für eine unorganische Bildung, noch weniger aber, einen ehemals flüssigen Zustand desselben anzunehmen.

Sobald das Eisen als Planetenkern angenommen wird, glaube ich es hiermit aber als im höchsten Grade wahrscheinlich aussprechen zu dürfen, dass der erste Anfang unseres und daher aller Planeten eine organische Bildungwar.
\clearpage
\section{Das Eisen von Ovifak\index{Ovifak}}
\paragraph{}
Durch die Güte des Herrn Professors Dr. von Nordenskjöld wurden mir 6 Stücke des Eisens von Ovifak und des Basalts, in welchem dasselbe gefunden wurde, zur Untersuchung gegeben.

[Friedrich] Wöhler (Neues Jahrbuch für Mineralogie 1869, S. 32) hält es auf Grund seiner chemischen Zusammensetzung nicht für meteoritisch. Das Vorkommen eines der mir vorliegenden Stücke in einer Kluft spricht ebenfalls nicht für meteoritischen Ursprung. Eisenteile mit Widmannstätten'schen Figuren finden sich auch im Basalt und im Olivingestein eingewachsen, und doch werden beide nicht als meteoritisch angesprochen. Endlich finden sich völlige Übergänge von Stein in Eisen, woraus hervorgeht, dass das Eisen nicht zufällig in den Basalt gefallen ist. Es wäre doch ein großes Wunder, wenn dieses Eisen gerade zu der Zeit, als der Basalt flüssig war, in denselben gefallen wäre, ganz abgesehen davon, dass dieses Eisen, wie festgestellt ist, sich kaum einige Jahre erhalten würde. — Und doch soll dieses Eisen seiner Struktur wegen meteoritisch sein.

Wir wissen aber, dass unser Erdkern mindestens von der Dichtigkeit dieses Metalls ist, und es wird derselbe wahrscheinlich auch aus Eisen von derselben Beschaffenheit bestehen, so dass die Wahrscheinlichkeit nahe läge, dass wir in dem Eisen von Ovifak den Eisenkern der Erde zu Tage treten sehen.

Damit wäre uns unendlich mehr gewonnen, als mit einem neuen Meteoriten.

Auf der Fläche dieses Eisens, das ich freilich, da ich dieses schreibe, anzugreifen die Erlaubnis noch nicht habe, finde ich Strukturen, welche denen der Crinoiden in den Chondriten sehr ähnlich sind.

Eine Untersuchung im Dünnschliffe aber muss ich auf die Zeit aufsparen, wo mir das Material zur freien Verfügung gestellt wird.
\clearpage
\section{Schlussfolgerungen}
\subsection{Ursprung der Meteorite}
\paragraph{}
Dass kleine Planeten, Planeten im Gewicht von 1/2 Kilogramm auf die Erde fallen und solche daher auch kreisen, ist ganz gewiss. Es lassen sich nun folgende Möglichkeiten denken:

1. die Meteorite kreisen außerhalb des Sonnensystems (ein solcher will einmal von Petit in Toulouse beobachtet worden sein),

2. die Meteorite kreisen innerhalb des Sonnensystems und zwar: für sich um die Sonne, — um die Sonne mit Planeten (vielleicht also auch einzelne mit der Erde) — um die Sonne, die Planeten und deren Trabanten,

3. die Meteorite kreisen in allen diesen Bahnen.

Man weis aus langjährigen Beobachtungen jetzt sicher, dass in gewissen Zeitabschnitten (10. August, 13. November) Schwärme von Meteoriten unserer Erde sich nähern und unsere Erdbahn schneiden; weis dass diese Schwärme in gewissen Jahren zahlreicher sind, als in andern, weis, dass einzelne Meteorite auf unsere Erde fallen, eine Tatsache, welche ihren Grund in der Anziehung der Erde hat. — Die Bahnen der Meteorite aber sind noch nicht festgestellt, weder die der Schwärme, noch die von einzelnen; weder von solchen, welche gefallen, noch von solchen, welche bloß an der Erde vorbeigezogen sind. Somit lässt sich aus den Bahnen, welche man nicht kennt, nichts für die Entstehung der Meteoriten ableiten.

Nun fragt es sich, was aus der Zusammensetzung der Meteorite folgt. Ihre chemischen Elemente sind dieselben, wie die unserer Erde. Diese Tatsache lässt sich nun auf gemeinsame Entstehung, also darauf deuten, dass die Erde mit den Meteoriten Eine Masse gebildet habe, wie darauf, dass die Entstehung und Entwicklung aller Planeten dieselbe sei. Die bloße Tatsache der chemischen Gleichheit lässt also verschiedene Folgerungen offen. Nun habe ich aber irdische Organismen in den Meteoriten nachgewiesen und es kann noch nicht einmal als gewiss angenommen werden, dass die nicht übereinstimmenden auf der Erde nicht auch vorkommen. — Zu meinem Bedauern muss ich es gestehen, dass die Zahl der Zweifel durch meine Entdeckung eben nur vermehrt worden ist.

Aufs Neue erheben sich jetzt die Fragen: Entstanden die Meteorite mit der Erde? Kommen sie von der Erde? Waren sie also von Anfang an mit der Erde eine Masse und wurden von ihr getrennt, so dass sie vielleicht eine Art unsichtbarer Trabanten derselben gewesen wären oder gar noch sind?

Ich hebe zunächst nur diese Fragen hervor, denn sie sind für die Geologie die wichtigsten. Das spezifische Gewicht der Erde und das Gestein von Ovifak machen es wahrscheinlich, dass die Erde ganz aus denselben Gesteinen zusammengesetzt ist wie die Meteorite, vorausgesetzt, dass Eisen- und Stein-Meteorite zusammengehören. Daraus ließe sich schließen, dass die Meteorite ursprünglich ein Teil der Erde gewesen, und zwar zur Zeit, als die Erdbildung bis zu den Olivinschichten vorgeschritten war, und dass sie jetzt erst von ihr losgelöst worden seien. Letzteres müsste geschehen sein in Folge des Stoßes eines Weltkörpers auf die Erde, denn ohne einen solchen wäre eine Trennung nicht zu erklären, es müsste denn die Erdanziehung plötzlich aufgehört, oder doch in so hohem Grade sich gemindert haben, dass ein Teil ihrer Masse aus ihrem Anziehungskreis hinausgeschleudert werden konnte. — An ein Zerspringen, also an einen Stoß von innen durch Gaskraft und dergleichen ist schwer zu glauben, obgleich auch das nicht völlig ausgeschlossen wäre.

Man kann also auch jetzt aus chemischen und morphologischen Gründen so wenig als aus der Gesteinsbeschaffenheit einen Schluss ziehen, ob die Meteorite Kinder oder Brüder der Erde sind und man ist zunächst auf den Ausspruch des Astronomen angewiesen.

Wenn nun aber dieser bestätigt, dass die Meteorite vermöge ihrer Bahnen nicht ein Teil der Erdmasse gewesen sein können, so treten zweitens die Fragen ein: wie verhalten sich die einzelnen Fälle zu einander? Sind die Steine und Eisen ursprünglich zusammengehörig, oder haben Steine und Eisen verschiedenen Ursprung? Und drittens wäre die Frage: haben wenigstens die chemisch und morphologisch gleichen Steine Einem Planeten angehört, welcher durch irgend eine Ursache in Trümmer ging?

Letzteres könnte auf den ersten Anblick eben aus der chemisch-morphologischen Ähnlichkeit gefolgert werden und in der Tat, die Sache schiene ganz einfach und klar. Aber es wäre doch noch eine andere Möglichkeit, die Möglichkeit, dass unter gleichen Bedingungen sich eine Unzahl kleiner Planeten bilden könnte und vielleicht heute noch bildet. Die Stücke wären dann nicht Trümmer, sondern eigene Weltkörper.

Eisen und Steine könnten nun eigene Weltkörper sein — die Größe allein stünde der Annahme nicht im Wege. — Wenn aber die kleinen Massen aus Wassergeschöpfen bestehen und sie bestehen ja auch aus einer bloß mikroskopischen Schöpfung — so fragt es sich: lebten diese im Wasser oder im Wasserdampf? Genügte ihnen ein fortwährender Niederschlag von Wasser, wie wir ihn sehr leicht uns denken können, da wir heute noch Gegenden auf unserer Erde haben, wo stets Regen fällt wie in anderen kein Regentropfen. Dieser Frage ist entgegen zu halten, dass auch zu der mikroskopischen Schöpfung Baustoffe notwendig waren, welche nicht unter, sondern über den Geschöpfen gesucht werden müssen, denn nur aus wässrigen Lösungen konnte sich die mikroskopische Tierwelt aufbauen.

Diese Tierwelt ist aber schon eine wenigstens zum Teil höher organisierte. Eine einzellige Pflanze, ein Hefenpilz mag der Anfang eines Planeten gewesen sein: ein Crinoid konnte es aus inneren Gründen nicht sein, denn hier müssen wir einen längeren Zeitraum und daher auch eine größere Masse uns denken, durch welche diese Stufe der Entwicklung erreicht werden konnte.

Diese Tatsachen leiten uns in Verbindung mit der Wahrscheinlichkeit, dass Eisen der Kern des Chondrit-Planeten gewesen sei, dahin: die Chondrite als Trümmer eines und desselben Weltkörpers anzusehen, Trümmer, welche nach der Zerstörung des Planeten kreisten, bis sie glücklicherweise in den Fallkreis unserer Erde kamen. Auch die Formen der Meteorite selbst sprechen endlich für Trümmer.

Wir haben also nur eine hypothetische Gewissheit: nämlich die Wahrscheinlichkeit der ursprünglichen Zusammengehörigkeit der zu uns gelangten Trümmer.

Sollten sie aber auch von unserer Erde gekommen, Teile derselben gewesen sein: ihre Zusammensetzung aus Organismen ist immerhin noch eine Tatsache, welche wichtig genug wäre für unsere Erdgeschichte. Stammen sie aber nicht von der Erde, so geben sie uns die Erklärung zweier Tatsachen: die Entstehung eines Planeten und die Wahrscheinlichkeit für die Art und Weise der Entstehung unserer Erde. Waren sie aber jeder ein Planet für sich, so bezeugen sie eine Schöpfungskraft, welche wirklich unsere Begriffe von der Entstehung organischer Formen und deren Verlauf weit hinter sich ließe.
\clearpage
\subsection{Die Erdbildung}
\paragraph{}
Anschließend an die bisherigen Resultate ließen sich auch für die Erdbildung einige Schlüsse ziehen. Höchst wahrscheinlich zeigt der Erddurchschnitt dieselbe Gesteins-Reihenfolge, wie die Meteorite, welche vom Eisen zum Pallasite (Olivin mit Eisen), von da zu Olivin-, Enstatit-, (Feldspat)-Gestein übergehen.

Auf der Erde folgt dem Olivin der Granit, ein Feldspatgestein: diese Reihenfolge entspricht auch dem spezifischen Gewicht der Minerale.

Es haben Hornblende 3-3,40, Olivin 3,35, Enstatit 3,10-3,29, Orthoklas 2,53-3,10, Quarz 2-2,80 spezifisches Gewicht. Das hohe spezifische Gewicht der Hornblende rührt offenbar noch von dem Eisengehalte her. Diese Aufeinanderfolge im Gewicht, wie in der Lagerung spricht ebenfalls entschieden für Bildung im Wasser, in wässeriger Lösung. Hier muss ich wiederholen, was ich schon in der \emph{Urzelle} sagte: die Schöpfung, d. h. die organische Bildung kann nicht mit den Krebsen (Trilobiten) angefangen haben. Wir finden ja überall in den späteren Schichten eine stete Entwicklungsreihe der Formen, warum sollte bloß im Anfang dieses Gesetz nicht gewaltet haben?

Schon dieses würde zu der Annahme des organischen Ursprungs der unmittelbaren Vorläufer des Silur, des Gneises und des Granits führen.

Mit dem Beweise der organischen Zusammensetzung der Chondrite ist das Hauptargument gefallen, welches bis daher im Wege stunde, den Granit für ein Wassergebilde anzusehen: beide Gesteine enthalten vorzugsweise Feldspat. — Was den Granit betrifft, so habe ich Formen darin gefunden, welche denen der Chondrite ähnlich sind.

Ich will hier zum Beweis des Ursprungs des Granits nicht nur aus Wasser, sondern aus Organismen, einige Punkte nachtragen. Feldspat und Quarz kristallisieren, ich möchte sagen, leidenschaftlich. Im Granit finden sich aber beide Minerale regelmäßig nicht kristallisiert; der Feldspat zeigt bloß einen Blätterbruch. Einen solchen zeigt aber auch jede in Kalk verwandelte Versteinerung, z. B. ein Crinoidenstiel. Warum kommt der Feldspat im Granit nicht kristallisiert vor? Weil er durch eine stärkere formbildende Kraft gebunden war. Der Feldspat des Granits (wo letzterer wirklich erhalten ist) zeigt ferner stets bestimmte, stets wiederkehrende Formen, nicht Konglomerat- oder Roll-, auch, wie ich bemerkte, keine Kristall-Formen. — Auch hier wächst immer eine Form aus der andern heraus. Diese Formen sind Schwammformen. Der Quarz füllt die Hohlräume.

Auch auf die Gebirgsbildung möchte ich hinweisen. Dr. [Friedrich Moritz] Stapff, welcher den Gebirgsbau im Gotthard-Tunnel gewiss zur Genüge beobachtet hat, erklärt (Neues Jahrbuch für Mineralogie 1869, S. 792), dass er keine Spur einer Massen-Hebung oder Zertrümmerung im Gotthard-Tunnel, dem größten Aufschluss des Erdinnern den man kennt, beobachtet habe. Dieses "`Urgebirge"' ist nach seiner Feststellung ein Sedimentgebirge. Ja! es ist sogar denkbar, dass es sich gebildet hat, als unsere Atmosphäre noch den größten Teil des Wassers in sich gefasst hielt, eine Atmosphäre, welche nicht durch Feuer im Erdinnern, wohl aber durch die chemische Wärme mehr erwärmt war als sie es heute ist. Ist dem aber so, so bleibt für die Entstehung der Urgesteine, wie Urgebirge kein Erklärungsgrund als das organische Leben.

Heute noch können niedere Tiere und Pflanzen einen Hitzegrad ertragen, welcher für andere Wesen absolut tödlich wirkt, somit steht auch der Annahme organischen Lebens bei erhöhtem Wärmegrad nichts im Wege. Apatit und Graphit können ebenfalls als Zeugen organischer Tätigkeit gelten. Mit dem Niederschlag der Kieselerde (Kieselsäure) war das Erdgerippe fertig: es bestand aus den Knochen der abgestorbenen Tiere; Ton, Kalk, Salz nebst Gasen und Wasser bildeten nun die Baustoffe für die fernere Tätigkeit auf der Erd-Oberfläche. Weil dieser (nicht Erstarrungs-, sondern Niederschlags-) Prozess in der Hauptsache abgeschlossen war, erhielt nun der Organismus Raum und Zeit zu einer höheren Entwicklung, welche bis dahin unmöglich war, denn jede neue Bildung begrub die kaum entstandene. Jetzt erst, nachdem eine schwer lösliche Verbindung als Mantel um die Erde gelegt war, konnte die Formen-Entwicklung in ihre Rechte eintreten. Die Erdperioden wurden jetzt länger; mit dem Vorrat an feineren Baustoffen kam das Gesetz der Symmetrie in Geltung. Aber noch eine weitere Ursache trat hinzu: die niedersten Organismen sind Kinder der Nacht; ein Pilz erstirbt im Licht der Sonne. Die ganze bisherige Schöpfung, bis zum Niederschlag der dichteren Baustoffe, war eine Nachtschöpfung: die fortwährenden chemischen Verbindungen mussten eine Wärme erzeugen, welche dem Wasser nicht gestattete, in dem Grade zum Meere zu werden wie heute. Endlich waren die chemischen Verbindungen in der Hauptsache abgeschlossen und es war dadurch eine Oberfläche, eine Art Schale geschaffen. Jetzt aber trat der Licht- und Wärmestrahl der Sonne in Wirkung, welchem bis dahin der Weg bis zur festeren Oberfläche durch eine hohe und dichte Atmosphäre verschlossen war. Es beginnt die Lichtschöpfung; das Königreich der Sonne hat das Reich der Nacht auf unserem Erdball überwunden, hat die Nacht in die Tiefen der Erde gebannt.

So, durch das Licht, erklärt sich nun auch das mit dem Silur plötzlich und mächtig hervortretende höhere Leben: es war der erste Ruhepunkt der Schöpfung. Unter dem Einfluss des Lichtes sehen wir nun eine Entwicklung beginnen, welche so weit von der früheren Abstand, als heute das Leben am Pol absteht von dem am Äquator. So erklärt sich auch die plötzliche Änderung. Hätte es sich bloß um Abkühlung gehandelt, so müsste die Schöpfung einen viel langsameren Übergang aufweisen. Was nach dem Niederschlag des Magnesium, Silicium, Kalium, Natrium noch im Wasser gelöst blieb, war verhältnismäßig wenig; hier konnte nun das Licht anfangen zu wirken. Durch diese Annahme erklärt sich allein, dass das Leben auf der ganzen Erde, dass auch auf ihrer ganzen Oberfläche Wasser war, sowie dass Wassertiere noch Gebirge aufbauen konnten, welche weit über den jetzigen Spiegel des Meeres reichen. Diese Gebirge sind nicht gehoben, auch nicht nach mechanischem Gesetze (durch Schwungkraft) hinaufgetrieben, ebensowenig durch Erkaltung der Oberfläche herausgepresst worden; denn als Letztere erkaltete (richtiger "`vertrocknete"'), konnten höchstens Sprünge und Klüfte entstehen und unter der Oberfläche war kein Brei, sondern feste Masse. Was ist nun nach meinen jetzigen Feststellungen Oberfläche, jetzt nachdem die Grenze des Urgebirgs und der folgenden Schichten aufgehoben ist?* Was diese Schichte hinsichtlich ihrer Schöpfung von dem Urgebirge scheidet, ist nur die Wirkung des Lichts, welche um so stärker werden musste, je mehr sich die Wasserdämpfe verdichteten und das Wasser die Klüfte des Erdballs ausfüllte.

Nun aber wären die Tage der Erde doch gezählt gewesen, wenn nicht eben durch das Licht gesorgt worden wäre, dass der Niederschlagsprocess sich nicht rasch vollendet, dass die einzigen noch übrigen chemischen Verbindungen sich nicht rasch vollzogen hätten und damit das Leben der Erde und auf der Erde für ewig zum Stillstand gebracht gewesen wäre. Die Schöpfungen des Lichts waren neue, höhere Organismen. Diese Organismen bauten sich auf aus den noch nicht in organische Verbindungen getretenen Abfallstoffen der bisherigen Schöpfung und dadurch wurde dem Tode Halt geboten. Dieser wäre eingetreten und die Erde wäre zur Wüste geworden, wenn nicht eben die durch das Licht geschaffenen Organismen mit ihrer Nahrung und durch ihre Einatmung Verbindungen eingingen und solche wieder lösten und so einen Kreislauf, Leben genannt, bewirkten. Es ist also das Licht, welches unsere Erde vor dem Tode schützt, der auf ihrem Satelliten schon eingetreten zu sein scheint. Das Licht aber wirkt durch das Wasser. Das Wasser verbindet den Stein und den Äther, und dies eröffnet uns den Blick in die Zukunft unseres Planeten.

*Man hat bei der Hebungstheorie vergessen, dass eine Gewalt, welche nötig wäre, um Gebirge zu heben, diese zugleich zermalmt hätte: bei der Pressungstheorie ist man nicht im Stande zu sagen, wo denn eigentlich das Gebirge geblieben ist, durch welches "`der Brei"' gepresst worden wäre! Die ganze Oberfläche kann doch nicht herausgepresst worden sein.
\clearpage
\subsection{Die Zukunft Unseres Planeten}
\paragraph{}
Der Fall von Planeten-Trümmern auf unsere Erde, (für diesen Ursprung der Meteorite sprechen die meisten Gründe) ließe ein mechanisches Enden, einen gewaltsamen Tod auch für unsere Erde fürchten. Geschah es jenem oder jenen Planeten, von welchen die Meteorite herrühren, dass sie zertrümmert wurden, und zwar wurden sie es wohl nicht durch eine Kraft von innen, sondern durch Anstoß von außen: so müssten wir darauf gefasst sein, dass auch unserer Erde einmal dieses Schicksal widerfahren werde, wenigstens drohte es uns. Ich muss es den Astronomen überlassen, sich und ihre Zeitgenossen darüber zu trösten.

Aber auch auf das andere, oben schon angedeutete Aufhören des Lebens auf der Oberfläche müssten wir gefasst sein, allerdings ein weniger blutiges, aber darum nicht tröstlicheres Ende, nämlich auf das Schicksal des allmählichen Absterbens, des Erlöschens der Lebenskraft durch die Verbindung der Baustoffe zu unlöslichen Verbindungen: wür müssten fürchten, es werde unsere Atmosphäre in der Bildung unlöslicher Verbindungen aus den noch übrigen Baustoffen fortfahren und es werde mit dem Verlust an verfügbarem Baustoff der Kreislauf ein stets schwächerer und langsamerer werden und endlich — aufhören.

Vor diesem sonst fast vorausberechenbaren Verlaufe bewahrt uns einzig und allein — das Wasser; das Wasser, welches unsere Erde in ihrer Bildung sich anzueignen und festzuhalten vermochte.

Dadurch, dass die geschaffenen Wesen selbst die Verbindungen wieder lösen, welche sich in ihren Körpern bilden — dass also insbesondere die Pflanze das was sie aufsaugt, selbst wieder zerlegt, während das Thier diese Ausscheidungen in sich aufnimmt, um sie dann alsbald wieder auszuscheiden und der Pflanze (nicht dem Boden) zurückzugeben: durch all diess ist ein Kreislauf geschaffen, dessen Ende nicht abzusehen ist.

Dieser Vorgang und nicht die Abkühlung der Erdrinde, von welcher so viel geredet worden ist, bildet die wahre Geschichte unserer Erdoberfläche. Allerdings scheinen wir an unserem Trabanten, dem Mond, ein schreckendes Beispiel zu haben: Dort, glaube ich, ist das Leben erloschen. Nicht Meere sind dort, wie man glaubte und nicht Vulkane waren es, sondern der Mangel oder der Verlust des Wassers wird es gewesen sein, was diesem Planeten einen vorzeitigen Tod bereitete, was das Leben bald nach der Geburt wieder verlöschen ließ.*

Die Wärme auf unserer Erdoberfläche scheint mir mehr von der Erhaltung der die Kälte des Weltraums abwehrenden Atmosphäre abzuhängen. Die größere Höhe der Erdatmosphäre am Äquator in Folge der Drehung der Erde und nicht der Ausfallswinkel der Sonnenstrahlen allein ist die Ursache der dort höheren und konstanten Wärme: sonst wäre unter dem Äquator 500m über dem Meere nicht schon eine Abkühlung von mehreren Graden Durchschnittswärme; sonst müsste die Schneemasse des Chimborasso sofort schmelzen.

Mag nun auch die Wärme in Folge der vom Wasser vermittelten chemischen Prozesse mit der Zeit abnehmen, soviel ist gewiss, dass unsere Erdoberfläche ohne den schützenden Mantel der Atmosphäre, trotzdem sie tagtäglich neue Sonnenwärme aufnimmt, doch schon bei Nacht einer so niederen Temperatur verfiele, dass sie das Leben nicht erhalten könnte, wie dies neuerer Zeit als Ursache des Erlöschens alles Lebens auf dem Monde behauptet wird.

Die Wärme strömt uns von der Sonne zu und wird durch die Atmosphäre zurückgehalten, so dass sie nicht sofort, wie sie da ist, wieder in den Weltraum ausströmen kann. So sind wir von einem doppelten, schützenden Mantel umgeben: der Erdrinde, welche die Wärme aufsaugt, und der Luft, welche sie zurückhält, (sie ist das Kleid der Erde), und zwischen beiden leben wir, lebt die ganze Schöpfung im steten Austausch der Stoffe. Hier lebt der Mensch, hier entstehen dieselben Wesen, welche einst den ersten Grundstein zum großen Bau der Erde gelegt haben. Und gerade diese niederen Wesen bezeugen heute noch durch ihre riesenhafte Vermehrung, durch ihre Erhaltung in einer Temperatur, in welcher höhere Wesen sofort sterben, dass sie fähig waren die ersten Bildner der Erde selbst zu sein.

Also nur, wenn die Quelle des Lichts und der Wärme selbst versiegte, müsste das Leben auf der Erde erstarren; vom Erlöschen des fraglich feurigen Erdkerns haben wir nichts zu fürchten. Für die Erhaltung des Lebens sorgt der Stoffwechsel unter dem Einfluss der licht- und wärmestrahlenden Sonne. Licht und Wärme sind also Vater und Mutter alles Lebendigen; sie verhindern, dass das Organische vor der Zeit zum Unorganischen werde, indem sie letzteres stets wieder zu neuen Verbindungen führen. Möchte aber auch noch so viel Licht und Wärme der Erde zuströmen, ohne die fortwährende Tätigkeit, ohne die Umbildung durch die organische Zelle wäre doch das Leben unseres Planeten nach Jahren zu zählen.

Der Anfang des Planeten war die Zelle, sie erhält ihn, so lange noch ein Lichtstrahl die Erde trifft.

Möglich ist es dass mit der Zeit doch Änderungen in der chemischen Zusammensetzung der Erdoberfläche und der Atmosphäre durch Niederschläge und feste Verbindungen eintreten, wodurch Baustoffe aus dem Kreislauf ausgeschieden werden. Sicher aber werden unter solchen veränderten Lebensbedingungen auch andere, ähnliche und (nach der bisherigen Erfahrung) höher organisierte Wesen entstehen. Ja es lässt sich denken, dass hier auf der Erde eine Verfeinerung der Organismen eintreten werde in demselben Verhältnis, wie sie nach der Olivin-Granitzeit eingetreten ist, dass Geschöpfe entstehen, welchen zu ihrer Erhaltung in höherem Masse Wasser und Gase genügen, was ja bei vielen Pflanzen jetzt schon nahezu der Fall ist.

*Nicht die Abnahme der Erdwärme oder der von der Sonne ausgestrahlten Wärme wäre das nächst drohende Schreckgespenst, sondern das Verschwinden unserer Atmosphäre.
\clearpage
\section{Erkl"arung der Tafeln}
\subsection{Vorbemerkung}
\paragraph{}
Die Steine, von welchen ich meine Dünnschliffe nahm, sind durchaus beglaubigte.

Die Dünnschliffe selbst sind von mir unter der unermüdlichen Beihilfe meiner Schwägerin, Fräulein Pauline Schloz, hergestellt. So beläuft sich meine Sammlung auf 560 Nummern (worunter 360 Knyahinya), wohl die größte Sammlung, welche es überhaupt gibt.

Bezüglich der Herstellung der Dünnschliffe muss ich eines Umstands erwähnen, welcher auf die Darstellung von Einfluss war.

Jeder, welcher Versteinerungen geschliffen hat, weis, dass nur ganz wenige einen dünnen Schliff gestatten. Nicht allein wegen des häufig opaken oder gar undurchsichtigen Materials (Kalk, Ton), sondern deshalb, weil die Struktur mit einem Male verschwindet, wenn sie bis zur (vermuteten) Durchsichtigkeit geschliffen werden.

Es hängt das mit der Art und Weise der jedem Versteinerungsprozess zu Grunde liegenden Umbildung zusammen.

So ist man vor die Wahl gestellt, entweder einen ziemlich dunklen Schliff vor sich zu haben, worin man wenig sieht, oder — von dem Wunsch nach schärferen Umrissen getrieben, wobei man stets vergeblich nach höheren Objektiven greift — einen Schliff zu bekommen, welcher nichts mehr zeigt.

Diese beiden Klippen konnten bei dem Meteoriten-Material (welches, beiläufig gesagt, wegen des Eisens dem Schliff ziemliche Schwierigkeiten entgegensetzt) nur dadurch vermieden werden, dass abwechslungsweise dünnere und dickere Schliffe gefertigt wurden.

Was die Auswahl der Formen betrifft, so werden künftige Forscher mich wohl entschuldigen, wenn ich diese und jene Form übersah. Meine Absicht freilich war, sämtliche Formen, welche in meinem Material enthalten sind, abzubilden. Die Abbildungen sollten nicht nur Bilder, sondern auch ein Gesamtbild geben: gerade darauf lege ich ja in der Schlussfrage über die Natur des Gesteins das größte Gewicht.

Was die Anordnung der Tafeln betrifft, so hängt diese mit der Anordnung des Stoffs zusammen. Da ich mir aber doch bewusst war, das ganze Material bei weitem noch nicht erschöpft zu haben: so gab ich mir auch nicht die Mühe, die einzelnen Formen zu bestimmen, oder Ansichten über den genetischen Zusammenhang derselben auszusprechen, zu begründen und hiernach die Anordnung zu treffen: es genügte wohl eine vorläufige Orientierung in dieser Richtung. Heute handelt es sich doch vorerst nur um den Beweis, dass das Gestein organisch und nicht darum, was alles darin ist.

Namen zu geben, vermied ich nicht aus Furcht damit der Kritik in die Hände zu fallen, sondern weil ich zur Einsicht kam, dass durch Namensgebung vorerst nichts oder nicht viel gewonnen ist.

Lange stand ich vor der Wahl, ob ich wirklich den Weg der photographischen Darstellung einschlagen solle. Ich kam aber mehr aus äußeren Rücksichten zu dem betreffenden Entschlüsse.

Es ist bei der Kritik meiner \emph{Urzelle} viel von Phantasie die Rede gewesen. Dass die Abbildungen nicht auf der Höhe der Zeit stehen — wusste ich: dass sie aber doch richtig sind, das mag z. B. die photographische Abbildung der Objekte meiner \emph{Urzelle} Tafel 32. Figur 5 verglichen mit Tafel 4. und 5. der \emph{Urzelle} ergeben.

Ich möchte hiebe noch Herrn Dr. Kuntze in Leipzig fragen, ob er solche künstliche Algen etwa beizubringen weis — zutreffendenfalls wäre ich sehr dankbar für Überlassung eines solchen Präparats um mich von einem Irrtum zu überzeugen.* Meines Wissens sind die Dendriten und "`künstlichen Algen"', welche mir so ohne alle Prüfung und Kenntnis entgegengehalten wurden, bloß Streifen ohne Gliederung und Absonderung. Ihrer Entstehung entsprechend ist es eine meist gleichmäßig verteilte, zusammenhängende Farbstoffmasse, welche zwischen zwei Stein-Platten liegt, also in einer vollkommenen Fläche, und so Pflanzenschatten gleicht.

Ich gebe zu, dass "`künstliche Algen"' nach den Begriffen gewisser Forscher von Algen gemacht werden können. Aber ich muss auch darauf hinweisen, dass alle Gebilde, welche fadenoder bandartig sind, ohne viel Besinnens bisher für Algen erklärt wurden. Um zu wissen, dass man eine Alge vor sich habe, gehört noch etwas mehr dazu. So sind Dinge für Pflanzen erklärt worden, welche sicher nicht halb so viel Form und Struktur zeigen, als meine Bilder in der \emph{Urzelle}. Nicht alle Faden- oder Bündelformen in Gesteinen oder anderen Massen würde ich, auf dieses Merkmal allein hin, für Algen erklären.

Meine Abbildungen in der \emph{Urzelle} zeigen deutliche Zellenwände und Zellen; wären diese Dinge künstliche Algen oder Dendriten, so könnten sie keine Querwände haben.

Hiermit kehre ich zu meinem Gegenstand zurück.

Die Photographie hat große Nachteile für die wissenschaftliche Darstellung, das weis jeder Forscher. Bei dem vorliegenden Gegenstand aber musste ich diesen Weg gehen, einfach weil mir sonst wieder von "`Phantasie"' hätte gesprochen werden können. Die Sonne und das Kollodium zusammen täuschen nicht und müssen jeden derartigen Vorwurf von vornherein von mir abwenden. — Wohl aber enthält das photographische Bild weniger als der Gegenstand. Das wurde besonders bei meinen besten Objekten fühlbar. Es konnte ferner besonders bei höheren Vergrößerungen nur ein Teil des Schliffs zur Darstellung gebracht werden, aber auch dieser nicht scharf, weil das darüber- und darunterliegende Gestein das eingestellte Bild verwischte. Zu hohe Vergrößerungen (das bemerke ich etwaigen Mitarbeitern an der Sache) taugen deshalb durchaus nicht für Gesteinsdünnschliffe. Ein weiterer hindernder Umstand ist, dass die Gesteine aus stark lichtbrechenden und das Licht verschieden brechenden Mineralien bestehen; dadurch entstehen Lichtreflexe der unangenehmsten Art, welche ein Ungeübter leicht für Formen ansehen kann. Um dies zu vermeiden, habe ich mich stets der schwächsten Vergrößerungen bedient und habe unvollkommene Strukturbilder zurückgelegt.

Die photographischen Bilder stehen also eher unter dem Objekt. Allein, wie gesagt, ich musste wegen der Glaubwürdigkeit der Darstellung diesen Weg gehen.

Eine Ursache weiterer besonders empfindlicher Mängel der photographischen Darstellung besteht in der Wirkung der Farben auf das Bild. Unter den schlimmen ist Gelb die schlimmste.

Wo Gelb im Präparat ist, erscheint statt aller Struktur ein schwarzer Fleck. Mit keinem Mittel war diesem Übel abzuhelfen. Und gerade das Gelb des Olivins ist dasjenige, welches absolut keinen Lichtstrahl durchlässt. Das macht sich am meisten geltend bei der Koralle, Tafel 1. Figur 6: der schwarze Ring auf dem Bilde ist ein lichtes Gelb (Eisen). — Dem Gelb folgt Braun, welches ebenfalls sehr dunkelt. Blau hat den entgegengesetzten Fehler, es wird zu licht, doch zeigt es noch Strukturen.

Dass der hohe Preis des Materials gewisse Sparsamkeit in den Präparaten auferlegt, ist selbstredend. Es ist dadurch die Auswahl beschränkt. Gerade dieser Umstand ist der Grund, dass die Schliffe von dem Forscher selbst hergestellt werden müssen. Das ist eine Aufgabe. Aber es ist auch nur so ein gründliches, freilich durch großen Zeitaufwand erschwertes Studium der Sache möglich.

Zur Vergrößerung und photographischen Darstellung habe ich mich des mittleren mikrophotographischen Apparats von Seibert \& Krafft in Wetzlar bedient und kann denselben nur rühmlich empfehlen. Die Bilder wurden unter meiner Leitung im photographischen Atelier der Herren Otto Lauer \& Carl Bossler hier gefertigt. Da wir alle noch keine Übung in dieser Art Aufnahmen hatten, so war die Beihilfe des Herrn Dr. Schreiner , Assistenten am chemischen Laboratorium in Tübingen eine äußerst erwünschte. Weitere Hilfe habe ich nicht zu verzeichnen, wohl aber glaube ich nicht unerwähnt lassen zu dürfen, die völlige Teilnahmslosigkeit aller derjenigen Gelehrten, welche die Sache am meisten berührt.

Bei der Anordnung des Stoffs habe ich die Schwämme vorangestellt, diesen die Korallen und dann die Crinoiden folgen lassen.

Entsprechend der Häufigkeit des Vorkommens habe ich auch die einzelnen Gattungen in der Zahl sich vertreten lassen. Leider musste ich manches bessere Objekt wegen der gelben Färbung zurücklegen. Wenn es sich bewährt, was Gümbel in seiner trefflichen Abhandlung über die bayerischen Meteoriten sagt, dass es ihm gelungen sei, die gelbe Farbe durch Säuren zu entfernen, so wäre viel gewonnen.

Was die Vergrößerungen betrifft, oder richtiger das Format der Vergrößerungen, so kam in Betracht, dass eben die Einrichtung der Kamera die Einhaltung eines bestimmten Formats auferlegt. Das führt zu dem Missstand, dass die Formen zuletzt alle gleich groß erscheinen.

Die Angabe der Vergrößerung, d. h. das Verhältnis der wahren Größe zum Durchmesser des dargestellten Bildes ist also ein sehr wenig bezeichnendes.

Ich habe daher vorgezogen mit der Angabe des Durchmessers jeder Form die wirkliche Größe des Objekts unmittelbar zu bezeichnen.

*Ähnlich ist Dr. Kuntze mit der Flora Columbiae von Dr. H. Karsten verfahren. Ehe sich derselbe gegen die Anschuldigung, welche Dr. W. Joos auf diese Kritiken hin gegen ihn erhoben, reinigt, hat er kein Recht mehr, in der Wissenschaft gehört zu werden.
\clearpage
\subsection{Tafelverzeichniss}
\begin{enumerate}
    \item Die Nummerierung der Abbildungen geschieht von links oben nach rechts unten.
    \item Abkürzungen: V. heißt Vergrößerung, D. heißt wirklicher Durchmesser, mm heißt Millimeter.
\end{enumerate}
\clearpage
\pagestyle{fancy}
\fancyhf{}
\rhead{Tafel 1: Zusammenstellung von Mineral-Strukturen mit organischen aus Chondriten\index{chondrite}}
\cfoot{\thepage}
\begin{figure}[b]
\includegraphics[width=\textwidth,height=\textheight,keepaspectratio]{figures/meteorite_1-1_edit-b2.jpg}
\caption{Tafel 1: Figur 1 --- Enstatit (-Bronzit) vom Kupferberg V.}
\centering
\end{figure}
\clearpage
\begin{figure}[t]
\includegraphics[width=\textwidth,height=\textheight,keepaspectratio]{figures/meteorite_1-2_edit-b.jpg}
\caption{Tafel 1: Figur 2 --- Enstatit von Texas V.}
\centering
\end{figure}
\clearpage
\begin{figure}[t]
\includegraphics[width=\textwidth,height=\textheight,keepaspectratio]{figures/meteorite_1-3_edit-b2.jpg}
\caption{Tafel 1: Figur 3 --- Spherulite-Liparite\index{spherulite} from Lipari M.}
\centering
\end{figure}
\clearpage
\begin{figure}[t]
\includegraphics[width=\textwidth,height=\textheight,keepaspectratio]{figures/meteorite_1-4_edit-b.jpg}
\caption{Tafel 1: Figur 4 --- ein Theil der Coralle Taf. 8, 9 und 10}
\centering
\end{figure}
\clearpage
\begin{figure}[t]
\includegraphics[width=\textwidth,height=\textheight,keepaspectratio]{figures/meteorite_1-5_edit-b2.jpg}
\caption{Tafel 1: Figur 5 --- Kettenkoralle D. 0,90 mm.}
\centering
\end{figure}
\clearpage
\begin{figure}[t]
\includegraphics[width=\textwidth,height=\textheight,keepaspectratio]{figures/meteorite_1-6_edit-b2.jpg}
\caption{Tafel 1: Figur 6 --- Crinoid D. 1,20 mm.}
\centering
\end{figure}
\clearpage
\rhead{Tafel 2: \emph{Urania}}
\begin{figure}[t]
\includegraphics[width=\textwidth,height=\textheight,keepaspectratio]{figures/meteorite_2-1_edit-b2.jpg}
\caption{Tafel 2: Figur 1 --- aus Knyahinya. Dieselbe Tafel 5. Fig. 1.}
\centering
\end{figure}
\clearpage
\rhead{Tafel 3: \emph{Urania}}
\begin{figure}[t]
\includegraphics[width=\textwidth,height=\textheight,keepaspectratio]{figures/meteorite_3-1_edit-b.jpg}
\caption{Tafel 3: Figur 1 --- aus Knyahinya D. 0,60 mm.}
\centering
\end{figure}
\clearpage
\begin{figure}[t]
\includegraphics[width=\textwidth,height=\textheight,keepaspectratio]{figures/meteorite_3-2_edit-b.jpg}
\caption{Tafel 3: Figur 2 --- aus Knyahinya D. 1,30 mm. (man übersehe nicht die prachtvollen Crinoidenglieder links oben!)}
\centering
\end{figure}
\clearpage
\begin{figure}[t]
\includegraphics[width=\textwidth,height=\textheight,keepaspectratio]{figures/meteorite_3-3_edit-b.jpg}
\caption{Tafel 3: Figur 3 --- aus Knyahinya D. 1 mm.}
\centering
\end{figure}
\clearpage
\begin{figure}[t]
\includegraphics[width=\textwidth,height=\textheight,keepaspectratio]{figures/meteorite_3-4_edit-b.jpg}
\caption{Tafel 3: Figur 4 --- aus Knyahinya D. 1 mm.}
\centering
\end{figure}
\clearpage
\begin{figure}[t]
\includegraphics[width=\textwidth,height=\textheight,keepaspectratio]{figures/meteorite_3-5_edit-b2.jpg}
\caption{Tafel 3: Figur 5 --- aus Knyahinya D. 1 mm. (zu beachten die Schichtung oben)}
\centering
\end{figure}
\clearpage
\begin{figure}[t]
\includegraphics[width=\textwidth,height=\textheight,keepaspectratio]{figures/meteorite_3-6_edit-b2.jpg}
\caption{Tafel 3: Figur 6 --- aus Knyahinya D. 1 mm. (Schichtung wie 5, doch im Bilde nicht wiedergegeben, 5 und 6 aus einem Dünnschliff.)}
\centering
\end{figure}
\clearpage
\rhead{Tafel 4: \emph{Urania}}
\begin{figure}[t]
\includegraphics[width=\textwidth,height=\textheight,keepaspectratio]{figures/meteorite_4-1_edit-b.jpg}
\caption{Tafel 4: Figur 1 --- aus Knyahinya D. 0,90 mm.}
\centering
\end{figure}
\clearpage
\begin{figure}[t]
\includegraphics[width=\textwidth,height=\textheight,keepaspectratio]{figures/meteorite_4-2_edit-b.jpg}
\caption{Tafel 4: Figur 2 --- aus Siena D. 3 mm. (der dunkle Strich rührt von gelber Färbung des Präparats)}
\centering
\end{figure}
\clearpage
\begin{figure}[t]
\includegraphics[width=\textwidth,height=\textheight,keepaspectratio]{figures/meteorite_4-3_edit-b.jpg}
\caption{Tafel 4: Figur 3 --- aus Knyahinya D. 0,60 mm.}
\centering
\end{figure}
\clearpage
\begin{figure}[t]
\includegraphics[width=\textwidth,height=\textheight,keepaspectratio]{figures/meteorite_4-4_edit-b.jpg}
\caption{Tafel 4: Figur 4 --- aus Knyahinya D. 0,90 mm. (Luftblase)}
\centering
\end{figure}
\clearpage
\begin{figure}[t]
\includegraphics[width=\textwidth,height=\textheight,keepaspectratio]{figures/meteorite_4-5_edit-b.jpg}
\caption{Tafel 4: Figur 5 --- aus Knyahinya D. 1,60 mm.}
\centering
\end{figure}
\clearpage
\begin{figure}[t]
\includegraphics[width=\textwidth,height=\textheight,keepaspectratio]{figures/meteorite_4-6_edit-b.jpg}
\caption{Tafel 4: Figur 6 --- aus Knyahinya D. 1,00 mm. (Luftblase)}
\centering
\end{figure}
\clearpage
\rhead{Tafel 5: \emph{Urania}}
\begin{figure}[t]
\includegraphics[width=\textwidth,height=\textheight,keepaspectratio]{figures/meteorite_5-1_edit-b.jpg}
\caption{Tafel 5: Figur 1 --- aus Knyahinya D. 1,40 mm. (siehe Tafel 2. Rings Crinoidendurchschnitte. Form unten links, vergl. Tafel 1. Fig. 6 und Tafel 25. 1, 2)}
\centering
\end{figure}
\clearpage
\begin{figure}[t]
\includegraphics[width=\textwidth,height=\textheight,keepaspectratio]{figures/meteorite_5-2_edit-b2.jpg}
\caption{Tafel 5: Figur 2 --- aus Knyahinya D. 1,80 mm.}
\centering
\end{figure}
\clearpage
\begin{figure}[t]
\includegraphics[width=\textwidth,height=\textheight,keepaspectratio]{figures/meteorite_5-3_edit-b.jpg}
\caption{Tafel 5: Figur 3 --- aus Knyahinya D. 1,80 mm.}
\centering
\end{figure}
\clearpage
\begin{figure}[t]
\includegraphics[width=\textwidth,height=\textheight,keepaspectratio]{figures/meteorite_5-4_edit-b.jpg}
\caption{Tafel 5: Figur 4 --- aus Knyahinya D. 1,30 mm. (undeutliches Bild)}
\centering
\end{figure}
\clearpage
\begin{figure}[t]
\includegraphics[width=\textwidth,height=\textheight,keepaspectratio]{figures/meteorite_5-5_edit-b2.jpg}
\caption{Tafel 5: Figur 5 --- aus Knyahinya D. 1,40 mm. (Luftblase)}
\centering
\end{figure}
\clearpage
\begin{figure}[t]
\includegraphics[width=\textwidth,height=\textheight,keepaspectratio]{figures/meteorite_5-6_edit-b2.jpg}
\caption{Tafel 5: Figur 6 --- aus Knyahinya D. 0,60 mm. (mangelhaftes Bild. Der weiße Ring ist der Durchschnitt)}
\centering
\end{figure}
\clearpage
\rhead{Tafel 6: \emph{Urania}}
\begin{figure}[t]
\includegraphics[width=\textwidth,height=\textheight,keepaspectratio]{figures/meteorite_6-1_edit-b2.jpg}
\caption{Tafel 6: Figur 1 --- aus Siena D. 4,00 mm.}
\centering
\end{figure}
\clearpage
\begin{figure}[t]
\includegraphics[width=\textwidth,height=\textheight,keepaspectratio]{figures/meteorite_6-2_edit-b.jpg}
\caption{Tafel 6: Figur 2 --- aus Knyahinya D. 0,80 mm.}
\centering
\end{figure}
\clearpage
\begin{figure}[t]
\includegraphics[width=\textwidth,height=\textheight,keepaspectratio]{figures/meteorite_6-3_edit-b.jpg}
\caption{Tafel 6: Figur 3 --- aus Siena D. 1,20 mm.}
\centering
\end{figure}
\clearpage
\begin{figure}[t]
\includegraphics[width=\textwidth,height=\textheight,keepaspectratio]{figures/meteorite_6-4_edit-b.jpg}
\caption{Tafel 6: Figur 4 --- aus Knyahinya D. 0,70 mm. (in der Mitte zu stark beleuchtet)}
\centering
\end{figure}
\clearpage
\begin{figure}[t]
\includegraphics[width=\textwidth,height=\textheight,keepaspectratio]{figures/meteorite_6-5_edit-b.jpg}
\caption{Tafel 6: Figur 5 --- aus Knyahinya D. 0,30 mm.}
\centering
\end{figure}
\clearpage
\begin{figure}[t]
\includegraphics[width=\textwidth,height=\textheight,keepaspectratio]{figures/meteorite_6-6_edit-b2.jpg}
\caption{Tafel 6: Figur 6 --- aus Knyahinya D. 0,90 mm. (Luftblase)}
\centering
\end{figure}
\clearpage
\rhead{Tafel 7: Schwämme}
\begin{figure}[t]
\includegraphics[width=\textwidth,height=\textheight,keepaspectratio]{figures/meteorite_7-1_edit-b.jpg}
\caption{Tafel 7: Figur 1 --- aus Knyahinya D. 2,30 mm.}
\centering
\end{figure}
\clearpage
\begin{figure}[t]
\includegraphics[width=\textwidth,height=\textheight,keepaspectratio]{figures/meteorite_7-2_edit-b.jpg}
\caption{Tafel 7: Figur 2 --- aus Knyahinya D. 1,80 mm. (ein Riss im Präparat. Die Nadeln)}
\centering
\end{figure}
\clearpage
\begin{figure}[t]
\includegraphics[width=\textwidth,height=\textheight,keepaspectratio]{figures/meteorite_7-3_edit-b.jpg}
\caption{Tafel 7: Figur 3 --- aus Knyahinya D. 2,10 mm.}
\centering
\end{figure}
\clearpage
\begin{figure}[t]
\includegraphics[width=\textwidth,height=\textheight,keepaspectratio]{figures/meteorite_7-4_edit-b.jpg}
\caption{Tafel 7: Figur 4 --- (Crinoid-Querschnitt?) aus Knyahinya D. 3,00 mm.}
\centering
\end{figure}
\clearpage
\begin{figure}[t]
\includegraphics[width=\textwidth,height=\textheight,keepaspectratio]{figures/meteorite_7-5_edit-b.jpg}
\caption{Tafel 7: Figur 5 --- Schwamm? D. 1,00 mm.}
\centering
\end{figure}
\clearpage
\begin{figure}[t]
\includegraphics[width=\textwidth,height=\textheight,keepaspectratio]{figures/meteorite_7-6_edit-b.jpg}
\caption{Tafel 7: Figur 6 --- Schwamm? D. 2,40 mm.}
\centering
\end{figure}
\clearpage
\rhead{Tafel 8: Korallen}
\begin{figure}[t]
\includegraphics[width=\textwidth,height=\textheight,keepaspectratio]{figures/meteorite_8-1_edit-b2.jpg}
\caption{Tafel 8: Figur 1 --- (Favosites) aus Knyahinya (vergl. Tafel 1: Figur 4)}
\centering
\end{figure}
\clearpage
\rhead{Tafel 9: Korallen}
\begin{figure}[t]
\includegraphics[width=\textwidth,height=\textheight,keepaspectratio]{figures/meteorite_9-1_edit-b3.jpg}
\caption{Tafel 9: Figur 1 --- Strukturbild aus links oben Tafel 8.}
\centering
\end{figure}
\clearpage
\rhead{Tafel 10: Korallen}
\begin{figure}[t]
\includegraphics[width=\textwidth,height=\textheight,keepaspectratio]{figures/meteorite_10-1_edit-b.jpg}
\caption{Tafel 10: Figur 1 --- aus Knyahinya Querschnitt D. 0,40 mm.}
\centering
\end{figure}
\clearpage
\begin{figure}[t]
\includegraphics[width=\textwidth,height=\textheight,keepaspectratio]{figures/meteorite_10-2_edit-b.jpg}
\caption{Tafel 10: Figur 2 --- Längsschnitt 0,50 mm.}
\centering
\end{figure}
\clearpage
\begin{figure}[t]
\includegraphics[width=\textwidth,height=\textheight,keepaspectratio]{figures/meteorite_10-3_edit-b2.jpg}
\caption{Tafel 10: Figur 3 --- aus Knyahinya D. 1,80 mm.}
\centering
\end{figure}
\clearpage
\begin{figure}[t]
%this figure has the same figure as Tafel 1: Figur 5, which has better quality
\includegraphics[width=\textwidth,height=\textheight,keepaspectratio]{figures/meteorite_1-5_edit-b2.jpg}
\caption{Tafel 10: Figur 4 --- aus Knyahinya D. 0,90 mm. (siehe Tafel 8. 9.)}
\centering
\end{figure}
\clearpage
\begin{figure}[t]
\includegraphics[width=\textwidth,height=\textheight,keepaspectratio]{figures/meteorite_10-5_edit-b.jpg}
\caption{Tafel 10: Figur 5 --- aus Knyahinya D. 0,30 mm.}
\centering
\end{figure}
\clearpage
\begin{figure}[t]
\includegraphics[width=\textwidth,height=\textheight,keepaspectratio]{figures/meteorite_10-6_edit-b.jpg}
\caption{Tafel 10: Figur 6 --- aus Knyahinya D. 0,80 mm.}
\centering
\end{figure}
\clearpage
\rhead{Tafel 11: Korallen}
\begin{figure}[t]
\includegraphics[width=\textwidth,height=\textheight,keepaspectratio]{figures/meteorite_11-1_edit-b.jpg}
\caption{Tafel 11: Figur 1 --- aus Knyahinya D. 1,20 mm.}
\centering
\end{figure}
\clearpage
\begin{figure}[t]
\includegraphics[width=\textwidth,height=\textheight,keepaspectratio]{figures/meteorite_11-2_edit-b.jpg}
\caption{Tafel 11: Figur 2 --- aus Knyahinya D. 1,00 mm.}
\centering
\end{figure}
\clearpage
\begin{figure}[t]
\includegraphics[width=\textwidth,height=\textheight,keepaspectratio]{figures/meteorite_11-3_edit-b.jpg}
\caption{Tafel 11: Figur 3 --- aus Knyahinya D. 1,80 mm.}
\centering
\end{figure}
\clearpage
\begin{figure}[t]
\includegraphics[width=\textwidth,height=\textheight,keepaspectratio]{figures/meteorite_11-4_edit-b.jpg}
\caption{Tafel 11: Figur 4 --- aus Knyahinya D. 1,20 mm.}
\centering
\end{figure}
\clearpage
\begin{figure}[t]
\includegraphics[width=\textwidth,height=\textheight,keepaspectratio]{figures/meteorite_11-5_edit-b.jpg}
\caption{Tafel 11: Figur 5 --- aus Parnallee D. 0,80 mm.}
\centering
\end{figure}
\clearpage
\begin{figure}[t]
\includegraphics[width=\textwidth,height=\textheight,keepaspectratio]{figures/meteorite_11-6_edit.jpg}
\caption{Tafel 11: Figur 6 --- aus Moung County D. 0,60 mm.}
\centering
\end{figure}
\clearpage
\rhead{Tafel 12: Korallen}
\begin{figure}[t]
\includegraphics[width=\textwidth,height=\textheight,keepaspectratio]{figures/meteorite_12-1_edit-b.jpg}
\caption{Tafel 12: Figur 1 --- aus Knyahinya D. 0,80 mm.}
\centering
\end{figure}
\clearpage
\begin{figure}[t]
\includegraphics[width=\textwidth,height=\textheight,keepaspectratio]{figures/meteorite_12-2_edit-b.jpg}
\caption{Tafel 12: Figur 2 --- aus Knyahinya D. 1,20 mm.}
\centering
\end{figure}
\clearpage
\begin{figure}[t]
\includegraphics[width=\textwidth,height=\textheight,keepaspectratio]{figures/meteorite_12-3_edit-b.jpg}
\caption{Tafel 12: Figur 3 --- aus Knyahinya D. 1,30 mm.}
\centering
\end{figure}
\clearpage
\begin{figure}[t]
\includegraphics[width=\textwidth,height=\textheight,keepaspectratio]{figures/meteorite_12-4_edit-b.jpg}
\caption{Tafel 12: Figur 4 --- aus Knyahinya D. 1,40 mm.}
\centering
\end{figure}
\clearpage
\begin{figure}[t]
\includegraphics[width=\textwidth,height=\textheight,keepaspectratio]{figures/meteorite_12-5_edit-b.jpg}
\caption{Tafel 12: Figur 5 --- aus Knyahinya D. 2,00 mm.}
\centering
\end{figure}
\clearpage
\begin{figure}[t]
\includegraphics[width=\textwidth,height=\textheight,keepaspectratio]{figures/meteorite_12-6_edit-b.jpg}
\caption{Tafel 12: Figur 6 --- aus Knyahinya D. 3,20 mm.}
\centering
\end{figure}
\clearpage
\rhead{Tafel 13: Korallen}
\begin{figure}[t]
\includegraphics[width=\textwidth,height=\textheight,keepaspectratio]{figures/meteorite_13-1_edit-b.jpg}
\caption{Tafel 13: Figur 1 --- aus Parnallee D. 0,20 mm.}
\centering
\end{figure}
\clearpage
\begin{figure}[t]
\includegraphics[width=\textwidth,height=\textheight,keepaspectratio]{figures/meteorite_13-2_edit-b.jpg}
\caption{Tafel 13: Figur 2 --- aus Knyahinya D. 0,80 mm.}
\centering
\end{figure}
\clearpage
\begin{figure}[t]
\includegraphics[width=\textwidth,height=\textheight,keepaspectratio]{figures/meteorite_13-3_edit-b.jpg}
\caption{Tafel 13: Figur 3 --- aus Siena D. 0,20 mm.}
\centering
\end{figure}
\clearpage
\begin{figure}[t]
\includegraphics[width=\textwidth,height=\textheight,keepaspectratio]{figures/meteorite_13-4_edit-b.jpg}
\caption{Tafel 13: Figur 4 --- aus Knyahinya D. 1,80 mm.}
\centering
\end{figure}
\clearpage
\begin{figure}[t]
\includegraphics[width=\textwidth,height=\textheight,keepaspectratio]{figures/meteorite_13-5_edit-b.jpg}
\caption{Tafel 13: Figur 5 --- aus Knyahinya D. 1,70 mm.}
\centering
\end{figure}
\clearpage
\begin{figure}[t]
\includegraphics[width=\textwidth,height=\textheight,keepaspectratio]{figures/meteorite_13-6_edit-b.jpg}
\caption{Tafel 13: Figur 6 --- aus Cabarras D. 0,30 mm.}
\centering
\end{figure}
\clearpage
\rhead{Tafel 14: Korallen}
\begin{figure}[t]
\includegraphics[width=\textwidth,height=\textheight,keepaspectratio]{figures/meteorite_14-1_edit-b2.jpg}
\caption{Tafel 14: Figur 1 --- Koralle D. 0,90 mm.}
\centering
\end{figure}
\clearpage
\rhead{Tafel 15: Korallen}
\begin{figure}[t]
\includegraphics[width=\textwidth,height=\textheight,keepaspectratio]{figures/meteorite_15-1_edit-b3.jpg}
\caption{Tafel 15: Figur 1 --- Koralle. Strukturbild von 14. Der linke obere Teil des Präparats, Vergrößerung 300, zeigt die Knospen-Kanäle.}
\centering
\end{figure}
\clearpage
\rhead{Tafel 16: Crinoiden}
\begin{figure}[t]
\includegraphics[width=\textwidth,height=\textheight,keepaspectratio]{figures/meteorite_16-1_edit-b2.jpg}
\caption{Tafel 16: Figur 1 --- aus Knyahinya D. 0,40 mm.}
\centering
\end{figure}
\clearpage
\rhead{Tafel 17: Crinoiden}
\begin{figure}[t]
\includegraphics[width=\textwidth,height=\textheight,keepaspectratio]{figures/meteorite_17-1_edit-b2.jpg}
\caption{Tafel 17: Figur 1 --- aus Knyahinya D. 2,00 mm.}
\centering
\end{figure}
\clearpage
\rhead{Tafel 18: Crinoiden}
\begin{figure}[t]
\includegraphics[width=\textwidth,height=\textheight,keepaspectratio]{figures/meteorite_18-1_edit-b2.jpg}
\caption{Tafel 18: Figur 1 --- aus Knyahinya, 4 Hauptarme durchschnitten, D. 2,20 mm.}
\centering
\end{figure}
\clearpage
\rhead{Tafel 19: Crinoiden}
\begin{figure}[t]
\includegraphics[width=\textwidth,height=\textheight,keepaspectratio]{figures/meteorite_19-1_edit-b2.jpg}
\caption{Tafel 19: Figur 1 --- Crinoid, vergl. Tafel 25. 1 und 2.}
\centering
\end{figure}
\clearpage
\rhead{Tafel 20: Crinoiden}
\begin{figure}[t]
\includegraphics[width=\textwidth,height=\textheight,keepaspectratio]{figures/meteorite_20-1_edit-b2.jpg}
\caption{Tafel 20: Figur 1 --- Crinoid und Koralle durchschnitten aus Knyahinya D. 1,20 mm.}
\centering
\end{figure}
\clearpage
\rhead{Tafel 21: Crinoiden}
\begin{figure}[t]
\includegraphics[width=\textwidth,height=\textheight,keepaspectratio]{figures/meteorite_21-1_edit-b.jpg}
\caption{Tafel 21: Figur 1 --- aus Knyahinya D. 0,80 mm.}
\centering
\end{figure}
\clearpage
\begin{figure}[t]
\includegraphics[width=\textwidth,height=\textheight,keepaspectratio]{figures/meteorite_21-2_edit-b.jpg}
\caption{Tafel 21: Figur 2 --- vergrößertes Bild von Figur 1}
\centering
\end{figure}
\clearpage
\begin{figure}[t]
\includegraphics[width=\textwidth,height=\textheight,keepaspectratio]{figures/meteorite_21-3_edit-b.jpg}
\caption{Tafel 21: Figur 3 --- aus Knyahinya D. 1,20 mm.}
\centering
\end{figure}
\clearpage
\begin{figure}[t]
\includegraphics[width=\textwidth,height=\textheight,keepaspectratio]{figures/meteorite_21-4_edit-b.jpg}
\caption{Tafel 21: Figur 4 --- vergrößertes Bild von Figur 3}
\centering
\end{figure}
\clearpage
\begin{figure}[t]
\includegraphics[width=\textwidth,height=\textheight,keepaspectratio]{figures/meteorite_21-5_edit-b.jpg}
\caption{Tafel 21: Figur 5 --- aus Knyahinya D. 1,80 mm. (ich bemerke die Ähnlichkeit mit Figur 1)}
\centering
\end{figure}
\clearpage
\begin{figure}[t]
\includegraphics[width=\textwidth,height=\textheight,keepaspectratio]{figures/meteorite_21-6_edit-b.jpg}
\caption{Tafel 21: Figur 6 --- aus Knyahinya D. 0,30 mm. (die Mundöffnung zwischen den Armen sichtbar)}
\centering
\end{figure}
\clearpage
\rhead{Tafel 22: Crinoiden}
\begin{figure}[t]
\includegraphics[width=\textwidth,height=\textheight,keepaspectratio]{figures/meteorite_22-1_edit-b.jpg}
\caption{Tafel 22: Figur 1 --- aus Knyahinya D. 0,50 mm.}
\centering
\end{figure}
\clearpage
\begin{figure}[t]
\includegraphics[width=\textwidth,height=\textheight,keepaspectratio]{figures/meteorite_22-2_edit-b.jpg}
\caption{Tafel 22: Figur 2 --- aus Knyahinya D. 0,60 mm.}
\centering
\end{figure}
\clearpage
\begin{figure}[t]
\includegraphics[width=\textwidth,height=\textheight,keepaspectratio]{figures/meteorite_22-3_edit-b.jpg}
\caption{Tafel 22: Figur 3 --- aus Knyahinya (Titelbild) D. 1,50 mm.}
\centering
\end{figure}
\clearpage
\begin{figure}[t]
\includegraphics[width=\textwidth,height=\textheight,keepaspectratio]{figures/meteorite_22-4_edit-b.jpg}
\caption{Tafel 22: Figur 4 --- aus Knyahinya D. 0,70 mm.}
\centering
\end{figure}
\clearpage
\begin{figure}[t]
\includegraphics[width=\textwidth,height=\textheight,keepaspectratio]{figures/meteorite_22-5_edit-b.jpg}
\caption{Tafel 22: Figur 5 --- aus Knyahinya D. 0,60 mm.}
\centering
\end{figure}
\clearpage
\begin{figure}[t]
\includegraphics[width=\textwidth,height=\textheight,keepaspectratio]{figures/meteorite_22-6_edit-b.jpg}
\caption{Tafel 22: Figur 6 --- aus Knyahinya D. 1,20 mm.}
\centering
\end{figure}
\clearpage
\rhead{Tafel 23: Crinoiden}
\begin{figure}[t]
\includegraphics[width=\textwidth,height=\textheight,keepaspectratio]{figures/meteorite_23-1_edit-b.jpg}
\caption{Tafel 23: Figur 1 --- aus Knyahinya D. 0,90 mm.}
\centering
\end{figure}
\clearpage
\begin{figure}[t]
\includegraphics[width=\textwidth,height=\textheight,keepaspectratio]{figures/meteorite_23-2_edit-b.jpg}
\caption{Tafel 23: Figur 2 --- aus Knyahinya D. 1,60 mm.}
\centering
\end{figure}
\clearpage
\begin{figure}[t]
\includegraphics[width=\textwidth,height=\textheight,keepaspectratio]{figures/meteorite_23-3_edit-b.jpg}
\caption{Tafel 23: Figur 3 --- aus Knyahinya D. 1,00 mm.}
\centering
\end{figure}
\clearpage
\begin{figure}[t]
\includegraphics[width=\textwidth,height=\textheight,keepaspectratio]{figures/meteorite_23-4_edit-b.jpg}
\caption{Tafel 23: Figur 4 --- aus Knyahinya D. 1,40 mm.}
\centering
\end{figure}
\clearpage
\begin{figure}[t]
\includegraphics[width=\textwidth,height=\textheight,keepaspectratio]{figures/meteorite_23-5_edit-b.jpg}
\caption{Tafel 23: Figur 5 --- aus Knyahinya D. 1,30 mm.}
\centering
\end{figure}
\clearpage
\begin{figure}[t]
\includegraphics[width=\textwidth,height=\textheight,keepaspectratio]{figures/meteorite_23-6_edit-b.jpg}
\caption{Tafel 23: Figur 6 --- aus Knyahinya D. 0,60 mm.}
\centering
\end{figure}
\clearpage
\rhead{Tafel 24: Crinoiden}
\begin{figure}[t]
\includegraphics[width=\textwidth,height=\textheight,keepaspectratio]{figures/meteorite_24-1_edit-b.jpg}
\caption{Tafel 24: Figur 1 --- aus Siena D. 0,80 mm.}
\centering
\end{figure}
\clearpage
\begin{figure}[t]
\includegraphics[width=\textwidth,height=\textheight,keepaspectratio]{figures/meteorite_24-2_edit-b.jpg}
\caption{Tafel 24: Figur 2 --- aus Knyahinya D. 2,80 mm.}
\centering
\end{figure}
\clearpage
\begin{figure}[t]
\includegraphics[width=\textwidth,height=\textheight,keepaspectratio]{figures/meteorite_24-3_edit-b.jpg}
\caption{Tafel 24: Figur 3 --- aus Knyahinya D. 1,00 mm.}
\centering
\end{figure}
\clearpage
\begin{figure}[t]
\includegraphics[width=\textwidth,height=\textheight,keepaspectratio]{figures/meteorite_24-4_edit-b.jpg}
\caption{Tafel 24: Figur 4 --- aus Knyahinya D. 2,00 mm.}
\centering
\end{figure}
\clearpage
\begin{figure}[t]
\includegraphics[width=\textwidth,height=\textheight,keepaspectratio]{figures/meteorite_24-5_edit-b.jpg}
\caption{Tafel 24: Figur 5 --- aus Knyahinya D. 1,50 mm.}
\centering
\end{figure}
\clearpage
\begin{figure}[t]
\includegraphics[width=\textwidth,height=\textheight,keepaspectratio]{figures/meteorite_24-6_edit-b.jpg}
\caption{Tafel 24: Figur 6 --- aus Cabarras D. 0,80 mm.}
\centering
\end{figure}
\clearpage
\rhead{Tafel 25: Crinoiden}
\begin{figure}[t]
\includegraphics[width=\textwidth,height=\textheight,keepaspectratio]{figures/meteorite_25-1_edit-b.jpg}
\caption{Tafel 25: Figur 1 --- aus Knyahinya D. 1,20 mm.}
\centering
\end{figure}
\clearpage
\begin{figure}[t]
\includegraphics[width=\textwidth,height=\textheight,keepaspectratio]{figures/meteorite_25-2_edit-b.jpg}
\caption{Tafel 25: Figur 2 --- aus Knyahinya D. 1,20 mm.}
\centering
\end{figure}
\clearpage
\begin{figure}[t]
\includegraphics[width=\textwidth,height=\textheight,keepaspectratio]{figures/meteorite_25-3_edit-b.jpg}
\caption{Tafel 25: Figur 3 --- aus Knyahinya D. 1,80 mm.}
\centering
\end{figure}
\clearpage
\begin{figure}[t]
\includegraphics[width=\textwidth,height=\textheight,keepaspectratio]{figures/meteorite_25-4_edit-b.jpg}
\caption{Tafel 25: Figur 4 --- aus Knyahinya D. 0,60 mm.}
\centering
\end{figure}
\clearpage
\begin{figure}[t]
\includegraphics[width=\textwidth,height=\textheight,keepaspectratio]{figures/meteorite_25-5_edit-b.jpg}
\caption{Tafel 25: Figur 5 --- aus Siena D. 1,80 mm.}
\centering
\end{figure}
\clearpage
\begin{figure}[t]
\includegraphics[width=\textwidth,height=\textheight,keepaspectratio]{figures/meteorite_25-6_edit-b.jpg}
\caption{Tafel 25: Figur 6 --- aus Knyahinya D. 1,40 mm. (Beide letztere Querschnitte von Crinoiden)}
\centering
\end{figure}
\clearpage
\rhead{Tafel 26: Crinoiden}
\begin{figure}[t]
\includegraphics[width=\textwidth,height=\textheight,keepaspectratio]{figures/meteorite_26-1_edit-b.jpg}
\caption{Tafel 26: Figur 1 --- aus Knyahinya D. 0,20 mm.}
\centering
\end{figure}
\clearpage
\begin{figure}[t]
\includegraphics[width=\textwidth,height=\textheight,keepaspectratio]{figures/meteorite_26-2_edit-b.jpg}
\caption{Tafel 26: Figur 2 --- aus Knyahinya D. 2,00 mm.}
\centering
\end{figure}
\clearpage
\begin{figure}[t]
\includegraphics[width=\textwidth,height=\textheight,keepaspectratio]{figures/meteorite_26-3_edit-b.jpg}
\caption{Tafel 26: Figur 3 --- aus Knyahinya D. 1,20 mm.}
\centering
\end{figure}
\clearpage
\begin{figure}[t]
\includegraphics[width=\textwidth,height=\textheight,keepaspectratio]{figures/meteorite_26-4_edit-b.jpg}
\caption{Tafel 26: Figur 4 --- aus Knyahinya D. 1,20 mm. (bis hierher gewundene Crinoiden)}
\centering
\end{figure}
\clearpage
\begin{figure}[t]
\includegraphics[width=\textwidth,height=\textheight,keepaspectratio]{figures/meteorite_26-5_edit-b.jpg}
\caption{Tafel 26: Figur 5 --- aus Knyahinya D. 2,00 mm.}
\centering
\end{figure}
\clearpage
\begin{figure}[t]
\includegraphics[width=\textwidth,height=\textheight,keepaspectratio]{figures/meteorite_26-6_edit-b.jpg}
\caption{Tafel 26: Figur 6 --- aus Knyahinya D. 2,20 mm. (die dunkle Linie in 5 und 6 ist der Nahrungskanal)}
\centering
\end{figure}
\clearpage
\rhead{Tafel 27: Crinoiden}
\begin{figure}[t]
\includegraphics[width=\textwidth,height=\textheight,keepaspectratio]{figures/meteorite_27-1_edit-b.jpg}
\caption{Tafel 27: Figur 1 --- aus Knyahinya D. 0,80 mm.}
\centering
\end{figure}
\clearpage
\begin{figure}[t]
\includegraphics[width=\textwidth,height=\textheight,keepaspectratio]{figures/meteorite_27-2_edit-b.jpg}
\caption{Tafel 27: Figur 2 --- aus Knyahinya D. 1,50 mm.}
\centering
\end{figure}
\clearpage
\begin{figure}[t]
\includegraphics[width=\textwidth,height=\textheight,keepaspectratio]{figures/meteorite_27-3_edit-b.jpg}
\caption{Tafel 27: Figur 3 --- aus Knyahinya D. 1,40 mm.}
\centering
\end{figure}
\clearpage
\begin{figure}[t]
\includegraphics[width=\textwidth,height=\textheight,keepaspectratio]{figures/meteorite_27-4_edit-b.jpg}
\caption{Tafel 27: Figur 4 --- aus Knyahinya D. 1,40 mm.}
\centering
\end{figure}
\clearpage
\begin{figure}[t]
\includegraphics[width=\textwidth,height=\textheight,keepaspectratio]{figures/meteorite_27-5_edit-b.jpg}
\caption{Tafel 27: Figur 5 --- aus Knyahinya D. 1,20 mm.}
\centering
\end{figure}
\clearpage
\begin{figure}[t]
\includegraphics[width=\textwidth,height=\textheight,keepaspectratio]{figures/meteorite_27-6_edit-b.jpg}
\caption{Tafel 27: Figur 6 --- aus Knyahinya D. 1,00 mm.}
\centering
\end{figure}
\clearpage
\rhead{Tafel 28: Crinoiden}
\begin{figure}[t]
\includegraphics[width=\textwidth,height=\textheight,keepaspectratio]{figures/meteorite_28-1_edit-b.jpg}
\caption{Tafel 28: Figur 1 --- aus Knyahinya (Coralle?) D. 3,00 mm. aus demselben Dünnschl. wie Tafel 18.}
\centering
\end{figure}
\clearpage
\begin{figure}[t]
\includegraphics[width=\textwidth,height=\textheight,keepaspectratio]{figures/meteorite_28-2_edit-b.jpg}
\caption{Tafel 28: Figur 2 --- aus Knyahinya D. 1,20 mm.}
\centering
\end{figure}
\clearpage
\begin{figure}[t]
\includegraphics[width=\textwidth,height=\textheight,keepaspectratio]{figures/meteorite_28-3_edit-b.jpg}
\caption{Tafel 28: Figur 3 --- aus Knyahinya D. 2,30 mm.}
\centering
\end{figure}
\clearpage
\begin{figure}[t]
\includegraphics[width=\textwidth,height=\textheight,keepaspectratio]{figures/meteorite_28-4_edit-b.jpg}
\caption{Tafel 28: Figur 4 --- aus Knyahinya D. 0,90 mm.}
\centering
\end{figure}
\clearpage
\begin{figure}[t]
\includegraphics[width=\textwidth,height=\textheight,keepaspectratio]{figures/meteorite_28-5_edit-b.jpg}
\caption{Tafel 28: Figur 5 --- aus Knyahinya D. 1,50 mm.}
\centering
\end{figure}
\clearpage
\begin{figure}[t]
\includegraphics[width=\textwidth,height=\textheight,keepaspectratio]{figures/meteorite_28-6_edit-b.jpg}
\caption{Tafel 28: Figur 6 --- aus Knyahinya D. 1,40 mm.}
\centering
\end{figure}
\clearpage
\rhead{Tafel 29: Crinoiden (1-3 von oben gesehen, 4 von unten.)}
\begin{figure}[t]
\includegraphics[width=\textwidth,height=\textheight,keepaspectratio]{figures/meteorite_29-1_edit-b.jpg}
\caption{Tafel 29: Figur 1 --- aus Knyahinya D. 0,20 mm.}
\centering
\end{figure}
\clearpage
\begin{figure}[t]
\includegraphics[width=\textwidth,height=\textheight,keepaspectratio]{figures/meteorite_29-2_edit-b.jpg}
\caption{Tafel 29: Figur 2 --- aus Knyahinya D. 0,90 mm.}
\centering
\end{figure}
\clearpage
\begin{figure}[t]
\includegraphics[width=\textwidth,height=\textheight,keepaspectratio]{figures/meteorite_29-3_edit-b.jpg}
\caption{Tafel 29: Figur 3 --- aus Tabor D. 2,10 mm.}
\centering
\end{figure}
\clearpage
\begin{figure}[t]
\includegraphics[width=\textwidth,height=\textheight,keepaspectratio]{figures/meteorite_29-4_edit-b.jpg}
\caption{Tafel 29: Figur 4 --- aus Knyahinya D. 1,10 mm.}
\centering
\end{figure}
\clearpage
\begin{figure}[t]
\includegraphics[width=\textwidth,height=\textheight,keepaspectratio]{figures/meteorite_29-5_edit-b.jpg}
\caption{Tafel 29: Figur 5 --- aus Borkut D. 1,50 mm.}
\centering
\end{figure}
\clearpage
\begin{figure}[t]
\includegraphics[width=\textwidth,height=\textheight,keepaspectratio]{figures/meteorite_29-6_edit-b.jpg}
\caption{Tafel 29: Figur 6 --- aus Knyahinya D. 1,30 mm. (zweifelhaft)}
\centering
\end{figure}
\clearpage
\rhead{Tafel 30: Crinoiden}
\begin{figure}[t]
\includegraphics[width=\textwidth,height=\textheight,keepaspectratio]{figures/meteorite_30-1_edit-b.jpg}
\caption{Tafel 30: Figur 1 --- aus Knyahinya D. 1,10 mm. (Koralle?)}
\centering
\end{figure}
\clearpage
\begin{figure}[t]
\includegraphics[width=\textwidth,height=\textheight,keepaspectratio]{figures/meteorite_30-2_edit-b.jpg}
\caption{Tafel 30: Figur 2 --- aus Knyahinya D. 1,40 mm. (Koralle und Crinoid, vergl. Tafel 20.)}
\centering
\end{figure}
\clearpage
\begin{figure}[t]
\includegraphics[width=\textwidth,height=\textheight,keepaspectratio]{figures/meteorite_30-3_edit-b.jpg}
\caption{Tafel 30: Figur 3 --- aus Knyahinya D. 0,30 mm. (die Arme nezförmig verschlungen)}
\centering
\end{figure}
\clearpage
\begin{figure}[t]
\includegraphics[width=\textwidth,height=\textheight,keepaspectratio]{figures/meteorite_30-4_edit-b.jpg}
\caption{Tafel 30: Figur 4 --- aus Knyahinya D. 1,85 mm. (Anschnitt)}
\centering
\end{figure}
\clearpage
\begin{figure}[t]
\includegraphics[width=\textwidth,height=\textheight,keepaspectratio]{figures/meteorite_30-5_edit-b.jpg}
\caption{Tafel 30: Figur 5 --- aus Knyahinya D. 0,70 mm. (Anschnitt)}
\centering
\end{figure}
\clearpage
\begin{figure}[t]
\includegraphics[width=\textwidth,height=\textheight,keepaspectratio]{figures/meteorite_30-6_edit-b.jpg}
\caption{Tafel 30: Figur 6 --- aus Knyahinya D. 0,40 mm. (Struktur dem des Schreibersits im Meteoreisen gleich)}
\centering
\end{figure}
\clearpage
\rhead{Tafel 31: \emph{Problematica}}
\begin{figure}[t]
\includegraphics[width=\textwidth,height=\textheight,keepaspectratio]{figures/meteorite_31-1_edit-b.jpg}
\caption{Tafel 31: Figur 1 --- aus Knyahinya D. 1,20 mm. (nicht ganz vollständiges Bild)}
\centering
\end{figure}
\clearpage
\begin{figure}[t]
\includegraphics[width=\textwidth,height=\textheight,keepaspectratio]{figures/meteorite_31-2_edit-b.jpg}
\caption{Tafel 31: Figur 2 --- aus Knyahinya D. 0,50 mm.}
\centering
\end{figure}
\clearpage
\begin{figure}[t]
\includegraphics[width=\textwidth,height=\textheight,keepaspectratio]{figures/meteorite_31-3_edit-b.jpg}
\caption{Tafel 31: Figur 3 --- aus Knyahinya D. 1,20 mm. (Drei übereinstimmende Formen aus 3 Dünnschliffen, in 1 und 2 beidemale der horizontale Ausschnitt)}
\centering
\end{figure}
\clearpage
\begin{figure}[t]
\includegraphics[width=\textwidth,height=\textheight,keepaspectratio]{figures/meteorite_31-4_edit-b.jpg}
\caption{Tafel 31: Figur 4 --- aus Knyahinya (ob Schwamm oder Koralle?) D. 0,90 mm.}
\centering
\end{figure}
\clearpage
\begin{figure}[t]
\includegraphics[width=\textwidth,height=\textheight,keepaspectratio]{figures/meteorite_31-5_edit-b.jpg}
\caption{Tafel 31: Figur 5 --- aus Knyahinya D. 1,50 mm.}
\centering
\end{figure}
\clearpage
\begin{figure}[t]
\includegraphics[width=\textwidth,height=\textheight,keepaspectratio]{figures/meteorite_31-6_edit-b.jpg}
\caption{Tafel 31: Figur 6 --- aus Knyahinya D. 1,40 mm.}
\centering
\end{figure}
\clearpage
\rhead{Tafel 32: Verschieden}
\begin{figure}[t]
\includegraphics[width=\textwidth,height=\textheight,keepaspectratio]{figures/meteorite_32-1_edit-b.jpg}
\caption{Tafel 32: Figur 1 --- aus Knyahinya (Einschluss) D. 1,50 mm.}
\centering
\end{figure}
\clearpage
\begin{figure}[t]
\includegraphics[width=\textwidth,height=\textheight,keepaspectratio]{figures/meteorite_32-2_edit-b.jpg}
\caption{Tafel 32: Figur 2 --- Borkutkugel D. 1,00 mm.}
\centering
\end{figure}
\clearpage
\begin{figure}[t]
\includegraphics[width=\textwidth,height=\textheight,keepaspectratio]{figures/meteorite_32-3_edit-b.jpg}
\caption{Tafel 32: Figur 3 --- Nummulit von Kempten. Die Kanäle sind (mit der Lupe) scharf zu erkennen}
\centering
\end{figure}
\clearpage
\begin{figure}[t]
\includegraphics[width=\textwidth,height=\textheight,keepaspectratio]{figures/meteorite_32-4_edit-b.jpg}
\caption{Tafel 32: Figur 4 --- Dünnschliff von Lias $\gamma\delta$. Dieser Dünnschliff ist der von mir zusammengestellten Sammlung von 30 Dünnschliffen von Sedimentgesteinen entnommen, gefertigt von Geognost Hildebrand in Ohmenhausen bei Reutlingen, welche ich zum Studium der mikroskopischen Beschaffenheit der Sedimentgesteine nebst Einschlüssen dringend empfehle.}
\centering
\end{figure}
\clearpage
\begin{figure}[t]
\includegraphics[width=\textwidth,height=\textheight,keepaspectratio]{figures/meteorite_32-5_edit-b.jpg}
\caption{Tafel 32: Figur 5 --- \emph{Eozoön canadense}\index{Eozoön}, angebliches Kanalsystem des \emph{Eozoön}\index{Eozoön}.}
\centering
\end{figure}
\clearpage
\begin{figure}[t]
\includegraphics[width=\textwidth,height=\textheight,keepaspectratio]{figures/meteorite_32-6_edit-b.jpg}
\caption{Tafel 32: Figur 6 --- desgl. Beide Gesteine, denen die Schliffe entnommen sind, von mir in Little Nation gesammelt. Man vergleiche Kanalsystem des Nummuliten Fig. 3 mit diesem angeblichen Kanalsystem! Bild 3 und 5 sollen dasselbe Ding sein. Zu Fig. 5 vergleiche \emph{Urzelle} Tafel 4. 5.}
\centering
\end{figure}
\clearpage
\pagestyle{plain}
\printindex
\clearpage
\end{document}
