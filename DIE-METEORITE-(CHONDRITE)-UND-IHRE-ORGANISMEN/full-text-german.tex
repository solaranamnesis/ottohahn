\documentclass[a4paper, 12pt, oneside]{article}
\usepackage[utf8]{inputenc}
\usepackage[T1]{fontenc}
\usepackage[ngerman]{babel}
\usepackage{fouriernc}
\usepackage{booktabs}
\usepackage{url}
\usepackage{graphicx}
\setlength{\emergencystretch}{15pt}
\graphicspath{ {./figures/} }
\usepackage[figurename=]{caption}
\usepackage{fancyhdr}
\usepackage{imakeidx}
\makeindex[columns=2, title=Alphabetical Index, intoc]
\begin{document}
\begin{titlepage} % Suppresses headers and footers on the title page
	\centering % Centre everything on the title page
	\scshape % Use small caps for all text on the title page

	%------------------------------------------------
	%	Title
	%------------------------------------------------
	
	\rule{\textwidth}{1.6pt}\vspace*{-\baselineskip}\vspace*{2pt} % Thick horizontal rule
	\rule{\textwidth}{0.4pt} % Thin horizontal rule
	
	\vspace{0.75\baselineskip} % Whitespace above the title
	
	{\LARGE DIE METEORITE (CHONDRITE)\\ UND\\ IHRE ORGANISMEN\\} % Title
	
	\vspace{0.75\baselineskip} % Whitespace below the title
	
	\rule{\textwidth}{0.4pt}\vspace*{-\baselineskip}\vspace{3.2pt} % Thin horizontal rule
	\rule{\textwidth}{1.6pt} % Thick horizontal rule
	
	\vspace{1\baselineskip} % Whitespace after the title block
	
	%------------------------------------------------
	%	Subtitle
	%------------------------------------------------
	
	{Dargestellt und Beschrieben\\ von\\ \scshape\Large Dr. Otto Hahn\\} % Subtitle or further description
	
	\vspace*{1\baselineskip} % Whitespace under the subtitle
	
    {\small 32 Tafeln mit 142 Abbildungen} % Subtitle or further description
    
	%------------------------------------------------
	%	Editor(s)
	%------------------------------------------------
	
	\vspace{1\baselineskip} % Whitespace below the editor list
	
    %------------------------------------------------
	%	Cover photo
	%------------------------------------------------
	
	\includegraphics[scale=0.95]{cover}
	
	%------------------------------------------------
	%	Publisher
	%------------------------------------------------
		
	\vspace{1\baselineskip} % Whitespace under the publisher logo
	
	1$^{st}$ Edition, Tübingen 1880 % Publication year
	
	{\small Verlag der H. Laupp'schen Buchhandlung } % Publisher

	\vspace{1\baselineskip} % Whitespace under the publisher logo

    2$^{nd}$ Edition, Internet Archive 2020 % Publication year
	
	{\small Namensnennung Nicht-kommerziell Weitergabe unter gleichen Bedingungen 4.0 International } % Publisher
\end{titlepage}
\setlength{\parskip}{1mm plus1mm minus1mm}
\setcounter{tocdepth}{2}
\setcounter{secnumdepth}{3}
\tableofcontents
\clearpage
\listoffigures
\clearpage
\section{Einleitung}
\subsection{Einleitung}
\paragraph{}
Nicht die zum Teil wenig sachlichen Angriffe auf meine \emph{Urzelle} waren es, welche mich in meinen Anstrengungen, gewisse neue geologische Tatsachen festzustellen — nicht ermüden ließen: es war die durch Beobachtungen gewonnene Überzeugung von der Unhaltbarkeit der bisherigen Anschauung in dem unstreitig wichtigsten Teile der geologischen Wissenschaft, in dem Teile, durch welchen er gerade mit dem Kosmos zusammenhängt — in der Lehre von den sogenannten plutonischen Gesteinen.

Hatte ich es im ersten Teile meiner \emph{Urzelle} noch mit Ergebung hingenommen, dass der Erdkern und damit auch die Erkenntnis der wirklichen Entstehungs-Geschichte unserer Erde uns stets verborgen bleiben werde: so bot sich doch am Schluss dieses Buchs schon ein Ausblick: der Meteorstein zeigte die ferne Durchfahrt, welche noch von keinem Forscher gewagt worden war.

Mit diesem Führer nun entschloss ich mich vorwärts zu schreiten.

Ich tat es, begleitet auf der einen Seite von dem bald leiser bald schärfer ausgesprochenen Spotte der Fachmänner, auf der andern Seite aufgemuntert durch die früher und nun täglich neu gewonnenen Ergebnisse und unterstützt von dem Rat weniger Freunde, welche zu überzeugen mir gelungen war.

Das was mir meine bei einem anstrengenden Beruf fast über Menschenkraft gehenden Arbeiten des letzten Jahres an Ergebnissen geliefert haben, ist in den folgenden Blättern niedergelegt.

Es ist die Tierwelt in einem Gesteine, welches auf unsere Erde herabfiel und uns Kunde brachte von kleinsten Wesen aus fernsten Räumen — eine Tierwelt, welche zu erblicken ein sterbliches Auge kaum hoffen konnte: eine Welt von Wesen, welche uns zeigt, dass dieselbe Schöpferkraft, welche unsere Erde aus einem Dunstnebel hat werden lassen, überall und gleichmäßig im Weltraum gewirkt und geschafft hat.

Freilich weist das Gestein der Meteorite und zwar der Chondrite — denn diese sind's, welche ich vorzugsweise zum Gegenstand meiner Untersuchung machte, — keine Tiere höheren Baus auf; Alles sind niedere Tiere — dieselben, welche in unseren Silurschichten vorherrschen — Schwämme, Korallen und Crinoiden, und auch in ihren Spezies-Merkmalen stimmen sie mit dieser Schöpfung.

Das Gestein der Chondrite, welche ich untersucht habe, ist ein Olivin-Enstatit-Gestein. Es hat von der Zeit seiner Entstehung, vom Tierknochen, bis es fiel, Verwandlungen durchgemacht, aber keine erhebliche: es ist nur von einer Silicatlösung durchtränkt worden, wie alle unsere Jurameer-Ablagerungen von einer Lösung von Kalk. Wahrscheinlich machte es, so lange es noch ein Teil eines Planeten war, noch mehr Planeten-Perioden durch, wie auch den tieferen Schichten unserer Erde andere gefolgt sind, unter deren Einfluss dann die früheren eine gewisse, freilich nicht so erhebliche Umwandlung als man gewöhnlich annimmt, erfahren haben.

Wesentlich geändert hat sich nur die Oberfläche des Gesteins und zwar im letzten Augenblick seines planetarischen Lebens durch den Einfluss der Reibungswärme, entstanden im Falle durch die Erdatmosphäre. Doch das Bild des ursprünglichen Gesteins ist im Wesentlichen geblieben. Wir sehen nun vor uns ein Stück Planeten wie er im Werden war, und damit ist uns die Geschichte unseres Erdkörpers aufgeschlossen, sofern wir ein Recht haben von der Bildung eines seiner Bewegung nach gleichartigen, in seiner chemischen Zusammensetzung gleichen Weltkörpers auf die gleiche Bildung der Erde zu schließen und umgekehrt.

Gleichzeitig war mir durch die Zusendung des "`Meteorite von Ovifak"' (ich verdanke ihn der Güte des Herrn Professors Dr. von Nordenskjöld) Gelegenheit geboten, dieses Gestein in die Untersuchung hereinzuziehen.

Ich halte es für irdisch — halte es für die tiefste Schichte unserer Erde, der Olivinschichte, die unter dem Granit lagert, angehörend. Ich nenne die ursprüngliche Schichte Olivin-Formation. Da das Gestein einem Meteorite sehr ähnlich ist, lag es nahe, dasselbe für einen solchen zu erklären. Die Gründe, warum ich es nicht für meteoritisch, sondern für den wahren Erdkern halte, sind in diesem Buche niedergelegt.

So haben wir zwei feste Punkte gewonnen, von welchen aus ein Hebel angesetzt werden kann.

Die Chondrite, ein Olivin-Feldspat-(Enstatit-)Gestein besteht aus einer Tierwelt, sie sind nicht ein Lager, nicht ein Konglomerat, sondern ein Filz von Tieren, ein Gewebe, dessen Maschen alle lebendige Wesen waren, und zwar Tiere der niedersten Art, Anfänge einer Schöpfung.

Ich konnte nun allerdings von dieser Tierwelt, welche uns in den Meteoriten erhalten ist, keine systematische Aufzählung machen: ich wollte nur nachweisen, dass sie ist — da ist. Ich bildete daher nur ganz unzweifelhaft organische Wesen ab, wobei ich mich damit begnügen musste, einerseits die Gattungen festzustellen, welche mit unseren terrestrischen Formen übereinstimmen, andererseits die spezifisch meteoritischen Formen auszusondern und beides künftiger Untersuchung in die Hand zu geben.

Es ist zu erwarten, dass meine Aufzählung durch weitere Forschung mit Hilfe eines reicheren Materials, als mir zu Gebot stand, bald sich vermehren und ergänzen werde. Es mussten daher insbesondere Untereinteilungen unterbleiben: jedes neu gefundene Wesen würde die Einteilung umgestoßen und damit die mühevolle, voreilige Arbeit auch zur vergeblichen gemacht haben.

Dies war der Grund warum ich nur die großen Abteilungen und diese nur insoweit gemacht habe, als dies zum Verständnis der Formen beiträgt: erschöpfend und abgeschlossen soll, das wiederhole ich, die Arbeit in dieser Richtung nicht sein.

Auch in anderer Richtung muss ich Nachsicht in Anspruch nehmen: in der Abgrenzung der Hauptabteilungen selbst.

Wer meine Formen nur oberflächlich überblickt, wird bald finden, dass sie eine wirkliche Entwicklungsgeschichte an die Hand geben. Alle die Übergänge vom Schwamm zur Koralle, von der Koralle zum Crinoiden sind da, so dass es wirklich zweifelhaft werden kann, will man nicht eine neue Tiergattung machen, wohin man diese Übergänge stellen soll.

In solchen Anfängen sind Irrtümer unvermeidlich, es ist daher nur eine Forderung der Billigkeit, sie zu verzeihen. Auch wollte ich die Veröffentlichung des Werkes nicht zu lange verzögern, und habe es daher eben so wie es jetzt vorliegt, abgeschlossen.
\clearpage
\subsection{Geschichte und Überblick}
$\Delta$o$\sigma$ $\mu$o$\iota$ $\chi\epsilon\nu\tau\rho$o$\nu$%Δός μοι χέντρον
\paragraph{}
Als ich im vorigen Jahre mein Tagebuch enthaltend gewisse neue Beobachtungen über die Zusammensetzung der Gesteine unserer Erde und schließlich auch der Meteorite, niederschrieb, war mir die Wichtigkeit der letzteren für unsere Erdkunde noch nicht völlig klar.

Erst als ich durch die Angriffe der Gegner gezwungen war, die Untersuchung aufs Neue in die Hand zu nehmen, trat es mir klar vor die Augen, welch' hohe Bedeutung eine sorgfältige Erforschung der Meteorite für die Geschichte unserer Erde haben müsse. Zuletzt kam ich zu der Überzeugung, dass bei dem jetzigen Stand unserer Erdkunde die Meteorite und nur die Meteorite den Punkt abgäben, von welchem aus unsere Erdgeschichte wenigstens mit ziemlicher Sicherheit erforscht werden könne.

Wenn ich also in meiner \emph{Urzelle} mit dem Granit die mögliche Grenze der Forschung erreicht zu haben glaubte, so wurde ich bald eines Bessern belehrt. Ich erwog, dass unser Erdkern vermöge seines spezifischen Gewichts ebenfalls mindestens aus gediegenem Eisen bestehen müsse, erwog ferner die sehr wahrscheinliche Reihenfolge in den Meteoriten, welche vom reinen Eisen bis zu den Feldspatgesteinen unserer Erde geht. Ich glaubte ferner, dass ein Rückschluss von unserer Erde auf die Meteorite gewagt werden dürfe, der Schluss, dass auch in den übrigen Planeten und in denjenigen oder demjenigen, deren (oder dessen) Trümmer wir wohl in den hunderttausend von kreisenden Meteoriten vor uns haben, eine Reihenfolge der Schichtung vom Schweren zum Leichten bestanden habe, eine Schicht-Folge, welche wir wahrscheinlich in der Reihe vom reinen Eisen durch die Halbeisen (Pallasite, Hainholz) hindurch zu den Chondriten und Eukriten, dann zu den Ton-(Kohle-)Meteoriten (Bokkefeld) vor uns haben.

Nachdem diese Wahrscheinlichkeit einmal gewonnen war, lag es nahe, die Meteorite einer genauen Prüfung hinsichtlich ihrer morphologischen Eigenschaften zu unterwerfen. Dies war auch in hohem Grade geboten, denn dass bisher in dieser Richtung so gut wie nichts geschehen ist, davon kann man sich durch Vergleichung meiner Abbildungen mit den etwa zwanzig dürftigen Bildern überzeugen, welche zusammen das heute vorliegende Material unserer Wissenschaft bilden. Die akademischen Schriften von Berlin, Wien, München haben je nur einige Tafeln aufzuweisen, die Zeichnungen sind klein, und wie sich sofort zeigt, von den am wenigsten für diese Richtung der Untersuchung geeigneten Meteoriten und ferner wahrscheinlich auch nicht von dem besten Teile, dem Innern, genommen.

Sollte also auch, dachte ich, meine frühere Behauptung: der Meteorstein von Knyahinya bestehe durchaus aus Pflanzen, durch meine neuen Untersuchungen sich nicht bestätigen, so wäre der Wissenschaft doch ein Dienst getan, wenn ich nur die wahre Form dieses Gesteins zur Darstellung bringen würde. Doch dieser Rückzug blieb mir glücklicherweise erspart, im Gegenteil: das Ergebnis der neuen Forschung war ein alle Erwartung weit übersteigendes — eine neue Welt tat sich auf.

Aber freilich — unsere Wissenschaft ist ungläubig — sie fordert mit Recht strengere Beweise, als ich in meiner \emph{Urzelle} geboten habe; ein Buch, das fast mehr im Stadium, ich möchte sagen, der Intuition geschrieben ist. — Heute lege ich Beweise vor.

Man überblicke die Tafeln dieses Werks und es wird sofort zur Gewissheit, dass es sich hier nicht um Mineral-, sondern um organische Formen handelt, dass wir die Bilder von Tieren vor uns haben, Bilder von Tieren der niedersten Stufe, einer Schöpfung, welche zum größeren Teile wenigstens ihre nächsten Verwandten auf unserer Erde findet; — hinsichtlich der Korallen und Crinoiden ist dies mit unbedingter Sicherheit festgestellt: die Schwämme aber haben wenigstens eine solche Ähnlichkeit mit den Formen der Erde aufzuweisen, wie sie eben innerhalb verwandter irdischer Gattungen besteht.

So war die Entstehung hinsichtlich der Teile festgestellt. Nun bestätigte sich aber auch bei meiner Untersuchung von 20 Chondriten (und 360 Dünnschliffen davon) die in meiner \emph{Urzelle} aufgestellte Behauptung, dass das Gestein der Chondrite nicht etwa nach Art der Sedimentgesteine unserer Erde nur ein Schlamm sei, in welchen die Versteinerungen eingelagert sind, dass es nicht eine Konglomeratbildung sei; ihre ganze Masse ist vielmehr völlig aus organischen Wesen gebildet, wie unsere Korallenfelsen. Also keine Pflanze, wie ich früher annahm, aber Pflanzentiere! Und der ganze Stein ein Leben: — ich denke, die Wissenschaft darf mir den ersten Irrtum gerne verzeihen.

Selbstredend war nun auch das Meteoreisen nochmals einer Prüfung zu unterwerfen. Hier blieb es bei meiner ersten Beobachtung.

Allerdings gestatteten mir Zeit und Umstände, insbesondere der Mangel an verfügbarem Material nicht, die Untersuchung darüber vor dieser Veröffentlichung abzuschließen. Wenn ich aber heute die erste Behauptung, dass das Meteoreisen nichts als ein Pflanzenfilz sei, in der Hauptsache wiederhole, so darf ich mich doch jetzt zu der Behauptung eher legitimiert ansehen, als zur Zeit, als ich die \emph{Urzelle} schrieb. Beizufügen habe ich, dass ich auch im Eisen Tierformen fand. Die Forscher, denen die Formen der Chondrite entgingen, welche ich abbilde, können auch übersehen haben, dass die sogenannten Widmannstätten'schen Figuren in der Tat größtenteils Pflanzenzellen und keine Kristalle sind.

Die bisherigen Untersuchungen auf dem ganzen Gebiete mit Ausnahme der Arbeit [Karl Wilhelm von] Gümbel's in den Schriften der Münchener Akademie sind, sowohl was Genauigkeit der Beobachtung, noch mehr aber was die auf solcher Beobachtung, auf unbewiesenen Hypothesen und leeren Voraussetzungen ruhende Deutung betrifft — wenig geeignet, als eine wissenschaftliche Feststellung angesehen zu werden. So war mir in der Tat das Feld noch völlig offen, wobei ich nur bedaure, dass ich bezüglich der Eisen vorerst noch keine Vorlage machen kann.

Ich komme nun zur Schlussfolgerung für unsere Erdkunde. Sind nämlich die Chondrite — also ein Olivin- und Enstatit-Gestein wirklich, was ich zur Gewissheit bringe, nur Stücke von Schwamm-Korallen-Crinoiden-Felsen, so ist für die Wissenschaft unserer Erde eine Tatsache von unermesslicher Tragweite gewonnen.

Ein Feldspatmineral ist reines Wasserprodukt, ist Versteinerungsmittel für Millionen von Organismen! Damit fallen alle Hypothesen über die metamorphischen und plutonischen Gesteine unserer Erde, damit fällt die Theorie von dem feuerflüssigen Erdinnern, — wenigstens kann aus dem Gestein kein Schluss mehr darauf gezogen werden.

Ich muss dies noch näher begründen. Die Vergleichung der Gesteine der Erde und der Meteorite zeigt, dass der Chondrit, wenigstens nach seiner chemischen Beschaffenheit, seine allernächsten Verwandten auf der Erde hat.

Das Olivingestein unserer Erde ist als Llerzolith ein Lagergestein, als Basalt sehen wir es den Granit durchbrechen; ich traf hier mit den Ergebnissen, welche [Gabriel Auguste] Daubrée gewonnen hatte, zusammen.

Der tieferliegende Granit ist also jedenfalls jünger als der Olivin. Ist aber das Olivingestein der Meteorite vermöge seiner Zusammensetzung ein Wassergestein, so wird es wohl der Granit unserer Erde auch sein; besteht das Olivingestein der Meteore aus niederen Tieren, so wird dasselbe bei dem Olivingestein der Erde der Fall: es wird wohl der Schluss uns erlaubt sein, dass auch dieses Gestein unserer Erde auf seiner ursprünglichen Lagerstätte aus denselben Tieren zusammengesetzt ist, wie der Chondrit. — Und aus demselben Grunde wird auch der Granit, als jüngeres Gestein, wohl denselben Ursprung haben. Haben wir in unserem (schwäbischen) Basalt nur Auslaugungen aus dem ursprünglichen Olivingestein zu erblicken, so ist doch die Lagerung des Llerzoliths unter dem Granit festgestellt. Und erscheint auch dieses Gestein als eine Wasserablagerung ohne unterscheidbare Formen, so hat doch das Eisen von Ovifak solche; dieses aber ist so sehr mit dem Basalt, so innig und nicht bloß mechanisch verbunden, dass beide als ein Gestein angesehen werden müssen. Dieses ist also das ursprüngliche Olivin-Lagergestein. Damit aber ist der Annahme einer Entstehung der Erde auf feurigem Wege der wissenschaftliche Grund entzogen.

Bestand die Oberfläche der Planeten oder des Planeten in den Schichten des Olivins aus Tieren, so ist dieselbe Schichte unserer Erde wohl auch nicht durch Feuer entstanden: wenigstens ist nicht der mindeste Grund zu dieser Vermutung mehr vorhanden, im Gegenteil, es ist anzunehmen, dass auch dieselbe Schichte der Erde eine Wasserbildung gewesen sei. — Hier traf ich nun auf die Kant-Laplace'sche Theorie.

Ich kann mir die Stoffe der Planeten (einschließlich des Wassers, welches gewöhnlich vergessen wird!) zur Zeit der ersten Massenbildung, wie [Immanuel] Kant und [Pierre-Simon] Laplace nur in Dunstform, aber freilich nicht als einen glühenden Dunst denken, sondern nur als Dunst- und Gasmasse im kalten Weltraum. Hier hat man aber den großen logischen Fehler in der genannten Theorie übersehen.

Die Massenanziehung sollte die Masse bilden! Die Wirkung sollte zugleich Ursache sein! Die Masse nämlich sollte sich durch Masseanziehung bilden, also dadurch entstehen, dass sie schon da war! Es ist zu bedauern, dass man diesen Denkfehler nicht früher entdeckt hat. Die Masse kann, wenn sie da ist, sich durch Anziehung vergrößern, aber nicht dadurch werden: es ist als ob Jemand sein eigne Vater sein sollte!

Also eine andere Kraft musste die Masse bilden: diese aber konnte nur entweder die Kristallisations-Kraft oder die organische Bildungskraft sein.

Erstere reicht zur Erklärung der Planetenbildung nicht hin, und es finden sich keine Kristalle: folglich bleibt bloß die zweite Kraft übrig — die organische. Hier erinnere ich an meine Beobachtungen der Struktur des Meteoreisens und so steht heute, für mich wenigstens, die Tatsache fest, dass der erste Anfang unserer Erde, wie der übrigen Planeten, eine organische Ursache hatte.

Erscheint der Satz auch etwas betäubend, so braucht man nur zu ganz Bekanntem zu greifen.

Erstens: Die Masse der Baustoffe, welche im Anfang der Planetenbildung zu Gebot stand, reicht vollständig hin, um die Bildung auch einer Planeten-Masse auf organischem Wege zu erklären.

Zweitens lehrt die Erfahrung von heute, in welch' kurzer Zeit sich die niedersten Pflanzen und Tiere vermehren, dass ihre Zahl, also auch ihre Masse, lediglich durch die Masse der Baustoffe bedingt ist, während ihre Organisation selbst eine Ausdehnung ins Unendliche (so lange nämlich Baustoffe da sind) möglich macht.

Was dieser Erklärung entgegen zu stehen scheint, ist nur die Erdwärme und die damit in Verbindung gebrachte Erscheinung der heute noch tätigen Vulkane. Allein bezüglich dieser beiden Tatsachen ist man längst auf eine andere Erklärung, als auf ein feuerflüssiges Erdinneres, zurückgeführt. Das Wasser wirkt auf Feldspat zersetzend ein. Bei diesem Zersetzungsprozess wird Wärme frei. Die Vulkane folgen dem Meere, weil das Wasser die Gase bilden hilft, welche, von oben entzündet, das anstehende Gestein und auch nur dieses schmelzen.

Wie sollte endlich ein feuriger Erdkern ohne Sauerstoff bestehen können! Und führt nicht eben auch das Dasein brennbarer Gase (denn solche sind die Ursachen der vulkanischen Erscheinungen,) insbesondere das der Schwefelgase auf organische im Erdinnern vorhandene Stoffe zurück? Hier bedarf es wahrhaft keiner neuen Beweise, sondern nur des Aufgebens gewisser Vorstellungen, welche sich der aus einigen augenfälligen Erscheinungen erregten Phantasie bemächtigt haben.

Dies sind die Schlussfolgerungen aus der Untersuchung über die Meteorite für unsere Erdbildung. Ungleich bedeutender aber sind die Tatsachen, welche die Astronomie daraus ableiten kann.

Die Dünnschliffe von 20 von mir untersuchten Meteoriten (Chondriten), von Fällen, welche über ein Jahrhundert auseinander liegen, zeigen dieselben Formen, ähnlich wie eine Leitmuschel überall in derselben Formation vorkommt; dies hat schon Gümbel, wenn er die Formen der Chondrite auch nicht richtig gedeutet hat, trefflich ausgesprochen.

Diese Chondrite stammen also wahrscheinlich von einem und demselben Weltkörper, einem Planeten. Oder ist gar bei verschiedenen Planeten die Entwicklung eine so sehr übereinstimmende gewesen?

Dieser Weltkörper trägt Wassertiere, ist also im Wasser und durch Wasser entstanden, auch nicht durch Feuer vergangen, denn Spuren des Feuers zeigen diese Gesteine nicht: der Meteorit ist zersprungen, seine Trümmer haben nur in ihrem kurzen Weg durch unsere Atmosphäre eine 1 mm dicke Schmelzrinde, in Folge der Reibungswärme, erhalten.

Die Tier-Schöpfung der Chondrite ist beinahe durchaus eine mikroskopische, Tiere sind es von 0,20 bis höchstens 3 mm Durchmesser, oft bedarf es einer Vergrößerung von 1000, um ihre zarte Struktur klar zu sehen, während bei solcher Vergrößerung unsere Petrefacten in eine gestaltlose Fläche sich auflösen.

So war mir durch die erste in meiner \emph{Urzelle} niedergelegte Beobachtung ein Weg geöffnet, auf welchem weite, weite Strecken unserer Wissenschaft gewonnen werden müssen.

Es bedurfte aber wahrlich gerade keiner Titanenkraft mehr, um das alte Gebäude umzustürzen, es war schon viel vorgearbeitet, nur nicht beachtet: es bedarf nur eines einzigen durchschlagenden Beweises und die Arbeit ist getan. Überlieferungen, auf ungenügende Beobachtungen gestützt, lösen sich in das auf, was sie sind, und nun hat die Wissenschaft wieder freie Bahn.
\clearpage
\subsection{Die Bisherigen Ansichten über die Meteorite}
\paragraph{}
Es folgt nun zunächst eine kurze Darstellung der bisherigen Ansichten über die Entstehung und Natur der Meteorite.

Nur die morphologischen Arbeiten über einzelne Meteorite, von der Zeit an, als man das Mikroskop in der Geologie anzuwenden begann, sollen aufgezählt werden.

Was das Mikroskop bis jetzt zur Deutung der Meteorite geliefert hat, das ist, abgesehen von den vergrößerten Olivinkristallen in [Nikolai Ivanovich] Kokscharow's \emph{Mineralien Russlands VI} Band S. 4, in folgenden Schriften enthalten:

    [Gustav] Tschermak: die Trümmerstruktur der Meteoriten von Orvinio und Chantonnay, vorgelegt in der Sitzung der K. Akademie der Wissenschaften (Wien) am 12. November 1874. (XX. Band der Sitzungsberichte der K. Akademie der Wissenschaften, I. Abteilung, November-Heft 1874. Mit 2 Tafeln.)

    [Alexander] Makowsky und G. Tschermak: Bericht über den Meteoritenfall bei Tieschitz in Mähren. Mit 5 Tafeln und 2 Holzschnitten, vorgelegt in der Sitzung der mathematisch-naturwissenschaftlichen Klasse (der Kgl. Akademie der Wissenschaften in Wien) am 21. November 1878. XXIX. Band der Denkschriften der genannten Klasse.

    [Johann Gottfried] Galle und [Arnold Constantin Peter Franz] von Lasaulx, vorgelegt von [Christian Friedrich Martin] Websky: Bericht über den Meteorsteinfall bei Gnadenfrei am 17. Mai 1879. Sitzung vom 31. Juli 1879. Monatsberichte der K. preußischen Akademie zu Berlin.

Die früheren Beschreibungen beschränken sich auf die Untersuchung mit bloßem Auge und der Lupe, sowie die chemische Analyse.

Sie stimmen alle dahin überein: Die Chondrite bestehen aus einer Grundmasse mit Kugeln von Enstatit (Bronzit), Olivin und Eisen, eingesprengtem Nickel- und Chromeisen.

Eine andere Stellung nimmt ein: Gümbel: Über die in Bayern gefundenen Steinmeteoriten; Sitzungsberichte der mathematisch-physikalischen Klasse der K. b. Akademie der Wissenschaften zu München 1878. Heft 1, S. 14 ff. In der Beschreibung der Meteorite von Eichstädt und Schöneberg erwähnte er "`Maschenstruktur"' (S. 27. 46.) Allerdings spricht er auch von "`Abkömmlingen zerbrochener größerer Chondren"' (S. 28). Das Bedeutende seiner Beobachtungen ist auf S. 58, welche ich hier folgen lasse:

"`Überblickt man die Resultate der Untersuchung dieser wenn auch beschränkten Gruppe von Steinmeteoriten, so drängt sich die Wahrnehmung in den Vordergrund, dass sie, trotz einiger Verschiedenheit in der Natur ihrer Gemengteile, doch von vollständig gleichen Strukturverhältnissen beherrscht sind. Alle sind unzweifelhafte Trümmergesteine, zusammengesetzt aus kleinen und größeren Mineralsplitterchen, aus den bekannten rundlichen Chondren, welche meist vollständig erhalten, aber oft auch in Stücke zersprungen vorkommen und aus Gräupchen von metallischen Substanzen Meteoreisen, Schwefeleisen, Chromeisen. Alle diese Fragmente sind aneinander geklebt, nicht durch eine Zwischensubstanz oder durch ein Bindemittel verkittet, wie sich überhaupt keine amorphen, glas- oder lavaartigen Beimengungen vorfinden. Nur die Schmelzrinde und die oft auf Klüften auftretenden, der Schmelzrinde ähnlich entstandenen schwarzen Überrindungen bestehen aus amorpher Glasmasse, die aber erst beim Niederfallen innerhalb unserer Atmosphäre nachträglich entstanden ist. In dieser Schmelzrinde sind die schwerer schmelzbaren und größeren Mineralkörnchen meist noch ungeschmolzen eingebettet. Die Mineralsplitterchen tragen durchaus keine Spuren einer Abrundung oder Abrollung an sich, sie sind scharfkantig und spitzeckig. Was die Chondren anbelangt, so ist ihre Oberfläche nie geglättet, wie sie sein müsste, wenn die Kügelchen das Produkt einer Abrollung wären, sie ist vielmehr stets höckerig uneben, maulbeerartig rau und warzig oder facettenartig mit einem Ansatz von Kristallflächen versehen. Viele derselben sind länglich, mit einer deutlichen Verjüngung oder Zuspitzung nach einer Richtung, wie es bei Hagelkörnern vorkommt. Oft begegnet man Stückchen, welche offenbar als Teile zertrümmerter oder zersprungener Chondren gelten müssen. Als Ausnahme kommen zwillingsartig verbundene Kügelchen vor, häufiger solche, in welchen Meteoreisenstückchen ein- oder angewachsen sind. Nach zahlreichen Dünnschliffen sind sie verschiedenartig zusammengesetzt. Am häufigsten findet sich eine exzentrisch strahlig faserige Struktur in der Art, dass von einer weit aus der Mitte nach dem sich verjüngenden oder etwas zugespitzten Teil hin verrückten Punkte aus ein Strahlenbüschel gegen Außen sich verbreitet. Da die in den verschiedensten Richtungen geführten Schnitte immer säulen- oder nadelförmige, nie blätter- oder lamellenartige Anordnung in der diesen Büschel bildenden Substanz erkennen lassen, so scheinen es in der Tat säulenförmige Fasern zu sein, aus welchen sich solche Chondren aufbauen. Bei gewissen Schnitten gewahrt man, dieser Annahme entsprechend, in den senkrecht zur Längenrichtung gehenden Querschnitten der Fasern nur unregelmäßig eckige, kleinste Feldchen, als ob das Ganze aus lauter kleinen polyedrischen Körnchen zusammengesetzt sei. Zuweilen sieht es aus, als ob in einem Kügelchen gleichsam mehrere nach verschiedener Richtung hin strahlende Systeme vorhanden wären oder als ob gleichsam der Ausstrahlungspunkt sich während ihrer Bildung geändert habe, wodurch bei Durchschnitten nach gewissen Richtungen eine scheinbar wirre, stängliche Struktur zum Vorschein kommt. Gegen die Außenseite hin, gegen welche der Viereinigungspunkt des Strahlenbüschels einseitig verschoben ist, zeigt sich die Faserstruktur meist undeutlich oder durch eine mehr körnige Aggregatbildung ersetzt. Bei keinem der zahlreichen angeschliffenen Chondren konnte ich beobachten, dass die Büschel so unmittelbar bis zum Rande verlaufen, als ob der Ausstrahlungspunkt gleichsam außerhalb des Kügelchens läge, sofern nur dasselbe vollständig erhalten und nicht etwa ein bloßes zersprungenes Stück vorhanden war. Die zierlich quergegliederten Fäserchen verlaufen meist nicht nach der ganzen Länge des Büschels in gleicher Weise, sondern sie spitzen sich allmählich zu, verästeln sich oder endigen, um andere an ihre Stelle treten zu lassen, so dass in dem Querschnitte eine mannichfache, maschenartige oder netzförmige Zeichnung entsteht. Diese Fäserchen bestehen, wie dies schon vielfach im Vorausgehenden geschildert wurde, aus einem meist helleren Kern und einer dunkleren Umhüllung, jener durch Säuren mehr oder weniger zerlegbar, letztere dagegen dieser Einwirkung widerstehend. Höchst merkwürdig sind die schalenförmigen Überrindungen, welche aus Meteoreisen zu bestehen scheinen und in der Regel nur über einen kleineren Teil der Kügelchen sich ausbreiten. Die gleichen einseitigen, im Durchschnitt mithin als bogenförmig gekrümmte Streifchen sichtbaren Überrindungen kommen auch im Innern der Chondren vor und liefern einen starken Gegenbeweis gegen die Annahme, dass die Chondren durch Abrollung irgend eines Materials entstanden seien, wie denn überhaupt die ganze Anordnung der büscheligen Struktur mit Entschiedenheit gegen ihre Entstehung durch Abrollung spricht. Doch nicht alle Chondren sind exzentrisch faserig; viele, namentlich die kleineren besitzen eine feinkörnige Zusammensetzung, als beständen sie aus einer zusammengeballten Staubmasse. Auch hierbei macht sich zuweilen die einseitige Ausbildung der Kügelchen durch eine exzentrisch größere Verdichtung der Staubteile bemerkbar"'.

Und ferner S. 61:

"`Der gewöhnliche Typus der Meteorite von steiniger Beschaffenheit ist soweit überwiegend derjenige der sog. Chondrite und die Zusammensetzung sowie die Struktur aller dieser Steine so sehr übereinstimmend, dass wir den gemeinsamen Ursprung und die uranfängliche Zusammengehörigkeit aller dieser Art Meteorite — wenn nicht aller — wohl nicht weiter in Zweifel ziehen können.

"`Der Umstand, dass sie sämtlich in höchst unregelmäßig geformten Stückchen in unsere Atmosphäre gelangen — abgesehen von dem Zerspringen innerhalb der letzteren in mehrere Fragmente, was zwar häufig vorkommt, aber doch nicht in allen Fällen angenommen werden kann, namentlich nicht, wenn durch direkte Beobachtung das Fallen nur eines Stückes konstatiert ist, — lässt weiter schließen, dass sie bereits in regellos zertrümmerten Stücken als Abkömmlinge von einem einzigen größeren Himmelskörper ihre Bahnen im Himmelsraume ziehen und in ihrer Zerstreutheit einzeln zuweilen in das Attraktionsbereich der Erde geraten zur Erde niederfallen. Der Mangel ursprünglicher, lavaartiger, amorpher Bestandteile in Verbindung mit der äußern unregelmäßigen Form dürfte von geo- oder kosmologischen Standpunkte aus die Annahme ausschließen, dass diese Meteorite Auswürfe von Mondvulkanen, wie vielfach behauptet wird, sein können."'

Gümbel fasst, nachdem er die Meteorite in die Olivingesteine unserer Erde eingestellt hat, seine Ansicht hinsichtlich der Entstehung (S. 64) in den Satz zusammen:

"`Es scheinen daher die Meteorite aus einer Art erstem Verschlackungsprozess der Himmelskörper, aber da sie metallisches Eisen enthalten — bei Mangel von Sauerstoff und Wasser hervorgegangen zu sein."'

"`So geistreich, fährt er (S. 68) fort, diese Hypothesen Daubrée's und Tschermak's sind (Entstehung aus zertrümmertem Vulkangestein), so kann ich mich doch in Bezug auf die Entstehung der Kügelchen (Chondren) ihrer Ansicht auf Grund meiner neuesten Untersuchungen nicht anschließen. Ich habe im Gegensatze zu Tschermak's Annahme nachzuweisen gesucht, dass das innere Gefüge der Chondren nicht außer Zusammenhang mit ihrer kugeligen Gestalt stehe, und dass man diese Kügelchen weder als Stücke eines Mineralkristalls, noch eines festen Gesteins ansehen könne. Spricht schon ihre nicht geglättete, nicht polierte Oberfläche, welche, wenn durch Abreibung oder Abrollung gebildet, bei solcher Härte des Materials spiegelglatt sein müsste, während sie rauh, höckerig, oft strichweise kristallinisch facettirt erscheint, gegen die Abreibungstheorie, so ist auch gar kein Grund einzusehen, weshalb nicht alle anderen Mineral splitterchen wie Sandkörner abgerundet seien und weshalb namentlich das Meteoreisen, das Schwefeleisen und das sehr harte Chromeisen, wie ich in dem Meteorit von L'Aigle mich überzeugt habe, stets nichtgerundete, oft äußerst fein zerschlitzte Formen besitzen. Wie wäre es zudem denkbar, dass, wie häufig beobachtet wird, innerhalb der Kügelchen konzentrische Anhäufung von Meteoreisen vorkommen? Auch erscheint die exzentrisch faserige Struktur der meisten Kügelchen in ihrem einseitig gelegenen Ausstrahlungspunkte in Bezug auf die Oberfläche nicht als zufällig, sondern der Art der Struktur der Hagelkörner nachgebildet. Dieses innere Gefüge steht im engsten Zusammenhang mit dem Akt ihrer Entstehung, welche nur als eine Verdichtung Mineral bildender Stoffe unter gleichzeitiger drehender Bewegung in Dämpfen, welche das Material zur Fortbildung lieferten, sich erklären lässt, wobei in der Richtung der Bewegung einseitig mehr Material sich ansetzte."'

Weiter freilich spricht Gümbel sich dahin aus, dass das Material, aus welchem die Chondrite bestehen, durch eine gestörte Kristallisation und Zertrümmerung in Folge von explosiven Vorgängen innerhalb eines Raumes sich gebildet habe, welcher von den die Mineralien bildenden Stoffe liefernden Dampf- und Wasserstoffgasen erfüllt war. Er schließt S. 72 bei Besprechung des Meteorites von Kaba:

"`Vielleicht gelingt es dennoch, die Anwesenheit organischer Wesen auf außerirdischen Körpern nachzuweisen."' Ich hoffe dies sei gelungen. — Aus seinen Abbildungen ersieht man, dass bei der Untersuchung ein schlechtes Material zu Gebot stand. Auch hätten immerhin mehr Dünnschliffe gefertigt werden müssen, zudem reicht die Vergrößerung bei Weitem nicht. Ich verweise hier auf das Folgende und die Beschreibung meiner Tafeln.

Was ich in dem Berichte Gümbels so hoch schätze, ist die gewissenhafte vorurteilsfreie, ich möchte sagen unparteiische Beobachtung. Ich habe mir erlaubt, die Schrift Gümbels wörtlich anzuführen, weil es mir in der Tat schwer wird, solche Darstellungen zusammenzufassen und Tatsachen und Deutung zu trennen.

Richtige Beobachtungen und unrichtige Erklärungen stehen so nahe beisammen, dass es unmöglich ist beides zu sondern. Ich glaubte, als ich die Gümbel'sche Abhandlung (nach dem Abschluss meiner Untersuchungen und meines Manuskripts) durchlas, in jedem Augenblick auf meine Resultate zu treten. Aber wie die Woge der Brandung den, welcher das Land gewinnen will, jedes mal dann wieder ergreift und zurückwirft, wenn er schon das Land gefasst zu haben glaubt, so auch hier: allemal reißt das alte Dogma den geehrten Forscher von der rettenden Klippe hinweg in den bodenlosen Strudel der Traditionen zurück.

Daubrée's verdienstvolles Werk \emph{Experimentalgeologie} erhielt ich erst in der Übersetzung zur Hand und ebenfalls nach Abschluss meiner Arbeit. Dass es diese widerlegte, wird wohl Niemand finden. Daubrée hat selbst Knyahinya abgebildet. M. hat gepresst, geschmolzen, aufgelöst, berechnet, nur nicht — gesehen.
\clearpage
\subsection{Die Meteorite und ihre Mineralogischen Eigenschaften}
\paragraph{}
Die Literatur der Meteorite ist eine sehr umfangreiche. Sie ist jedoch, was die Art und Zahl, chemische Zusammensetzung betrifft, so bekannt, dass ich auf diesen Teil derselben, also insbesondere die früheren Arbeiten, nicht einzugehen brauche.

Die Meteorite werden eingeteilt in Eisen und Steine, zwischen beiden steht jedoch noch eine Klasse: Halbeisen, d. h. eine Verbindung von gediegenem Eisen und Stein — die Pallasite. Während die Eisen eine ziemliche Übereinstimmung, sowohl in ihrer chemischen Zusammensetzung, als in der Form ihrer Struktur zeigen, sind die Pallasite (je nach dem Vorwiegend des Eisens) sehr verschieden. Aber es finden sich noch weitere Verschiedenheiten darunter. Hainholz z. B. hat neben Eisen und Olivin ein blaues Mineral (Enstatit) und in diesem einen großen Reichtum von Tierformen. — Die Steine werden eingeteilt in Chondrite, Stannerite [eukriten], Luotolaxer [howarditen], Bokkefelder [karbonatisch], Bishopvillit [aubriten], (Quenstedt, Klar und Wahr S. 280 folg.)

Ich habe mich vorzugsweise mit den Chondriten beschäftigt und, wo ich von Meteoriten rede, rede ich von dieser allerdings auch am zahlreichsten vertretenen Klasse von Stein-Meteoriten.
\paragraph{}
Ich habe untersucht:
\begin{center}
\begin{tabular}{ l r }
 Tabor, Böhmen [Czech Republic] & July 3, 1753\index{meteorite!Tabor} \\
 Siena, Toskana [Italy] & June 16, 1794\index{meteorite!Siena} \\
 L'Aigle, Normandy [France] & April 26, 1803\index{meteorite!L'Aigle} \\
 Weston, Connecticut [USA] & December 14, 1807\index{meteorite!Weston} \\
 Tipperary, Ireland & November 23, 1810\index{meteorite!Tipperary} \\
 Blansko, Brünn [Czech Republic] & November 25, 1833\index{meteorite!Blansko} \\
 Château-Renard, Loiret [France] & July 12, 1841\index{meteorite!Château-Renard} \\
 Linn [Marion] County, Iowa [USA] & February 25, 1847\index{meteorite!Marion County}\index{meteorite!Linn} \\
 Cabarras [Monroe] County, North Carolina [USA] & October 31, 1849\index{meteorite!Monroe County}\index{meteorite!Cabarras} \\
 Mezö-Madaras [Romania] & September 4, 1852\index{meteorite!Mezö-Madaras} \\
 Borkut, Hungary & October 13, 1852\index{meteorite!Borkut} \\
 Bremervörde, Hanover [Germany] & May 13, 1855\index{meteorite!Bremervörde} \\
 Parnallee, East India [Tamil Nadu] & February 28, 1857\index{meteorite!Parnallee} \\
 Heredia, Costa Rica & April 1, 1857\index{meteorite!Heredia} \\
 New Concord, Ohio [USA] & May 1, 1860\index{meteorite!New Concord} \\
 Knyahinya, Hungary & June 9, 1866\index{meteorite!Knyahinya} \\
 Pultusk, Warsaw [Poland] & January 30, 1868\index{meteorite!Pultusk} \\
 Orvinio [Italy] & August 31, 1872\index{meteorite!Olvinio} \\
 Simbirsk [Russia] & [1838]\index{meteorite!Simbirsk} \\
\end{tabular}
\end{center}
\clearpage
\paragraph{}
Alle Gesteine sind durchaus beglaubigt. Ich habe hier vor Allem der Liberalität meines verehrten Lehrers, Herrn Professor Dr. [Friedrich August] von Quenstedt, mit welcher er mir die vorzügliche Tübinger Universitäts-Sammlung (welche bekanntlich zum größten Teil vom Freiherrn [Karl Ludwig] von Reichenbach in Wien stammt) dankend zu gedenken.

Von Knyahinya besitze ich 360 Dünnschliffe, von L'Aigle 6, von Pultusk 6, von den übrigen 1-3. Ich werde sämtliche Steine kurz nach dem Fallort benennen. Bei Herstellung der Dünnschliffe habe ich die Schnitte in 2 Richtungen genommen. Es ergab sich nämlich nach mehreren Versuchen an Knyahinya, dass derselbe nach einer bestimmten Richtung bricht.

Es konnte dies aus den Einschlüssen entnommen werden, welche, nachdem einmal die Stellung gefunden war, regelmäßig bestimmte Formen-Durchschnitte ergaben, welchen dann die Formen in einem senkrecht auf diese Stellung gefertigten Schnitte entsprachen.

Waren die Formen an diesem Steine gestellt, so wäre wohl dieselbe Stellung in den übrigen Steinen zu erhalten gewesen, vorausgesetzt natürlich, dass das Material zu Gebot gestanden hätte. Bei einzelnen ergab sich dieselbe zufällig — bei anderen nicht, es musste aber aus den angeführten Gründen auf weitere Feststellung in dieser Richtung verzichtet werden.

Ich fertigte ferner die Schliffe absichtlich in dreierlei Dicke: schwer durchsichtig, um die ganzen Einschlüsse möglichst vollständig zu bekommen: sehr dünn, um die Strukturverhältnisse klar zu stellen; den größten Teil aber so, dass beides noch zur Anschauung kam.

Ich reihe hier eine Bemerkung an, welche mir Jeder bestätigen wird, welcher sich mit Dünnschliffen von Petrefacten beschäftigt hat.

Nur in seltenen Fällen ist in völlig durchsichtigen, also ganz dünnen Schliffen, noch die Struktur sichtbar. Wer seinen Schliff, wenn er halbdurchsichtig, im Mikroskop betrachtet, ist im höchsten Grad erfreut über die schönen Formen und Linien. In der Freude darüber will er die Sache noch besser machen und erwartet bei fortgesetztem Schleifen ein vollendetes Bild. Aber wenn er den Schliff zum zweiten Mal unter das Mikroskop legt — ist nichts mehr da als eine fast strukturlose Fläche, kaum angedeutete, sogar in den Umrissen verschwommene Formen, aus welchen nun das, was man vorher schon mit der Lupe wahrnahm, nicht einmal mehr mit dem Mikroskop zu ersehen ist. Diese Erscheinung hängt aber mit der Art der Metamorphose des Gesteins und der darin eingeschlossenen Formen zusammen. Die Sache ist jedoch bekannt und bedarf deshalb keiner weiteren Ausführung. Ich musste der Tatsache nur deshalb erwähnen, damit solche, welche Beobachtungen erst anstellen wollen, ohne dass sie dieselbe kennen, nicht überrascht werden und ihre Beobachtungsweise verbessern können.

Dass die Chondrite zum größten Teile aus Bronzit-Enstatit (Augit) und Olivin sowie Magnetkies bestehen, ist eine in der Wissenschaft angenommen Tatsache. Quenstedt, \emph{Handbuch der Mineralogie} S. 722.

Insbesondere aber sind die Einschlüsse, welche ich für Korallen erkläre, für Enstatit angesprochen worden. Damit glaubte man die Struktur derselben erklären zu können. Andere gingen noch weiter und erklärten die Einschlüsse zum Teil für Gläser: (Tschermak).

Ehe ich also an die Begründung meiner Ansicht komme, muss die mikroskopische Erscheinung des hauptsächlich vorkommenden Minerals, des Enstatits, genau festgestellt werden.

Ich erlaube mir hier Kürze halber dasjenige anzuführen, was [Karl Heinrich Ferdinand] Rosenbusch in seinem Buch: \emph{Mikroskopische Physiographie der petrographisch wichtigen Mineralien} Stuttgart 1873 S. 252, über Enstatit (und Bronzit) sagt:

"`Bekanntlich hat man seit den optischen Untersuchungen von [Alfred] Des Cloizeaux den Enstatit, Bronzit und Hypersthen als rhombisch kristallisierend vom Pyroxen getrennt und sie in eine eigene Gruppe zusammengestellt. Dieselben zeigen neben der Spaltung nach dem Prisma von 87° noch weitere Spaltungen nach den vertikalen Pinakoiden, über deren relative Vollkommenheit die Angaben der verschiedenen Forscher nicht genau übereinstimmen. Chemisch bilden diese 3 Mineralien eine ununterbrochene Reihe, an deren Anfange der fast eisenfreie Enstatit und an deren Ende der sehr eisenreiche Hypersthen steht. Enstatit und Bronzit sind sich überdies auch in allen physikalischen Eigenschaften so ähnlich, dass eine Trennung derselben in zwei Spezies kaum durchzuführen sein dürfte. Der Hypersthen dagegen zeigt eine verschiedene optische Orientierung und mag daher immerhin eine eigene Spezies bilden. Interessant ist die von Tschermak gegebene Zusammenstellung der negativen Winkel der optischen Achsen und des Eisengehaltes der drei genannten Mineralien, wobei es sich ergibt, dass mit zunehmendem Gehalte an FeO der Winkel der optischen Achsen stetig abnimmt. Die Mikrostruktur aller Mineralien der Enstatit-Gruppe ist im Allgemeinen eine so ähnliche, dass im speziellen Falle eine sichere Entscheidung unter ihnen nur durch chemische und genaue optische Analyse gegeben werden kann."'

"`Enstatit und Bronzit finden sich in den Gesteinen nicht als Kristalle, sondern fast nur in unregelmäßig begrenzten Kristallkörnern, welche meistens eine sehr dichte Streifung erkennen lassen, die bei dem Enstatit mehr geradlinig, bei dem Bronzit mehr sanft wellig gewunden verläuft. Doch ist dieser Unterschied kein durchgreifender. Die gleiche Streifung zeigt auch der monokline Diallag und der rhombische Bastit, der sich aber durch andere, später zu besprechende, optische Erscheinungen nicht unschwer vom Bronzit trennen lässt. Traf der Schliff den Enstatit oder Bronzit stark geneigt zu seiner Hauptspaltungsfläche, so ist die Oberfläche nicht in gleicher Weise feinfaserig, sondern treppenförmig rauh. Querliegeende Absonderungsflächen und Zierbrechungen sind nicht selten."'

"`An fremdartigen Einlagerungen sind beide verhältnismäßig arm; ja sie fehlen z. B. im Enstatit aus dem Pseudophit des Aloysthals in Mähren und in manchen Enstatiten oder Bronziten der Lherzolithe und Olivinfelsen ganz. Ersterer ist nur von häufigen Adern des Pseudophit durchzogen, von welchen aus in senkrechter Richtung feinfaserige Zersetzungsprodukte in den Enstatit eindringen. Andere Vorkommnisse und selbst andere Individuen desselben Handstücks enthalten dagegen oft massenhafte Einschlüsse von grünen oder braunen Lamellen, Leistchen und Körnern (je nach der Lage der Schliffebene), welche ausnahmslos der vollkommensten Spaltungsrichtung parallel gelagert sind. Der Gedanke liegt nahe, dass die verschiedenen Angaben über die relative Vollkommenheit der pinakoidalen ($\infty$P$\infty$) Spaltung gegenüber der prismatischen vielleicht auf die mehr oder weniger massenhafte Anwesenseit dieser Interpositionen zurückzuführen seien, die zweifellos auch den Metalloiden Schiller auf dem Brachypinakoid bedingen. Dann wäre aber die Leichtigkeit der Trennung in der genannten Richtung mehr eine Absonderung, als eine eigentliche Spaltbarkeit."'

"`Der Enstatit ohne und der Bronzit mit metallischem Schimmer auf der brachypinakoidalen Spaltungsfläche finden sich in Serpentinen von Aloysthal in Mähren (Enstatit) und Mont Brésouars in den Vogesen, in den Lherzolithen und Olivinfelsen, in manchen Olivingabbros, in Streng's Enstatitfels vom Radauthal bei Harzburg und in den Olivinbomben des Dreiser Weihers, sowie in manchen Meteoriten; also stets in Gesellschaft des Olivin und in veränderten Olivingesteinen."'

Für diejenigen, welchen das Buch nicht zu Gebote steht, gebe ich 2 Abbildungen, die eine von Bronzit vom Kupferberg Tafel 1. 1, die andere von Enstatit von Texas Tafel 1. 2, welche mit den Rosenbusch'schen ziemlich übereinstimmen.

Was den Olivin betrifft, so bedarf es keiner Abbildung, da die Formen dieses Gesteins durch Zirkel vollständig erschöpft sind. Es genügt zu sagen, dass reiner frischer Olivin keine Struktur zeigt. Struktur zeigt der Olivin bloß, wenn man seine Einschlüsse oder Anwachsstellen des Kristalls oder Zersetzungserscheinungen (Serpentinbildung) Struktur nennen wollte. Aber sicher findet sich in keinem Kristall etwas, was meinen Formen auch nur ähnlich sieht. Was die Behauptung betrifft, die Kugeln seien Gläser, so wird nicht einmal unterschieden, welche chemische Zusammensetzung diese Gläser gegenüber Enstatit, Bronzit und Olivin haben sollen. Offenbar werden alle Formen zusammen geworfen und für Gläser erklärt, obgleich Enstatit nach Quenstedt (Mineralogie S. 318) unschmelzbar, nach Naumann-Zirkel S. 585 wenigstens schwer schmelzbar ist. Es wird sogar behauptet, dass diese Gläser erst im Fallen entstanden seien. Allein Feuereinwirkungen finden sich bloß in der Rinde. Die Schmelzrinde der meisten Meteorite hat kaum 2 mm Durchmesser.

Die Behauptung, es seien Gläser, wurde der Mitteilung meiner ersten Dünnschliffe entgegengehalten und dabei auf die Ähnlichkeit der meteoritischen Form mit solchen Gläsern in dem Gesteine unserer Erde hingewiesen. So wurde ich von [Ferdinand] Zirkel auf einen Sphaerulit-Liparit verwiesen, dessen Abbildung ich Tafel 1. Figur 3 gebe. Diese Form sollte dartun, dass meine Urania eine Täuschung sei. Ich halte die Form im Liparit für eine Kristallit-Bildung (wahrscheinlich Zeolith). Nun betrachte man die Strukturbilder daneben Tafel 1, Figur 4, 5, 6!

Unsere Forscher, mit Ausnahme Gümbels, sprechen von den Meteoriten als vulkanischen Bomben, erklären das Gestein als identisch mit dem Vulkangesteine der Erde, zählen also den Meteorstein ohne Bedenken zu den vulkanischen. Der Gegenbeweis ist der Gegenstand dieses Buchs.

Richtig allein hat Quenstedt die Frage für eine offene erklärt und gesagt: es sei dem Mikroskop vorbehalten, das Rätsel der Zusammensetzung der Meteorite zu lösen! \emph{Handbuch der Mineralogie} S. 722.
\clearpage
\section{Die Organische Natur der Chondrite}
\subsection{Organisch oder Unorganisch?}
\paragraph{}
Um den Beweis zu führen, dass ein pflanzlicher oder tierischer Organismus vorliege, halte ich für notwendig darzutun:
\begin{enumerate}
    \item eine geschlossene Form,
    \item eine wiederkehrende Form,
    \item wiederkehrend in Entwicklungsstufen,
    \item Struktur und zwar entweder Zellen oder Gefäße,
    \item Ähnlichkeit mit bekannten Formen.
\end{enumerate}
\paragraph{}
Sind diese Erfordernisse da, so bleibt nur noch zu entscheiden, ob Pflanze oder Tier? Nun fragt sich, erfüllen meine Formen diese Forderungen?

Ich glaube, ehe ich an den positiven Beweis gehe, den negativen Beweis führen zu sollen.

Der Beweis nämlich, den ich für das Dasein organischer Wesen antrete, ist ein doppelter: ein negativer, indem ich dartue, dass die meteoritischen Formen nicht dem Mineralreich angehören: ein positiver, indem ich die Übereinstimmung derselben mit den Formen unserer Erde, sei es lebender oder ausgestorbener, begründe: das erste also, was zu beweisen, ist der Satz:

Die Einschlüsse der Meteoriten sind keine Mineralbildungen.

1. Unsere Mineralogen erklären die Einschlüsse der Chondrite für Enstatit, Bronzit, Olivin.

Olivin hat keinen sichtbaren Blätterbruch, Enstatit und Bronzit einen deutlichen. Ich bilde einen Bronzit von Kupferberg, Tafel 1. 1. einen Enstatit von Texas, Tafel 1. 2. (Dünnschliff bei 75 facher Vergrößerung) ab. Figur 2. zeigt einen der besten Blätterbrüche. Man vergleiche nun damit zuerst Tafel 1. Figur 4, einen Teil eines Favositen des Meteorsteins von Knyahinya (etwa 250 mal vergrößert) und man wird wohl nicht mehr davon reden, dass der Blätterbruch die Ursache der Strukturerscheinungen der Chondrite sei. Nun betrachte man aber noch sämtliche Tafeln und es wird diese Erklärung ein für allemal abgetan sein.

2. Wenn die Einschlüsse der Chondrite nach der bisherigen Deutung aus Enstatit oder Olivin bestehen, oder wenn es Gläser wären: wie wäre es, frage ich, möglich, dass dasselbe Mineral oder Glas im Ganzen in so verschiedenen Formen (Umrissen und Strukturen), und verschiedene Minerale in so scharf übereinstimmenden Formen auftreten? Man betrachte einmal einen Hypersthen, eine Hornblende, einen Augit! Abgesehen von einigen sichtbaren, leicht zu erklärenden Einschlüssen — (und um diese handelt es sich ja hier nicht) immer dasselbe Bild! Von höchstens 3 Mineralen hundert verschiedene Bilder!

Das Mineral ist einfach, muss seinem Begriff nach einfach sein und daher stets das Bild einer homogenen Masse (Fläche) geben, höchstens mit einigen Einschlüssen. Wie sollte nun dasselbe Mineral in so verschiedenen Strukturen, dabei in so übereinstimmenden von den Kristallformen abweichenden Umrissen vorkommen?

3. Die Minerale sind entweder kristallisiert oder nicht kristallisiert. — In dem ersten Zustand haben sie bestimmte gesetzmäßige also wiederkehrende Formen: sie rühren von Flächen, welche im Durchschnitt sich als gerade Linien projizieren. Diese Formen (Linien und Winkel) sind wiederkehrend, wechseln bloß der Größe, nicht dem Verhältnis nach. Solche Formen finden sich unter den von mir als organisch angesprochenen Formen nicht. Hier ist keine Form mit einer Fläche oder mit einem Winkel; Alle sind Kugeln, Ellipsen mit Abweichungen von der mathematischen Form, Abweichungen, welche aber doch konstante sind. Also ganz abgesehen von der übereinstimmenden Struktur, zeigt sich eine Konstanz der Umrisse, aber andere Formen als die Kristallformen des Enstatits, des Olivins sie geben müssten.

Allerdings kommen seltene, kleine Stellen mit wirklichen Kristallen vor, aber in einer Weise, welche durchaus auf den Beweiswert dieser Tatsachen nicht einwirkt. Hierüber siehe unten und Tafel 32. Figur 2.

4. Waren die Minerale ursprünglich kristallisiert, haben aber durch mechanische Gewalt ihre Kristall-Form verloren, so ist die einzige Form, welche hier sich wiederholen könnte, die Kugel oder eine dieser sich nähernde Form, etwa die Ellipse. Hier wäre eine Wiederholung möglich, ohne dass aus der Form ein Schluss gezogen werden könnte. In den Rollsteinen schneidet die Oberfläche den Körper in einer Weise, dass sofort die Einwirkung der mechanischen Gewalt hervortritt, — insbesondere werden Einschlüsse ganz willkürlich getroffen.

In den Meteoreinschlüssen aber ist die Struktur im Stein stets, ich möchte sagen: symmetrisch, im Einklang mit den Umrissen.

5. Bei Verwitterung von Kristallen ändern sich die Schichten von außen nach innen — konzentrisch: — von Verwitterung aber ist keine Spur in den Einschlüssen der Chondrite zu sehen und die Strukturen sind stets exzentrisch.

6. Was die Einschlüsse der Mineralien betrifft, so können diese je nach ihrer Beschaffenheit verschiedene Bilder geben. Es kommen ganz willkürliche Formen der Einlagerung vor, wie Glas-Flüssigkeits-Einschlüsse, Kristalliten.

Wo aber ein Formengesetz in der Einlagerung auftritt, richtet sich dieses stets nach der Kristallform. Beides trifft bei den Meteoritformen nicht zu. Keine Spur von Einlagerung nach einer Kristallform!

7. Ein Blätterbruch wird nur sichtbar, wenn durch mechanische Gewalt Spalten und nun Lichtbrechungserscheinungen auf den Spaltungsflächen entstehen. Ohne diese ist er nicht wahrnehmbar. Spaltungsflächen sind nicht da, Lichtbrechungserscheinungen zeigen die Meteorit-Einschlüsse auch nicht, bloß „Einstäubungen“.

Es finden sich in den terrestrischen Mineralien Interpositionen parallel mit dem Blätterbruch eingelagert: diese zeigen die Meteoriten nicht.

Ich glaube, der Anblick meiner Formen wird eine weitere Auseinandersetzung über ihre Verschiedenheit von Mineral- und insbesondere von Kristallbildern nicht notwendig machen.

8. Es ist aber soviel von Kristalliten, von Kristallkonkretionen gesprochen worden.

Für solche wurden die Enstatit-Bronzit-Olivin-Kugeln bisher gehalten. Gümbel wies dementgegen darauf hin, dass es keine Kugel gebe, wo der Mittelpunkt nicht exzentrisch liege!

Hier gerade tritt der wesentliche Unterschied zwischen den Meteorit-Formen und den Kristalliten recht deutlich hervor.

Die Kristalliten legen sich stets um einen Punkt (konzentrisch) an. Die Formen in den Meteoriten sind alle elliptisch und birnenförmig: wenn die äußere Form aber auch kugelig ist, sind die angeblichen Einschlüsse exzentrisch geordnet und zwar liegt der Mittelpunkt an der Peripherie, (sogar jenseits derselben, nämlich dann, wenn er weggeschliffen ist, was Gümbel übersah) — eine Erscheinung, welche nie im Mineralreich vorkommt. Es ist eben die Bedingung der Kristalliten- d. h. Kugelbildung, dass die Kristalle um Einen Kristall gleichmassig sich anlegen, wodurch dann notwendig die konzentrische Form entsteht.

Wären also die Kugeln in den Meteoriten Kristalliten, so müssten sie, wenigstens nach dem Gesetz der Erde, konzentrische Bildungen aufweisen.

9. Schließlich muss ich einen Widerspruch aufzeigen, in welchen die Wissenschaft mit sich geriet, wenn sie die Struktur der Chondriten aus der Mineral-Eigenschaft erklären wollte. Dies ist das optische Verhalten dieser Einschlüsse.

Wären sie Kristalle und wäre der Blätterbruch (freilich Olivin hat keinen, und doch finden sich auch in den angeblichen Olivin-Kugeln Strukturen, also Blätterbruch!) die Ursache der Struktur, so müsste das Mineral notwendig das Licht brechen. Bei den meisten der Einschlüsse zeigt sich aber keine Lichtbrechung, nicht einmal Aggregat-Polarisation! — So können sie also weder einfache Mineralien noch Kristalle sein, am allerwenigsten ließe sich die Struktur aus Blätterbrüchen erklären. Diese Tatsache, das optische Verhalten, sollte allein schon zur richtigen Deutung geführt haben.

All diese Beweise sind freilich dem Botaniker und Zoologen fremd, während sie jeder Mineraloge kennt: daher muss ich diesen bitten dem Kollegen Botaniker und Zoologen das eben Vorgetragene zu bestätigen, zu bestätigen was meine Lichtbilder zeigen: Diese Formen sind keine Mineralformen. Damit hat der Mineraloge seinen Anteil an der Arbeit getan und nunmehr geht sie in die Hand des Paläontologen, oder richtiger des Zoologen über und es beginnt die positive Beweisführung.
\clearpage
\subsection{The Individual Forms: Sponges --- \emph{Urania}\index{Urania}}
\paragraph{}
Rounded, lobed bodies with an obvious place of growth. Table 2 gives a larger image of an \emph{Urania}\index{Urania} (compared with Table 5: Fig. 1, the same picture). One sees here: the acute general form, the outermost lobed edge (white, on the left), the folds, which developed while contracting, the place of growth. Even more clearly is the latter as a chalice, Table 4: Figure 3.

Consolidated spiral-form \emph{Urania}\index{Urania} Table 3: Figure 5 and 6.

In comprehending the threads of Table 4: Figure 1 the structure consists of an outer membrane enclosing lamellar layers, Table 3: Figure 4. Table 4: Figure 6 (the latter can be seen with a magnifying glass). Median diameter of \emph{Urania}\index{Urania} 1 mm, color slate gray.

This structure was maintained to be a breakage of the bronzite\index{bronzite} sheet! Whether Table 4: Figure 4 belongs to the \emph{Urania}\index{Urania} is doubtful. The form and color suggest as much. The trim cuts on both sides show clear structure.

Table 5: Figure 5 shows entirely winding lobes. Either it is a hoisted spiral-form body, or it is several lobes, of which the outer one surrounds the inner.

Table 4: Figure 6 is a cross section, which does not show much. In the object itself you can see an average uncolored outer thin shell.

Table 5: Figure 2 shows such clear stratification, that if the outer form did not exist, one might attempt to place the form as coral\index{coral}.

Table 4: Figure 5 shows cross sections through both vanes of the lobes.

Table 6: Figure 3 lamellar structure. Figure 5 and 6 may also contain the simplest crinoids\index{crinoid}, whose arms have been laid out, on each other. Regarding the transitions of forms, I must refer to the chapter on that question.

The most incredible is Table 6: Figure 1 and 2. In Figure 1, the dull spot in the specimen is yellow, the striped blue. I have situated Figure 2 next, which clearly shows two lobes, connected like two shells in one place and at first sight also makes the impression of a double shell. (It is not a mere cut.) If you think a shell, the dull spot of Figure 1 would be the stone piece. But the structure is \emph{Urania}-like\index{Urania}.

Table 5: Figure 3: Two individuals show the structure most clearly, as well as the growth points. In Figure 4 (which is a bad photo), several individuals lie together in a fan-like manner.

In Table 3: Figure 3 and Table 4: Figures 1 and 2, it is believed to be seen the round mouth opening as implied from above.

After all this, I think \emph{Urania}\index{Urania} is a sessile sponge that contracts in a spiral form, absorbing and expelling water like our living sponges.

\emph{Urania} composes three twentieths of the rock mass.
\clearpage
\subsection{The Individual Forms: Sponges --- Needle Sponges\index{sponge}}
\paragraph{}
In Table 7 the forms of Figures 1, 2, 3, 5, and 6 show a spicule framework. Figure 1 points to \emph{Astrospongia}\index{Astrospongia}. The needles are regularly crossed. Figure 6 is an irregularly massive spicule framework with a cavity, which from the picture suggests is very delicate. These two forms seem unquestionable to me.

Almost certain are Figures 2 and 5 (in Figure 2, the white line is a rock crack).

The shape of Figure 4 I kept in the arrangement of tables as a sponge. After changing the arrangement was no longer possible, I realized this form was the skewed average of a crinoid and what I initially considered to be needles --- are fine crinoid arms. I note that the determination is very difficult because of the exceptionally plain meteoritic crinoid forms, which means a decision must be avoided pending further investigation. The cavity of the needle sponge can be confused for the food channel of the crinoid\index{crinoid} arms, when the latter are stretched straight and the limbs are no longer clearly preserved. This fact of the matter, however unpleasant for the investigator of individual forms, is more rewarding for the one who pursues the development of the forms --- for proving the development of one form to another. It is always enough one to the other. This puts us in a more favorable position.
\clearpage
\subsection{The Individual Forms: Corals\index{coral}}
\paragraph{}
Here we have such well-preserved terrestrial forms that not a doubt is left remaining.

Table 8 shows a sample image, Table 9 its channel structure: obvious bud channels that are tubular connections (for there are such). In addition, there is the curvature of the channels, which absolutely cannot be mistaken for a sheet breakage, plus there is the very clear tube openings and finally an equally clear growth site. (Table 1: Figure 4 shows an even sharper picture of the same object.) Regrettably, staining of the specimen gives the structure pictured in Table 9, such appalling shadows. The bud channels are 0.003 mm apart. Of course, everything you can ask for from a \emph{Favosites}\index{Favosites} structure.

Table 10: Figures 3 and 4 shows the image of \emph{Favosites multiformis}\index{Favosites} from the Silurian\index{Silurian}, in this one cannot even separate the species.

In Table 11: Figures 1, 2, and 3 (where 2 also shows growth points), any researcher will easily recognize the image of living coral\index{coral} forms, the more so as the cup shape (cavity) is indicated in Figure 1 above. The same object also shows the cross partitions of the tubes, which clearly emerge. Unfortunately, part of the picture is obscured by black in the photograph due to the yellow coloring of the specimen.

Table 10: Figures 1 and 2 show less well-preserved cross-wise and longitudinal sections, though the exact same repetition of both in several sections raises doubts that they are organic forms, and if they are such, then they can only be corals\index{coral}. Figure 3 seems to be a cup coral, Figure 4 has grown the same. The fact that Figure 6 has a coral\index{coral} structure does not require proof. This form recurs several times.

Table 11: Figure 4: This form also recurs several times. Peculiar coral\index{coral} forms are shown in Figures 5 and 6. Figure 5 is formed of tubular rings and most likely also Figure 6. I note that this shape appears hundreds of times.

At high magnification, partitions show: Table 11: Figures 1, 2, 3, and 6.

Table 12: Figures 1, 2, and 3 show clear lamellar structure. The transverse groove in Figure 4 is reminiscent of \emph{Fungia}\index{Fungia}. Table 30: Figures 1 and 2 and Table 20 probably also belong here.

The coincidence of the structure in Table 20 with that in Table 30: Figure 1 (in two different cut preparations) would alone suffice to exclude any possible thought of inorganic formation. Moreover, the form occurs about twenty times in 350 cuts.

Table 12: Figure 5 I found only once. In the original there are clear lamellae, which in the picture appear only in the lower part. Figure 6 is a milky white object, hence indistinct. I believe I recognize the star shape and have therefore placed the form here as a star coral\index{coral}.

Table 13: Figures 1, 2, 3, and 4 are corals which undoubtedly belong with the tubular corals. In the original, one can clearly distinguish: glassy like intermediate masses, black tube walls, yellow tubular filling material, occasionally the latter is also black. This form occurs a hundredfold in all the chondrites\index{chondrite}. Figure 5 is composed of lamellas showing clear cavities and Figure 6 has tubes with partitions. These forms belong with the largest of forms: they have diameters of up to 3 mm.

In Table 25: Figures 1 and 2 the form is here so well-preserved that the existence of an organism cannot be doubted, the more so because it occurs in two cuts and otherwise recurs frequently. See Table 2, lower left, Table 5: Figure 6 has the form, Table 1: Figure 6 and Table 25: Figures 1 and 2 are posed in sequence with the crinoids; the channels are unquestionable, the cross lines can also be interpreted as crinoid links. You can see incisions, furthermore the arms are broken, which can only be associated with crinoids\index{crinoid}.

Broken or kinked arms also appear in Table 25: Figure 4, with this form there are multiple examples which give precisely the same image.

All coral\index{coral} forms throughout make up about a twentieth the total volume of the chondrite\index{chondrite} rock, but constituting the remaining sixteen twentieths, that which is by far the greatest part of the whole mass, is the:
\clearpage
\subsection{The Individual Forms: Crinoids\index{crinoid}}
\paragraph{}
They are found in the simplest form, from their articulately divided arms to the developed crinoid\index{crinoid} with stem, crown, main and auxiliary arms. Their preservation is good for the most part. The difficulty lies only in the thousands of possible directions of cutting, which always give different perspectives of the same object. The pear-shaped bodies, which are regarded as glass are crinoids --- their crowns.

I present four crinoids\index{crinoid} in an upright position and in high quality in Tables 16, 17, 18, and 19 and in profile in Table 20.

Table 21: Figures 1, 2, 3, 4, and 5 show average vertical sections of more developed crinoids. These are the main arms with auxiliary arms and distinct joint surfaces.

Table 21: Figure 3 shows stem and crown. (Figures 2 and 4 have double the magnification of 1 and 3.) Figure 5, from another thin section, is shown to display the conformity of the forms. In Figure 6 I believe one can perceive the mouth opening in the cusp between the arms.

Table 22: Figures 1, 3, 4, and 5, and Table 23: Figures 1 and 2, show five as the number of arms, as well as with the auxiliary arms.

In Table 23: Figures 2 and 3 shows the kinking of arms due to pressure from above.

Table 22: Figures 2 and 4 call to mind Comatulida\index{Comatulida}.

There are particular species of crinoids, which consist only of a number of arms. These are seen in Table 23: Figures 4 and 5, Table 24: Figures 4, 5, and 6 and Table 26 (The picture on Table 24: Figure 6 is a smaller scale of the coral\index{coral} from Cabarras\index{meteorite!Cabarras}, Table 13: Figure 6.)

Table 29: Figures 1, 2, 3, 4, 5, and 6 and Table 27: Figure 3 show pictures of crinoids\index{crinoid} as seen from above.

Table 27: Figure 2 and Table 29: Figure 4 show crinoids\index{crinoid} from below: here the base of the stem emerges as a bright spot. The cross-sectional cuts give dozens of cases showing a consistent form. (See also Table 3: Figure 2, top left. Finer results could probably not have been asked for: the muscle layers are clearly visible here.)

Peculiar entanglements are shown in Table 26: Figures 1, 2, 3, and 4.

The clearest profiles are given in Table 25: Figures 5 and 6. Table 27: Figure 3 is a longitudinal profile with broken arms.

Table 24: Figures 1 and 2 are forms which I first viewed as coral\index{coral}.

Table 28: Figure 1 could, nevertheless, be added to the latter. (The structure should be more clearly preserved for a final decision to be made).

A little clearer is Table 27: Figure 1: an apparent outer casing, which is nothing but regular closed main arms.

An exceptionally nice picture is given in Table 30: Figure 3; whether crinoid?\index{crinoid} this is doubtful. I only take notice, the two parts are symmetrical, and the arms are not placed beside each other, rather they cross.

Table 30: Figure 5 with a cut, I had at first placed as \emph{Urania}. It shall be added to the crinoids.

Table 31: Figures 1, 2, and 3 appear to be similar forms. In Figures 1 and 3 one can perceive a distinct furrow, perhaps this is the place where two crinoid arms lie against one another. With the polarization device, the furrow appears even more clearly. In Figure 4 two individuals are merged, leaving it open to interpretation as either sponge or coral. Figure 5 has a structure in the middle part, some structural tissue, showing the upper arms as distinct structures. Do these belong together? Since the form only occurs once, I dare not make a final decision. The resemblance of the central image with the structure of the schreibersite\index{schreibersite} in meteorites\index{meteorite} is striking. Figure 6 is found twice, so that I consider both parts as related.

The same mesh structure is shown in Table 30: Figure 6 at increased magnification. The structure of both agrees, as suggested before, with the structure of the schreibersite\index{schreibersite} in the meteorites\index{meteorite} and makes an appearance several times.

As I already noted at the beginning, I do not consider my task here to enumerate species. My task is only to establish the existence of organisms by proving unified recurring forms with undoubtedly organic structures. I think that I have done this, and I think that no one should have even the slightest doubt (especially after viewing the originals in thin section) that these do not act as minerals. Even if only five organic forms were verified without a doubt, the other less well-preserved forms would also be organic.

The final determination of the genera and even the species requires more material and years of investigation. (I will be grateful for the former.) Above all, I should have more time than the current night hours and more strength than my current strenuous profession leaves me to finish my work. I think I have given the required points asked for, on which one can stand. In conclusion, I refer to the table commentary.

Thus, the forms are presented. I have been pursuing a plan, of making a statistical study on the occurrence of the forms, to count out something such as the occurrence of same forms that one finds in 500 thin sections. I bring this up, because I felt I had to say, that I did not think such would have great value. Each multiplication of my collection by twelve new ones would change the ratio. I therefore preferred to give an approximate numerical ratio for the individual forms.
\clearpage
\subsection{All Life}
\paragraph{}
The individual forms were brought to view in the previous sections. All these forms are not buried upon death, but one grows upon another and, in truth, they are buried alive by life. Here of course only our vision can provide conviction. To this purpose one should look at all the pictures with the individual forms within their surroundings!

What at first glance appears as a bright spot, upon closer examination shows on the average a sponge, a coral, or a crinoid part. Nowhere are there, as Gümbel\index{Gümbel} has quite rightly observed, disassembled tumbled forms and fragments --- also there is not a binder between them. Only soft tissues are missing, everything else is preserved, just as it was when the life was in water. The crinoid forms show this clearly. For these are, at most, curved on a side, winding, and seldom broken; one sees also that there was only a weak mechanical resistance against neighboring heads. But everything together, grown apart --- nothing laid down, nothing buried. There is also no mass available that could have constituted a grave.

The fact, that there is nothing inorganic in the chondrite\index{chondrite} rocks and not a single place without life in them, I consider to be as important as the existence of the organisms themselves. First, this fact casts full light on the emergence of planets. If one adds to this, that the rock that includes these formations consists of minerals belonging to the purported primary mountains [Urgestein]\index{Urgestein}, yes ``volcanism'' associated with the mountains: then our geology must take a different path in the explanation of the facts. My belief is by no means that the sponges, corals, and crinoids are from minerals we have here, that constitute forms today. The original organisms must have been composed differently; they must have endured a transformation.

It is so much, I think, beyond all doubt that what is nowadays hornblende, augite, and olivine\index{olivine} are what filled the referred-to forms, formerly these minerals must have been in a different condition, namely a liquid water one, a water solution.

Now we find these minerals in our primary mountains as forms, which are not crystals, but are like the meteoritic ones. We find mountain masses composed of such forms. So here too it is highly probable that organic forms, subsequently transformed, are what we now call rocks. These rocks, however, point to a layer that is undoubtedly close to the meteoritic (chondritic)\index{chondrite}, indeed they are closely related. Under this must lie the iron. This testifies to the specific weight of the Earth\index{Earth}. Again, the identical situation appears in the fallen iron meteorites\index{meteorite!iron}: here, as in the Ovifak\index{Ovifak} rock, we find transitions, compositions of iron and olivine\index{olivine}.

This gives us the greatest baseline for geology --- we have the chronological development of the body of the Earth\index{Earth}. The development of form --- the reason for the growth of the forms themselves is at the same time open. If the organism in the lowest layer, that we know of, was the source of mass creation then it could also have been the initial cause for the beginning of the planet itself. The assumption of mere mass-attraction, the mechanical formation of the Earth\index{Earth} and the heavenly bodies would in general be thereby refuted.

Admittedly organisms in iron, in the Earth's core\index{Earth!core}, and in the meteoritic iron must also be detected. It is this task which I set for myself in what follows next. The previous results allow for a hopeful solution.
\clearpage
\subsection{Stone in the Stone}
\paragraph{}
When I said that the chondrite\index{chondrite} is nothing but an animal-fabric, an animal-felt, a qualification must be sustained.

There are, however, very small, sharply outlined places in this animal-bone stone which could probably (but not necessarily) be from the first rocks. These are slate-blue, uncommon inclusions with 3-5 mm. diameters lacking definite recurring forms which include distinct crystals in their grayish mass, these are on average either squares or rhombuses while in other places it includes hexagons. This mineral can be either augite or olivine\index{olivine}. Here the crystalline form is pronounced in favor of a mineral. The sole existence of this speaks for my views. Why have the crystals not grown themselves identically everywhere? And why should there not be hollow cavities remaining in the organisms? It is known that fillers in organic forms later crystallize. And in the final-filled organic forms, cavities are found in which their outlines look like surfaces recessing at an angle.

The reason why I acknowledge that these inclusions are inorganic parts of the chondrites\index{chondrite}, as distinct from actual meteoritic stone (stone in the stone), is because the outlines do not give the indication, that is, their form does not address itself as being organic. These inclusions may be deposits of an already developed rocky mass or they may have only developed in the cavities.

This situation is possible, even probable, that it was a falling-in of pieces of already deposited rock that were fully developed and does not need to be denied: it does not knock on the fact that in the olivine\index{olivine} strata formations exist and that these are the cause of the construction of the planet bodies, their self-constructed development and complex composition.

In all cases, however, the ratio in the chondritic rock is the opposite as that in the sedimentary layers of Earth\index{Earth}. In the latter the organisms are interred and the rock strata enclose them; in the first there are only organisms and the rock strata are masses of such. I put an image of an actual rock-piece from Borkut [Ukraine], Table 32: Figure 2, next to that (Figure 1) I have depicted a form, slate-blue like \emph{Urania}, however, without a set structure its outlines are inconsistent which could be from the lack of filler. If it were an organic form, it would be of the lowest nature. For comparison I show in Table 32: Figure 4 a thin section of Lias $\gamma\delta$\index{Lias} [Early Jurassic] (Zwischenkalk), here shells are located in limestone but most parts are merely pieces of shells; the parts are crushed into all sizes and, regarding their origin, they are tumbled beyond any recognition. In the chondrite\index{chondrite} there is no place remaining that can leave a doubt as to their composition.
\clearpage
\subsection{Reproduction}
\paragraph{}
In the stone there are found a multitude of round and pear-shaped forms with 0.10-0.50 mm. diameters, which barely indicate structure. I hold these forms to be the first developmental forms. Among the many forms, the most outstanding are the transparent spherical forms of rock in the center of which are channel openings. Here one finds these channels within spheres, with two further below and a larger above, and so forth on up to the forms of Table 13: Figures 1, 2, 3, and 4. The case is here, I believe, secure. Not only is this form evident in all the chondrites\index{chondrite}, but in each of them one also finds full developmental stages with up to twenty or more channels: they are common and at the same time certain because of their self-evident channel structure. They have been preserved in those chondrites\index{chondrite} which hardly show the forms on the left. The development suggested here is that the channels reproduce.

Of course, there are many faint spherical and pear-shapes which indicate structure. They appear to have been made of sarcode when they were suddenly interred. I would not dare to bring these forms up if they did not indicate a definite structure. They consist of two, three, four, and five lobed-form branches and are probably the beginnings of crinoids. That the observation of developmental forms is difficult is well-known. Hence, I do not allow myself to act prematurely here. What I say here should only be considered as a pointer towards future research.

Good preservation is an impossibility. This is because meteoritic forms face the same destiny as living animals: it is always the ultimate labor to find that first beginning of development, the embryo.

I will refer to a single fact here, which is a considerable point of proof for the organic nature of the forms: the ever occurring association of the individual forms. Many forms that one finds collectively resemble each other: a few stand individually and at the same time as a unit. I hold this as highly significant. If several individuals of the same species come together, it goes to follow from this that there exists mother or sibling relationships. The same phenomenon is known to occur in the terrestrial types. This would seem to signify, as minerals often do, to which form it belongs, as undoubtedly the same applies to other species' mineral fillings, so that a mineralogical ground from which the different derivatives of structure could be inferred.
\clearpage
\subsection{Development}
\paragraph{}
After having depicted the individual forms, I must now discuss their relations to each other, the development of the unfolding of forms.

That \emph{Urania} is the simplest form, this is certain. However, it establishes the inception of what follows.

These layers in the hemispherical lobes, these tubular layers, they part themselves crosswise --- that which today would constitute an arm connects a channel. It develops a crown between the arms and the growth point and the simplest crinoid is there. If this seems like a twisted chain of events, the forms involuntarily demand it. But just as we always find somewhere in living forms a line of development so should we also not find that the same changes have taken place here? Certainly. Only, I believe, they are found with more quantity and with much greater visibility of transitions in the meteoritic forms. One can find the ancestor of the \emph{Pentacrinus briareus} nowhere else on Earth\index{Earth} except with the corals, and one can see the origin of the coral in the sponge form: it is decidedly a lower form than that of the coral.

What this meteorite-creation\index{meteorite} gives of such great importance to the evolutionary theory is not only the occurrence of animal forms in the deepest strata, but also consistent types for all meteoritic organisms. This becomes clear after viewing hundreds of thin sections one after the other.

The scale of the organisms is uniform, at least one thousand times smaller than the ones of Earth\index{Earth}: the development of the individual forms attains an approximately equal high level. The construction of the forms corresponds perfectly with the circumstances under which they grew, namely an extremely shortened lifespan, which was an experience it had: it is a hasty, relatively incomplete creation. The crinoid is the highest representative of this animal world. I hold that the most advanced is the form in Table 22: Figures 1, 3, 5, and 6, because it really embodies the number five.

One will not want to go so far, however, as to derive the crinoids through the corals, thus the form of \emph{Urania} must offer some clue. I show some forms which have the loose branches. They are indicated in their descriptions. I find at high magnification overlying arms.

Even here an adequate observation of a single is not enough for a complete conclusion.
\clearpage
\section{The Iron Meteorites\index{meteorite!iron}}
\paragraph{}
As I have already indicated in \emph{Primordial Cell}\index{Primordial Cell}, the structure of the iron meteorites\index{meteorite} is nothing other than a single mat of unicellular plants. The so-called Widmanstätten figures are, for the most part nothing other than these unicellular plants.

A piece of the Toluca iron meteorite\index{meteorite!Toluca}\index{meteorite!iron} lies in front of me in which the cylindrical cells alternately emerge from each other, the two are often copulated. The individual cells show a double cell wall (iron band), show cross partitioning, show clear round root points; in some there is a marrow substance (which it is really called), indeed, in the inside of the cell there is yet more structure. All of the cells lie in a mat of filler (iron-filler).

Compare these figures with the forms of the Lias slate, especially \emph{Algacites} [\emph{Fucoides}] \emph{granulatus} and ask yourself, of the two, which one shows a plant structure clearest, the Toluca iron or the \emph{Algacites} from Lias-Epsilon?

These forms are cylindrical, from time to time one sees (on average) approximately polyhedral surfaces: they have walls. What especially distinguishes them from crystals (which can be foreseen from the round forms) are the growth sites.

Crystals, which grow together, set themselves against one crystal surface as well along surfaces, (dendrites of silver, copper); they place themselves along the surfaces of another, without entering them, but in the meteoritic iron one finds penetration instead. The cross section is not a straight line (crystal surface), but a curve.

Here end all similarities with crystals, unless one assumes that there could be cylindrical crystals, which grow out of each other. The claim, that these figures have fixed mathematical positions, may be correct here and there by chance; all researchers accept this fact, that nowhere are the angles constant, which with dendrites is always the case. If one finds a place, out of which an octahedron, a cube, or a different regular crystal form derive their location, even a rhombohedron: immediately the order compared with another is quite different. And how can one speak of crystal laws, when from identical minerals not once has this fixed crystal system been repeated? Because one finds, as I have said, rhombohedral slices next to regular ones.

I find just two objections that seem to be justified:
\begin{enumerate}
    \item The objection, that the figures are occasional sheets:
    
    Against this I want to object that, once a cylindrical form is verified, the forms are just not crystalline and now the conclusion is not that they are cylindrical crystals, but on the contrary, that the plates, which bear the same structure, are not crystals.
    \item The second objection is this: How is it supposed to be that plants transform themselves into iron? This objection is not difficult to refute. One has only to think of our many petrifacts, especially the fossilized stems in the Lias; one recalls the so-called Mansfeld [buds] ears in the Zechstein (Cupressites ulmanni), where cypresses are transformed thru silver-bearing copper. One should think that such an objection could be made.
\end{enumerate}
\paragraph{}
But now I am well by uniting with a revered friend, Professor Dr. H. [Gustav Karl Wilhelm Hermann] Karsten in Schaffhausen, who presently is in a position to furnish evidence for the transformation of plants into iron. Karsten has already proven in the year 1869 that our lowest plants absorb iron through entirely outstanding means; I owe the iron plants of today to his kindness. With his permission I include an excerpt from his excellent work, \emph{The Chemistry of Plant Cells}, Vienna 1869, p. 53 which here follows:

``Bring \emph{Oidium lactis} or yeast in heavy moist air (not under liquid) that has for some time been in contact with lactose together with metallic iron by scattering iron filings on the vegetated milk yeast via a glass objective, at first some of the iron touches the cells, later many are vaguely situated then more or less a rapid intense red color soon comes to a surprising size.''

``One would be constrained to suppose that the cause of this strange and exceptional, often very accelerated enlargement, which alone should cause one to search for a mechanical swell up of the cell membranes if one did not also witness simultaneously, within the layered part of the thickened mother cell under the above indicated cultivation ratios, that the available daughter cells multiply at a modest rate and fill up the mother cell completely.''

``The membranes of the daughter cells also produce an acid, as seen in the iron reaction; their shape is according to the connection of their skin with that of the iron, which is very similar to the previously described protein-crystalloid; such as those located on the surface, 3-4-5-sided, though with fewer sharp edges and angular plates; irregularly juxtaposed, they completely fill the size of the cell cavity, but decrease when the skin of the mother cell breaks, as they fall out more or less together.''

``Similar metamorphoses are experienced by the \emph{Oidium} mycelia, especially the dissecting branches rising in the air, which will, when they are brought under similar conditions and indeed this type often expand unequally from the dissimilar member cells, for the most part primarily the upper more than the lower, and usually a round stem remains, with some stretched, whereby these branches with their head-shaped swollen end-cells Mucor- then fruit- or flower-like will, when the top ones enlarge at the well-defined parting top, or from above to below starts to tear open. The membrane of the primary and secondary cells tears apart, each in its own peculiar manner.''

``Even in regard to the organization of plant cells in general, these vegetations of are of great interest.''

``Those namely, which the above described crystalloid cells contain, are also on the inner surface of each of both the nested cell membranes, which the wall forms, with one minor layer occupied that is either laid and flattened closely together or vaguely with some of each other, and gives to the entire cell system the view and small reticular structure, of a tubercular or porous thickened parenchym cell. \emph{De Cella Vitali} 1843, supplement page 37 and 437. These cells, equivalent morphologically to the secretion cells of the composite plant, grow simultaneously with their mother cell close by, they lie between the primary and secondary and form an epidermis. The whole cell system is highly similar to the envelope of many Pollen- and Diatomaceae- (\emph{Gallionella}, \emph{Biddulphia}, \emph{Coscinodiscus}, \emph{Triceratium}, \emph{Amphitetras} etc.) cells.''

``If one records such a cell system colored red by iron and places it into a new mixture made from the above-mentioned nutrient solution without iron, it will quickly decompose into its elements. The cells, which are similarly assembled, with both the crystalloid cell content and also with the epidermis start to round themselves and enlarge; new generations are originated in them and, finally becoming free as their special mother cell liquefies, one sees through months of continued observation the way that the bottom yeast microsporum, through the development of suitor daughter cells, multiplies.''

``The warty thickened \emph{Oidium} cells permeated with lactic acid iron were the ones which grew forth highly long-shaped contents, from or next to the cells which display a reticular warty epidermis, which one would notice, is in the manner of \emph{Micrococcus}, the \emph{Vibrio} spores.''

``Hyphomycetes, particularly \emph{Penicillium} and \emph{Botrytis}, as well as \emph{Rhizopus}, also give, once they have been vegetated and nourished with lactose for some time and brought into contact with metallic iron, a very interesting preparation, partly like those of \emph{Oidium} with swollen gonidium chains or hyphaloid cells. The gonidia chains of \emph{Penicillium} have a rule in which the gonidium original ancestors at first swell up followed in succession by others down to the youngest. The \emph{Penicillium} gonidia, saturated with nutrient salts in a lactose solution after contact with iron soon slowly swell and develop numerous cells on the inner surface of their progressively enlarged and thickened outer skin, giving it a reticulated or porous appearance, so that forms are similar to those described above with \emph{Oidium}, porous and thick-walled. In other cases, the daughter cells fill the cavity more and become like a mucor-head filled with gonidia.''

``Very often are found, as in the case of \emph{Oidium} when it is poorly cultivated, empty cells with very smooth walls. Quite often the inner cell, impregnated with lactic acid iron, breaks through the outer cellular-warty-etc. thickened membrane, which peels or splits as it grows out.''

``The culture used for this purpose should not be kept moist, because undertaken in humid air these vegetations, which are permeated with acidic iron salts, are very susceptible to decay. Even without such a precaution for the culture, I have seen the member cells and gonidia of mold, as well as \emph{Micrococcus} cells and vibrion germs contained in dust, swell as described when brought into contact with polished metallic iron, no doubt because these cells contain acids or acidic salts.''

``It becomes apparent from the phenomena of the growth of these fungal cells that the cause of their abnormal enlargement is to be found in the subsequent association of this acid with the neutral lactic acid iron to an acidic salt, so that the whole phenomenon of peculiar malformation is based on a purely chemical process that changes those cells vegetating under normal conditions in such a way that normal development becomes pathological and causes the ultimate destruction of the organism.''

``Against the idea that the acid here in the fungi as well as the resin, wax, etc. is produced by the assimilation activity of the cell membrane, could be raised the concern that it may be the secretion cells (microgonidia, vibrion germs) alone that are between these membranes of the cellular system (the cells nested in each other in the 1$^{st}$, 2$^{nd}$, 3$^{rd}$, etc. degrees), as noted above these organic acids produce by their vegetative activity, especially since, without doubt, the vibrions that develop from them, even in the total absence of more developed cell forms, are very energetic producers of acids, e.g. milk, butter, and acetic acid. However, those cells enlarged by the absorption of iron in the same way, whose walls are quite structure-less, i.e. without recognizable cellular organizations between the two composing membranes of the cells nested in each other and without enclosed free cells in their cavity; furthermore, the fact that \emph{Oidium} mycelium and its yeast cells, if they are submerged, first have their membranes blackened followed by the liquid contents of the nucleus and are blackened by iron and sulfur ammonium. Against other metals, like aluminum, magnesium, zinc, cobalt, nickel, even against copper, these lactic acid cells behave similarly as with the iron, but with the same colorless or only slightly colored, partly (especially with copper) fragile organizations. Therefore, these metals are less favorable to experiments with this acid yeast.''

I think that if iron plants can be produced before our eyes, then we should not raise concerns against the assumption of the same process at work in an earlier time, at a time when all the materials of organic formation were available. We have mass formations before us here in the atolls of the calm seas, we have in the chondrites\index{chondrite} a composition of similar animals: what stands in the way of assuming such previous plant-mass formations?

At last, through yeast production, we have a process that is completely analogous, once the fiery heat idea goes away.

Here I come back to the Kant-Laplace hypothesis about mass formation. I have already proven their logical error. How do you seek to bring forth a glowing ball from a vapor mass that also surely included water? Or shall the Earth\index{Earth} only come to embers after it has been formed? By what? Experience speaks only for mass formation through organic means. Apparently, only the sight of the volcanoes has led to the assumption of a liquid fire interior of the Earth\index{Earth}, and this notion led to the assumption that the whole Earth\index{Earth} had once been in this state and that the plutonic rocks were the products of this period. Also, it is by no means certain that the thermal radiation of the Sun comes from a liquid fire body. However, the fact of free water on our Earth\index{Earth}, and also the fact of the Moon (without atmosphere!), indicates that from the beginning mass could not have been in a liquid fire state.

In any case, it is certain that meteoritic iron is not a smelting product, for what should have put the meteorite\index{meteorite} into blaze? I also found crinoid and sponge forms in the meteoritic iron. There is no doubt that Hainholz\index{meteorite!Hainholz} shows such.

As already the Pallasites\index{meteorite!Pallasite} show organic and even animal forms, rocks that form the transition from the pure iron to the chondrite\index{chondrite}, there is thus no reason to assume the pure iron is an inorganic formation and much less as being formerly liquid.

Once the iron is assumed to be the nucleus of planets, I believe it then becomes most probable that the first beginnings of our planet, and therefore of all planets, was an organic formation.
\clearpage
\section{The Iron of Ovifak\index{Ovifak}}
\paragraph{}
Through the kindness of Professor Dr. von Nordenskjöld, I was given six pieces of the iron of Ovifak\index{Ovifak} and a basalt, in which the same was found, for examination.

[Friedrich] Wöhler (New Yearbook for Mineralogy 1869, p. 32) does not consider it to be meteoritic because of its chemical composition. The occurrence of an item in a cleft in one of my pieces does not speak for a meteoritic origin either. Iron parts with Widmannstätten's figures are also found in the basalt and olivine\index{olivine}, and yet both are not addressed as meteoritic. Finally, there are transitions from stone to iron, indicating that the iron did not fall into the basalt by chance. It would be a great miracle if this iron had fallen into it just at the time when the basalt was liquid, quite apart from the fact that this iron would hardly be preserved for more than a few years. And yet this iron is said to be meteoritic because of its structure.

We know, however, that Earth's core\index{Earth!core} is at least the density of this metal, and it probably consists of iron of the same nature, thus the likelihood of us seeing the iron core of the Earth\index{Earth} in Ovifak's\index{Ovifak} iron would be obvious.

That would have won us infinitely more than a new meteorite\index{meteorite}.

On the surface of this iron, which, of course, I do not yet have the permission to assail, I find structures very similar to those of the crinoids\index{crinoid} in the chondrites\index{chondrite}.

However, I must save a thin section investigation until the time when the material is made available to me.
\clearpage
\section{Conclusions}
\subsection{The Origin of Meteorites\index{meteorite!origin}}
\paragraph{}
It is quite certain that small planets, weighing half of the Earth's\index{Earth} kilograms, fall and therefore revolve. One can now think of the following options:
\begin{itemize}
    \item The meteorites\index{meteorite} revolve outside the solar system (one such might have been observed by [Frédéric] Petit in Toulouse)
    \item The meteorites\index{meteorite} revolve within the solar system: by themselves around the sun --- around the Sun with the planets (perhaps even individuals with the Earth\index{Earth}) --- around the sun, the planets, and their satellites.
    \item The meteorites\index{meteorite} revolve in all these paths.
\end{itemize}
\paragraph{}
It is known, from many years of observation, that at certain periods (August 10th, November 13th) swarms of meteorites\index{meteorite} approach our planet and intersect with its orbit; it is known that these swarms are more numerous in certain years than others and that also single meteorites\index{meteorite} fall upon the Earth\index{Earth}, both facts have their cause in the attraction of the Earth\index{Earth}. The orbits of the meteorites\index{meteorite}, however, are not known, neither those of the swarms nor of the individuals; neither those which have fallen nor of those which have merely passed the Earth\index{Earth}. Thus, nothing for the formation of the meteorites can be derived from their orbits.

We now come to wonder what follows from the composition of the meteorites\index{meteorite}. Their chemical elements are the same as those of our Earth\index{Earth}. This fact points to a common origin, that is, the mass of the Earth\index{Earth} formed together with the meteorites\index{meteorite} and the formation and development of all planets was the same. The mere fact of chemical equality leads to various conclusions. I have demonstrated, however, Earthly organisms in the meteorites\index{meteorite} and it cannot be assumed as certain that the dissimilar ones do not occur on Earth\index{Earth}. To my regret, I must admit that the number of doubts has been increased by my discovery.

These questions now arise: did the meteorites\index{meteorite} arise with the Earth\index{Earth}? Are they from the Earth\index{Earth}? Thus, from the beginning, were they a mass along with the Earth\index{Earth} and then separated from it, so that they might be or still are a kind of invisible satellite of the Earth\index{Earth}?

First, I only raise these questions because they are the most important for geology. The specific gravity of the Earth\index{Earth} and the rock of Ovifak\index{Ovifak} make it likely that the Earth\index{Earth} is entirely composed of the same rocks as the meteorites\index{meteorite}, provided that the iron and the stone meteorites\index{meteorite} belong together. It could be concluded that the meteorites\index{meteorite} had originally been part of the Earth\index{Earth} at the time that its formation had progressed to the olivine\index{olivine} layers, and that they had then become detached from it. The latter would have happened as a result of an impact between a world body with the Earth\index{Earth}, for without such, a separation could not be explained unless the gravity of the Earth\index{Earth} suddenly stopped or diminished to such a degree that part of its mass could have been thrown out from its circle of attraction. It is difficult to believe in a shattering from the inside, from gas power or the like, although this too cannot be completely ruled out.

So, for chemical and morphological reasons, it is not possible to draw conclusions from the rock as to whether the meteorites\index{meteorite} are children or brothers of the Earth\index{Earth}, and one must rely on the pronouncement of the astronomer.

But if the latter confirms, by virtue of their orbits, that the meteorites\index{meteorite} could not have been part of the Earth's\index{Earth} mass, then a second question arises: how do the individual cases relate to one another? Are the stones and the irons originally related, or do the stones and the irons have different origins? And a third question: do the chemically and morphologically identical stones belong to a planet which was destroyed by some cause?

The latter, at first sight, could be deduced from the chemical and morphological similarities, and in fact, the matter seems quite simple and clear. But there is another possibility, the possibility that under the same conditions a myriad of small planets could form and perhaps still form today. The pieces would then not be rubble but their own world bodies.

The irons and the stones would now be their own world bodies --- size alone would not stand in the way of the hypothesis. But if the small masses consist of water creatures and they being a mere microscopic creation, then it is natural to wonder: did they live in water or water vapor? Provided they had a continuous source of water, which we can easily imagine since today we have areas on Earth\index{Earth} where rain is always falling and others where there is none. The question must be countered by the fact that the necessary building materials for the microscopic creation must be sought not under but above the creatures, because only aqueous solutions could have built up this microscopic animal world.

This animal world is already at least partially organized. A unicellular plant, a yeast fungus, may have been the beginning of a planet: it could not have been crinoids that organized it because we have to think of the long periods of time, and therefore the much greater mass that this stage of development must have required.

These facts, in connection with the likelihood that the irons were the core of the chondrite\index{chondrite!origin} planet, lead us to regard the chondrites\index{chondrite} as the debris of one and the same world body, debris that has been orbiting, following the destruction of this planet, until it fortunately falls into the path of our Earth\index{Earth}. The forms of the meteorites\index{meteorite} suggest themselves as being rubble.

So, we have only one hypothetical certainty: the likelihood of the original unity of the debris that reaches us.

But if they came from Earth\index{Earth}, then they have been parts of it: the composition of organisms is still a fact that is important for our geological history. However, if they do not come from Earth\index{Earth} they illustrate two facts: the origin of a planet and the probability of the way in which our Earth\index{Earth} was born. But if they were each a planet they testify to a creative power that leaves our concepts about the origin of organic forms and their development far behind.
\clearpage
\subsection{The Formation of the Earth\index{Earth}}
\paragraph{}
Going off the results so far, some conclusions could also be drawn regarding the formation of the Earth\index{Earth}. It is most likely, on average, that the Earth\index{Earth} shows the same sequence of rocks as the meteorites\index{meteorite}, which pass from the iron to the pallasite (olivine\index{olivine} with iron) and from there to the olivine\index{olivine}, enstatite\index{enstatite}, and (feldspar) rocks.

On the Earth\index{Earth}, olivine\index{olivine} is followed by granite, a feldspar rock: this order also corresponds with the specific gravity of the mineral.

The specific gravity of hornblende is 3-3.40, olivine\index{olivine} 3.35, enstatite\index{enstatite} 3.10-3.29, orthoclase 2.53-3.10, and quartz 2-2.80. The high specific gravity of hornblende seems to stem from its iron content. This sequence of specific gravity, just as in their stratification, strongly suggests mineral formation in water, i.e. in an aqueous solution. Here I must repeat what I have already said in \emph{Primordial Cell}\index{Primordial Cell}: that creation, i.e. organic formation, could not have started with crabs (Trilobites)\index{Trilobite}. We find a constant series of forms everywhere in the later strata, so why should this law not continue all the way down to the very beginning?

This alone should lead one to the assumption that the immediate precursors to the Silurian, gneiss, and granite have an organic origin.

With the evidence for the organic composition of the chondrites\index{chondrite} no argument stands in the way for considering the granite as a water structure: both rocks contain mainly feldspar. As concerns the granite, I have found forms in it which are like those of the chondrites\index{chondrite}.

I would like to add some points here to prove that the origin of the granite was not only from water, but from organisms. Feldspar and quartz crystallize, I would say, fervently. In the granite, however, both minerals are regularly not crystallized; feldspar merely shows sheet fractures. This is also seen in lime petrification\index{petrifact}, e.g. a crinoid stalk. Why does feldspar in granite not appear crystallized? Because it is bound by a stronger formative force. The feldspar in granite (where the latter is truly preserved) always shows definite recurring forms, not conglomerated or tumbled, nor, as I have noticed, crystal forms. Here also one form always grows out of an another. These forms are sponge shapes. The quartz fills the cavities.

I would also like to point out the formation of the mountains. Dr. [Friedrich Moritz] Stapff\index{Stapff}, who has sufficiently observed mountain structure from the Gotthard Tunnel\index{Gotthard Tunnel}, explains (New Yearbook of Mineralogy 1869, p. 792) that there is no sign of mass uplift or fragmentation in the Gotthard Tunnel\index{Gotthard Tunnel}, the greatest insight into the Earth's interior\index{Earth!interior} that is known. This ``primordial mountain'' is, according to the findings, a sedimentary mountain. Yes! It is even conceivable that it was formed when our atmosphere still held most of the water, an atmosphere that was not heated by fire in the Earth's interior\index{Earth!interior}, but rather by chemical heat, as it is today. But if this is the case then there remains no reason against explaining the origin of the primitive rocks, and the primordial mountains, by organic life.

Even today lower animals and plants can endure a degree of heat which is fatal for other beings, so there is nothing standing in the way of accepting organic life with an increased degree of heat. Apatite and graphite can also be considered a witness of organic activity. With the precipitation of silica the Earth's\index{Earth} body was finished: it consisted of the bones of dead animals; clay, lime, and salt together with gases and water formed the building materials for further activity on the Earth's\index{Earth} surface. Because this (not solidification, but precipitation) process was mostly completed, the organism obtained space and time for higher development, which was until then impossible, for every new formation buried the barely formed one. Only after a sparingly soluble compound was laid as a coat around the Earth\index{Earth} could the development of forms enter their own right. The Earth's\index{Earth} periods grew longer; with the supply of finer building materials the law of symmetry came into effect. But another cause helped: the lowest organisms are children of the night; a fungus dies in the light of the Sun. The whole of the previous creation, up until the precipitation of the denser building materials, was a nighttime creation: the continuous chemical coupling had to have produced a heat that prevented water from becoming the ocean that it is today. Finally, the chemical coupling was essentially completed, creating a surface, a kind of shell. But now, the light and heat rays of the Sun came into effect, which, until then, had been hitherto blocked by the tall and dense atmosphere. The light creation begins; the kingdom of the Sun overcame the kingdom of the night on our planet, capturing the night into the depths of the Earth\index{Earth}.

Thus, through light, the higher life that suddenly and powerfully emerges with the Silurian is explained: it was the first resting place of creation. Under the influence of light, we now see a development begin, which is so far removed from the earlier forms as life today at the pole differs from that at the equator. This explains the sudden change. If it had merely been a matter of cooling, creation would show a much slower transition. What remained dissolved in the water after the precipitation of magnesium, silicon, potassium, and sodium was relatively little; light could now begin to work. This assumption explains how life arose on the whole Earth\index{Earth}, that there was water on its entire surface, and that aquatic animals could build mountains that would extend far above the current level of the sea. These mountains have not been lifted, nor driven upwards through mechanical force (by momentum), nor squeezed out by the cooling of the surface; because as the latter cooled (more correctly ``dried up''), at most only cracks and clefts could have arisen, for under the surface there was no slurry, but solid mass. According to my current findings, what is the surface, now that the boundary of the ``primordial mountains'' and the succeeding strata has been abolished?\footnote{It has been forgotten in the theory of uplifting that a force which would be necessary to lift mountains would at the same time have crushed them: in the theory of pressing one is unable to say where the mountain has actually remained, through which the semi-solid would have been pressed! The whole surface could not have been squeezed out.} What separates this layer from the ``primordial mountains'' is only the effect of light, which became stronger as the water vapors condensed and filled the fissures of the globe.

But the days of the Earth\index{Earth} would have been numbered if the light had not ended the process of precipitation quickly enough, because the dwindling chemical coupling would have not have taken place quickly enough and life on Earth\index{Earth} and the Earth\index{Earth} itself would have been brought to a standstill forever. These creations of light were new, higher organisms. These organisms were built from the waste materials of the previous creation, which had not yet ceased their organic coupling, and thus halting death. This would have occurred and the Earth\index{Earth} become a desert had it not been for the very reason that the organisms created by the light, with their nourishment and through their respiration, entered into a coupling and once again dissolved the waste, thus creating a cycle called life. So it is light that protects our Earth\index{Earth} from a death that had already occurred on its satellite. But the light works through the water. The water connects the stone and the air and this opens for us a glimpse into the future of our planet.
\clearpage
\subsection{The Future of Our Planet}
\paragraph{}
The fall of planetary fragments upon our Earth\index{Earth} (for this is what the existence of meteorites\index{meteorite} suggests) could cause a physical destruction, a violent death for Earth\index{Earth!destruction} to fear. If it happened to this planet or that planet from which the meteorites\index{meteorite!origin} originate, that it was pulverized, and probably not due to a force from the inside but by an impetus from the outside: so we should be prepared for this fate on Earth\index{Earth}, at least it does threaten us. I will leave it to the astronomers to comfort themselves and their contemporaries.

But we should also be prepared for the previously mentioned cessation of life on the surface, a less bloody but no less comforting end, namely the fate of gradual death, the termination of the coupling of insoluble compounds with the life force and the building materials: we have to worry that our atmosphere will continue to form insoluble compounds from the remaining building materials and thus the cycle will become weaker and slower, and finally --- stopping.

The only thing saving us from this almost certain fate is water; the water that our Earth\index{Earth} was able to acquire and retain in its formation.

The fact that these created beings release the compounds that formed their bodies and that the plant in particular decomposes what it absorbs, while the animal absorbs these excretions within itself and then excretes them immediately again and again, then returning them to the plant (not the soil): through all this, a cycle is created whose end cannot be foreseen.

This process, not the cooling of the Earth's crust\index{Earth!crust}, of which so much has been spoken, constitutes the true story of the Earth's\index{Earth} surface. However, we seem to have a frightening example in the Moon: there, I think, life is extinct. There are neither seas, as it was believed, nor volcanoes; the lack or loss of water was what caused this planet's premature death, which made life extinct soon after its birth.

The heat on our surface seems to depend mostly on the preservation of the atmosphere, which defends against the cold of space. The greater height of the Earth's atmosphere\index{Earth!atmosphere} at the equator, due to the rotation of the Earth\index{Earth} and not just the angle of the Sun's rays, causes a higher and more constant heat: or else, 500 meters above sea level at the equator would experience a cooling of several degrees from the average heat; and otherwise the glacial mass of Chimborazo would melt immediately.

Although heat, as a result of the chemical processes mediated by water, may decrease with time, it is certain that without the protective coat of the atmosphere the Earth's\index{Earth} surface, although it absorbs new solar heat each day, will succumb to such low temperatures at night that it could not sustain life, as has recently been claimed as the cause for the extinction of all life on the Moon.

Heat flows to us from the Sun and is trapped by the atmosphere so that it cannot immediately emanate back into space. Thus, we are surrounded by a double protective mantle: the crust which absorbs heat and the air that holds it back (it is the jacket of the Earth\index{Earth}), and between the two we live, the whole of creation lives in a constant exchange of substances. Here man lives, here the same beings arose which once laid the first foundation stone for the great construction of the Earth\index{Earth!formation}. These lower beings even today testify, by their enormous multiplication and preservation in a temperature in which higher beings would immediately die, to their being the first sculptors of the Earth\index{Earth} itself.

Thus, only if the source of light and heat itself were destroyed would life on Earth\index{Earth} freeze; we have nothing to fear from the extinction of the fiery core of the Earth\index{Earth!core}. For the preservation and metabolism of life, the Sun provides radiating light and heat. Light and heat are therefore mother and father to all living things; from before time they have prevented the organic from becoming inorganic, constantly forming new compounds. But even if so much light and heat should flow to the Earth\index{Earth}, without the continuous activity and transformation of the organic cell life on our planet would be numbered in years.\footnote{The loss of geothermal heat or heat radiated by the Sun would not be the next threatening nightmare, but the disappearance of our atmosphere.}

The origin of the planets is the cell, it is maintained so long as light rays hit the Earth\index{Earth}.

It is possible that over time changes in the chemical composition of the Earth's\index{Earth} surface and atmosphere will occur due to the precipitation of solid compounds, whereby building materials are removed from the cycle. Certainly, under such modified living conditions, other similar, and (according to previous experience) higher organized beings will emerge. Indeed, it can be imagined that there will be a refinement of organisms here on Earth\index{Earth}, in the same proportion as occurred after the olivine-granite\index{olivine} period, and that creatures will arise that consume high amounts of water and gas for their preservation, as is almost the case with many plants.
\clearpage
\section{Explanation of the Tables}
\subsection{Preliminary Note}
\paragraph{}
The stones from which I made my thin sections were thoroughly certified.

The thin sections themselves were made by me with the untiring support of my sister-in-law, Miss Pauline Schloz\index{Schloz}. My collection numbers at 560 (including 360 of Knyahinya\index{meteorite!Knyahinya}), probably the largest collection that is available.

Regarding the manufacture of thin sections, I must mention the circumstances which influenced their appearance.

Anyone who has polished petrifacts\index{petrifact} knows that very few allow a thin slice. Not only because of the often opaque or difficult material (lime, clay), but because structures disappear when ground to (presumed) transparency.

It depends on the way in which the process of petrification occurs in each.

Thus, one is faced with the choice of either having a rather dim cut, in which one sees little, or, driven by the desire for sharper outlines, getting a cut that no longer shows anything, resorting to higher objectives in vain.

Both obstacles can be avoided in the meteorite\index{meteorite} material (which, incidentally, because of the iron, is difficult to grind) only by alternately making thinner and thicker cuts.

Regarding the choice of forms, future researchers will excuse me if I overlooked this or that form. My intention, of course, was to depict all the forms contained in my material. The figures should not only give pictures but also an overall view: I placed the greatest weight on concluding the matter of the nature of the rock.

As far as the order of the tables is concerned, it is related to the order of the material. Since I was aware that I had not yet exhausted the entire material, I did not bother to determine individual forms or to express views on their genetic links to justify them and their order: it was sufficient only to make a preliminary orientation in this direction. And for the present time, it is only a proof of organic rock, not about what everything is.

I avoided giving names not for fear of falling into the hands of critics, but because I came to the realization that by naming, nothing, or not much, is gained.

For a long time, I was faced with the choice of whether I should really take the path of photographic representation. However, I arrived at the decision in question more so out of thoughtfulness for the outsider.

There was a lot of talk regarding imagination in the criticism of \emph{Primordial Cell}\index{Primordial Cell}. I realize that the illustrations were not exact, that might be, but they are correct. For example, see the photographic depiction of the objects in \emph{Primordial Cell}\index{Primordial Cell} on Table 32: Figure 5 compared to Table 4 and 5 in \emph{Primordial Cell}\index{Primordial Cell}.

I would like to ask Dr. Kuntze\index{Kuntze} in Leipzig\index{Leipzig} whether he teaches of such synthetic algae\index{algae} --- if so, I would be very grateful for the provisioning of such a preparation to convince me of an error.\footnote{A similar treatment of Dr. Kuntze\index{Kuntze} with Dr. H. Karsten's\index{Karsten} \emph{Flora Columbiae}. Until he cleanses himself of the accusation Dr. W. Joos\index{Joos} raised against him on these criticisms, he has no right to be heard in science.} As far as I know, the dendrites and ``synthetic algae'', which were thus held against me without any examination or knowledge, are merely stripes not structures and secretions. In accordance with its formation it is usually a uniformly distributed continuously stained bulk, which lies between two stone slabs, i.e. as a perfect surface and so resembles plant shadows.

I admit that ``synthetic algae'' can be made from algae, as some researchers have said. But I must also point out that all structures that are thread or band-like have been explained as algae without much thought. To know that you have an alga in front of you, something more is needed. Things have been explained as plants that certainly do not show half as much form or structure as my pictures in \emph{Primordial Cell}\index{Primordial Cell}. Not all thread or sheaf shapes in rocks or other masses would I explain, using only these features, as algae.

My illustrations in \emph{Primordial Cell}\index{Primordial Cell} clearly show cell walls and cells; if these things were artificial algae or dendrites, they would not have any transverse walls.

With this I return to my subject.

Photography has significant drawbacks for scientific representation, as every researcher knows. For the present subject I had to follow this path simply because I would otherwise have been accused of ``imagination'' again. The Sun and collodion\index{collodion} together do not fool and must ward off any such accusation from the start. But the photographic image incorporates the object to a lesser extent. This was especially felt with my best subjects. In addition, especially at the higher magnifications, only a part of the thin section could be displayed and it was not sharp because of higher and lower rocks blurring the focus of the image. Too high of a magnification (I note this matter for any colleagues) is therefore not suitable in rock thin sections. Another obstacle is that the rocks consist of highly refractive material and the light of mineral fractures must be overcome; this creates light reflections of the most unpleasant kind that an untrained person could easily mistake for forms. To avoid this, I always work with the weakest magnifications to put aside the imperfect structural images.

The photographic focus is more likely to be below the object. The credibility of representation, as I have said, was the only reason for taking this path.

One particularly sensitive cause of additional shortcoming in the photographic representation is the effect of colors on the image. Of all the bad ones, yellow is the worst.

Where yellow is present in the preparation a black stain appears instead of structure. There was no means to rectify this evil. And it is the yellow of the olivine\index{olivine} that does not allow any ray of light through. This is most pronounced in the coral\index{coral} in Table 1: Figure 6: the black ring in the picture is a light yellow (iron). Brown follows yellow, which is also very dark. Blue has the opposite shortcoming, it becomes too light, but it still shows structures.

It goes without saying that the high price of the material imposes a certain economy in the preparations. This limits the selection. It is precisely for this reason why the thin sections must be made by the researcher himself. It is his duty. Admittedly this complicates things by the great amount of time required but it is the only possible way to thoroughly study the subjects.

For magnification and photographic representation, I have the intermediate microphotographic apparatus of Seibert\index{Seibert} \& Krafft\index{Krafft} from Wetzlar\index{Wetzlar} and can commend it as praiseworthy. The pictures were produced under my direction here in the photographic studio of Messrs. Otto Lauer\index{Lauer} \& Carl Bossler\index{Bossler}. Since we all had no practice in this sort of shooting, the contribution of Dr. Schreiner\index{Schreiner}, assistant at the chemical laboratory in Tübingen, was highly welcomed. I did not have additional help, but I think it should not go without mentioning the complete lack of participation from all those scholars to whom this matter most concerns.

In the ordering of the material, I put the sponges\index{sponge} first, followed by the corals\index{coral} and then the crinoids\index{crinoid}.

I have also represented the individual genera numerically in accordance with their frequency of occurrence. Unfortunately, I had to put aside some of the better objects because of their yellow coloring. If Gümbel\index{Gümbel}, as he says in his excellent essay on the Bavarian meteorites\index{meteorite}, proves correct in removing the yellow color by acids, much would be gained.

As for the magnifications, or more correctly the exact size of the magnifications, it came into consideration that the camera imposes a certain observance size. This leads to the bad state of affairs in which all the forms seem equally large.

The magnification specification, i.e. the ratio of the true size to the diameter of the displayed image is thus of very little significance.

I therefore preferred to denote the real size of the object by directly stating the diameter of each shape.
\clearpage
\subsection{Table Index}
\begin{enumerate}
    \item Pictures are numbered from top left to bottom right.
    \item Abbreviations: M. indicates magnification, D. indicates real diameter, mm indicates millimeter.
\end{enumerate}
\clearpage
\pagestyle{fancy}
\fancyhf{}
\rhead{Table 1: Mineral structures along with organic ones from the chondrites\index{chondrite}}
\cfoot{\thepage}
\begin{figure}[b]
\includegraphics[width=\textwidth,height=\textheight,keepaspectratio]{figures/meteorite_1-1_edit-b2.jpg}
\caption{Table 1: Figure 1 --- Enstatite\index{enstatite} (-Bronzite)\index{bronzite} from Kupferberg\index{Kupferberg} M.}
\centering
\end{figure}
\clearpage
\begin{figure}[t]
\includegraphics[width=\textwidth,height=\textheight,keepaspectratio]{figures/meteorite_1-2_edit-b.jpg}
\caption{Table 1: Figure 2 --- Enstatite\index{enstatite} from Texas\index{Texas} M.}
\centering
\end{figure}
\clearpage
\begin{figure}[t]
\includegraphics[width=\textwidth,height=\textheight,keepaspectratio]{figures/meteorite_1-3_edit-b2.jpg}
\caption{Table 1: Figure 3 --- Spherulite-Liparite\index{spherulite} from Lipari M.}
\centering
\end{figure}
\clearpage
\begin{figure}[t]
\includegraphics[width=\textwidth,height=\textheight,keepaspectratio]{figures/meteorite_1-4_edit-b.jpg}
\caption{Table 1: Figure 4 --- A part of the coral\index{coral} from Table 8, 9, and 10}
\centering
\end{figure}
\clearpage
\begin{figure}[t]
\includegraphics[width=\textwidth,height=\textheight,keepaspectratio]{figures/meteorite_1-5_edit-b2.jpg}
\caption{Table 1: Figure 5 --- Chain coral\index{coral} D. 0.90 mm.}
\centering
\end{figure}
\clearpage
\begin{figure}[t]
\includegraphics[width=\textwidth,height=\textheight,keepaspectratio]{figures/meteorite_1-6_edit-b2.jpg}
\caption{Table 1: Figure 6 --- Crinoid\index{crinoid} D. 1.20 mm.}
\centering
\end{figure}
\clearpage
\rhead{Table 2: \emph{Urania}}
\begin{figure}[t]
\includegraphics[width=\textwidth,height=\textheight,keepaspectratio]{figures/meteorite_2-1_edit-b2.jpg}
\caption{Table 2: Figure 1 --- Knyahinya\index{meteorite!Knyahinya}, same as Table 5: Figure 1.}
\centering
\end{figure}
\clearpage
\rhead{Table 3: \emph{Urania}}
\begin{figure}[t]
\includegraphics[width=\textwidth,height=\textheight,keepaspectratio]{figures/meteorite_3-1_edit-b.jpg}
\caption{Table 3: Figure 1 --- Knyahinya\index{meteorite!Knyahinya} D. 0.60 mm.}
\centering
\end{figure}
\clearpage
\begin{figure}[t]
\includegraphics[width=\textwidth,height=\textheight,keepaspectratio]{figures/meteorite_3-2_edit-b.jpg}
\caption{Table 3: Figure 2 --- Knyahinya\index{meteorite!Knyahinya} D. 1.30 mm. (do not overlook the magnificent crinoid\index{crinoid} limbs on the top left!)}
\centering
\end{figure}
\clearpage
\begin{figure}[t]
\includegraphics[width=\textwidth,height=\textheight,keepaspectratio]{figures/meteorite_3-3_edit-b.jpg}
\caption{Table 3: Figure 3 --- Knyahinya\index{meteorite!Knyahinya} D. 1 mm.}
\centering
\end{figure}
\clearpage
\begin{figure}[t]
\includegraphics[width=\textwidth,height=\textheight,keepaspectratio]{figures/meteorite_3-4_edit-b.jpg}
\caption{Table 3: Figure 4 --- Knyahinya\index{meteorite!Knyahinya} D. 1 mm.}
\centering
\end{figure}
\clearpage
\begin{figure}[t]
\includegraphics[width=\textwidth,height=\textheight,keepaspectratio]{figures/meteorite_3-5_edit-b2.jpg}
\caption{Table 3: Figure 5 --- Knyahinya\index{meteorite!Knyahinya} D. 1 mm. (notice the stratification at the top)}
\centering
\end{figure}
\clearpage
\begin{figure}[t]
\includegraphics[width=\textwidth,height=\textheight,keepaspectratio]{figures/meteorite_3-6_edit-b2.jpg}
\caption{Table 3: Figure 6 --- Knyahinya\index{meteorite!Knyahinya} D. 1 mm. (Stratification like 5, but not reproduced in the image, 5 and 6 of a thin section)}
\centering
\end{figure}
\clearpage
\rhead{Table 4: \emph{Urania}}
\begin{figure}[t]
\includegraphics[width=\textwidth,height=\textheight,keepaspectratio]{figures/meteorite_4-1_edit-b.jpg}
\caption{Table 4: Figure 1 --- Knyahinya\index{meteorite!Knyahinya} D. 0.90 mm.}
\centering
\end{figure}
\clearpage
\begin{figure}[t]
\includegraphics[width=\textwidth,height=\textheight,keepaspectratio]{figures/meteorite_4-2_edit-b.jpg}
\caption{Table 4: Figure 2 --- Siena\index{meteorite!Siena} D. 3 mm. (the dark line is due to the yellow color of the preparation)}
\centering
\end{figure}
\clearpage
\begin{figure}[t]
\includegraphics[width=\textwidth,height=\textheight,keepaspectratio]{figures/meteorite_4-3_edit-b.jpg}
\caption{Table 4: Figure 3 --- Knyahinya\index{meteorite!Knyahinya} D. 0.60 mm.}
\centering
\end{figure}
\clearpage
\begin{figure}[t]
\includegraphics[width=\textwidth,height=\textheight,keepaspectratio]{figures/meteorite_4-4_edit-b.jpg}
\caption{Table 4: Figure 4 --- Knyahinya\index{meteorite!Knyahinya} D. 0.90 mm. (air bubble)}
\centering
\end{figure}
\clearpage
\begin{figure}[t]
\includegraphics[width=\textwidth,height=\textheight,keepaspectratio]{figures/meteorite_4-5_edit-b.jpg}
\caption{Table 4: Figure 5 --- Knyahinya\index{meteorite!Knyahinya} D. 1.60 mm.}
\centering
\end{figure}
\clearpage
\begin{figure}[t]
\includegraphics[width=\textwidth,height=\textheight,keepaspectratio]{figures/meteorite_4-6_edit-b.jpg}
\caption{Table 4: Figure 6 --- Knyahinya\index{meteorite!Knyahinya} D. 1.00 mm. (air bubble)}
\centering
\end{figure}
\clearpage
\rhead{Table 5: \emph{Urania}}
\begin{figure}[t]
\includegraphics[width=\textwidth,height=\textheight,keepaspectratio]{figures/meteorite_5-1_edit-b.jpg}
\caption{Table 5: Figure 1 --- Knyahinya\index{meteorite!Knyahinya} D. 1.40 mm. (see Table 2. All around average crinoid\index{crinoid}. Form bottom left, magnification. Table 1: Figure 6 and Table 25: Figures 1 and 2)}
\centering
\end{figure}
\clearpage
\begin{figure}[t]
\includegraphics[width=\textwidth,height=\textheight,keepaspectratio]{figures/meteorite_5-2_edit-b2.jpg}
\caption{Table 5: Figure 2 --- Knyahinya\index{meteorite!Knyahinya} D. 1.80 mm.}
\centering
\end{figure}
\clearpage
\begin{figure}[t]
\includegraphics[width=\textwidth,height=\textheight,keepaspectratio]{figures/meteorite_5-3_edit-b.jpg}
\caption{Table 5: Figure 3 --- Knyahinya\index{meteorite!Knyahinya} D. 1.80 mm.}
\centering
\end{figure}
\clearpage
\begin{figure}[t]
\includegraphics[width=\textwidth,height=\textheight,keepaspectratio]{figures/meteorite_5-4_edit-b.jpg}
\caption{Table 5: Figure 4 --- Knyahinya\index{meteorite!Knyahinya} D. 1.30 mm. (blurred picture)}
\centering
\end{figure}
\clearpage
\begin{figure}[t]
\includegraphics[width=\textwidth,height=\textheight,keepaspectratio]{figures/meteorite_5-5_edit-b2.jpg}
\caption{Table 5: Figure 5 --- Knyahinya\index{meteorite!Knyahinya} D. 1.40 mm. (air bubble)}
\centering
\end{figure}
\clearpage
\begin{figure}[t]
\includegraphics[width=\textwidth,height=\textheight,keepaspectratio]{figures/meteorite_5-6_edit-b2.jpg}
\caption{Table 5: Figure 6 --- Knyahinya\index{meteorite!Knyahinya} D. 0.60 mm. (poor picture. The white circle is the average)}
\centering
\end{figure}
\clearpage
\rhead{Table 6: \emph{Urania}}
\begin{figure}[t]
\includegraphics[width=\textwidth,height=\textheight,keepaspectratio]{figures/meteorite_6-1_edit-b2.jpg}
\caption{Table 6: Figure 1 --- Siena\index{meteorite!Siena} D. 4.00 mm.}
\centering
\end{figure}
\clearpage
\begin{figure}[t]
\includegraphics[width=\textwidth,height=\textheight,keepaspectratio]{figures/meteorite_6-2_edit-b.jpg}
\caption{Table 6: Figure 2 --- Knyahinya\index{meteorite!Knyahinya} D. 0.80 mm.}
\centering
\end{figure}
\clearpage
\begin{figure}[t]
\includegraphics[width=\textwidth,height=\textheight,keepaspectratio]{figures/meteorite_6-3_edit-b.jpg}
\caption{Table 6: Figure 3 --- Siena\index{meteorite!Siena} D. 1.20 mm.}
\centering
\end{figure}
\clearpage
\begin{figure}[t]
\includegraphics[width=\textwidth,height=\textheight,keepaspectratio]{figures/meteorite_6-4_edit-b.jpg}
\caption{Table 6: Figure 4 --- Knyahinya\index{meteorite!Knyahinya} D. 0.70 mm. (the center is heavily illuminated)}
\centering
\end{figure}
\clearpage
\begin{figure}[t]
\includegraphics[width=\textwidth,height=\textheight,keepaspectratio]{figures/meteorite_6-5_edit-b.jpg}
\caption{Table 6: Figure 5 --- Knyahinya\index{meteorite!Knyahinya} D. 0.30 mm.}
\centering
\end{figure}
\clearpage
\begin{figure}[t]
\includegraphics[width=\textwidth,height=\textheight,keepaspectratio]{figures/meteorite_6-6_edit-b2.jpg}
\caption{Table 6: Figure 6 --- Knyahinya\index{meteorite!Knyahinya} D. 0.90 mm. (air bubble)}
\centering
\end{figure}
\clearpage
\rhead{Table 7: Sponges}
\begin{figure}[t]
\includegraphics[width=\textwidth,height=\textheight,keepaspectratio]{figures/meteorite_7-1_edit-b.jpg}
\caption{Table 7: Figure 1 --- Knyahinya\index{meteorite!Knyahinya} D. 2.30 mm.}
\centering
\end{figure}
\clearpage
\begin{figure}[t]
\includegraphics[width=\textwidth,height=\textheight,keepaspectratio]{figures/meteorite_7-2_edit-b.jpg}
\caption{Table 7: Figure 2 --- Knyahinya\index{meteorite!Knyahinya} D. 1.80 mm. (a crack in the preparation. Needle)}
\centering
\end{figure}
\clearpage
\begin{figure}[t]
\includegraphics[width=\textwidth,height=\textheight,keepaspectratio]{figures/meteorite_7-3_edit-b.jpg}
\caption{Table 7: Figure 3 --- Knyahinya\index{meteorite!Knyahinya} D. 2.10 mm.}
\centering
\end{figure}
\clearpage
\begin{figure}[t]
\includegraphics[width=\textwidth,height=\textheight,keepaspectratio]{figures/meteorite_7-4_edit-b.jpg}
\caption{Table 7: Figure 4 --- (Crinoid\index{crinoid} cross section?) of Knyahinya\index{meteorite!Knyahinya} D. 3.00 mm.}
\centering
\end{figure}
\clearpage
\begin{figure}[t]
\includegraphics[width=\textwidth,height=\textheight,keepaspectratio]{figures/meteorite_7-5_edit-b.jpg}
\caption{Table 7: Figure 5 --- Sponge? D. 1.00 mm.\index{sponge}}
\centering
\end{figure}
\clearpage
\begin{figure}[t]
\includegraphics[width=\textwidth,height=\textheight,keepaspectratio]{figures/meteorite_7-6_edit-b.jpg}
\caption{Table 7: Figure 6 --- Sponge? D. 2.40 mm.\index{sponge}}
\centering
\end{figure}
\clearpage
\rhead{Table 8: Corals}
\begin{figure}[t]
\includegraphics[width=\textwidth,height=\textheight,keepaspectratio]{figures/meteorite_8-1_edit-b2.jpg}
\caption{Table 8: Figure 1 --- (\emph{Favosites}\index{Favosites}) of Knyahinya\index{meteorite!Knyahinya} (see Table 1: Figure 4)}
\centering
\end{figure}
\clearpage
\rhead{Table 9: Corals}
\begin{figure}[t]
\includegraphics[width=\textwidth,height=\textheight,keepaspectratio]{figures/meteorite_9-1_edit-b3.jpg}
\caption{Table 9: Figure 1 --- Structure picture from top left of Table 8.}
\centering
\end{figure}
\clearpage
\rhead{Table 10: Corals}
\begin{figure}[t]
\includegraphics[width=\textwidth,height=\textheight,keepaspectratio]{figures/meteorite_10-1_edit-b.jpg}
\caption{Table 10: Figure 1 --- Knyahinya\index{meteorite!Knyahinya} cross section D. 0.40 mm.}
\centering
\end{figure}
\clearpage
\begin{figure}[t]
\includegraphics[width=\textwidth,height=\textheight,keepaspectratio]{figures/meteorite_10-2_edit-b.jpg}
\caption{Table 10: Figure 2 --- Longitudinal section 0.50 mm.}
\centering
\end{figure}
\clearpage
\begin{figure}[t]
\includegraphics[width=\textwidth,height=\textheight,keepaspectratio]{figures/meteorite_10-3_edit-b2.jpg}
\caption{Table 10: Figure 3 --- Knyahinya\index{meteorite!Knyahinya} D. 1.80 mm.}
\centering
\end{figure}
\clearpage
\begin{figure}[t]
%this figure has the same figure as Table 1: Figure 5, which has better quality
\includegraphics[width=\textwidth,height=\textheight,keepaspectratio]{figures/meteorite_1-5_edit-b2.jpg}
\caption{Table 10: Figure 4 --- Knyahinya\index{meteorite!Knyahinya} D. 0.90 mm. (see Table 8 and 9.)}
\centering
\end{figure}
\clearpage
\begin{figure}[t]
\includegraphics[width=\textwidth,height=\textheight,keepaspectratio]{figures/meteorite_10-5_edit-b.jpg}
\caption{Table 10: Figure 5 --- Knyahinya\index{meteorite!Knyahinya} D. 0.30 mm.}
\centering
\end{figure}
\clearpage
\begin{figure}[t]
\includegraphics[width=\textwidth,height=\textheight,keepaspectratio]{figures/meteorite_10-6_edit-b.jpg}
\caption{Table 10: Figure 6 --- Knyahinya\index{meteorite!Knyahinya} D. 0.80 mm.}
\centering
\end{figure}
\clearpage
\rhead{Table 11: Corals}
\begin{figure}[t]
\includegraphics[width=\textwidth,height=\textheight,keepaspectratio]{figures/meteorite_11-1_edit-b.jpg}
\caption{Table 11: Figure 1 --- Knyahinya\index{meteorite!Knyahinya} D. 1.20 mm.}
\centering
\end{figure}
\clearpage
\begin{figure}[t]
\includegraphics[width=\textwidth,height=\textheight,keepaspectratio]{figures/meteorite_11-2_edit-b.jpg}
\caption{Table 11: Figure 2 --- Knyahinya\index{meteorite!Knyahinya} D. 1.00 mm.}
\centering
\end{figure}
\clearpage
\begin{figure}[t]
\includegraphics[width=\textwidth,height=\textheight,keepaspectratio]{figures/meteorite_11-3_edit-b.jpg}
\caption{Table 11: Figure 3 --- Knyahinya\index{meteorite!Knyahinya} D. 1.80 mm.}
\centering
\end{figure}
\clearpage
\begin{figure}[t]
\includegraphics[width=\textwidth,height=\textheight,keepaspectratio]{figures/meteorite_11-4_edit-b.jpg}
\caption{Table 11: Figure 4 --- Knyahinya\index{meteorite!Knyahinya} D. 1.20 mm.}
\centering
\end{figure}
\clearpage
\begin{figure}[t]
\includegraphics[width=\textwidth,height=\textheight,keepaspectratio]{figures/meteorite_11-5_edit-b.jpg}
\caption{Table 11: Figure 5 --- Parnallee\index{meteorite!Parnallee} D. 0.80 mm.}
\centering
\end{figure}
\clearpage
\begin{figure}[t]
\includegraphics[width=\textwidth,height=\textheight,keepaspectratio]{figures/meteorite_11-6_edit.jpg}
\caption{Table 11: Figure 6 --- Moung County\index{meteorite!Moung County} D. 0.60 mm.}
\centering
\end{figure}
\clearpage
\rhead{Table 12: Corals}
\begin{figure}[t]
\includegraphics[width=\textwidth,height=\textheight,keepaspectratio]{figures/meteorite_12-1_edit-b.jpg}
\caption{Table 12: Figure 1 --- Knyahinya\index{meteorite!Knyahinya} D. 0.80 mm.}
\centering
\end{figure}
\clearpage
\begin{figure}[t]
\includegraphics[width=\textwidth,height=\textheight,keepaspectratio]{figures/meteorite_12-2_edit-b.jpg}
\caption{Table 12: Figure 2 --- Knyahinya\index{meteorite!Knyahinya} D. 1.20 mm.}
\centering
\end{figure}
\clearpage
\begin{figure}[t]
\includegraphics[width=\textwidth,height=\textheight,keepaspectratio]{figures/meteorite_12-3_edit-b.jpg}
\caption{Table 12: Figure 3 --- Knyahinya\index{meteorite!Knyahinya} D. 1.30 mm.}
\centering
\end{figure}
\clearpage
\begin{figure}[t]
\includegraphics[width=\textwidth,height=\textheight,keepaspectratio]{figures/meteorite_12-4_edit-b.jpg}
\caption{Table 12: Figure 4 --- Knyahinya\index{meteorite!Knyahinya} D. 1.40 mm.}
\centering
\end{figure}
\clearpage
\begin{figure}[t]
\includegraphics[width=\textwidth,height=\textheight,keepaspectratio]{figures/meteorite_12-5_edit-b.jpg}
\caption{Table 12: Figure 5 --- Knyahinya\index{meteorite!Knyahinya} D. 2.00 mm.}
\centering
\end{figure}
\clearpage
\begin{figure}[t]
\includegraphics[width=\textwidth,height=\textheight,keepaspectratio]{figures/meteorite_12-6_edit-b.jpg}
\caption{Table 12: Figure 6 --- Knyahinya\index{meteorite!Knyahinya} D. 3.20 mm.}
\centering
\end{figure}
\clearpage
\rhead{Table 13: Corals}
\begin{figure}[t]
\includegraphics[width=\textwidth,height=\textheight,keepaspectratio]{figures/meteorite_13-1_edit-b.jpg}
\caption{Table 13: Figure 1 --- Parnallee\index{meteorite!Parnallee} D. 0.20 mm.}
\centering
\end{figure}
\clearpage
\begin{figure}[t]
\includegraphics[width=\textwidth,height=\textheight,keepaspectratio]{figures/meteorite_13-2_edit-b.jpg}
\caption{Table 13: Figure 2 --- Knyahinya\index{meteorite!Knyahinya} D. 0.80 mm.}
\centering
\end{figure}
\clearpage
\begin{figure}[t]
\includegraphics[width=\textwidth,height=\textheight,keepaspectratio]{figures/meteorite_13-3_edit-b.jpg}
\caption{Table 13: Figure 3 --- Siena\index{meteorite!Siena} D. 0.20 mm.}
\centering
\end{figure}
\clearpage
\begin{figure}[t]
\includegraphics[width=\textwidth,height=\textheight,keepaspectratio]{figures/meteorite_13-4_edit-b.jpg}
\caption{Table 13: Figure 4 --- Knyahinya\index{meteorite!Knyahinya} D. 1.80 mm.}
\centering
\end{figure}
\clearpage
\begin{figure}[t]
\includegraphics[width=\textwidth,height=\textheight,keepaspectratio]{figures/meteorite_13-5_edit-b.jpg}
\caption{Table 13: Figure 5 --- Knyahinya\index{meteorite!Knyahinya} D. 1.70 mm.}
\centering
\end{figure}
\clearpage
\begin{figure}[t]
\includegraphics[width=\textwidth,height=\textheight,keepaspectratio]{figures/meteorite_13-6_edit-b.jpg}
\caption{Table 13: Figure 6 --- Cabarras\index{meteorite!Cabarras} D. 0.30 mm.}
\centering
\end{figure}
\clearpage
\rhead{Table 14: Corals}
\begin{figure}[t]
\includegraphics[width=\textwidth,height=\textheight,keepaspectratio]{figures/meteorite_14-1_edit-b2.jpg}
\caption{Table 14: Figure 1 --- Coral\index{coral} D. 0.90 mm.}
\centering
\end{figure}
\clearpage
\rhead{Table 15: Corals}
\begin{figure}[t]
\includegraphics[width=\textwidth,height=\textheight,keepaspectratio]{figures/meteorite_15-1_edit-b3.jpg}
\caption{Table 15: Figure 1 --- Coral\index{coral}. Structure picture from 14. The upper left part of the preparation, magnification 300, shows the bud canals.}
\centering
\end{figure}
\clearpage
\rhead{Table 16: Crinoids}
\begin{figure}[t]
\includegraphics[width=\textwidth,height=\textheight,keepaspectratio]{figures/meteorite_16-1_edit-b2.jpg}
\caption{Table 16: Figure 1 --- Knyahinya\index{meteorite!Knyahinya} D. 0.40 mm.}
\centering
\end{figure}
\clearpage
\rhead{Table 17: Crinoids}
\begin{figure}[t]
\includegraphics[width=\textwidth,height=\textheight,keepaspectratio]{figures/meteorite_17-1_edit-b2.jpg}
\caption{Table 17: Figure 1 --- Knyahinya\index{meteorite!Knyahinya} D. 2.00 mm.}
\centering
\end{figure}
\clearpage
\rhead{Table 18: Crinoids}
\begin{figure}[t]
\includegraphics[width=\textwidth,height=\textheight,keepaspectratio]{figures/meteorite_18-1_edit-b2.jpg}
\caption{Table 18: Figure 1 --- Knyahinya\index{meteorite!Knyahinya}, cut through four main arms, D. 2.20 mm.}
\centering
\end{figure}
\clearpage
\rhead{Table 19: Crinoids}
\begin{figure}[t]
\includegraphics[width=\textwidth,height=\textheight,keepaspectratio]{figures/meteorite_19-1_edit-b2.jpg}
\caption{Table 19: Figure 1 --- Crinoid\index{crinoid}, see Table 25: Figures 1 and 2.}
\centering
\end{figure}
\clearpage
\rhead{Table 20: Crinoids}
\begin{figure}[t]
\includegraphics[width=\textwidth,height=\textheight,keepaspectratio]{figures/meteorite_20-1_edit-b2.jpg}
\caption{Table 20: Figure 1 --- Cut through crinoid\index{crinoid} and coral\index{coral} in Knyahinya\index{meteorite!Knyahinya} D. 1.20 mm.}
\centering
\end{figure}
\clearpage
\rhead{Table 21: Crinoids}
\begin{figure}[t]
\includegraphics[width=\textwidth,height=\textheight,keepaspectratio]{figures/meteorite_21-1_edit-b.jpg}
\caption{Table 21: Figure 1 --- Knyahinya\index{meteorite!Knyahinya} D. 0.80 mm.}
\centering
\end{figure}
\clearpage
\begin{figure}[t]
\includegraphics[width=\textwidth,height=\textheight,keepaspectratio]{figures/meteorite_21-2_edit-b.jpg}
\caption{Table 21: Figure 2 --- magnified image from Figure 1}
\centering
\end{figure}
\clearpage
\begin{figure}[t]
\includegraphics[width=\textwidth,height=\textheight,keepaspectratio]{figures/meteorite_21-3_edit-b.jpg}
\caption{Table 21: Figure 3 --- Knyahinya\index{meteorite!Knyahinya} D. 1.20 mm.}
\centering
\end{figure}
\clearpage
\begin{figure}[t]
\includegraphics[width=\textwidth,height=\textheight,keepaspectratio]{figures/meteorite_21-4_edit-b.jpg}
\caption{Table 21: Figure 4 --- magnified image from Figure 3}
\centering
\end{figure}
\clearpage
\begin{figure}[t]
\includegraphics[width=\textwidth,height=\textheight,keepaspectratio]{figures/meteorite_21-5_edit-b.jpg}
\caption{Table 21: Figure 5 --- Knyahinya\index{meteorite!Knyahinya} D. 1.80 mm. (I notice resemblance with Figure 1)}
\centering
\end{figure}
\clearpage
\begin{figure}[t]
\includegraphics[width=\textwidth,height=\textheight,keepaspectratio]{figures/meteorite_21-6_edit-b.jpg}
\caption{Table 21: Figure 6 --- Knyahinya\index{meteorite!Knyahinya} D. 0.30 mm. (the mouth opening between the arms is visible)}
\centering
\end{figure}
\clearpage
\rhead{Table 22: Crinoids}
\begin{figure}[t]
\includegraphics[width=\textwidth,height=\textheight,keepaspectratio]{figures/meteorite_22-1_edit-b.jpg}
\caption{Table 22: Figure 1 --- Knyahinya\index{meteorite!Knyahinya} D. 0.50 mm.}
\centering
\end{figure}
\clearpage
\begin{figure}[t]
\includegraphics[width=\textwidth,height=\textheight,keepaspectratio]{figures/meteorite_22-2_edit-b.jpg}
\caption{Table 22: Figure 2 --- Knyahinya\index{meteorite!Knyahinya} D. 0.60 mm.}
\centering
\end{figure}
\clearpage
\begin{figure}[t]
\includegraphics[width=\textwidth,height=\textheight,keepaspectratio]{figures/meteorite_22-3_edit-b.jpg}
\caption{Table 22: Figure 3 --- Knyahinya\index{meteorite!Knyahinya} (Cover picture) D. 1.50 mm.}
\centering
\end{figure}
\clearpage
\begin{figure}[t]
\includegraphics[width=\textwidth,height=\textheight,keepaspectratio]{figures/meteorite_22-4_edit-b.jpg}
\caption{Table 22: Figure 4 --- Knyahinya\index{meteorite!Knyahinya} D. 0.70 mm.}
\centering
\end{figure}
\clearpage
\begin{figure}[t]
\includegraphics[width=\textwidth,height=\textheight,keepaspectratio]{figures/meteorite_22-5_edit-b.jpg}
\caption{Table 22: Figure 5 --- Knyahinya\index{meteorite!Knyahinya} D. 0.60 mm.}
\centering
\end{figure}
\clearpage
\begin{figure}[t]
\includegraphics[width=\textwidth,height=\textheight,keepaspectratio]{figures/meteorite_22-6_edit-b.jpg}
\caption{Table 22: Figure 6 --- Knyahinya\index{meteorite!Knyahinya} D. 1.20 mm.}
\centering
\end{figure}
\clearpage
\rhead{Table 23: Crinoids}
\begin{figure}[t]
\includegraphics[width=\textwidth,height=\textheight,keepaspectratio]{figures/meteorite_23-1_edit-b.jpg}
\caption{Table 23: Figure 1 --- Knyahinya\index{meteorite!Knyahinya} D. 0.90 mm.}
\centering
\end{figure}
\clearpage
\begin{figure}[t]
\includegraphics[width=\textwidth,height=\textheight,keepaspectratio]{figures/meteorite_23-2_edit-b.jpg}
\caption{Table 23: Figure 2 --- Knyahinya\index{meteorite!Knyahinya} D. 1.60 mm.}
\centering
\end{figure}
\clearpage
\begin{figure}[t]
\includegraphics[width=\textwidth,height=\textheight,keepaspectratio]{figures/meteorite_23-3_edit-b.jpg}
\caption{Table 23: Figure 3 --- Knyahinya\index{meteorite!Knyahinya} D. 1.00 mm.}
\centering
\end{figure}
\clearpage
\begin{figure}[t]
\includegraphics[width=\textwidth,height=\textheight,keepaspectratio]{figures/meteorite_23-4_edit-b.jpg}
\caption{Table 23: Figure 4 --- Knyahinya\index{meteorite!Knyahinya} D. 1.40 mm.}
\centering
\end{figure}
\clearpage
\begin{figure}[t]
\includegraphics[width=\textwidth,height=\textheight,keepaspectratio]{figures/meteorite_23-5_edit-b.jpg}
\caption{Table 23: Figure 5 --- Knyahinya\index{meteorite!Knyahinya} D. 1.30 mm.}
\centering
\end{figure}
\clearpage
\begin{figure}[t]
\includegraphics[width=\textwidth,height=\textheight,keepaspectratio]{figures/meteorite_23-6_edit-b.jpg}
\caption{Table 23: Figure 6 --- Knyahinya\index{meteorite!Knyahinya} D. 0.60 mm.}
\centering
\end{figure}
\clearpage
\rhead{Table 24: Crinoids}
\begin{figure}[t]
\includegraphics[width=\textwidth,height=\textheight,keepaspectratio]{figures/meteorite_24-1_edit-b.jpg}
\caption{Table 24: Figure 1 --- Siena\index{meteorite!Siena} D. 0.80 mm.}
\centering
\end{figure}
\clearpage
\begin{figure}[t]
\includegraphics[width=\textwidth,height=\textheight,keepaspectratio]{figures/meteorite_24-2_edit-b.jpg}
\caption{Table 24: Figure 2 --- Knyahinya\index{meteorite!Knyahinya} D. 2.80 mm.}
\centering
\end{figure}
\clearpage
\begin{figure}[t]
\includegraphics[width=\textwidth,height=\textheight,keepaspectratio]{figures/meteorite_24-3_edit-b.jpg}
\caption{Table 24: Figure 3 --- Knyahinya\index{meteorite!Knyahinya} D. 1.00 mm.}
\centering
\end{figure}
\clearpage
\begin{figure}[t]
\includegraphics[width=\textwidth,height=\textheight,keepaspectratio]{figures/meteorite_24-4_edit-b.jpg}
\caption{Table 24: Figure 4 --- Knyahinya\index{meteorite!Knyahinya} D. 2.00 mm.}
\centering
\end{figure}
\clearpage
\begin{figure}[t]
\includegraphics[width=\textwidth,height=\textheight,keepaspectratio]{figures/meteorite_24-5_edit-b.jpg}
\caption{Table 24: Figure 5 --- Knyahinya\index{meteorite!Knyahinya} D. 1.50 mm.}
\centering
\end{figure}
\clearpage
\begin{figure}[t]
\includegraphics[width=\textwidth,height=\textheight,keepaspectratio]{figures/meteorite_24-6_edit-b.jpg}
\caption{Table 24: Figure 6 --- Cabarras\index{meteorite!Cabarras} D. 0.80 mm.}
\centering
\end{figure}
\clearpage
\rhead{Table 25: Crinoids}
\begin{figure}[t]
\includegraphics[width=\textwidth,height=\textheight,keepaspectratio]{figures/meteorite_25-1_edit-b.jpg}
\caption{Table 25: Figure 1 --- Knyahinya\index{meteorite!Knyahinya} D. 1.20 mm.}
\centering
\end{figure}
\clearpage
\begin{figure}[t]
\includegraphics[width=\textwidth,height=\textheight,keepaspectratio]{figures/meteorite_25-2_edit-b.jpg}
\caption{Table 25: Figure 2 --- Knyahinya\index{meteorite!Knyahinya} D. 1.20 mm.}
\centering
\end{figure}
\clearpage
\begin{figure}[t]
\includegraphics[width=\textwidth,height=\textheight,keepaspectratio]{figures/meteorite_25-3_edit-b.jpg}
\caption{Table 25: Figure 3 --- Knyahinya\index{meteorite!Knyahinya} D. 1.80 mm.}
\centering
\end{figure}
\clearpage
\begin{figure}[t]
\includegraphics[width=\textwidth,height=\textheight,keepaspectratio]{figures/meteorite_25-4_edit-b.jpg}
\caption{Table 25: Figure 4 --- Knyahinya\index{meteorite!Knyahinya} D. 0.60 mm.}
\centering
\end{figure}
\clearpage
\begin{figure}[t]
\includegraphics[width=\textwidth,height=\textheight,keepaspectratio]{figures/meteorite_25-5_edit-b.jpg}
\caption{Table 25: Figure 5 --- Siena\index{meteorite!Siena} D. 1.80 mm.}
\centering
\end{figure}
\clearpage
\begin{figure}[t]
\includegraphics[width=\textwidth,height=\textheight,keepaspectratio]{figures/meteorite_25-6_edit-b.jpg}
\caption{Table 25: Figure 6 --- Knyahinya\index{meteorite!Knyahinya} D. 1.40 mm. (Both latter are cross sections of crinoids)}
\centering
\end{figure}
\clearpage
\rhead{Table 26: Crinoids}
\begin{figure}[t]
\includegraphics[width=\textwidth,height=\textheight,keepaspectratio]{figures/meteorite_26-1_edit-b.jpg}
\caption{Table 26: Figure 1 --- Knyahinya\index{meteorite!Knyahinya} D. 0.20 mm.}
\centering
\end{figure}
\clearpage
\begin{figure}[t]
\includegraphics[width=\textwidth,height=\textheight,keepaspectratio]{figures/meteorite_26-2_edit-b.jpg}
\caption{Table 26: Figure 2 --- Knyahinya\index{meteorite!Knyahinya} D. 2.00 mm.}
\centering
\end{figure}
\clearpage
\begin{figure}[t]
\includegraphics[width=\textwidth,height=\textheight,keepaspectratio]{figures/meteorite_26-3_edit-b.jpg}
\caption{Table 26: Figure 3 --- Knyahinya\index{meteorite!Knyahinya} D. 1.20 mm.}
\centering
\end{figure}
\clearpage
\begin{figure}[t]
\includegraphics[width=\textwidth,height=\textheight,keepaspectratio]{figures/meteorite_26-4_edit-b.jpg}
\caption{Table 26: Figure 4 --- Knyahinya\index{meteorite!Knyahinya} D. 1.20 mm. (here twisted crinoids)}
\centering
\end{figure}
\clearpage
\begin{figure}[t]
\includegraphics[width=\textwidth,height=\textheight,keepaspectratio]{figures/meteorite_26-5_edit-b.jpg}
\caption{Table 26: Figure 5 --- Knyahinya\index{meteorite!Knyahinya} D. 2.00 mm.}
\centering
\end{figure}
\clearpage
\begin{figure}[t]
\includegraphics[width=\textwidth,height=\textheight,keepaspectratio]{figures/meteorite_26-6_edit-b.jpg}
\caption{Table 26: Figure 6 --- Knyahinya\index{meteorite!Knyahinya} D. 2.20 mm. (the dark line in 5 and 6 is the food channel)}
\centering
\end{figure}
\clearpage
\rhead{Table 27: Crinoids}
\begin{figure}[t]
\includegraphics[width=\textwidth,height=\textheight,keepaspectratio]{figures/meteorite_27-1_edit-b.jpg}
\caption{Table 27: Figure 1 --- Knyahinya\index{meteorite!Knyahinya} D. 0.80 mm.}
\centering
\end{figure}
\clearpage
\begin{figure}[t]
\includegraphics[width=\textwidth,height=\textheight,keepaspectratio]{figures/meteorite_27-2_edit-b.jpg}
\caption{Table 27: Figure 2 --- Knyahinya\index{meteorite!Knyahinya} D. 1.50 mm.}
\centering
\end{figure}
\clearpage
\begin{figure}[t]
\includegraphics[width=\textwidth,height=\textheight,keepaspectratio]{figures/meteorite_27-3_edit-b.jpg}
\caption{Table 27: Figure 3 --- Knyahinya\index{meteorite!Knyahinya} D. 1.40 mm.}
\centering
\end{figure}
\clearpage
\begin{figure}[t]
\includegraphics[width=\textwidth,height=\textheight,keepaspectratio]{figures/meteorite_27-4_edit-b.jpg}
\caption{Table 27: Figure 4 --- Knyahinya\index{meteorite!Knyahinya} D. 1.40 mm.}
\centering
\end{figure}
\clearpage
\begin{figure}[t]
\includegraphics[width=\textwidth,height=\textheight,keepaspectratio]{figures/meteorite_27-5_edit-b.jpg}
\caption{Table 27: Figure 5 --- Knyahinya\index{meteorite!Knyahinya} D. 1.20 mm.}
\centering
\end{figure}
\clearpage
\begin{figure}[t]
\includegraphics[width=\textwidth,height=\textheight,keepaspectratio]{figures/meteorite_27-6_edit-b.jpg}
\caption{Table 27: Figure 6 --- Knyahinya\index{meteorite!Knyahinya} D. 1.00 mm.}
\centering
\end{figure}
\clearpage
\rhead{Table 28: Crinoids}
\begin{figure}[t]
\includegraphics[width=\textwidth,height=\textheight,keepaspectratio]{figures/meteorite_28-1_edit-b.jpg}
\caption{Table 28: Figure 1 --- Knyahinya\index{meteorite!Knyahinya} (Coral?) D. 3.00 mm. from the same thin section as Table 18.}
\centering
\end{figure}
\clearpage
\begin{figure}[t]
\includegraphics[width=\textwidth,height=\textheight,keepaspectratio]{figures/meteorite_28-2_edit-b.jpg}
\caption{Table 28: Figure 2 --- Knyahinya\index{meteorite!Knyahinya} D. 1.20 mm.}
\centering
\end{figure}
\clearpage
\begin{figure}[t]
\includegraphics[width=\textwidth,height=\textheight,keepaspectratio]{figures/meteorite_28-3_edit-b.jpg}
\caption{Table 28: Figure 3 --- Knyahinya\index{meteorite!Knyahinya} D. 2.30 mm.}
\centering
\end{figure}
\clearpage
\begin{figure}[t]
\includegraphics[width=\textwidth,height=\textheight,keepaspectratio]{figures/meteorite_28-4_edit-b.jpg}
\caption{Table 28: Figure 4 --- Knyahinya\index{meteorite!Knyahinya} D. 0.90 mm.}
\centering
\end{figure}
\clearpage
\begin{figure}[t]
\includegraphics[width=\textwidth,height=\textheight,keepaspectratio]{figures/meteorite_28-5_edit-b.jpg}
\caption{Table 28: Figure 5 --- Knyahinya\index{meteorite!Knyahinya} D. 1.50 mm.}
\centering
\end{figure}
\clearpage
\begin{figure}[t]
\includegraphics[width=\textwidth,height=\textheight,keepaspectratio]{figures/meteorite_28-6_edit-b.jpg}
\caption{Table 28: Figure 6 --- Knyahinya\index{meteorite!Knyahinya} D. 1.40 mm.}
\centering
\end{figure}
\clearpage
\rhead{Table 29: Crinoids (1-3 viewed from above, 4 from below)}
\begin{figure}[t]
\includegraphics[width=\textwidth,height=\textheight,keepaspectratio]{figures/meteorite_29-1_edit-b.jpg}
\caption{Table 29: Figure 1 --- Knyahinya\index{meteorite!Knyahinya} D. 0.20 mm.}
\centering
\end{figure}
\clearpage
\begin{figure}[t]
\includegraphics[width=\textwidth,height=\textheight,keepaspectratio]{figures/meteorite_29-2_edit-b.jpg}
\caption{Table 29: Figure 2 --- Knyahinya\index{meteorite!Knyahinya} D. 0.90 mm.}
\centering
\end{figure}
\clearpage
\begin{figure}[t]
\includegraphics[width=\textwidth,height=\textheight,keepaspectratio]{figures/meteorite_29-3_edit-b.jpg}
\caption{Table 29: Figure 3 --- Tabor\index{meteorite!Tabor} D. 2.10 mm.}
\centering
\end{figure}
\clearpage
\begin{figure}[t]
\includegraphics[width=\textwidth,height=\textheight,keepaspectratio]{figures/meteorite_29-4_edit-b.jpg}
\caption{Table 29: Figure 4 --- Knyahinya\index{meteorite!Knyahinya} D. 1.10 mm.}
\centering
\end{figure}
\clearpage
\begin{figure}[t]
\includegraphics[width=\textwidth,height=\textheight,keepaspectratio]{figures/meteorite_29-5_edit-b.jpg}
\caption{Table 29: Figure 5 --- Borkut\index{Borkut} D. 1.50 mm.}
\centering
\end{figure}
\clearpage
\begin{figure}[t]
\includegraphics[width=\textwidth,height=\textheight,keepaspectratio]{figures/meteorite_29-6_edit-b.jpg}
\caption{Table 29: Figure 6 --- Knyahinya\index{meteorite!Knyahinya} D. 1.30 mm. (questionable)}
\centering
\end{figure}
\clearpage
\rhead{Table 30: Crinoids}
\begin{figure}[t]
\includegraphics[width=\textwidth,height=\textheight,keepaspectratio]{figures/meteorite_30-1_edit-b.jpg}
\caption{Table 30: Figure 1 --- Knyahinya\index{meteorite!Knyahinya} D. 1.10 mm. (Coral?)}
\centering
\end{figure}
\clearpage
\begin{figure}[t]
\includegraphics[width=\textwidth,height=\textheight,keepaspectratio]{figures/meteorite_30-2_edit-b.jpg}
\caption{Table 30: Figure 2 --- Knyahinya\index{meteorite!Knyahinya} D. 1.40 mm. (Coral and Crinoid, see Table 20)}
\centering
\end{figure}
\clearpage
\begin{figure}[t]
\includegraphics[width=\textwidth,height=\textheight,keepaspectratio]{figures/meteorite_30-3_edit-b.jpg}
\caption{Table 30: Figure 3 --- Knyahinya\index{meteorite!Knyahinya} D. 0.30 mm. (the arms entwined like a mesh)}
\centering
\end{figure}
\clearpage
\begin{figure}[t]
\includegraphics[width=\textwidth,height=\textheight,keepaspectratio]{figures/meteorite_30-4_edit-b.jpg}
\caption{Table 30: Figure 4 --- Knyahinya\index{meteorite!Knyahinya} D. 1.85 mm. (first slice)}
\centering
\end{figure}
\clearpage
\begin{figure}[t]
\includegraphics[width=\textwidth,height=\textheight,keepaspectratio]{figures/meteorite_30-5_edit-b.jpg}
\caption{Table 30: Figure 5 --- Knyahinya\index{meteorite!Knyahinya} D. 0.70 mm. (first slice)}
\centering
\end{figure}
\clearpage
\begin{figure}[t]
\includegraphics[width=\textwidth,height=\textheight,keepaspectratio]{figures/meteorite_30-6_edit-b.jpg}
\caption{Table 30: Figure 6 --- Knyahinya\index{meteorite!Knyahinya} D. 0.40 mm. (Structure like the Schreibersite in the iron meteorites)}
\centering
\end{figure}
\clearpage
\rhead{Table 31: Problematic}
\begin{figure}[t]
\includegraphics[width=\textwidth,height=\textheight,keepaspectratio]{figures/meteorite_31-1_edit-b.jpg}
\caption{Table 31: Figure 1 --- Knyahinya\index{meteorite!Knyahinya} D. 1.20 mm. (not quite complete picture)}
\centering
\end{figure}
\clearpage
\begin{figure}[t]
\includegraphics[width=\textwidth,height=\textheight,keepaspectratio]{figures/meteorite_31-2_edit-b.jpg}
\caption{Table 31: Figure 2 --- Knyahinya\index{meteorite!Knyahinya} D. 0.50 mm.}
\centering
\end{figure}
\clearpage
\begin{figure}[t]
\includegraphics[width=\textwidth,height=\textheight,keepaspectratio]{figures/meteorite_31-3_edit-b.jpg}
\caption{Table 31: Figure 3 --- Knyahinya\index{meteorite!Knyahinya} D. 1.20 mm. (Three corresponding forms of three thin sections, in both 1 and 2 horizontal cuts)}
\centering
\end{figure}
\clearpage
\begin{figure}[t]
\includegraphics[width=\textwidth,height=\textheight,keepaspectratio]{figures/meteorite_31-4_edit-b.jpg}
\caption{Table 31: Figure 4 --- Knyahinya\index{meteorite!Knyahinya} (whether sponge or coral?) D. 0.90 mm.}
\centering
\end{figure}
\clearpage
\begin{figure}[t]
\includegraphics[width=\textwidth,height=\textheight,keepaspectratio]{figures/meteorite_31-5_edit-b.jpg}
\caption{Table 31: Figure 5 --- Knyahinya\index{meteorite!Knyahinya} D. 1.50 mm.}
\centering
\end{figure}
\clearpage
\begin{figure}[t]
\includegraphics[width=\textwidth,height=\textheight,keepaspectratio]{figures/meteorite_31-6_edit-b.jpg}
\caption{Table 31: Figure 6 --- Knyahinya\index{meteorite!Knyahinya} D. 1.40 mm.}
\centering
\end{figure}
\clearpage
\rhead{Table 32: Miscellaneous}
\begin{figure}[t]
\includegraphics[width=\textwidth,height=\textheight,keepaspectratio]{figures/meteorite_32-1_edit-b.jpg}
\caption{Table 32: Figure 1 --- Knyahinya\index{meteorite!Knyahinya} (inclusion) D. 1.50 mm.}
\centering
\end{figure}
\clearpage
\begin{figure}[t]
\includegraphics[width=\textwidth,height=\textheight,keepaspectratio]{figures/meteorite_32-2_edit-b.jpg}
\caption{Table 32: Figure 2 --- Borkut\index{Borkut} sphere D. 1.00 mm.}
\centering
\end{figure}
\clearpage
\begin{figure}[t]
\includegraphics[width=\textwidth,height=\textheight,keepaspectratio]{figures/meteorite_32-3_edit-b.jpg}
\caption{Table 32: Figure 3 --- Nummulite\index{nummulite} from Kempten\index{Kempten}. The channel is clearly visible (with the magnifying glass).}
\centering
\end{figure}
\clearpage
\begin{figure}[t]
\includegraphics[width=\textwidth,height=\textheight,keepaspectratio]{figures/meteorite_32-4_edit-b.jpg}
\caption{Table 32: Figure 4 --- Thin section from Lias\index{Lias} $\gamma\delta$. This thin section is taken from the assembled collection of 30 thin sections of sedimentary rocks, manufactured by geologist Hildebrand\index{Hildebrand} in Ohmenhausen\index{Ohmenhausen} near Reutlingen\index{Reutlingen}, which I strongly recommend for studying the microscopic nature of sedimentary rocks and inclusions.}
\centering
\end{figure}
\clearpage
\begin{figure}[t]
\includegraphics[width=\textwidth,height=\textheight,keepaspectratio]{figures/meteorite_32-5_edit-b.jpg}
\caption{Table 32: Figure 5 --- \emph{Eozoön canadense}\index{Eozoön}, so-called channel system of \emph{Eozoön}\index{Eozoön}.}
\centering
\end{figure}
\clearpage
\begin{figure}[t]
\includegraphics[width=\textwidth,height=\textheight,keepaspectratio]{figures/meteorite_32-6_edit-b.jpg}
\caption{Table 32: Figure 6 --- ditto. Both cuts taken from rocks collected by me in Little Nation. Compare the channel system of the numulites\index{nummulite} in Figure 3 with this alleged channel system! Picture 3 and 5 should be the same object. Compare to Figure 5 from \emph{Primordial Cell}\index{Primordial Cell} Table 4 and 5.}
\centering
\end{figure}
\clearpage
\pagestyle{plain}
\printindex
\clearpage
\end{document}
