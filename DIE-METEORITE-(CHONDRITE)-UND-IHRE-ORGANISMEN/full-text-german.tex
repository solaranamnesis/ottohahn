\documentclass[a4paper, 11pt, oneside]{article}
\usepackage[utf8]{inputenc}
\usepackage[T1]{fontenc}
\usepackage[ngerman]{babel}
\usepackage{fbb} %Derived from Cardo, provides a Bembo-like font family in otf and pfb format plus LaTeX font support files
\usepackage{booktabs}
\usepackage{url}
\usepackage{graphicx}
\setlength{\emergencystretch}{15pt}
\graphicspath{ {./figures/} }
\usepackage[figurename=]{caption}
\usepackage{fancyhdr}
\usepackage{imakeidx}
\makeindex[columns=2, title=Alphabetischer Index, intoc]
\begin{document}
\begin{titlepage} % Suppresses headers and footers on the title page
	\centering % Centre everything on the title page
	\scshape % Use small caps for all text on the title page

	%------------------------------------------------
	%	Title
	%------------------------------------------------
	
	\rule{\textwidth}{1.6pt}\vspace*{-\baselineskip}\vspace*{2pt} % Thick horizontal rule
	\rule{\textwidth}{0.4pt} % Thin horizontal rule
	
	\vspace{0.75\baselineskip} % Whitespace above the title
	
	{\LARGE DIE METEORITE (CHONDRITE)\\ UND\\ IHRE ORGANISMEN\\} % Title
	
	\vspace{0.75\baselineskip} % Whitespace below the title
	
	\rule{\textwidth}{0.4pt}\vspace*{-\baselineskip}\vspace{3.2pt} % Thin horizontal rule
	\rule{\textwidth}{1.6pt} % Thick horizontal rule
	
	\vspace{1\baselineskip} % Whitespace after the title block
	
	%------------------------------------------------
	%	Subtitle
	%------------------------------------------------
	
	{Dargestellt und Beschrieben\\ von\\ \scshape\Large Dr. Otto Hahn\\} % Subtitle or further description
	
	\vspace*{1\baselineskip} % Whitespace under the subtitle
	
    {\small 32 Tafeln mit 142 Abbildungen} % Subtitle or further description
    
	%------------------------------------------------
	%	Editor(s)
	%------------------------------------------------
	
	\vspace{1\baselineskip} % Whitespace below the editor list
	
    %------------------------------------------------
	%	Cover photo
	%------------------------------------------------
	
	\includegraphics[scale=1]{cover}
	
	%------------------------------------------------
	%	Publisher
	%------------------------------------------------
		
	\vspace*{\fill}% Whitespace under the publisher logo
	
	1$^{st}$ Edition, T"ubingen 1880 % Publication year
	
	{\small Verlag der H. Laupp'schen Buchhandlung } % Publisher

	\vspace{1\baselineskip} % Whitespace under the publisher logo

    Internet Archive Online Edition  % Publication year
	
	{\small Namensnennung Nicht-kommerziell Weitergabe unter gleichen Bedingungen 4.0 International } % Publisher
\end{titlepage}
\setlength{\parskip}{1mm plus1mm minus1mm}
\setcounter{tocdepth}{2}
\setcounter{secnumdepth}{3}
\tableofcontents
\clearpage
\listoffigures
\clearpage
\section{Einleitung}
\subsection{Einleitung}
\paragraph{}
Nicht die zum Teil wenig sachlichen Angriffe auf meine \emph{Urzelle} waren es, welche mich in meinen Anstrengungen, gewisse neue geologische Tatsachen festzustellen --- nicht erm"uden lie"sen: es war die durch Beobachtungen gewonnene "Uberzeugung von der Unhaltbarkeit der bisherigen Anschauung in dem unstreitig wichtigsten Teile der geologischen Wissenschaft, in dem Teile, durch welchen er gerade mit dem Kosmos zusammenh"angt --- in der Lehre von den sogenannten plutonischen Gesteinen.

Hatte ich es im ersten Teile meiner \emph{Urzelle} noch mit Ergebung hingenommen, dass der Erdkern und damit auch die Erkenntnis der wirklichen Entstehungs-Geschichte unserer Erde uns stets verborgen bleiben werde: so bot sich doch am Schluss dieses Buchs schon ein Ausblick: der Meteorstein zeigte die ferne Durchfahrt, welche noch von keinem Forscher gewagt worden war.

Mit diesem F"uhrer nun entschloss ich mich vorw"arts zu schreiten.

Ich tat es, begleitet auf der einen Seite von dem bald leiser bald sch"arfer ausgesprochenen Spotte der Fachm"anner, auf der andern Seite aufgemuntert durch die fr"uher und nun t"aglich neu gewonnenen Ergebnisse und unterst"utzt von dem Rat weniger Freunde, welche zu "uberzeugen mir gelungen war.

Das was mir meine bei einem anstrengenden Beruf fast "uber Menschenkraft gehenden Arbeiten des letzten Jahres an Ergebnissen geliefert haben, ist in den folgenden Bl"attern niedergelegt.

Es ist die Tierwelt in einem Gesteine, welches auf unsere Erde herabfiel und uns Kunde brachte von kleinsten Wesen aus fernsten R"aumen --- eine Tierwelt, welche zu erblicken ein sterbliches Auge kaum hoffen konnte: eine Welt von Wesen, welche uns zeigt, dass dieselbe Sch"opferkraft, welche unsere Erde aus einem Dunstnebel hat werden lassen, "uberall und gleichm"a"sig im Weltraum gewirkt und geschafft hat.

Freilich weist das Gestein der Meteorite\index{Meteorite} und zwar der Chondrite --- denn diese sind's, welche ich vorzugsweise zum Gegenstand meiner Untersuchung machte, --- keine Tiere h"oheren Baus auf; Alles sind niedere Tiere --- dieselben, welche in unseren Silurschichten vorherrschen --- Schw"amme, Korallen und Crinoiden\index{Crinoid}, und auch in ihren Spezies-Merkmalen stimmen sie mit dieser Sch"opfung.

Das Gestein der Chondrite, welche ich untersucht habe, ist ein Olivin-Enstatit-Gestein. Es hat von der Zeit seiner Entstehung, vom Tierknochen, bis es fiel, Verwandlungen durchgemacht, aber keine erhebliche: es ist nur von einer Silicatl"osung durchtr"ankt worden, wie alle unsere Jurameer-Ablagerungen von einer L"osung von Kalk. Wahrscheinlich machte es, so lange es noch ein Teil eines Planeten war, noch mehr Planeten-Perioden durch, wie auch den tieferen Schichten unserer Erde andere gefolgt sind, unter deren Einfluss dann die fr"uheren eine gewisse, freilich nicht so erhebliche Umwandlung als man gew"ohnlich annimmt, erfahren haben.

Wesentlich ge"andert hat sich nur die Oberfl"ache des Gesteins und zwar im letzten Augenblick seines planetarischen Lebens durch den Einfluss der Reibungsw"arme, entstanden im Falle durch die Erdatmosph"are. Doch das Bild des urspr"unglichen Gesteins ist im Wesentlichen geblieben. Wir sehen nun vor uns ein St"uck Planeten wie er im Werden war, und damit ist uns die Geschichte unseres Erdk"orpers aufgeschlossen, sofern wir ein Recht haben von der Bildung eines seiner Bewegung nach gleichartigen, in seiner chemischen Zusammensetzung gleichen Weltk"orpers auf die gleiche Bildung der Erde zu schlie"sen und umgekehrt.

Gleichzeitig war mir durch die Zusendung des "`Meteorite von Ovifak"' (ich verdanke ihn der G"ute des Herrn Professors Dr. von Nordenskj"old) Gelegenheit geboten, dieses Gestein in die Untersuchung hereinzuziehen.

Ich halte es f"ur irdisch --- halte es f"ur die tiefste Schichte unserer Erde, der Olivinschichte, die unter dem Granit lagert, angeh"orend. Ich nenne die urspr"ungliche Schichte Olivin-Formation. Da das Gestein einem Meteorite\index{Meteorite} sehr "ahnlich ist, lag es nahe, dasselbe f"ur einen solchen zu erkl"aren. Die Gr"unde, warum ich es nicht f"ur meteoritisch, sondern f"ur den wahren Erdkern halte, sind in diesem Buche niedergelegt.

So haben wir zwei feste Punkte gewonnen, von welchen aus ein Hebel angesetzt werden kann.

Die Chondrite, ein Olivin-Feldspat-(Enstatit-)Gestein besteht aus einer Tierwelt, sie sind nicht ein Lager, nicht ein Konglomerat, sondern ein Filz von Tieren, ein Gewebe, dessen Maschen alle lebendige Wesen waren, und zwar Tiere der niedersten Art, Anf"ange einer Sch"opfung.

Ich konnte nun allerdings von dieser Tierwelt, welche uns in den Meteoriten erhalten ist, keine systematische Aufz"ahlung machen: ich wollte nur nachweisen, dass sie ist --- da ist. Ich bildete daher nur ganz unzweifelhaft organische Wesen ab, wobei ich mich damit begn"ugen musste, einerseits die Gattungen festzustellen, welche mit unseren terrestrischen Formen "ubereinstimmen, andererseits die spezifisch meteoritischen Formen auszusondern und beides k"unftiger Untersuchung in die Hand zu geben.

Es ist zu erwarten, dass meine Aufz"ahlung durch weitere Forschung mit Hilfe eines reicheren Materials, als mir zu Gebot stand, bald sich vermehren und erg"anzen werde. Es mussten daher insbesondere Untereinteilungen unterbleiben: jedes neu gefundene Wesen w"urde die Einteilung umgesto"sen und damit die m"uhevolle, voreilige Arbeit auch zur vergeblichen gemacht haben.

Dies war der Grund warum ich nur die gro"sen Abteilungen und diese nur insoweit gemacht habe, als dies zum Verst"andnis der Formen beitr"agt: ersch"opfend und abgeschlossen soll, das wiederhole ich, die Arbeit in dieser Richtung nicht sein.

Auch in anderer Richtung muss ich Nachsicht in Anspruch nehmen: in der Abgrenzung der Hauptabteilungen selbst.

Wer meine Formen nur oberfl"achlich "uberblickt, wird bald finden, dass sie eine wirkliche Entwicklungsgeschichte an die Hand geben. Alle die "Uberg"ange vom Schwamm zur Koralle, von der Koralle zum Crinoiden\index{Crinoid} sind da, so dass es wirklich zweifelhaft werden kann, will man nicht eine neue Tiergattung machen, wohin man diese "Uberg"ange stellen soll.

In solchen Anf"angen sind Irrt"umer unvermeidlich, es ist daher nur eine Forderung der Billigkeit, sie zu verzeihen. Auch wollte ich die Ver"offentlichung des Werkes nicht zu lange verz"ogern, und habe es daher eben so wie es jetzt vorliegt, abgeschlossen.
\clearpage
\subsection{Geschichte und "Uberblick}
$\Delta$o$\sigma$ $\mu$o$\iota$ $\chi\epsilon\nu\tau\rho$o$\nu$%Δός μοι χέντρον
\paragraph{}
Als ich im vorigen Jahre mein Tagebuch enthaltend gewisse neue Beobachtungen "uber die Zusammensetzung der Gesteine unserer Erde und schlie"slich auch der Meteorite\index{Meteorite}, niederschrieb, war mir die Wichtigkeit der letzteren f"ur unsere Erdkunde noch nicht v"ollig klar.

Erst als ich durch die Angriffe der Gegner gezwungen war, die Untersuchung aufs Neue in die Hand zu nehmen, trat es mir klar vor die Augen, welch' hohe Bedeutung eine sorgf"altige Erforschung der Meteorite\index{Meteorite} f"ur die Geschichte unserer Erde haben m"usse. Zuletzt kam ich zu der "Uberzeugung, dass bei dem jetzigen Stand unserer Erdkunde die Meteorite\index{Meteorite} und nur die Meteorite\index{Meteorite} den Punkt abg"aben, von welchem aus unsere Erdgeschichte wenigstens mit ziemlicher Sicherheit erforscht werden k"onne.

Wenn ich also in meiner \emph{Urzelle} mit dem Granit die m"ogliche Grenze der Forschung erreicht zu haben glaubte, so wurde ich bald eines Bessern belehrt. Ich erwog, dass unser Erdkern verm"oge seines spezifischen Gewichts ebenfalls mindestens aus gediegenem Eisen bestehen m"usse, erwog ferner die sehr wahrscheinliche Reihenfolge in den Meteorite\index{Meteorite}n, welche vom reinen Eisen bis zu den Feldspatgesteinen unserer Erde geht. Ich glaubte ferner, dass ein R"uckschluss von unserer Erde auf die Meteorite\index{Meteorite} gewagt werden d"urfe, der Schluss, dass auch in den "ubrigen Planeten und in denjenigen oder demjenigen, deren (oder dessen) Tr"ummer wir wohl in den hunderttausend von kreisenden Meteoriten vor uns haben, eine Reihenfolge der Schichtung vom Schweren zum Leichten bestanden habe, eine Schicht-Folge, welche wir wahrscheinlich in der Reihe vom reinen Eisen durch die Halbeisen (Pallasite, Hainholz) hindurch zu den Chondriten und Eukriten, dann zu den Ton-(Kohle-)Meteoriten (Bokkefeld) vor uns haben.

Nachdem diese Wahrscheinlichkeit einmal gewonnen war, lag es nahe, die Meteorite\index{Meteorite} einer genauen Pr"ufung hinsichtlich ihrer morphologischen Eigenschaften zu unterwerfen. Dies war auch in hohem Grade geboten, denn dass bisher in dieser Richtung so gut wie nichts geschehen ist, davon kann man sich durch Vergleichung meiner Abbildungen mit den etwa zwanzig d"urftigen Bildern "uberzeugen, welche zusammen das heute vorliegende Material unserer Wissenschaft bilden. Die akademischen Schriften von Berlin, Wien, M"unchen haben je nur einige Tafeln aufzuweisen, die Zeichnungen sind klein, und wie sich sofort zeigt, von den am wenigsten f"ur diese Richtung der Untersuchung geeigneten Meteoriten und ferner wahrscheinlich auch nicht von dem besten Teile, dem Innern, genommen.

Sollte also auch, dachte ich, meine fr"uhere Behauptung: der Meteorstein von Knyahinya\index{Meteorite!Knyahinya} bestehe durchaus aus Pflanzen, durch meine neuen Untersuchungen sich nicht best"atigen, so w"are der Wissenschaft doch ein Dienst getan, wenn ich nur die wahre Form dieses Gesteins zur Darstellung bringen w"urde. Doch dieser R"uckzug blieb mir gl"ucklicherweise erspart, im Gegenteil: das Ergebnis der neuen Forschung war ein alle Erwartung weit "ubersteigendes --- eine neue Welt tat sich auf.

Aber freilich --- unsere Wissenschaft ist ungl"aubig --- sie fordert mit Recht strengere Beweise, als ich in meiner \emph{Urzelle} geboten habe; ein Buch, das fast mehr im Stadium, ich m"ochte sagen, der Intuition geschrieben ist. --- Heute lege ich Beweise vor.

Man "uberblicke die Tafeln dieses Werks und es wird sofort zur Gewissheit, dass es sich hier nicht um Mineral-, sondern um organische Formen handelt, dass wir die Bilder von Tieren vor uns haben, Bilder von Tieren der niedersten Stufe, einer Sch"opfung, welche zum gr"o"seren Teile wenigstens ihre n"achsten Verwandten auf unserer Erde findet; --- hinsichtlich der Korallen und Crinoiden\index{Crinoid} ist dies mit unbedingter Sicherheit festgestellt: die Schw"amme aber haben wenigstens eine solche "ahnlichkeit mit den Formen der Erde aufzuweisen, wie sie eben innerhalb verwandter irdischer Gattungen besteht.

So war die Entstehung hinsichtlich der Teile festgestellt. Nun best"atigte sich aber auch bei meiner Untersuchung von 20 Chondriten (und 360 D"unnschliffen davon) die in meiner \emph{Urzelle} aufgestellte Behauptung, dass das Gestein der Chondrite nicht etwa nach Art der Sedimentgesteine unserer Erde nur ein Schlamm sei, in welchen die Versteinerungen eingelagert sind, dass es nicht eine Konglomeratbildung sei; ihre ganze Masse ist vielmehr v"ollig aus organischen Wesen gebildet, wie unsere Korallenfelsen. Also keine Pflanze, wie ich fr"uher annahm, aber Pflanzentiere! Und der ganze Stein ein Leben: --- ich denke, die Wissenschaft darf mir den ersten Irrtum gerne verzeihen.

Selbstredend war nun auch das Meteoreisen nochmals einer Pr"ufung zu unterwerfen. Hier blieb es bei meiner ersten Beobachtung.

Allerdings gestatteten mir Zeit und Umst"ande, insbesondere der Mangel an verf"ugbarem Material nicht, die Untersuchung dar"uber vor dieser Ver"offentlichung abzuschlie"sen. Wenn ich aber heute die erste Behauptung, dass das Meteoreisen nichts als ein Pflanzenfilz sei, in der Hauptsache wiederhole, so darf ich mich doch jetzt zu der Behauptung eher legitimiert ansehen, als zur Zeit, als ich die \emph{Urzelle} schrieb. Beizuf"ugen habe ich, dass ich auch im Eisen Tierformen fand. Die Forscher, denen die Formen der Chondrite entgingen, welche ich abbilde, k"onnen auch "ubersehen haben, dass die sogenannten Widmannst"atten'schen Figuren in der Tat gr"o"stenteils Pflanzenzellen und keine Kristalle sind.

Die bisherigen Untersuchungen auf dem ganzen Gebiete mit Ausnahme der Arbeit [Karl Wilhelm von] G"umbel's in den Schriften der M"unchener Akademie sind, sowohl was Genauigkeit der Beobachtung, noch mehr aber was die auf solcher Beobachtung, auf unbewiesenen Hypothesen und leeren Voraussetzungen ruhende Deutung betrifft --- wenig geeignet, als eine wissenschaftliche Feststellung angesehen zu werden. So war mir in der Tat das Feld noch v"ollig offen, wobei ich nur bedaure, dass ich bez"uglich der Eisen vorerst noch keine Vorlage machen kann.

Ich komme nun zur Schlussfolgerung f"ur unsere Erdkunde. Sind n"amlich die Chondrite --- also ein Olivin- und Enstatit-Gestein wirklich, was ich zur Gewissheit bringe, nur St"ucke von Schwamm-Korallen-Crinoiden-Felsen, so ist f"ur die Wissenschaft unserer Erde eine Tatsache von unermesslicher Tragweite gewonnen.

Ein Feldspatmineral ist reines Wasserprodukt, ist Versteinerungsmittel f"ur Millionen von Organismen! Damit fallen alle Hypothesen "uber die metamorphischen und plutonischen Gesteine unserer Erde, damit f"allt die Theorie von dem feuerfl"ussigen Erdinnern, --- wenigstens kann aus dem Gestein kein Schluss mehr darauf gezogen werden.

Ich muss dies noch n"aher begr"unden. Die Vergleichung der Gesteine der Erde und der Meteorite\index{Meteorite} zeigt, dass der Chondrit, wenigstens nach seiner chemischen Beschaffenheit, seine allern"achsten Verwandten auf der Erde hat.

Das Olivingestein unserer Erde ist als Llerzolith ein Lagergestein, als Basalt sehen wir es den Granit durchbrechen; ich traf hier mit den Ergebnissen, welche [Gabriel Auguste] Daubrée gewonnen hatte, zusammen.

Der tieferliegende Granit ist also jedenfalls j"unger als der Olivin. Ist aber das Olivingestein der Meteorite\index{Meteorite} verm"oge seiner Zusammensetzung ein Wassergestein, so wird es wohl der Granit unserer Erde auch sein; besteht das Olivingestein der Meteore aus niederen Tieren, so wird dasselbe bei dem Olivingestein der Erde der Fall: es wird wohl der Schluss uns erlaubt sein, dass auch dieses Gestein unserer Erde auf seiner urspr"unglichen Lagerst"atte aus denselben Tieren zusammengesetzt ist, wie der Chondrit. --- Und aus demselben Grunde wird auch der Granit, als j"ungeres Gestein, wohl denselben Ursprung haben. Haben wir in unserem (schw"abischen) Basalt nur Auslaugungen aus dem urspr"unglichen Olivingestein zu erblicken, so ist doch die Lagerung des Llerzoliths unter dem Granit festgestellt. Und erscheint auch dieses Gestein als eine Wasserablagerung ohne unterscheidbare Formen, so hat doch das Eisen von Ovifak solche; dieses aber ist so sehr mit dem Basalt, so innig und nicht blo"s mechanisch verbunden, dass beide als ein Gestein angesehen werden m"ussen. Dieses ist also das urspr"ungliche Olivin-Lagergestein. Damit aber ist der Annahme einer Entstehung der Erde auf feurigem Wege der wissenschaftliche Grund entzogen.

Bestand die Oberfl"ache der Planeten oder des Planeten in den Schichten des Olivins aus Tieren, so ist dieselbe Schichte unserer Erde wohl auch nicht durch Feuer entstanden: wenigstens ist nicht der mindeste Grund zu dieser Vermutung mehr vorhanden, im Gegenteil, es ist anzunehmen, dass auch dieselbe Schichte der Erde eine Wasserbildung gewesen sei. --- Hier traf ich nun auf die Kant-Laplace'sche Theorie.

Ich kann mir die Stoffe der Planeten (einschlie"slich des Wassers, welches gew"ohnlich vergessen wird!) zur Zeit der ersten Massenbildung, wie [Immanuel] Kant und [Pierre-Simon] Laplace nur in Dunstform, aber freilich nicht als einen gl"uhenden Dunst denken, sondern nur als Dunst- und Gasmasse im kalten Weltraum. Hier hat man aber den gro"sen logischen Fehler in der genannten Theorie "ubersehen.

Die Massenanziehung sollte die Masse bilden! Die Wirkung sollte zugleich Ursache sein! Die Masse n"amlich sollte sich durch Masseanziehung bilden, also dadurch entstehen, dass sie schon da war! Es ist zu bedauern, dass man diesen Denkfehler nicht fr"uher entdeckt hat. Die Masse kann, wenn sie da ist, sich durch Anziehung vergr"o"sern, aber nicht dadurch werden: es ist als ob Jemand sein eigne Vater sein sollte!

Also eine andere Kraft musste die Masse bilden: diese aber konnte nur entweder die Kristallisations-Kraft oder die organische Bildungskraft sein.

Erstere reicht zur Erkl"arung der Planetenbildung nicht hin, und es finden sich keine Kristalle: folglich bleibt blo"s die zweite Kraft "ubrig --- die organische. Hier erinnere ich an meine Beobachtungen der Struktur des Meteoreisens und so steht heute, f"ur mich wenigstens, die Tatsache fest, dass der erste Anfang unserer Erde, wie der "ubrigen Planeten, eine organische Ursache hatte.

Erscheint der Satz auch etwas bet"aubend, so braucht man nur zu ganz Bekanntem zu greifen.

Erstens: Die Masse der Baustoffe, welche im Anfang der Planetenbildung zu Gebot stand, reicht vollst"andig hin, um die Bildung auch einer Planeten-Masse auf organischem Wege zu erkl"aren.

Zweitens lehrt die Erfahrung von heute, in welch' kurzer Zeit sich die niedersten Pflanzen und Tiere vermehren, dass ihre Zahl, also auch ihre Masse, lediglich durch die Masse der Baustoffe bedingt ist, w"ahrend ihre Organisation selbst eine Ausdehnung ins Unendliche (so lange n"amlich Baustoffe da sind) m"oglich macht.

Was dieser Erkl"arung entgegen zu stehen scheint, ist nur die Erdw"arme und die damit in Verbindung gebrachte Erscheinung der heute noch t"atigen Vulkane. Allein bez"uglich dieser beiden Tatsachen ist man l"angst auf eine andere Erkl"arung, als auf ein feuerfl"ussiges Erdinneres, zur"uckgef"uhrt. Das Wasser wirkt auf Feldspat zersetzend ein. Bei diesem Zersetzungsprozess wird W"arme frei. Die Vulkane folgen dem Meere, weil das Wasser die Gase bilden hilft, welche, von oben entz"undet, das anstehende Gestein und auch nur dieses schmelzen.

Wie sollte endlich ein feuriger Erdkern ohne Sauerstoff bestehen k"onnen! Und f"uhrt nicht eben auch das Dasein brennbarer Gase (denn solche sind die Ursachen der vulkanischen Erscheinungen,) insbesondere das der Schwefelgase auf organische im Erdinnern vorhandene Stoffe zur"uck? Hier bedarf es wahrhaft keiner neuen Beweise, sondern nur des Aufgebens gewisser Vorstellungen, welche sich der aus einigen augenf"alligen Erscheinungen erregten Phantasie bem"achtigt haben.

Dies sind die Schlussfolgerungen aus der Untersuchung "uber die Meteorite\index{Meteorite} f"ur unsere Erdbildung. Ungleich bedeutender aber sind die Tatsachen, welche die Astronomie daraus ableiten kann.

Die D"unnschliffe von 20 von mir untersuchten Meteoriten (Chondriten), von F"allen, welche "uber ein Jahrhundert auseinander liegen, zeigen dieselben Formen, "ahnlich wie eine Leitmuschel "uberall in derselben Formation vorkommt; dies hat schon G"umbel, wenn er die Formen der Chondrite auch nicht richtig gedeutet hat, trefflich ausgesprochen.

Diese Chondrite stammen also wahrscheinlich von einem und demselben Weltk"orper, einem Planeten. Oder ist gar bei verschiedenen Planeten die Entwicklung eine so sehr "ubereinstimmende gewesen?

Dieser Weltk"orper tr"agt Wassertiere, ist also im Wasser und durch Wasser entstanden, auch nicht durch Feuer vergangen, denn Spuren des Feuers zeigen diese Gesteine nicht: der Meteorit ist zersprungen, seine Tr"ummer haben nur in ihrem kurzen Weg durch unsere Atmosph"are eine 1 mm dicke Schmelzrinde, in Folge der Reibungsw"arme, erhalten.

Die Tier-Sch"opfung der Chondrite ist beinahe durchaus eine mikroskopische, Tiere sind es von 0,20 bis h"ochstens 3 mm Durchmesser, oft bedarf es einer Vergr"o"serung von 1000, um ihre zarte Struktur klar zu sehen, w"ahrend bei solcher Vergr"o"serung unsere Petrefacten in eine gestaltlose Fl"ache sich aufl"osen.

So war mir durch die erste in meiner \emph{Urzelle} niedergelegte Beobachtung ein Weg ge"offnet, auf welchem weite, weite Strecken unserer Wissenschaft gewonnen werden m"ussen.

Es bedurfte aber wahrlich gerade keiner Titanenkraft mehr, um das alte Geb"aude umzust"urzen, es war schon viel vorgearbeitet, nur nicht beachtet: es bedarf nur eines einzigen durchschlagenden Beweises und die Arbeit ist getan. "Uberlieferungen, auf ungen"ugende Beobachtungen gest"utzt, l"osen sich in das auf, was sie sind, und nun hat die Wissenschaft wieder freie Bahn.
\clearpage
\subsection{Die Bisherigen Ansichten "uber die Meteorite\index{Meteorite}}
\paragraph{}
Es folgt nun zun"achst eine kurze Darstellung der bisherigen Ansichten "uber die Entstehung und Natur der Meteorite\index{Meteorite}.

Nur die morphologischen Arbeiten "uber einzelne Meteorite\index{Meteorite}, von der Zeit an, als man das Mikroskop in der Geologie anzuwenden begann, sollen aufgez"ahlt werden.

Was das Mikroskop bis jetzt zur Deutung der Meteorite\index{Meteorite} geliefert hat, das ist, abgesehen von den vergr"o"serten Olivinkristallen in [Nikolai Ivanovich] Kokscharow's \emph{Mineralien Russlands VI} Band S. 4, in folgenden Schriften enthalten:
\begin{enumerate}
\item Gustav Tschermak: "`die Tr"ummerstruktur der Meteoriten von Orvinio\index{meteorite!Orvinio} und Chantonnay"', vorgelegt in der Sitzung der K. Akademie der Wissenschaften (Wien) am 12. November 1874. (XX. Band der Sitzungsberichte der K. Akademie der Wissenschaften, I. Abteilung, November-Heft 1874. Mit 2 Tafeln.)
\item Alexander Makowsky\index{Makowsky} und G. Tschermak: "`Bericht "uber den Meteoritenfall bei Tieschitz in M"ahren"'. Mit 5 Tafeln und 2 Holzschnitten, vorgelegt in der Sitzung der mathematisch-naturwissenschaftlichen Klasse (der Kgl. Akademie der Wissenschaften in Wien) am 21. November 1878. XXIX. Band der Denkschriften der genannten Klasse.
\item Johann Gottfried Galle und [Arnold Constantin Peter Franz] von Lasaulx\index{Lasaulx}, vorgelegt von [Christian Friedrich Martin] Websky: "`Bericht "uber den Meteorsteinfall bei Gnadenfrei am 17. Mai 1879"'. Sitzung vom 31. Juli 1879. Monatsberichte der K. preu"sischen Akademie zu Berlin.
\end{enumerate}
\paragraph{}
Die fr"uheren Beschreibungen beschr"anken sich auf die Untersuchung mit blo"sem Auge und der Lupe, sowie die chemische Analyse.

Sie stimmen alle dahin "uberein: Die Chondrite bestehen aus einer Grundmasse mit Kugeln von Enstatit (Bronzit), Olivin und Eisen, eingesprengtem Nickel- und Chromeisen.

Eine andere Stellung nimmt ein: G"umbel: "Uber die in Bayern gefundenen Steinmeteoriten; Sitzungsberichte der mathematisch-physikalischen Klasse der K. b. Akademie der Wissenschaften zu M"unchen 1878. Heft 1, S. 14 ff. In der Beschreibung der Meteorite\index{Meteorite} von Eichst"adt und Sch"oneberg erw"ahnte er "`Maschenstruktur"' (S. 27. 46.) Allerdings spricht er auch von "`Abk"ommlingen zerbrochener gr"o"serer Chondren"' (S. 28). Das Bedeutende seiner Beobachtungen ist auf S. 58, welche ich hier folgen lasse:

"`"Uberblickt man die Resultate der Untersuchung dieser wenn auch beschr"ankten Gruppe von Steinmeteoriten, so dr"angt sich die Wahrnehmung in den Vordergrund, dass sie, trotz einiger Verschiedenheit in der Natur ihrer Gemengteile, doch von vollst"andig gleichen Strukturverh"altnissen beherrscht sind. Alle sind unzweifelhafte Tr"ummergesteine, zusammengesetzt aus kleinen und gr"o"seren Mineralsplitterchen, aus den bekannten rundlichen Chondren, welche meist vollst"andig erhalten, aber oft auch in St"ucke zersprungen vorkommen und aus Gr"aupchen von metallischen Substanzen Meteoreisen, Schwefeleisen, Chromeisen. Alle diese Fragmente sind aneinander geklebt, nicht durch eine Zwischensubstanz oder durch ein Bindemittel verkittet, wie sich "uberhaupt keine amorphen, glas- oder lavaartigen Beimengungen vorfinden. Nur die Schmelzrinde und die oft auf Kl"uften auftretenden, der Schmelzrinde "ahnlich entstandenen schwarzen "Uberrindungen bestehen aus amorpher Glasmasse, die aber erst beim Niederfallen innerhalb unserer Atmosph"are nachtr"aglich entstanden ist. In dieser Schmelzrinde sind die schwerer schmelzbaren und gr"o"seren Mineralk"ornchen meist noch ungeschmolzen eingebettet. Die Mineralsplitterchen tragen durchaus keine Spuren einer Abrundung oder Abrollung an sich, sie sind scharfkantig und spitzeckig. Was die Chondren anbelangt, so ist ihre Oberfl"ache nie gegl"attet, wie sie sein m"usste, wenn die K"ugelchen das Produkt einer Abrollung w"aren, sie ist vielmehr stets h"ockerig uneben, maulbeerartig rau und warzig oder facettenartig mit einem Ansatz von Kristallfl"achen versehen. Viele derselben sind l"anglich, mit einer deutlichen Verj"ungung oder Zuspitzung nach einer Richtung, wie es bei Hagelk"ornern vorkommt. Oft begegnet man St"uckchen, welche offenbar als Teile zertr"ummerter oder zersprungener Chondren gelten m"ussen. Als Ausnahme kommen zwillingsartig verbundene K"ugelchen vor, h"aufiger solche, in welchen Meteoreisenst"uckchen ein- oder angewachsen sind. Nach zahlreichen D"unnschliffen sind sie verschiedenartig zusammengesetzt. Am h"aufigsten findet sich eine exzentrisch strahlig faserige Struktur in der Art, dass von einer weit aus der Mitte nach dem sich verj"ungenden oder etwas zugespitzten Teil hin verr"uckten Punkte aus ein Strahlenb"uschel gegen Au"sen sich verbreitet. Da die in den verschiedensten Richtungen gef"uhrten Schnitte immer s"aulen- oder nadelf"ormige, nie bl"atter- oder lamellenartige Anordnung in der diesen B"uschel bildenden Substanz erkennen lassen, so scheinen es in der Tat s"aulenf"ormige Fasern zu sein, aus welchen sich solche Chondren aufbauen. Bei gewissen Schnitten gewahrt man, dieser Annahme entsprechend, in den senkrecht zur L"angenrichtung gehenden Querschnitten der Fasern nur unregelm"a"sig eckige, kleinste Feldchen, als ob das Ganze aus lauter kleinen polyedrischen K"ornchen zusammengesetzt sei. Zuweilen sieht es aus, als ob in einem K"ugelchen gleichsam mehrere nach verschiedener Richtung hin strahlende Systeme vorhanden w"aren oder als ob gleichsam der Ausstrahlungspunkt sich w"ahrend ihrer Bildung ge"andert habe, wodurch bei Durchschnitten nach gewissen Richtungen eine scheinbar wirre, st"angliche Struktur zum Vorschein kommt. Gegen die Au"senseite hin, gegen welche der Viereinigungspunkt des Strahlenb"uschels einseitig verschoben ist, zeigt sich die Faserstruktur meist undeutlich oder durch eine mehr k"ornige Aggregatbildung ersetzt. Bei keinem der zahlreichen angeschliffenen Chondren konnte ich beobachten, dass die B"uschel so unmittelbar bis zum Rande verlaufen, als ob der Ausstrahlungspunkt gleichsam au"serhalb des K"ugelchens l"age, sofern nur dasselbe vollst"andig erhalten und nicht etwa ein blo"ses zersprungenes St"uck vorhanden war. Die zierlich quergegliederten F"aserchen verlaufen meist nicht nach der ganzen L"ange des B"uschels in gleicher Weise, sondern sie spitzen sich allm"ahlich zu, ver"asteln sich oder endigen, um andere an ihre Stelle treten zu lassen, so dass in dem Querschnitte eine mannichfache, maschenartige oder netzf"ormige Zeichnung entsteht. Diese F"aserchen bestehen, wie dies schon vielfach im Vorausgehenden geschildert wurde, aus einem meist helleren Kern und einer dunkleren Umh"ullung, jener durch S"auren mehr oder weniger zerlegbar, letztere dagegen dieser Einwirkung widerstehend. H"ochst merkw"urdig sind die schalenf"ormigen "Uberrindungen, welche aus Meteoreisen zu bestehen scheinen und in der Regel nur "uber einen kleineren Teil der K"ugelchen sich ausbreiten. Die gleichen einseitigen, im Durchschnitt mithin als bogenf"ormig gekr"ummte Streifchen sichtbaren "Uberrindungen kommen auch im Innern der Chondren vor und liefern einen starken Gegenbeweis gegen die Annahme, dass die Chondren durch Abrollung irgend eines Materials entstanden seien, wie denn "uberhaupt die ganze Anordnung der b"uscheligen Struktur mit Entschiedenheit gegen ihre Entstehung durch Abrollung spricht. Doch nicht alle Chondren sind exzentrisch faserig; viele, namentlich die kleineren besitzen eine feink"ornige Zusammensetzung, als best"anden sie aus einer zusammengeballten Staubmasse. Auch hierbei macht sich zuweilen die einseitige Ausbildung der K"ugelchen durch eine exzentrisch gr"o"sere Verdichtung der Staubteile bemerkbar"'.

Und ferner S. 61:

"`Der gew"ohnliche Typus der Meteorite\index{Meteorite} von steiniger Beschaffenheit ist soweit "uberwiegend derjenige der sog. Chondrite und die Zusammensetzung sowie die Struktur aller dieser Steine so sehr "ubereinstimmend, dass wir den gemeinsamen Ursprung und die uranf"angliche Zusammengeh"origkeit aller dieser Art Meteorite\index{Meteorite} --- wenn nicht aller --- wohl nicht weiter in Zweifel ziehen k"onnen.

"`Der Umstand, dass sie s"amtlich in h"ochst unregelm"a"sig geformten St"uckchen in unsere Atmosph"are gelangen --- abgesehen von dem Zerspringen innerhalb der letzteren in mehrere Fragmente, was zwar h"aufig vorkommt, aber doch nicht in allen F"allen angenommen werden kann, namentlich nicht, wenn durch direkte Beobachtung das Fallen nur eines St"uckes konstatiert ist, --- l"asst weiter schlie"sen, dass sie bereits in regellos zertr"ummerten St"ucken als Abk"ommlinge von einem einzigen gr"o"seren Himmelsk"orper ihre Bahnen im Himmelsraume ziehen und in ihrer Zerstreutheit einzeln zuweilen in das Attraktionsbereich der Erde geraten zur Erde niederfallen. Der Mangel urspr"unglicher, lavaartiger, amorpher Bestandteile in Verbindung mit der "au"sern unregelm"a"sigen Form d"urfte von geo- oder kosmologischen Standpunkte aus die Annahme ausschlie"sen, dass diese Meteorite\index{Meteorite} Ausw"urfe von Mondvulkanen, wie vielfach behauptet wird, sein k"onnen."'

G"umbel fasst, nachdem er die Meteorite\index{Meteorite} in die Olivingesteine unserer Erde eingestellt hat, seine Ansicht hinsichtlich der Entstehung (S. 64) in den Satz zusammen:

"`Es scheinen daher die Meteorite\index{Meteorite} aus einer Art erstem Verschlackungsprozess der Himmelsk"orper, aber da sie metallisches Eisen enthalten --- bei Mangel von Sauerstoff und Wasser hervorgegangen zu sein."'

"`So geistreich, f"ahrt er (S. 68) fort, diese Hypothesen Daubrée's und Tschermak's sind (Entstehung aus zertr"ummertem Vulkangestein), so kann ich mich doch in Bezug auf die Entstehung der K"ugelchen (Chondren) ihrer Ansicht auf Grund meiner neuesten Untersuchungen nicht anschlie"sen. Ich habe im Gegensatze zu Tschermak's Annahme nachzuweisen gesucht, dass das innere Gef"uge der Chondren nicht au"ser Zusammenhang mit ihrer kugeligen Gestalt stehe, und dass man diese K"ugelchen weder als St"ucke eines Mineralkristalls, noch eines festen Gesteins ansehen k"onne. Spricht schon ihre nicht gegl"attete, nicht polierte Oberfl"ache, welche, wenn durch Abreibung oder Abrollung gebildet, bei solcher H"arte des Materials spiegelglatt sein m"usste, w"ahrend sie rauh, h"ockerig, oft strichweise kristallinisch facettirt erscheint, gegen die Abreibungstheorie, so ist auch gar kein Grund einzusehen, weshalb nicht alle anderen Mineral splitterchen wie Sandk"orner abgerundet seien und weshalb namentlich das Meteoreisen, das Schwefeleisen und das sehr harte Chromeisen, wie ich in dem Meteorit von L'Aigle mich "uberzeugt habe, stets nichtgerundete, oft "au"serst fein zerschlitzte Formen besitzen. Wie w"are es zudem denkbar, dass, wie h"aufig beobachtet wird, innerhalb der K"ugelchen konzentrische Anh"aufung von Meteoreisen vorkommen? Auch erscheint die exzentrisch faserige Struktur der meisten K"ugelchen in ihrem einseitig gelegenen Ausstrahlungspunkte in Bezug auf die Oberfl"ache nicht als zuf"allig, sondern der Art der Struktur der Hagelk"orner nachgebildet. Dieses innere Gef"uge steht im engsten Zusammenhang mit dem Akt ihrer Entstehung, welche nur als eine Verdichtung Mineral bildender Stoffe unter gleichzeitiger drehender Bewegung in D"ampfen, welche das Material zur Fortbildung lieferten, sich erkl"aren l"asst, wobei in der Richtung der Bewegung einseitig mehr Material sich ansetzte."'

Weiter freilich spricht G"umbel sich dahin aus, dass das Material, aus welchem die Chondrite bestehen, durch eine gest"orte Kristallisation und Zertr"ummerung in Folge von explosiven Vorg"angen innerhalb eines Raumes sich gebildet habe, welcher von den die Mineralien bildenden Stoffe liefernden Dampf- und Wasserstoffgasen erf"ullt war. Er schlie"st S. 72 bei Besprechung des Meteorites von Kaba:

"`Vielleicht gelingt es dennoch, die Anwesenheit organischer Wesen auf au"serirdischen K"orpern nachzuweisen."' Ich hoffe dies sei gelungen. --- Aus seinen Abbildungen ersieht man, dass bei der Untersuchung ein schlechtes Material zu Gebot stand. Auch h"atten immerhin mehr D"unnschliffe gefertigt werden m"ussen, zudem reicht die Vergr"o"serung bei Weitem nicht. Ich verweise hier auf das Folgende und die Beschreibung meiner Tafeln.

Was ich in dem Berichte G"umbels so hoch sch"atze, ist die gewissenhafte vorurteilsfreie, ich m"ochte sagen unparteiische Beobachtung. Ich habe mir erlaubt, die Schrift G"umbels w"ortlich anzuf"uhren, weil es mir in der Tat schwer wird, solche Darstellungen zusammenzufassen und Tatsachen und Deutung zu trennen.

Richtige Beobachtungen und unrichtige Erkl"arungen stehen so nahe beisammen, dass es unm"oglich ist beides zu sondern. Ich glaubte, als ich die G"umbel'sche Abhandlung (nach dem Abschluss meiner Untersuchungen und meines Manuskripts) durchlas, in jedem Augenblick auf meine Resultate zu treten. Aber wie die Woge der Brandung den, welcher das Land gewinnen will, jedes mal dann wieder ergreift und zur"uckwirft, wenn er schon das Land gefasst zu haben glaubt, so auch hier: allemal rei"st das alte Dogma den geehrten Forscher von der rettenden Klippe hinweg in den bodenlosen Strudel der Traditionen zur"uck.

Daubrée's verdienstvolles Werk \emph{Experimentalgeologie} erhielt ich erst in der "Ubersetzung zur Hand und ebenfalls nach Abschluss meiner Arbeit. Dass es diese widerlegte, wird wohl Niemand finden. Daubrée hat selbst Knyahinya\index{Meteorite!Knyahinya} abgebildet. M. hat gepresst, geschmolzen, aufgel"ost, berechnet, nur nicht --- gesehen.
\clearpage
\subsection{Die Meteorite\index{Meteorite} und ihre Mineralogischen Eigenschaften}
\paragraph{}
Die Literatur der Meteorite\index{Meteorite} ist eine sehr umfangreiche. Sie ist jedoch, was die Art und Zahl, chemische Zusammensetzung betrifft, so bekannt, dass ich auf diesen Teil derselben, also insbesondere die fr"uheren Arbeiten, nicht einzugehen brauche.

Die Meteorite\index{Meteorite} werden eingeteilt in Eisen und Steine, zwischen beiden steht jedoch noch eine Klasse: Halbeisen, d. h. eine Verbindung von gediegenem Eisen und Stein --- die Pallasite. W"ahrend die Eisen eine ziemliche "Ubereinstimmung, sowohl in ihrer chemischen Zusammensetzung, als in der Form ihrer Struktur zeigen, sind die Pallasite (je nach dem Vorwiegend des Eisens) sehr verschieden. Aber es finden sich noch weitere Verschiedenheiten darunter. Hainholz z. B. hat neben Eisen und Olivin ein blaues Mineral (Enstatit) und in diesem einen gro"sen Reichtum von Tierformen. --- Die Steine werden eingeteilt in Chondrite, Stannerite [eukriten], Luotolaxer [howarditen], Bokkefelder [karbonatisch], Bishopvillit [aubriten], (Quenstedt, Klar und Wahr S. 280 folg.)

Ich habe mich vorzugsweise mit den Chondriten besch"aftigt und, wo ich von Meteoriten rede, rede ich von dieser allerdings auch am zahlreichsten vertretenen Klasse von Stein-Meteoriten.
\paragraph{}
Ich habe untersucht:
\begin{center}
\begin{tabular}{ l r }
 Tabor, B"ohmen [Tschechische Republik] & July 3, 1753\index{Meteorite!Tabor} \\
 Siena, Toskana [Italienische Republik] & June 16, 1794\index{Meteorite!Siena} \\
 L'Aigle, Normandy [Französische Republik] & April 26, 1803\index{Meteorite!L'Aigle} \\
 Weston, Connecticut [Vereinigte Staaten] & December 14, 1807\index{Meteorite!Weston} \\
 Tipperary, Irland & November 23, 1810\index{Meteorite!Tipperary} \\
 Blansko, Br"unn [Tschechische Republik] & November 25, 1833\index{Meteorite!Blansko} \\
 Château-Renard, Loiret [Französische Republik] & July 12, 1841\index{Meteorite!Château-Renard} \\
 Linn [Marion] County, Iowa [Vereinigte Staaten] & February 25, 1847\index{Meteorite!Marion County}\index{Meteorite!Linn} \\
 Cabarras [Monroe] County, North Carolina [Vereinigte Staaten] & October 31, 1849\index{Meteorite!Monroe County}\index{Meteorite!Cabarras} \\
 Mez"o-Madaras [Romania] & September 4, 1852\index{Meteorite!Mez"o-Madaras} \\
 Borkut, Ungarn & October 13, 1852\index{Meteorite!Borkut} \\
 Bremerv"orde, Hannover [Deutschland] & May 13, 1855\index{Meteorite!Bremerv"orde} \\
 Parnallee, Osten Indien [Tamil Nadu] & February 28, 1857\index{Meteorite!Parnallee} \\
 Heredia, Kostarika & April 1, 1857\index{Meteorite!Heredia} \\
 New Concord, Ohio [Vereinigte Staaten] & May 1, 1860\index{Meteorite!New Concord} \\
 Knyahinya, Ungarn & June 9, 1866\index{Meteorite!Knyahinya} \\
 Pultusk, Warschau [Republik Polen] & January 30, 1868\index{Meteorite!Pultusk} \\
 Orvinio [Italienische Republik] & August 31, 1872\index{Meteorite!Olvinio} \\
 Simbirsk [Russland] & [1838]\index{Meteorite!Simbirsk} \\
\end{tabular}
\end{center}
\clearpage
\paragraph{}
Alle Gesteine sind durchaus beglaubigt. Ich habe hier vor Allem der Liberalit"at meines verehrten Lehrers, Herrn Professor Dr. [Friedrich August] von Quenstedt, mit welcher er mir die vorz"ugliche T"ubinger Universit"ats-Sammlung (welche bekanntlich zum gr"o"sten Teil vom Freiherrn [Karl Ludwig] von Reichenbach in Wien stammt) dankend zu gedenken.

Von Knyahinya besitze ich 360 D"unnschliffe, von L'Aigle 6, von Pultusk 6, von den "ubrigen 1-3. Ich werde s"amtliche Steine kurz nach dem Fallort benennen. Bei Herstellung der D"unnschliffe habe ich die Schnitte in 2 Richtungen genommen. Es ergab sich n"amlich nach mehreren Versuchen an Knyahinya, dass derselbe nach einer bestimmten Richtung bricht.

Es konnte dies aus den Einschl"ussen entnommen werden, welche, nachdem einmal die Stellung gefunden war, regelm"a"sig bestimmte Formen-Durchschnitte ergaben, welchen dann die Formen in einem senkrecht auf diese Stellung gefertigten Schnitte entsprachen.

Waren die Formen an diesem Steine gestellt, so w"are wohl dieselbe Stellung in den "ubrigen Steinen zu erhalten gewesen, vorausgesetzt nat"urlich, dass das Material zu Gebot gestanden h"atte. Bei einzelnen ergab sich dieselbe zuf"allig --- bei anderen nicht, es musste aber aus den angef"uhrten Gr"unden auf weitere Feststellung in dieser Richtung verzichtet werden.

Ich fertigte ferner die Schliffe absichtlich in dreierlei Dicke: schwer durchsichtig, um die ganzen Einschl"usse m"oglichst vollst"andig zu bekommen: sehr d"unn, um die Strukturverh"altnisse klar zu stellen; den gr"o"sten Teil aber so, dass beides noch zur Anschauung kam.

Ich reihe hier eine Bemerkung an, welche mir Jeder best"atigen wird, welcher sich mit D"unnschliffen von Petrefacten besch"aftigt hat.

Nur in seltenen F"allen ist in v"ollig durchsichtigen, also ganz d"unnen Schliffen, noch die Struktur sichtbar. Wer seinen Schliff, wenn er halbdurchsichtig, im Mikroskop betrachtet, ist im h"ochsten Grad erfreut "uber die sch"onen Formen und Linien. In der Freude dar"uber will er die Sache noch besser machen und erwartet bei fortgesetztem Schleifen ein vollendetes Bild. Aber wenn er den Schliff zum zweiten Mal unter das Mikroskop legt --- ist nichts mehr da als eine fast strukturlose Fl"ache, kaum angedeutete, sogar in den Umrissen verschwommene Formen, aus welchen nun das, was man vorher schon mit der Lupe wahrnahm, nicht einmal mehr mit dem Mikroskop zu ersehen ist. Diese Erscheinung h"angt aber mit der Art der Metamorphose des Gesteins und der darin eingeschlossenen Formen zusammen. Die Sache ist jedoch bekannt und bedarf deshalb keiner weiteren Ausf"uhrung. Ich musste der Tatsache nur deshalb erw"ahnen, damit solche, welche Beobachtungen erst anstellen wollen, ohne dass sie dieselbe kennen, nicht "uberrascht werden und ihre Beobachtungsweise verbessern k"onnen.

Dass die Chondrite zum gr"o"sten Teile aus Bronzit-Enstatit (Augit) und Olivin sowie Magnetkies bestehen, ist eine in der Wissenschaft angenommen Tatsache. Quenstedt, \emph{Handbuch der Mineralogie} S. 722.

Insbesondere aber sind die Einschl"usse, welche ich f"ur Korallen erkl"are, f"ur Enstatit angesprochen worden. Damit glaubte man die Struktur derselben erkl"aren zu k"onnen. Andere gingen noch weiter und erkl"arten die Einschl"usse zum Teil f"ur Gl"aser: (Tschermak).

Ehe ich also an die Begr"undung meiner Ansicht komme, muss die mikroskopische Erscheinung des haupts"achlich vorkommenden Minerals, des Enstatits, genau festgestellt werden.

Ich erlaube mir hier K"urze halber dasjenige anzuf"uhren, was [Karl Heinrich Ferdinand] Rosenbusch in seinem Buch: \emph{Mikroskopische Physiographie der petrographisch wichtigen Mineralien} Stuttgart 1873 S. 252, "uber Enstatit (und Bronzit) sagt:

"`Bekanntlich hat man seit den optischen Untersuchungen von [Alfred] Des Cloizeaux den Enstatit, Bronzit und Hypersthen als rhombisch kristallisierend vom Pyroxen getrennt und sie in eine eigene Gruppe zusammengestellt. Dieselben zeigen neben der Spaltung nach dem Prisma von 87$^{\circ}$ noch weitere Spaltungen nach den vertikalen Pinakoiden, "uber deren relative Vollkommenheit die Angaben der verschiedenen Forscher nicht genau "ubereinstimmen. Chemisch bilden diese 3 Mineralien eine ununterbrochene Reihe, an deren Anfange der fast eisenfreie Enstatit und an deren Ende der sehr eisenreiche Hypersthen steht. Enstatit und Bronzit sind sich "uberdies auch in allen physikalischen Eigenschaften so "ahnlich, dass eine Trennung derselben in zwei Spezies kaum durchzuf"uhren sein d"urfte. Der Hypersthen dagegen zeigt eine verschiedene optische Orientierung und mag daher immerhin eine eigene Spezies bilden. Interessant ist die von Tschermak gegebene Zusammenstellung der negativen Winkel der optischen Achsen und des Eisengehaltes der drei genannten Mineralien, wobei es sich ergibt, dass mit zunehmendem Gehalte an FeO der Winkel der optischen Achsen stetig abnimmt. Die Mikrostruktur aller Mineralien der Enstatit-Gruppe ist im Allgemeinen eine so "ahnliche, dass im speziellen Falle eine sichere Entscheidung unter ihnen nur durch chemische und genaue optische Analyse gegeben werden kann."'

"`Enstatit und Bronzit finden sich in den Gesteinen nicht als Kristalle, sondern fast nur in unregelm"a"sig begrenzten Kristallk"ornern, welche meistens eine sehr dichte Streifung erkennen lassen, die bei dem Enstatit mehr geradlinig, bei dem Bronzit mehr sanft wellig gewunden verl"auft. Doch ist dieser Unterschied kein durchgreifender. Die gleiche Streifung zeigt auch der monokline Diallag und der rhombische Bastit, der sich aber durch andere, sp"ater zu besprechende, optische Erscheinungen nicht unschwer vom Bronzit trennen l"asst. Traf der Schliff den Enstatit oder Bronzit stark geneigt zu seiner Hauptspaltungsfl"ache, so ist die Oberfl"ache nicht in gleicher Weise feinfaserig, sondern treppenf"ormig rauh. Querliegeende Absonderungsfl"achen und Zierbrechungen sind nicht selten."'

"`An fremdartigen Einlagerungen sind beide verh"altnism"a"sig arm; ja sie fehlen z. B. im Enstatit aus dem Pseudophit des Aloysthals in M"ahren und in manchen Enstatiten oder Bronziten der Lherzolithe und Olivinfelsen ganz. Ersterer ist nur von h"aufigen Adern des Pseudophit durchzogen, von welchen aus in senkrechter Richtung feinfaserige Zersetzungsprodukte in den Enstatit eindringen. Andere Vorkommnisse und selbst andere Individuen desselben Handst"ucks enthalten dagegen oft massenhafte Einschl"usse von gr"unen oder braunen Lamellen, Leistchen und K"ornern (je nach der Lage der Schliffebene), welche ausnahmslos der vollkommensten Spaltungsrichtung parallel gelagert sind. Der Gedanke liegt nahe, dass die verschiedenen Angaben "uber die relative Vollkommenheit der pinakoidalen ($\infty$P$\infty$) Spaltung gegen"uber der prismatischen vielleicht auf die mehr oder weniger massenhafte Anwesenseit dieser Interpositionen zur"uckzuf"uhren seien, die zweifellos auch den Metalloiden Schiller auf dem Brachypinakoid bedingen. Dann w"are aber die Leichtigkeit der Trennung in der genannten Richtung mehr eine Absonderung, als eine eigentliche Spaltbarkeit."'

"`Der Enstatit ohne und der Bronzit mit metallischem Schimmer auf der brachypinakoidalen Spaltungsfl"ache finden sich in Serpentinen von Aloysthal in M"ahren (Enstatit) und Mont Brésouars in den Vogesen, in den Lherzolithen und Olivinfelsen, in manchen Olivingabbros, in Streng's Enstatitfels vom Radauthal bei Harzburg und in den Olivinbomben des Dreiser Weihers, sowie in manchen Meteoriten; also stets in Gesellschaft des Olivin und in ver"anderten Olivingesteinen."'

F"ur diejenigen, welchen das Buch nicht zu Gebote steht, gebe ich 2 Abbildungen, die eine von Bronzit vom Kupferberg Tafel 1. 1, die andere von Enstatit von Texas Tafel 1. 2, welche mit den Rosenbusch'schen ziemlich "ubereinstimmen.

Was den Olivin betrifft, so bedarf es keiner Abbildung, da die Formen dieses Gesteins durch Zirkel vollst"andig ersch"opft sind. Es gen"ugt zu sagen, dass reiner frischer Olivin keine Struktur zeigt. Struktur zeigt der Olivin blo"s, wenn man seine Einschl"usse oder Anwachsstellen des Kristalls oder Zersetzungserscheinungen (Serpentinbildung) Struktur nennen wollte. Aber sicher findet sich in keinem Kristall etwas, was meinen Formen auch nur "ahnlich sieht. Was die Behauptung betrifft, die Kugeln seien Gl"aser, so wird nicht einmal unterschieden, welche chemische Zusammensetzung diese Gl"aser gegen"uber Enstatit, Bronzit und Olivin haben sollen. Offenbar werden alle Formen zusammen geworfen und f"ur Gl"aser erkl"art, obgleich Enstatit nach Quenstedt (Mineralogie S. 318) unschmelzbar, nach Naumann-Zirkel S. 585 wenigstens schwer schmelzbar ist. Es wird sogar behauptet, dass diese Gl"aser erst im Fallen entstanden seien. Allein Feuereinwirkungen finden sich blo"s in der Rinde. Die Schmelzrinde der meisten Meteorite hat kaum 2 mm Durchmesser.

Die Behauptung, es seien Gl"aser, wurde der Mitteilung meiner ersten D"unnschliffe entgegengehalten und dabei auf die "ahnlichkeit der meteoritischen Form mit solchen Gl"asern in dem Gesteine unserer Erde hingewiesen. So wurde ich von [Ferdinand] Zirkel auf einen Sphaerulit-Liparit verwiesen, dessen Abbildung ich Tafel 1. Figur 3 gebe. Diese Form sollte dartun, dass meine \emph{Urania}\index{Urania} eine T"auschung sei. Ich halte die Form im Liparit f"ur eine Kristallit-Bildung (wahrscheinlich Zeolith). Nun betrachte man die Strukturbilder daneben Tafel 1, Figur 4, 5, 6!

Unsere Forscher, mit Ausnahme G"umbels, sprechen von den Meteoriten als vulkanischen Bomben, erkl"aren das Gestein als identisch mit dem Vulkangesteine der Erde, z"ahlen also den Meteorstein ohne Bedenken zu den vulkanischen. Der Gegenbeweis ist der Gegenstand dieses Buchs.

Richtig allein hat Quenstedt die Frage f"ur eine offene erkl"art und gesagt: es sei dem Mikroskop vorbehalten, das R"atsel der Zusammensetzung der Meteorite zu l"osen! \emph{Handbuch der Mineralogie} S. 722.
\clearpage
\section{Die Organische Natur der Chondrite}
\subsection{Organisch oder Unorganisch?}
\paragraph{}
Um den Beweis zu f"uhren, dass ein pflanzlicher oder tierischer Organismus vorliege, halte ich f"ur notwendig darzutun:
\begin{enumerate}
    \item eine geschlossene Form,
    \item eine wiederkehrende Form,
    \item wiederkehrend in Entwicklungsstufen,
    \item Struktur und zwar entweder Zellen oder Gef"a"se,
    \item "ahnlichkeit mit bekannten Formen.
\end{enumerate}
\paragraph{}
Sind diese Erfordernisse da, so bleibt nur noch zu entscheiden, ob Pflanze oder Tier? Nun fragt sich, erf"ullen meine Formen diese Forderungen?

Ich glaube, ehe ich an den positiven Beweis gehe, den negativen Beweis f"uhren zu sollen.

Der Beweis n"amlich, den ich f"ur das Dasein organischer Wesen antrete, ist ein doppelter: ein negativer, indem ich dartue, dass die meteoritischen Formen nicht dem Mineralreich angeh"oren: ein positiver, indem ich die "Ubereinstimmung derselben mit den Formen unserer Erde, sei es lebender oder ausgestorbener, begr"unde: das erste also, was zu beweisen, ist der Satz:

Die Einschl"usse der Meteoriten sind keine Mineralbildungen.
\begin{enumerate}
\item Unsere Mineralogen erkl"aren die Einschl"usse der Chondrite f"ur Enstatit, Bronzit, Olivin.

Olivin hat keinen sichtbaren Bl"atterbruch, Enstatit und Bronzit einen deutlichen. Ich bilde einen Bronzit von Kupferberg, Tafel 1. 1. einen Enstatit von Texas, Tafel 1. 2. (D"unnschliff bei 75 facher Vergr"o"serung) ab. Figur 2. zeigt einen der besten Bl"atterbr"uche. Man vergleiche nun damit zuerst Tafel 1. Figur 4, einen Teil eines Favositen des Meteorsteins von Knyahinya (etwa 250 mal vergr"o"sert) und man wird wohl nicht mehr davon reden, dass der Bl"atterbruch die Ursache der Strukturerscheinungen der Chondrite sei. Nun betrachte man aber noch s"amtliche Tafeln und es wird diese Erkl"arung ein f"ur allemal abgetan sein.
\item Wenn die Einschl"usse der Chondrite nach der bisherigen Deutung aus Enstatit oder Olivin bestehen, oder wenn es Gl"aser w"aren: wie w"are es, frage ich, m"oglich, dass dasselbe Mineral oder Glas im Ganzen in so verschiedenen Formen (Umrissen und Strukturen), und verschiedene Minerale in so scharf "ubereinstimmenden Formen auftreten? Man betrachte einmal einen Hypersthen, eine Hornblende, einen Augit! Abgesehen von einigen sichtbaren, leicht zu erkl"arenden Einschl"ussen --- (und um diese handelt es sich ja hier nicht) immer dasselbe Bild! Von h"ochstens 3 Mineralen hundert verschiedene Bilder!

Das Mineral ist einfach, muss seinem Begriff nach einfach sein und daher stets das Bild einer homogenen Masse (Fl"ache) geben, h"ochstens mit einigen Einschl"ussen. Wie sollte nun dasselbe Mineral in so verschiedenen Strukturen, dabei in so "ubereinstimmenden von den Kristallformen abweichenden Umrissen vorkommen?
\item Die Minerale sind entweder kristallisiert oder nicht kristallisiert. --- In dem ersten Zustand haben sie bestimmte gesetzm"a"sige also wiederkehrende Formen: sie r"uhren von Fl"achen, welche im Durchschnitt sich als gerade Linien projizieren. Diese Formen (Linien und Winkel) sind wiederkehrend, wechseln blo"s der Gr"o"se, nicht dem Verh"altnis nach. Solche Formen finden sich unter den von mir als organisch angesprochenen Formen nicht. Hier ist keine Form mit einer Fl"ache oder mit einem Winkel; Alle sind Kugeln, Ellipsen mit Abweichungen von der mathematischen Form, Abweichungen, welche aber doch konstante sind. Also ganz abgesehen von der "ubereinstimmenden Struktur, zeigt sich eine Konstanz der Umrisse, aber andere Formen als die Kristallformen des Enstatits, des Olivins sie geben m"ussten.

Allerdings kommen seltene, kleine Stellen mit wirklichen Kristallen vor, aber in einer Weise, welche durchaus auf den Beweiswert dieser Tatsachen nicht einwirkt. Hier"uber siehe unten und Tafel 32. Figur 2.
\item Waren die Minerale urspr"unglich kristallisiert, haben aber durch mechanische Gewalt ihre Kristall-Form verloren, so ist die einzige Form, welche hier sich wiederholen k"onnte, die Kugel oder eine dieser sich n"ahernde Form, etwa die Ellipse. Hier w"are eine Wiederholung m"oglich, ohne dass aus der Form ein Schluss gezogen werden k"onnte. In den Rollsteinen schneidet die Oberfl"ache den K"orper in einer Weise, dass sofort die Einwirkung der mechanischen Gewalt hervortritt, --- insbesondere werden Einschl"usse ganz willk"urlich getroffen.

In den Meteoreinschl"ussen aber ist die Struktur im Stein stets, ich m"ochte sagen: symmetrisch, im Einklang mit den Umrissen.
\item Bei Verwitterung von Kristallen "andern sich die Schichten von au"sen nach innen --- konzentrisch: --- von Verwitterung aber ist keine Spur in den Einschl"ussen der Chondrite zu sehen und die Strukturen sind stets exzentrisch.
\item Was die Einschl"usse der Mineralien betrifft, so k"onnen diese je nach ihrer Beschaffenheit verschiedene Bilder geben. Es kommen ganz willk"urliche Formen der Einlagerung vor, wie Glas-Fl"ussigkeits-Einschl"usse, Kristalliten.

Wo aber ein Formengesetz in der Einlagerung auftritt, richtet sich dieses stets nach der Kristallform. Beides trifft bei den Meteoritformen nicht zu. Keine Spur von Einlagerung nach einer Kristallform!
\item Ein Bl"atterbruch wird nur sichtbar, wenn durch mechanische Gewalt Spalten und nun Lichtbrechungserscheinungen auf den Spaltungsfl"achen entstehen. Ohne diese ist er nicht wahrnehmbar. Spaltungsfl"achen sind nicht da, Lichtbrechungserscheinungen zeigen die Meteorit-Einschl"usse auch nicht, blo"s "`Einst"aubungen"'.

Es finden sich in den terrestrischen Mineralien Interpositionen parallel mit dem Bl"atterbruch eingelagert: diese zeigen die Meteoriten nicht.

Ich glaube, der Anblick meiner Formen wird eine weitere Auseinandersetzung "uber ihre Verschiedenheit von Mineral- und insbesondere von Kristallbildern nicht notwendig machen.
\item Es ist aber soviel von Kristalliten, von Kristallkonkretionen gesprochen worden.

F"ur solche wurden die Enstatit-Bronzit-Olivin-Kugeln bisher gehalten. G"umbel wies dementgegen darauf hin, dass es keine Kugel gebe, wo der Mittelpunkt nicht exzentrisch liege!

Hier gerade tritt der wesentliche Unterschied zwischen den Meteorit-Formen und den Kristalliten recht deutlich hervor.

Die Kristalliten legen sich stets um einen Punkt (konzentrisch) an. Die Formen in den Meteoriten sind alle elliptisch und birnenf"ormig: wenn die "au"sere Form aber auch kugelig ist, sind die angeblichen Einschl"usse exzentrisch geordnet und zwar liegt der Mittelpunkt an der Peripherie, (sogar jenseits derselben, n"amlich dann, wenn er weggeschliffen ist, was G"umbel "ubersah) --- eine Erscheinung, welche nie im Mineralreich vorkommt. Es ist eben die Bedingung der Kristalliten- d. h. Kugelbildung, dass die Kristalle um Einen Kristall gleichmassig sich anlegen, wodurch dann notwendig die konzentrische Form entsteht.

W"aren also die Kugeln in den Meteoriten Kristalliten, so m"ussten sie, wenigstens nach dem Gesetz der Erde, konzentrische Bildungen aufweisen.
\item Schlie"slich muss ich einen Widerspruch aufzeigen, in welchen die Wissenschaft mit sich geriet, wenn sie die Struktur der Chondriten aus der Mineral-Eigenschaft erkl"aren wollte. Dies ist das optische Verhalten dieser Einschl"usse.

W"aren sie Kristalle und w"are der Bl"atterbruch (freilich Olivin hat keinen, und doch finden sich auch in den angeblichen Olivin-Kugeln Strukturen, also Bl"atterbruch!) die Ursache der Struktur, so m"usste das Mineral notwendig das Licht brechen. Bei den meisten der Einschl"usse zeigt sich aber keine Lichtbrechung, nicht einmal Aggregat-Polarisation! --- So k"onnen sie also weder einfache Mineralien noch Kristalle sein, am allerwenigsten lie"se sich die Struktur aus Bl"atterbr"uchen erkl"aren. Diese Tatsache, das optische Verhalten, sollte allein schon zur richtigen Deutung gef"uhrt haben.
\end{enumerate}
\paragraph{}
All diese Beweise sind freilich dem Botaniker und Zoologen fremd, w"ahrend sie jeder Mineraloge kennt: daher muss ich diesen bitten dem Kollegen Botaniker und Zoologen das eben Vorgetragene zu best"atigen, zu best"atigen was meine Lichtbilder zeigen: Diese Formen sind keine Mineralformen. Damit hat der Mineraloge seinen Anteil an der Arbeit getan und nunmehr geht sie in die Hand des Pal"aontologen, oder richtiger des Zoologen "uber und es beginnt die positive Beweisf"uhrung.
\clearpage
\subsection{Die Einzelnen Formen: Schw"amme --- \emph{Urania}\index{Urania}}
\paragraph{}
Rundlappige K"orper mit deutlicher Anwachsstelle. Tafel 2. gibt ein gr"o"seres Normalbild einer \emph{Urania}\index{Urania} (vergleiche Tafel 5. Figur 1, dasselbe Bild). Man sieht hier: die Gesamtform scharf, den "au"sersten Lappenrand angeschnitten (weis links), die Falten, welche beim Zusammenziehen entstehen, die Anwachsstelle. Noch deutlicher ist letztere mit Kelch, Tafel 4. Figur 3.

\emph{Urania}\index{Urania} spiralf"ormig zusammengelegt Tafel 3. Figur 5, 6.

In der Windung begriffen Tafel 4. Figur 1: die Struktur besteht in einer Au"senhaut "uber lamellaren Schichten Tafel 3. Figur 4. Tafel 4. Figur 6 (letztere mit der Lupe zu betrachten). Mittlerer Durchmesser der \emph{Urania}\index{Urania} 1 mm, Farbe smalteblau.

Diese Struktur wurde f"ur den Bl"atterbruch des Bronzits gehalten! Ob Tafel 4. Figur 4 zu den Uranien geh"ort, ist zweifelhaft. "au"sern Form und Farbe sprechen daf"ur. Die Anschnitte an beiden Seiten zeigen deutliche Struktur.

Tafel 5. Figur 5 zeigt vollst"andig gewundene Lappen. Entweder ist es ein K"orper spiralf"ormig aufgewunden oder sind es mehrere Lappen, von welchen der "au"sere die inneren mantelartig umgibt.

Tafel 4. Figur 6 ist ein Querschnitt, welcher allerdings wenig zeigt. Im Objekt selbst sieht man den Durchschnitt der Au"senhaut weis.

Tafel 5. Figur 2 zeigt so deutliche Schichtung, dass wenn die "au"sere Form nicht w"are, man versucht sein k"onnte, die Form zu den Korallen zu stellen.

Tafel 4. Figur 5 zeigt Querschnitte durch beide Fl"ugel der Lappen.

Tafel 6. Figur 3 Lamellen-Struktur. Figur 5 und 6 k"onnen auch die einfachsten Crinoiden\index{Crinoid} sein, deren Arme sich an einander angelegt haben. Hinsichtlich des "Ubergangs der Formen in andere muss ich auf das betreffende Kapitel verweisen.

Am r"atselhaften ist Tafel 6. Figur 1 und 2. Bei Figur 1 ist die matte Stelle im Pr"aparat gelb, die gestreifte blau. Ich habe sie neben Figur 2 gestellt, diese zeigt deutlich zwei Lappen, welche wie zwei Muschelschalen an einer Stelle verbunden sind und beim ersten Anblick auch vollkommen den Eindruck eines Zweischaligen machen. (Es ist nicht ein blo"ser Anschnitt.) Denkt man an Muscheln, so k"onnte die matte Stelle von Figur 1 der Steinkern sein. Allein die Struktur ist eben Uranienartig.

Tafel 5. Figur 3. 2 Individuen zeigen die Struktur "uberaus deutlich, ebenso die Anwachsstellen. In Figur 4 (welche ein schlechtes Bild gibt) legen sich mehrere Individuen f"acherartig aneinander.

Bei Tafel 3. Figur 3, IV. 1, 2 glaubt man oben eine runde Mund"offnung angedeutet zu sehen.

Hiernach halte ich die \emph{Urania}\index{Urania} f"ur einen festgewachsenen Schwamm, welcher sich spiralf"ormig zusammenzieht, hiebe Wasser einsaugt und austreibt, wie unsere lebenden Schw"amme.

\emph{Urania}\index{Urania} nimmt etwa 3/20 der Gesteins-Masse ein.
\clearpage
\subsection{Die Einzelnen Formen: Schw"amme --- Nadel-Schw"amme}
\paragraph{}
Tafel 7. Die Formen Figur 1, 2, 3, 5, 6 zeigen ein Nadelger"uste. Figur 1 stelle ich zu Astrospongia. Die Nadeln liegen regelm"a"sig gekreuzt. Figur 6 ist ein unregelm"a"siges Nadelger"uste mit einem Hohlraum, welchen das Bild allerdings sehr schwach andeutet. Diese beiden Formen scheinen mir unzweifelhaft zu sein.

Ann"ahernd sicher sind Figur 2 und 5 (in Figur 2 ist der wei"se Strich ein Gesteinsriss).

Die Form Figur 4 habe ich bei der Zusammenstellung der Tafeln f"ur einen Schwamm gehalten. Nachdem eine "anderung der Anordnung nicht mehr m"oglich war, erkannte ich in dieser Form den schiefen Durchschnitt eines Crinoiden\index{Crinoid} und was ich Anfangs f"ur Nadeln hielt --- als feine Crinoidenarme. Ich bemerke, dass die Bestimmung sehr schwierig ist wegen der au"serordentlich einfachen meteoritischen Crinoidenformen, weshalb eine Entscheidung weiterer Untersuchung aufgespart bleiben muss. Es l"asst sich der Hohlraum der Schwammnadeln mit dem Nahrungskanal der Crinoidenarme verwechseln, wenn letztere gerade gestreckt liegen und die Glieder nicht mehr deutlich erhalten sind. Diese Tatsache, so wenig angenehm sie f"ur den Untersucher der einzelnen Formen ist, ist um so lohnender f"ur denjenigen, welcher dem Zusammenhang der Formen nachgeht --- f"ur den Nachweis der Entwickelung einer Form aus der andern. Es reicht immer eine an die andere hin. In g"unstigere Lage versetzen uns:
\clearpage
\subsection{Die Einzelnen Formen: Die Korallen\index{koralle}}
\paragraph{}
Hier haben wir so wohl erhaltene terrestrische Formen, dass ein Zweifel nicht "ubrig bleibt.

Tafel 8. zeigt ein Musterbild, Tafel 9. dessen Kanalstruktur: deutliche Knospen-Kan"ale, welche die R"ohren (denn solche sind es) verbinden. Dazu kommt die mit einem Bl"atterbruch absolut nicht zu verwechselnde Kurvenrichtung der Kan"ale, dazu kommen die ganz deutlichen R"ohren"offnungen und endlich die ebenso deutliche Anwachsstelle. (Tafel 1. Figur 4 zeigt ein noch sch"arferes Bild desselben Objekts.) Leider geben F"arbungen des Pr"aparats dem Struktur-Bild Tafel 9. widerw"artigen Schatten. Die Knospen-Kan"ale stehen 0,003 mm von einander ab. Gewiss alles, was man von einer Struktur eines Favositen verlangen kann.

Tafel 10. Figur 3, 4 zeigen uns das Bild des \emph{Favosites multiformis} aus dem Silur so, dass man hier auch nicht einmal Spezies zu trennen verm"ochte.

Auf Tafel 11. in Figur 1, 2, 3 (wo 2 auch die Anwachsstellen zeigt) wird jeder Forscher das Bild lebender Korallenformen leicht erkennen, umso mehr als in Figur 1 oben noch die Becherform (Hohlraum) angedeutet ist. Dasselbe Objekt zeigt ferner in den R"ohren Querscheidew"ande, die klar hervortreten. Leider ist ein Teil des Bildes in Folge der gelben F"arbung des Pr"aparats in der Photographie durch Schwarz verdeckt.

Tafel 10. Figur 1 und 2 zeigen weniger gut erhaltene Quer- und L"angsschnitte, doch hebt die ganz gleiche Wiederholung beider in mehreren Schliffen den Zweifel daran, dass es organische Formen sind, und sind es solche, so k"onnen es blo"s Korallen sein. Figur 3 scheint eine Becher-Koralle zu sein, Figur 4 ist an dieselbe angewachsen. Dass Figur 6 Korallenstruktur hat, bedarf wohl keines Nachweises. Diese Form kehrt mehrfach wieder.

Tafel 11. Figur 4. Diese Form kehrt ebenfalls mehrfach wieder. Eigent"umliche Korallenformen zeigen Figur 5 und 6. --- Figur 5 ist gebildet aus R"ohrenringen und h"ochstwahrscheinlich auch Figur 6. Ich bemerke, dass diese Form hundertmal wiederkehrt.

Bei h"oherer Vergr"o"serung zeigen Zwischenw"ande Tafel 11. Figur 1, 2, 3, 6.

Tafel 12. Figur 1, 2, 3 zeigen deutliche Lammellarstruktur. Die Querfurche in Figur 4 erinnert an Fungia. Wahrscheinlich geh"oren auch hierher Tafel 30. Figur 1, 2 und Tafel 20.

Die "Ubereinstimmung der Struktur in Tafel 20. mit Tafel 30. Figur 1 (in zwei verschiedenen Schliffen) w"urde allein hinreichen jeden Gedanken an eine unorganische Bildung auszuschlie"sen. "Uberdies kehrt die Form in 350 Schliffen etwa zwanzigmal wieder.

Tafel 12. Figur 5 habe ich nur einmal gefunden. Im Original sind deutliche Lamellen, welche im Bilde blo"s am unteren Teil hervortreten. Figur 6 ist ein milchwei"ses Objekt, daher undeutlich. Ich glaube Sternform zu erkennen und habe die Form deshalb als Sternkoralle hierher gestellt.

Tafel 13. Figur 1, 2, 3, 4 sind Korallen, welche ganz unzweifelhaft den R"ohrenkorallen angeh"oren. Es sind im Original deutlich zu unterscheiden: Glasartige Zwischenmasse, schwarze R"ohrenwand, gelbe F"ullmasse der R"ohren, zuweilen sind beide letztere schwarz. Diese Form kommt hundertf"altig vor und zwar in allen Chondriten. Figur 5 aus Lamellen zusammengesetzt zeigt deutliche Hohlr"aume und Figur 6 R"ohren mit Zwischenw"anden. Die Formen geh"oren zu den gr"o"sten Formen: sie haben bis zu 3 mm. Durchmesser.

Tafel 25. 1 und 2. Die Form ist hier so ausgezeichnet erhalten, dass an dem Vorhandensein eines Organismus nicht gezweifelt werden kann, um so weniger, als sie in zwei Schliffen "ubereinstimmend vorkommt und auch sonst h"aufig wiederkehrt. Vergl. Tafel 2. links unten, Tafel 5. Figur 6. Ich habe die Formen Tafel 1. Figur 6 und Tafel 25. Figur 1, 2 in der Folge zu den Crinoiden\index{Crinoid} gestellt; die Kan"ale sind unzweifelhaft, die Querlinien lassen sich auch als Crinoiden-Glieder deuten. Man sieht Einschnitte, ferner sind die Arme geknickt, was sich blo"s bei Crinoiden\index{Crinoid} denken l"asst.

Geknickte Arme zeigt auch Tafel 25. Figur 4. Von dieser Form sind mehrere Exemplare da, welche genau dasselbe Bild geben.

W"ahrend die Korallenformen etwa 1/20 des Volumens der Gesamtmasse des Chondrit-Gesteins einnehmen, bilden den Rest mit 16/20 --- also den bei weitem gr"o"sten Teil der ganzen Masse:
\clearpage
\subsection{Die Einzelnen Formen: Crinoiden\index{Crinoid}}
\paragraph{}
Sie finden sich von der einfachsten Form eines gegliederten Armes bis zum ausgebildeten Crinoiden\index{Crinoid} mit Stiel, Krone, Haupt- und Hilfsarmen. Ihre Erhaltung ist gr"o"stenteils sehr gut. Die Schwierigkeit liegt blo"s in den tausenderlei Richtungen der Schnitte, welche immer verschiedene Bilder desselben Objekts geben. Die birnenf"ormigen K"orper, welche man als Gl"aser ansah, sind Crinoiden-Kronen.

Ich stelle 4 Crinoiden\index{Crinoid} in aufrechter Stellung und in gro"sem Format in Tafel 16, XVII, XVIII, 19 dar und einen im Querschnitt Tafel 20.

Tafel 21. Figur 1, 2, 3, 4, 5 zeigt senkrechte Durchschnitte eines schon h"oher entwickelten Crinoiden\index{Crinoid}. Es sind Hauptarme mit Hilfsarmen und deutlichen Gelenkfl"achen.

Tafel 21. Figur 3 zeigt Stiel und Krone. (2 und 4 doppelte Vergr"o"serung von 1 und 3.) Figur 5, aus einem andern D"unnschliffe, ist da, um die "Ubereinstimmung der Formen zu zeigen. In Figur 6 glaube ich die Mund"offnung in dem H"ocker zwischen den Armen erhalten zu sehen.

Tafel 22. Figur 1, 3, 4, 5 und Tafel 23. Figur 1, 2 zeigen die Zahl 5 der Arme, sowie die Hilfsarme.

In Tafel 23. Figur 2 und 3 sieht man die Knickung der Arme durch Druck von oben.

Tafel 22. Figur 2 und 4 erinnern an Comatula.

Eine besondere Art sind die Crinoiden\index{Crinoid}, welche blo"s aus einer beliebigen Anzahl von Armen bestehen. Zu diesen rechne ich Tafel 23. Figur 4, 5, Tafel 24. 4, 5, 6, Tafel 26. (Es ist auf dem Bilde Tafel 24. Figur 6 in kleinerem Ma"sstab die Koralle aus Cabarras, Tafel 13. Figur 6.)

Tafel 29. Figur 1, 2, 3, 4, 5, 6 und Tafel 27. Figur 3 geben Bilder von Crinoiden\index{Crinoid} von oben gesehen.

Tafel 27. Figur 2 und Tafel 29. Figur 4 zeigen Crinoiden\index{Crinoid} von unten: hier tritt der Stielansatz als heller Punkt hervor. Diese Querschnitte kehren in dutzend F"allen in "ubereinstimmender Form wieder. (Man vergleiche auch Tafel 3. Figur 2 links oben. Bessere Durchschnitte kann man wohl nicht fordern: die Muskelschichten sind hier deutlich sichtbar.)

Eigent"umliche Verschlingungen zeigen Tafel 26. Figur 1, 2, 3, 4.

Die deutlichsten Querschnitte geben Tafel 25. Figur 5 und 6. Ein L"angsschnitt ist Tafel 27. Figur 3 mit geknickten Armen.

Tafel 24 Figur 1 und 2 sind Formen, welche ich anfangs f"ur Korallen ansah.

Tafel 28. Figur 1 k"onnte doch diesen letzteren zuzuz"ahlen sein (die Struktur sollte deutlicher erhalten sein, um endg"ultig zu entscheiden).

Etwas deutlicher ist Tafel 27. Figur 1: eine scheinbare Au"senwand, welche aber nichts als der Durchschnitt des regelm"a"sig gelagerten Hauptarms ist.

Ein sehr sch"ones Bild gibt Tafel 30. Figur 3; ob Crinoid\index{Crinoid}? ist zweifelhaft. Nur bemerke ich, dass die beiden Teile symmetrisch und die Arme nicht aneinander gelegt sind, sondern sich kreuzen.

Tafel 30. Figur 5 mit einem Anschnitt hatte ich anfangs zu den Uranien gestellt. Sie wird den Crinoiden\index{Crinoid} zuzuz"ahlen sein.

Tafel 31. Figur 1, 2, 3 sind offenbar dieselben Formen. In Figur 1 und 3 ist eine deutliche Furche wahrzunehmen, vielleicht die Stelle wo zwei Crinoiden-Arme sich aneinander legen. Im Polarisationsapparat tritt diese Furche noch deutlicher hervor. Figur 4, zwei Individuen zusammengelegt, lie"se die Deutung auf Schwamm oder Koralle offen. Figur 5 mit Maschenstruktur in dem mittleren Teil, ein Gewebe von Gliedern, zeigt oben Arme mit deutlicher Struktur. Geh"oren diese Dinge zu stammen? Da die Form nur einmal vorkommt, wage ich keine Entscheidung. Auffallend ist nur die "ahnlichkeit des Mittelbildes mit der Struktur des Schreibersits im Meteoreisen. Figur 6 findet sich zweimal, weshalb ich beide Teile als zusammenh"angend angesehen habe.

Dieselbe Maschenstruktur zeigt Tafel 30. Figur 6 bei Lupenvergr"o"serung. Die Struktur beider stimmt, wie erw"ahnt, mit der Struktur des Schreibersits in dem Meteoreisen und kehrt mehrmals wieder.

Wie ich schon im Eingang bemerkte, halte ich es nicht f"ur meine Aufgabe Spezies zu machen. Meine Aufgabe war nur das Dasein von Organismen mit dem Nachweise geschlossener wiederkehrender Formen von organischer Struktur unzweifelhaft festzustellen. Dies glaube ich getan zu haben und ich denke, es sollte Niemand auch nur den mindesten Zweifel mehr hegen, (insbesondere nach dem Anblick eines D"unnschliffes im Original), dass es sich hier nicht um Mineralformen handle. Sind aber nur 5 organische Formen unzweifelhaft nachgewiesen, so sind auch die "ubrigen weniger gut erhaltenen Formen organisch.

Um endg"ultig Genera und gar Spezies festzustellen, geh"ort mehr Material und jahrelange Untersuchung dazu. (F"ur ersteres werde ich dankbar sein.) Vor Allem m"usste ich mehr Zeit haben, als die Nachtstunden und mehr Kraft, als mir mein anstrengender Beruf "ubrig l"asst, um die Arbeit zu vollenden. Doch meine ich den geforderten Punkt gegeben zu haben, auf welchem man stehen kann.

Zum Schluss verweise ich auf die Tafelerkl"arung.

Damit sind die Formen vorgef"uhrt. Ich habe eine Zeitlang den Plan verfolgt, eine f"ormliche Statistik "uber das Vorkommen der Formen in meiner D"unnschliffsammlung zu machen, aufzuz"ahlen, wie oft ein und dieselbe Form in den 500 D"unnschliffen sich findet. Ich stand davon ab, weil ich mir sagen musste, dass es doch keinen gro"sen Wert haben werde. Jede Vermehrung meiner Sammlung um nur 12 Nummern w"urde die Verh"altniszahl "andern. Ich zog daher vor, bei einzelnen Formen das Zahlenverh"altnis ann"ahernd anzugeben.
\clearpage
\subsection{Alles Leben}
\paragraph{}
Es sind im Vorstehenden die einzelnen Formen zur Anschauung gebracht. Alle diese Formen sind nicht tot eingebettet, sondern die eine aus der anderen gewachsen und in Wahrheit lebend vom Leben begraben. Hier kann freilich nur die Anschauung "Uberzeugung geben. --- Zu diesem Zweck betrachte man in s"amtlichen Bildern die einzelnen Formen mit ihrer Umgebung!

Was auf den ersten Blick auch nur als ein heller Fleck erscheint, bei genauerer Untersuchung zeigt es den Durchschnitt eines Schwamms, einer Koralle, oder eines Crinoidengliedes. Nirgends sind, wie G"umbel ganz richtig beobachtet hat, Zierst"uck, zerbrochene, abgerollte Formen, Splitter --- auch ist kein Bindemittel zwischen denselben. Nur die Weichteile fehlen, alles Andere ist erhalten, wie es sich im Leben im Wasser bewegte. Die Crinoidenformen zeigen dies am deutlichsten. Denn auch diese sind h"ochstens auf die Seite gebogen, gewunden, selten geknickt; man sieht auch den nur schwachen mechanischen Wiederstand gegen den "uber dem Haupt entstandenen Nachbar. --- Aber Alles aneinander, auseinander gewachsen, Nichts niedergelegt, Nichts tot eingebettet. Da ist auch keine Masse, welche ein Grab h"atte bilden k"onnen.

Die Tatsache, dass nichts Unorganisches in dem Chondrit-Gestein und kein Raum ohne Leben darin ist, halte ich f"ur ebenso bedeutend, als das Dasein der Organismen selbst. Diese Tatsache erst wirft auf die Entstehung des Planeten das volle Licht. Nimmt man hinzu, dass das Gestein, welche diese Bildungen einschlie"st, aus Mineralen besteht, welche dem sogenannten Urgebirge, ja "`vulkanischem"' Gebirge angeh"oren: so muss unsere Geologie notwendig einen andern Weg in der Erkl"arung der Tatsachen einschlagen. Ich glaube nun freilich keineswegs, dass es Schw"amme, Korallen, Crinoiden\index{Crinoid} aus den Mineralen gegeben habe, welche heute die Formen bilden. Die Organismen m"ussen urspr"unglich anders zusammengesetzt gewesen sein, m"ussen also eine Umwandlung erlitten haben.

So viel ist, denke ich, "uber allen Zweifel erhaben, dass das, was jetzt Hornblende, Augit, Olivin ist und die genannten Formen ausf"ullt, fr"uher in einem andern Zustand gewesen sein muss, n"amlich eine fl"ussige, und zwar wasserfl"ussige L"osung.

Nun finden wir aber diese Minerale in unserem Urgebirge in Formen, welche nicht Kristalle, wohl aber den meteoritischen "ahnlich sind. Wir finden Gebirgsmassen aus solchen Formen zusammengesetzt. Also waren es auch hier h"ochst wahrscheinlich organische Formen, nachher verwandelt in das, was wir jetzt Gestein nennen. Dieses Gestein weist aber auf eine Schichte, welche ganz unzweifelhaft mit der meteoritischen (den Chondriten) n"aher, ja n"achst verwandt ist --- den Olivin. Und unter diesem muss Eisen liegen: das bezeugt das spezifische Gewicht der Erde. --- Wieder eine gleiche Tatsache sehen wir in den gefallenen Eisen-Meteoriten: hier, wie im Ovifak-Gestein finden wir "Uberg"ange, Zusammensetzung von Eisen und Olivin.

Damit sind uns die gr"o"sten Grundlinien der Geologie gegeben --- wir haben die zeitliche Entwicklung des Erdk"orpers. Die Formentwicklung --- die Ursache der Entwicklung der Formen selbst ist damit zugleich aufgeschlossen. Ist der Organismus in den untersten Schichten, die wir kennen, die Ursache der Massenbildung, so wird er auch die Ursache des Anfangs des Planeten selbst gewesen sein.

Die Annahme einer blo"sen Massenanziehung, der mechanische Anfang der Erde und der Weltk"orper "uberhaupt, w"are damit widerlegt.

Allerdings m"ussten auch noch Organismen im Eisen, im Erdkern, in dem Meteoreisen nachgewiesen werden. Diese Aufgabe habe ich mir als n"achste gestellt; die bisherigen Resultate lassen ihre L"osung hoffen.
\clearpage
\subsection{Stein im Stein}
\paragraph{}
Wenn ich gesagt habe: die Chondrite sind nichts als ein Tiergewebe, ein Tierfilz, so leidet dies eine Einschr"ankung.

Es kommen allerdings in diesem Tierknochengesteine ganz kleine, scharf umschriebene Stellen vor, welche von Anfang an wahrscheinlich (aber nicht notwendig) Gestein sind. Das sind blaugraue, seltene Einschl"usse von 3-5 mm. Durchmesser ohne bestimmt wiederkehrende Form, welche in der grauen Masse deutliche Kristalle eines gelbgr"unlichen Minerals, dessen Durchschnitte das einmal Quadrate oder Rhomben, das andermal Sechsecke sind, einschlie"sen. Dieses Mineral kann Augit oder Olivin sein. Hier spricht die Kristallform f"ur ein Mineral. Allein das Vorhandensein solcher Teile spricht auch f"ur meine Ansicht. Warum h"atten sich die Kristalle nicht "uberall gleich gebildet? Und warum sollten nicht auch Hohlr"aume neben Organismen "ubrig bleiben? Sodann ist bekannt, dass auch blo"se F"ullmassen in organischen Formen nachtr"aglich kristallisieren. Endlich finden sich aber auch in organischen Formen Ausf"ullungen von H"ohlen, welche sich in ihren Umrissen dem Aussehen von Fl"achen und Winkeln n"ahern.

Der Grund, warum ich diese Einschl"usse aber doch als unorganische Teile der Chondrite zugebe, als eigentlichen Meteorstein (Stein im Stein), ist, weil die Umrisse einen Anhaltspunkt nicht geben, um die Form als organische anzusprechen. Diese Einschl"usse k"onnen Einlagerungen einer schon gebildeten Gesteinsmasse sein oder k"onnten sie sich in den Hohlr"aumen erst gebildet haben.

Dass eine Schlammablagerung m"oglich, dass ein Hineinfallen von Teilen eines schon abgelagerten, also fertig gebildeten Gesteins m"oglich, sogar wahrscheinlich sei, braucht nicht geleugnet zu werden: es st"o"st die Tatsache nicht um, dass in den Olivinschichten organische Bildungen vorhanden und dass diese den Aufbau des Planetenk"orpers bewirkt, den Bau selbst gebildet und zusammengesetzt haben.

Unter allen Umst"anden aber ist im Chondrit-Gestein das Verh"altnis das umgekehrte wie bei den Sedimentschichten unserer Erde. In diesen sind die Organismen eingelagert, das Gestein umschlie"st sie; jenes ist eben nichts als Organismen und das Gestein ist eine Masse solcher. Ich f"uge ein Bild eines wirklichen Gesteinteils aus Borkut bei. Tafel 32. Figur 2. Daneben (Figur 1) habe ich eine Form abgebildet, graublau wie \emph{Urania}\index{Urania}, aber ohne bestimmte Struktur, auch in ihren Umrissen unbest"andig, weshalb sie eine blo"se F"ullmasse sein k"onnte. W"are sie eine organische Form, so w"are sie die eines niedersten Wesens. Zur Vergleichung bilde ich in Tafel 32. Figur 4 einen D"unnschliff von Lias $\gamma\delta$ (Zwischenkalk) ab. Hier liegen die Schalen zum Teil ganz im Kalke, gr"o"stenteils aber sind es blo"s St"ucke von Schalen; die Teile sind in alle Gr"o"sen zerschlagen, und, was ihre Herkunft betrifft, gerollt bis zur Unkenntlichkeit. Im Chondrite bleibt fast keine Stelle, welche Zweifel "uber ihre Zusammensetzung "ubrig lie"se.
\clearpage
\subsection{Fortpflanzung}
\paragraph{}
In den Steinen findet sich eine Unzahl runder und birnenf"ormiger Formen von 0,10 mm. --- 0,50 mm. Durchmesser, mit kaum angedeuteter Struktur. Ich halte diese f"ur die ersten Entwicklungsformen. Unter diesen hebt sich am meisten hervor eine Kugelform aus durchsichtigem Gestein, in der Mitte die Anf"ange von Kan"alen. Da finden sich Kugeln mit einem Kanal, mit zwei weiteren unterhalb und oberhalb des gr"o"seren, und so fort bis zu den Formen Tafel 13. Figur 1, 2, 3, 4. Die Sache ist hier, glaube ich, sicher. Diese Form l"asst sich nicht nur in allen Chondriten nachweisen; in allen finden sich auch alle Entwicklungsstufen von einem bis zu 20 und mehr Kan"alen: sie ist die h"aufigste und zugleich, wegen der deutlichen Struktur der Kan"ale, sicherste. Sie hat sich deshalb auch in denjenigen Chondriten erhalten, welche die "ubrigen Formen kaum mehr zeigen. Die Entwicklung besteht also darin, dass sich die Kan"ale vermehren.

Nun finden sich aber eine Menge von Kugel- und Birnenformen mit schwach angedeuteter Struktur. Sie scheinen aus Sarcode bestanden zu haben, als sie einst begraben wurden. Ich w"urde es nicht wagen, diese Formen hereinzuziehen, wenn sie nicht doch eine bestimmte Gliederung zeigten. Sie bestehen aus zwei, drei, vier, f"unf lappenf"ormigen Armen und sind wahrscheinlich die Anf"ange von Crinoiden\index{Crinoid}.

Dass die Feststellung von Entwicklungsformen am schwierigsten ist, ist bekannt. Ich erlaube mir daher hier auch nicht zu weit vorzugreifen. Was ich hier sage, kann nur ein Fingerzeig f"ur k"unftige Forschung sein.

Die gute Erhaltung ist eine Unm"oglichkeit. Die meteoritischen Formen werden daher auch zum mindesten das Schicksal der lebenden teilen: es ist immer die letzte Arbeit, die ersten Anf"ange der Entwicklung, die Embryonen festzustellen.

Nur einer Tatsache will ich hier noch erw"ahnen, welche zugleich ein erhebliches Beweismoment f"ur die organische Natur der Formen ist: die immer auftretende Vergesellschaftung der einzelnen Formen. Die meisten Formen finden sich mit gleichen zusammen: wenige stehen einzeln und zugleich als Unica da. Ich halte dies f"ur sehr wichtig. Wenn mehrere Individuen der gleichen Spezies sich zusammenfinden, so geht daraus hervor, dass sie im Mutter- oder Geschwisterverh"altnisse stehen. Dieselbe Erscheinung tritt auch bei den terrestrischen Arten auf. Dies wird um so bedeutender, als oft das Mineral, aus welchem eine Form besteht, unzweifelhaft das gleiche ist mit dem eine andere Spezies ausf"ullenden Mineral, also ein mineralogischer Grund nicht da ist, aus welchem die Verschiedenheit der Struktur abgeleitet werden k"onnte.
\clearpage
\subsection{Entwickelung}
\paragraph{}
Nachdem ich die einzelnen Formen dargestellt habe, habe ich auch ihr Verh"altnis zu einander, die Entwickelung der Formen aus einander, zu besprechen.

Dass \emph{Urania}\index{Urania} die einfachste Form ist, ist sicher. Diese Form bildet aber auch den Anfang zu den folgenden.

Der halbrunde Lappen teilt sich in Schichten, diese Schichten in R"ohren, die R"ohren teilen sich quer --- jetzt bilden sich Arme, welche ein Kanal verbindet. Es entwickelt sich eine Krone zwischen Armen und Anwachsstelle und der einfachste Crinoid\index{Crinoid} ist da. --- Mag diese Kette allzuk"uhn geschlungen erscheinen, die Formen fordern unwillk"urlich dazu auf. --- Aber muss denn, wenn wir nur irgendwo in unseren lebenden Formen eine Entwicklungsreihe feststellen wollen, nicht auch hier dieselbe Wandlung vor sich gegangen sein? --- Sicher. Nur, glaube ich, finden sich in den meteoritischen Formen mehr und viel sichtbarere "Uberg"ange. Man kann den Stammvater des Pentacrinus Briareus auf unserer Erde nirgends anders suchen, als in den Korallen und gewiss darf man den Anfang der Korallen selbst in der Schwammform erblicken: sie ist entschieden eine niederer Form als die der Korallen.

Was der Meteorsch"opfung die gr"o"ste Wichtigkeit f"ur die Entwicklungslehre gibt, ist nicht nur das Vorkommen von Tierformen in den tiefsten Schichten, sondern der einheitliche Typus aller meteoritischen Organismen. Dieses wird klar, wenn man hunderte von D"unn-Schliffen nach einander betrachtet. Die Gr"o"se der Organismen ist eine gleichartige, verh"altnism"a"sig mindestens 1000 mal kleinere als die der Erde: die Entwickelung der einzelnen Formen erreicht ann"ahernd einen gleichen H"ohepunkt. Der Aufbau der Formen entspricht vollkommen den Umst"anden, unter welchen sie entstanden, n"amlich der "uberaus kurzen Lebenszeit, welche sie gehabt haben k"onnen: es ist eine hastige, relativ unvollkommene Sch"opfung. Der Crinoid\index{Crinoid} ist der h"ochste Repr"asentant dieser Tierwelt. Ich halte f"ur den h"ochstentwickelten die Form Tafel 22. Figur 1, 3, 5, 6, weil er schon die F"unfzahl enth"alt.

Will man aber nicht so weit gehen, die Crinoiden\index{Crinoid} nicht durch die Korallen hindurch ableiten, so bietet die Form der \emph{Urania}\index{Urania} selbst Anhaltspunkte. Ich habe noch einige Formen abgebildet, welche lose Glieder zeigen. Sie sind in der Beschreibung bezeichnet. Insbesondere fand ich bei h"oherer Vergr"o"serung "ubereinanderliegende Arme.

Auch hier reicht die Beobachtung im Einzelnen noch nicht hin, um abschlie"sen zu k"onnen.
\clearpage
\section{Das Meteoreisen\index{Meteorite!eisen}}
\paragraph{}
Ich habe schon in meiner \emph{Urzelle} darauf hingewiesen, dass die Struktur des Meteoreisens nichts anders sei, als die eines Filzes von einzelligen Pflanzen. Die sogenannten Widmannst"atten'schen Figuren sind gr"o"stenteils nichts anderes als einzellige Pflanzen.

Ein St"uck Meteoreisen von Toluca liegt mir vor, in welchen die zylindrischen Zellen eine aus der andern hervorgehen, h"aufig sind zwei kopuliert. Die einzelnen Zellen zeigen doppelte Zellw"ande (Bandeisen), zeigen Querscheidew"ande, zeigen deutliche runde Ansatzstellen; in manchen hat die Marksubstanz (wie man sie gar nannte), wirklich im Innern der Zellen noch Struktur. Die ganzen Zellen selbst liegen in einer matten F"ullmasse (F"ulleisen).

Man vergleiche mit diesen Figuren die Formen aus dem Liasschiefer, insbesondere Algacites granulatus und frage sich, welche von beiden Formen die Pflanzen-Struktur deutlicher zeigt, Toluca-Eisen oder die Alge aus Lias-Epsilon.

Diese Formen sind zylindrisch, mitunter sieht man (im Durchschnitt) ann"ahernd polyedrische Fl"achen: sie haben Wandungen. Was sie aber ganz besonders von Kristallen unterscheidet (abgesehen von ihrer runden Form), sind die Anwachsstellen.

Kristalle, welche aneinander wachsen, setzen sich stets mit einer bestimmten Kristallfl"ache an eine andere ebenso bestimmte Fl"ache an, (Dendriten von Silber, Kupfer). Sie legen sich an die Fl"ache des andern an, ohne in sie einzudringen, Im Meteoreisen aber findet ein Eindringen statt. Der Querschnitt ist nicht eine gerade Linie (Kristallfl"ache), sondern eine Kurve.

Damit h"ort alle "ahnlichkeit mit Kristallen auf, au"ser man n"ahme an, dass es auf andern Planeten Zylinder-Kristalle g"abe, welche auseinander hervorwachsen. Die Behauptung, dass die Figuren bestimmte mathematische Lagen haben, mag stellenweise zuf"allig zutreffen; allein alle Forscher geben zu, dass die Winkel nirgends konstante sind, was bei den Dendriten stets der Fall ist. Findet man auch eine Stelle, woraus man ein Oktaeder, einen W"urfel, oder eine andere regul"are Kristallform, oder auch ein Rhomboeder abzuleiten im Stande w"are: sofort ist die Ordnung daneben eine ganz andere. Und wie wollte man noch von Kristallgesetzen sprechen, wenn von demselben Mineral nicht einmal ein bestimmtes Kristall-System eingehalten w"are? Denn es finden sich, wie gesagt, rhomboedrische Schnitte neben regul"aren.

Ich finde nur zwei Einw"urfe scheinbar begr"undet:
\begin{enumerate}
\item den Einwurf, dass die Figuren zuweilen Platten sind. ---

Hiergegen m"ochte ieh einwenden, dass, wenn einmal Zylinderform nachgewiesen ist, die Formen eben keine Kristalle sind, und dass nun die Folge nicht ist, dass jene Zylinder Kristalle, sondern umgekehrt, dass die Platten, welche dieselbe Struktur tragen, keine Kristalle sind.
\item Der zweite Einwurf ist der: Wie sollen sich Pflanzen in Eisen verwandeln?

Dieser Einwurf ist nicht schwer zu widerlegen. Man denke nur an die meisten unserer verkieselten Versteinerungen, insbesondere die verkieselten St"amme im Lias; man erinnere sich der sogenannten Mansfelder "ahren im Zechstein (Cupressites Ulmanni), wo Cypressen in silberhaltiges Kupfer verwandelt sind. Man sollte meinen, ein solcher Einwand k"onne nicht gemacht werden.
\end{enumerate}
\paragraph{}
Nun bin ich aber durch einen verehrten Freund, Professor Dr. H. Karsten in Schaffhausen, in der Lage, f"ur die Verwandlung von Pflanzen in Eisen einen schlagenden Beweis aus der Jetztzeit beizubringen. Karsten hat schon im Jahre 1869 nachgewiesen, dass unsere niedersten Pflanzen in ganz hervorragender Weise Eisen aufnehmen; seiner G"ute verdanke ich Eisenpflanzen von heute. Mit seiner Erlaubnis lasse ich einen Auszug aus seiner ausgezeichneten Schrift: \emph{Der Chemismus der Pflanzenzelle}, Wien 1869, S. 53 hier folgen:

"`Bringt man Oidium lactis oder Hefe, welche einige Zeit in m"a"sig feuchter Luft (nicht unter Fl"ussigkeit) mit Milchzucker in Ber"uhrung war, mit metallischem Eisen zusammen, indem man "uber die auf dem Objekttr"ager vegetierende Milchhefe Eisenfeilsp"ahne streut, so nehmen zuerst manche dieser das Eisen ber"uhrenden Zellen, sp"ater auch viele von demselben entferntliegenden, mehr oder minder rasch eine intensiv rote Farbe und bald auch eine erstaunliche Gr"o"se an."'

"`Man w"urde sich gezwungen glauben, die Ursache der merkw"urdigen und au"serordentlichen, oft sehr beschleunigten Vergr"o"serung allein nur in einem mechanischen Aufquellen der Zellh"aute zu suchen, s"ahe man nicht zugleich die im Innern der hiebe zum Teil schichtig verdickten Mutterzelle unter den oben angedeuteten Kulturverh"altnissen vorhandenen Tochterzellchen verh"altnism"a"sig mit heranwachsen und sich so vermehren, dass sie die Mutterzelle g"anzlich ausf"ullen."'

"`Auch die Haut der Tochterzellchen produziert S"aure, wie die Eisenreaktion erkennen l"asst; ihre Gestalt ist nach der Verbindung ihrer Haut mit dem Eisen derjenigen der oben beschriebenen Protein-Kristalloide sehr "ahnlich; wie diese sind sie flache, 3-4-5seitige, wenn auch weniger scharfkantige und eckige T"afelchen; unregelm"a"sig neben einanderliegend, f"ullen sie die gro"se Zellh"ohlung v"ollig aus, fallen aber, wenn die Haut der Mutterzelle zerbrochen wird, mehr oder minder mit einander vereinigt aus derselben hervor."'

"`"ahnliche Metamorphosen erfahren auch die Oidiummycellen, besonders die in die Luft hineinragenden zergliedernden "aste, wenn sie in "ahnliche Verh"altnisse gebracht werden, und zwar der Art, dass die verschiedenen Gliedzellen sich oft ungleich ausdehnen, meistens die oberen zuerst und mehr als die unteren, gew"ohnlich stielrund bleibenden, sich etwas streckenden, wodurch diese Zweige mit ihren knopff"ormig angeschwollenen Endzellen Mucor- oder sp"ater frucht- oder blumen"ahnlich werden, wenn sie die oberste vergr"o"serte Zelle am Scheidel deckelartig, oder von oben nach unten klappig anrei"send zu "offnen beginnt. Die H"aute der prim"aren und sekund"aren Zellen zerrei"sen, jede in ihrer eigent"umlichen Weise."'

"`Auch in R"ucksicht auf die Organisation der Pflanzenzelle im Allgemeinen sind manche dieser Vegetationen der Milchs"aurezellen von gro"sem Interesse."'

"`Diejenigen n"amlich, welche die oben beschriebenen Kristalloid-Zellchen enthalten, sind auch an der inneren Oberfl"ache jeder der beiden in einander geschachtelten Zellh"aute, welche die Wandung bilden, mit einer Schichte kleiner Zellchen belegt, die, entweder eng beisammen liegend und an einander abgeplattet, oder etwas von einander entfernt, dem ganzen Zellsysteme das Ansehen und die Struktur einer kleinnetzig, warzig oder por"os verdickten Parenchym-Zelle geben. \emph{De Cella Vitali} 1843. Ges. Beilage pag. 37 und 437. Diese Zellchen, morphologisch den Sekretion-Zellchen der zusammengesetzten Pflanze gleichwertig, wachsen gleichzeitig mit ihrer Mutterzelle zu der Gr"o"se heran, dass die zwischen der prim"aren und sekund"aren Zelle liegenden eine Epidermis bilden. Das ganze Zellsystem ist oft h"ochst "ahnlich, mit der Au"senhaut vieler Pollen- und Diatomaceen- (\emph{Gallionella}, \emph{Biddulphia}, \emph{Coscinodiscus}, \emph{Triceratium}, \emph{Amphitetras} etc.) Zellen."'

"`Wird ein solches von aufgenommenem Eisen rotgef"arbtes Zellsystem in eine neue Mischung der oben bezeichneten N"ahrstoffl"osung ohne Eisen gelegt, so zerf"allt es bald in seine Elemente. Die Zellchen, welche dasselbe zusammensetzen, sowohl die kristalloidischen Inhaltszellchen als auch die der Oberhaut beginnen sich abzurunden und sich etwas zu Vergr"o"serern; es entstehen neue Generationen in ihnen, die endlich frei werden, indem ihre Spezialmutterzelle verfl"ussigt wird, und so sieht man sie bei Monate hindurch fortgesetzter Beobachtung sich in der Weise der Unterhefe mikrosporonartig, d. h. durch Entwicklung freier Tochterzelien vermehren."'

"`Diese mit milchsaurem Eisen durchdrungenen, warzig verdickten Oidiumszellen waren es auch, an welchen ein Hervorwachsen von sehr langgestielten Inhaltszellchen, aus oder neben den Zellchen, welche die netzig-warzige Oberhaut darstellen, beobachtet wurde, nach Art des \emph{Micrococcus}, der Vibrionenkeime."'

"`Auch Hyphomyzeten, besonders \emph{Penicillium} und \emph{Botrytis}, sowie \emph{Rhizopus} gaben, nachdem sie einige Zeit mit Milchzucker ern"ahrt vegetierten und darauf mit metallischem Eisen in Ber"uhrung gebracht wurden, sehr interessante Pr"aparate, zum Teil "ahnlich denen des \emph{Oidium} mit angeschwollenen Gonidienketten oder Hyphengliedzellen. An den Gonidienketten von \emph{Penicillium} schwellen in der Regel die obersten "altesten Gonidien zuerst etwas an, dann folgen nach und nach die unteren. Die in Milchzuckerl"osung mit N"ahrstoffsalzen ges"attigten und bald darauf mit Eisen in Ber"uhrung gebrachten \emph{Penicillium}-Gonidien schwellen langsam an und entwickeln an der inneren Oberfl"ache ihrer nach und nach au"serordentlich vergr"o"serten und verdickten Au"senhaut zahlreiche Zellchen, die derselben ein netzigoder por"os verdicktes Ansehen geben, so dass dadurch Formen entstehen, die den oben von \emph{Oidium} beschriebenen, por"os dickwandigen "ahnlich sind. In andern F"allen f"ullen die Tochterzellen mehr die H"ohlung an und werden einem mit Gonidien gef"ullten Mucork"opfchen "ahnlich."'

"`Sehr h"aufig finden sich auch hier wie bei \emph{Oidium}, wenn es mager kultiviert war, inhaltsleere Zellen mit ganz glatten Wandungen. Nicht selten durchbricht die innere, mit milchsaurem Eisen durchtr"ankte Zelle die "au"sere einfache oder auch zellig-warzig-etc. verdickte Haut, welche abbl"attert oder zerspaltet, w"ahrend jene hervorw"achst."'

"`Die f"ur diesen Zweck angestellten Kulturen d"urfen nicht feucht gehalten, nur in feuchter Luft unternommen werden, da diese mit saurem Eisensalze durchdrungenen Vegetationen dem Zerflie"sen sehr ausgesetzt sind. Auch ohne solche vorg"angige Kultur habe ich die Gliedzellchen und Gonidien genannter Schimmel, sowie im Staube enthaltene \emph{Micrococcus}-Zellen und Vibrionenkeime in beschriebener Weise anschwellen sehen, wenn sie mit poliertem metallischem Eisen in Ber"uhrung gebracht wurden, ohne Zweifel, weil diese Zellchen S"auren oder saure Salze enthielten."'

"`Wird es aus den eben mitgeteilten Erscheinungen des Wachstums dieser Pilzzellen ersichtlich, dass es deren assimilierende Membranen sind, welche die zerflie"sende S"aure bilden, so ist die Ursache der abnormen Vergr"o"serung dieser Zellen in der nachtr"aglichen Verbindung dieser S"aure mit dem neutralen milchsauren Eisen zu einem sauren Salze zu suchen, so dass also die ganze Erscheinung der merkw"urdigen Missbildung auf einem rein chemischen Prozesse beruht, der denjenigen, welcher in den unter normalen Bedingungen vegetierenden Zellen stattfindet, in der Weise "andert, dass die normale Entwicklung eine krankhafte wird, welche die endliche Zerst"orung des Organismus herbeif"uhrt."'

"`Gegen die Idee, dass die S"aure hier bei den Pilzen ebenso wie das Harz, Wachs etc. durch die Assimilitations-T"atigkeit der Zellmenbran entstehe, k"onnte noch das Bedenken erhoben werden, dass es vielleicht die Sekretionszellchen (Microgonidien, Vibrionenkeime) allein seien, welche zwischen diesen Membranen des Zellensystems (der in einander geschachtelten Zellen 1., 2., 3. etc. Grads) wie oben bemerkt eingeschlossen, diese organischen S"auren durch ihre vegetative T"atigkeit erzeugen, um so mehr, da ohne Zweifel die Vibrionen, die sich aus ihnen entwickeln, auch bei v"olliger Abwesenheit von entwickelteren Zellenformen sehr energische Erzeuger von S"auren, z. B. von Milch-, Butter-, Essigs"aure sind. Dagegen sprechen jedoch diejenigen durch Aufnahme von Eisen in gleicher Weise vergr"o"serten Zellen, deren Wandung durchaus strukturlos ist, d. h. ohne erkennbar zellige Organisationen zwischen den beiden sie zusammensetzenden Membranen der in einander geschachtelten Zellen und ohne eingeschlossene freie Zellchen in ihrer H"ohlung; ferner die Tatsache, dass von dem \emph{Oidium}-Mycelium und deren Hefezellen, wenn dieselben untergetaucht sich entwickeln, zuerst die Membranen, dann erst der fl"ussige Inhalt, der sich au"serhalb der Kernzelle befindet, durch Eisen- und Schwefel-Ammonium geschw"arzt werden. Gegen andere Metalle, gegen Aluminium, Magnesium, Zink, Kobalt, Nickel, selbst gegen Kupfer verhalten sich diese Milchs"aurezellen "ahnlich wie gegen Eisen, bilden mit demselben jedoch farblose oder nur schwach gef"arbte, zum Teil (besonders mit Kupfer) sehr leicht zerflie"sliche Organisationen. Zu Versuchen mit dieser S"aurehefe sind daher diese Metalle weniger g"unstig."'

Ich denke, wenn vor unserem Auge Eisenpflanzen entstehen, sollte man ein Bedenken gegen die Annahme desselben Vorgangs zu einer fr"uheren Zeit, zu einer Zeit, als s"amtliche Stoffe der organischen Bildung zur Verf"ugung waren, nicht erheben. Haben wir heute noch Massenbildungen vor uns in den Atollen des stillen Meeres, haben wir in den Chondriten die Zusammensetzung aus "ahnlichen Tieren, wie dort nachgewiesen: was steht im Wege, vorhergehende Pflanzenmassenbildungen anzunehmen?

Endlich haben wir in der Hefebildung einen Vorgang, welcher vollst"andig analog ist, sobald nur die Gluthitze weggedacht wird.

Ich komme hier auf die Kant-Laplace'sche Hypothese von der Massenbildung zur"uck. Oben schon habe ich ihren logischen Fehler erwiesen. Wie will man aus der Dunstmasse, welche sicher auch das Wasser einschloss, einen gl"uhenden Ball herausbringen? Oder soll die Erde erst, nachdem sie gebildet war, in Glut gekommen sein? Nun wodurch? Die Erfahrung spricht blo"s f"ur Massenbildung auf organischem Wege. Offenbar hat nur der Anblick der Vulkane dazu gef"uhrt, ein feuerfl"ussiges Erdinneres anzunehmen, und diese Vorstellung f"uhrte zu der Annahme, dass die ganze Erde einmal in diesem Zustande gewesen und dass die plutonischen Gesteine die Produkte jener Periode seien. Auch ist es ja keineswegs gewiss, dass der W"armestrahl der Sonne von einem feuerfl"ussigen K"orper herr"uhre. Wenn aber auch, so spricht eben die Tatsache der Losl"osung unserer Erde mit dem Wasser und insbesondere des Mondes (ohne Atmosph"are!) daf"ur, dass die Masse von Anfang an eine feuerfl"ussige feste Masse nicht gewesen und eine solche auch nicht geworden sein kann.

Soviel ist jedenfalls gewiss, dass das Meteoreisen nicht ein Schmelzprodukt ist, und was sollte das Meteoreisen in Glut versetzt haben? Ich habe auch im Meteoreisen Crinoiden- und Schwammformen gefunden. Ganz unzweifelhaft zeigt Hainholz solche.

Zeigen aber schon die Pallasite organische und sogar tierische Formen, Gesteine, welche den "Ubergang von reinem Eisen zum Chondrit bilden, so ist auch kein Grund vorhanden, das reine Eisen f"ur eine unorganische Bildung, noch weniger aber, einen ehemals fl"ussigen Zustand desselben anzunehmen.

Sobald das Eisen als Planetenkern angenommen wird, glaube ich es hiermit aber als im h"ochsten Grade wahrscheinlich aussprechen zu d"urfen, dass der erste Anfang unseres und daher aller Planeten eine organische Bildungwar.
\clearpage
\section{Das Eisen von Ovifak\index{Ovifak}}
\paragraph{}
Durch die G"ute des Herrn Professors Dr. von Nordenskj"old wurden mir 6 St"ucke des Eisens von Ovifak und des Basalts, in welchem dasselbe gefunden wurde, zur Untersuchung gegeben.

[Friedrich] W"ohler (Neues Jahrbuch f"ur Mineralogie 1869, S. 32) h"alt es auf Grund seiner chemischen Zusammensetzung nicht f"ur meteoritisch. Das Vorkommen eines der mir vorliegenden St"ucke in einer Kluft spricht ebenfalls nicht f"ur meteoritischen Ursprung. Eisenteile mit Widmannst"atten'schen Figuren finden sich auch im Basalt und im Olivingestein eingewachsen, und doch werden beide nicht als meteoritisch angesprochen. Endlich finden sich v"ollige "Uberg"ange von Stein in Eisen, woraus hervorgeht, dass das Eisen nicht zuf"allig in den Basalt gefallen ist. Es w"are doch ein gro"ses Wunder, wenn dieses Eisen gerade zu der Zeit, als der Basalt fl"ussig war, in denselben gefallen w"are, ganz abgesehen davon, dass dieses Eisen, wie festgestellt ist, sich kaum einige Jahre erhalten w"urde. --- Und doch soll dieses Eisen seiner Struktur wegen meteoritisch sein.

Wir wissen aber, dass unser Erdkern mindestens von der Dichtigkeit dieses Metalls ist, und es wird derselbe wahrscheinlich auch aus Eisen von derselben Beschaffenheit bestehen, so dass die Wahrscheinlichkeit nahe l"age, dass wir in dem Eisen von Ovifak den Eisenkern der Erde zu Tage treten sehen.

Damit w"are uns unendlich mehr gewonnen, als mit einem neuen Meteoriten.

Auf der Fl"ache dieses Eisens, das ich freilich, da ich dieses schreibe, anzugreifen die Erlaubnis noch nicht habe, finde ich Strukturen, welche denen der Crinoiden\index{Crinoid} in den Chondriten sehr "ahnlich sind.

Eine Untersuchung im D"unnschliffe aber muss ich auf die Zeit aufsparen, wo mir das Material zur freien Verf"ugung gestellt wird.
\clearpage
\section{Schlussfolgerungen}
\subsection{Ursprung der Meteorite\index{Meteorite}}
\paragraph{}
Dass kleine Planeten, Planeten im Gewicht von 1/2 Kilogramm auf die Erde fallen und solche daher auch kreisen, ist ganz gewiss. Es lassen sich nun folgende M"oglichkeiten denken:
\begin{enumerate}
\item die Meteorite\index{Meteorite} kreisen au"serhalb des Sonnensystems (ein solcher will einmal von Petit in Toulouse beobachtet worden sein),
\item die Meteorite\index{Meteorite} kreisen innerhalb des Sonnensystems und zwar: f"ur sich um die Sonne, --- um die Sonne mit Planeten (vielleicht also auch einzelne mit der Erde) --- um die Sonne, die Planeten und deren Trabanten,
\item die Meteorite\index{Meteorite} kreisen in allen diesen Bahnen.
\end{enumerate}
\paragraph{}
Man weis aus langj"ahrigen Beobachtungen jetzt sicher, dass in gewissen Zeitabschnitten (10. August, 13. November) Schw"arme von Meteoriten unserer Erde sich n"ahern und unsere Erdbahn schneiden; weis dass diese Schw"arme in gewissen Jahren zahlreicher sind, als in andern, weis, dass einzelne Meteorite\index{Meteorite} auf unsere Erde fallen, eine Tatsache, welche ihren Grund in der Anziehung der Erde hat. --- Die Bahnen der Meteorite\index{Meteorite} aber sind noch nicht festgestellt, weder die der Schw"arme, noch die von einzelnen; weder von solchen, welche gefallen, noch von solchen, welche blo"s an der Erde vorbeigezogen sind. Somit l"asst sich aus den Bahnen, welche man nicht kennt, nichts f"ur die Entstehung der Meteoriten ableiten.

Nun fragt es sich, was aus der Zusammensetzung der Meteorite\index{Meteorite} folgt. Ihre chemischen Elemente sind dieselben, wie die unserer Erde. Diese Tatsache l"asst sich nun auf gemeinsame Entstehung, also darauf deuten, dass die Erde mit den Meteoriten Eine Masse gebildet habe, wie darauf, dass die Entstehung und Entwicklung aller Planeten dieselbe sei. Die blo"se Tatsache der chemischen Gleichheit l"asst also verschiedene Folgerungen offen. Nun habe ich aber irdische Organismen in den Meteoriten nachgewiesen und es kann noch nicht einmal als gewiss angenommen werden, dass die nicht "ubereinstimmenden auf der Erde nicht auch vorkommen. --- Zu meinem Bedauern muss ich es gestehen, dass die Zahl der Zweifel durch meine Entdeckung eben nur vermehrt worden ist.

Aufs Neue erheben sich jetzt die Fragen: Entstanden die Meteorite\index{Meteorite} mit der Erde? Kommen sie von der Erde? Waren sie also von Anfang an mit der Erde eine Masse und wurden von ihr getrennt, so dass sie vielleicht eine Art unsichtbarer Trabanten derselben gewesen w"aren oder gar noch sind?

Ich hebe zun"achst nur diese Fragen hervor, denn sie sind f"ur die Geologie die wichtigsten. Das spezifische Gewicht der Erde und das Gestein von Ovifak machen es wahrscheinlich, dass die Erde ganz aus denselben Gesteinen zusammengesetzt ist wie die Meteorite\index{Meteorite}, vorausgesetzt, dass Eisen- und Stein-Meteorite\index{Meteorite} zusammengeh"oren. Daraus lie"se sich schlie"sen, dass die Meteorite\index{Meteorite} urspr"unglich ein Teil der Erde gewesen, und zwar zur Zeit, als die Erdbildung bis zu den Olivinschichten vorgeschritten war, und dass sie jetzt erst von ihr losgel"ost worden seien. Letzteres m"usste geschehen sein in Folge des Sto"ses eines Weltk"orpers auf die Erde, denn ohne einen solchen w"are eine Trennung nicht zu erkl"aren, es m"usste denn die Erdanziehung pl"otzlich aufgeh"ort, oder doch in so hohem Grade sich gemindert haben, dass ein Teil ihrer Masse aus ihrem Anziehungskreis hinausgeschleudert werden konnte. --- An ein Zerspringen, also an einen Sto"s von innen durch Gaskraft und dergleichen ist schwer zu glauben, obgleich auch das nicht v"ollig ausgeschlossen w"are.

Man kann also auch jetzt aus chemischen und morphologischen Gr"unden so wenig als aus der Gesteinsbeschaffenheit einen Schluss ziehen, ob die Meteorite\index{Meteorite} Kinder oder Br"uder der Erde sind und man ist zun"achst auf den Ausspruch des Astronomen angewiesen.

Wenn nun aber dieser best"atigt, dass die Meteorite\index{Meteorite} verm"oge ihrer Bahnen nicht ein Teil der Erdmasse gewesen sein k"onnen, so treten zweitens die Fragen ein: wie verhalten sich die einzelnen F"alle zu einander? Sind die Steine und Eisen urspr"unglich zusammengeh"orig, oder haben Steine und Eisen verschiedenen Ursprung? Und drittens w"are die Frage: haben wenigstens die chemisch und morphologisch gleichen Steine Einem Planeten angeh"ort, welcher durch irgend eine Ursache in Tr"ummer ging?

Letzteres k"onnte auf den ersten Anblick eben aus der chemisch morphologischen "ahnlichkeit gefolgert werden und in der Tat, die Sache schiene ganz einfach und klar. Aber es w"are doch noch eine andere M"oglichkeit, die M"oglichkeit, dass unter gleichen Bedingungen sich eine Unzahl kleiner Planeten bilden k"onnte und vielleicht heute noch bildet. Die St"ucke w"aren dann nicht Tr"ummer, sondern eigene Weltk"orper.

Eisen und Steine k"onnten nun eigene Weltk"orper sein --- die Gr"o"se allein st"unde der Annahme nicht im Wege. --- Wenn aber die kleinen Massen aus Wassergesch"opfen bestehen und sie bestehen ja auch aus einer blo"s mikroskopischen Sch"opfung --- so fragt es sich: lebten diese im Wasser oder im Wasserdampf? Gen"ugte ihnen ein fortw"ahrender Niederschlag von Wasser, wie wir ihn sehr leicht uns denken k"onnen, da wir heute noch Gegenden auf unserer Erde haben, wo stets Regen f"allt wie in anderen kein Regentropfen. Dieser Frage ist entgegen zu halten, dass auch zu der mikroskopischen Sch"opfung Baustoffe notwendig waren, welche nicht unter, sondern "uber den Gesch"opfen gesucht werden m"ussen, denn nur aus w"assrigen L"osungen konnte sich die mikroskopische Tierwelt aufbauen.

Diese Tierwelt ist aber schon eine wenigstens zum Teil h"oher organisierte. Eine einzellige Pflanze, ein Hefenpilz mag der Anfang eines Planeten gewesen sein: ein Crinoid\index{Crinoid} konnte es aus inneren Gr"unden nicht sein, denn hier m"ussen wir einen l"angeren Zeitraum und daher auch eine gr"o"sere Masse uns denken, durch welche diese Stufe der Entwicklung erreicht werden konnte.

Diese Tatsachen leiten uns in Verbindung mit der Wahrscheinlichkeit, dass Eisen der Kern des Chondrit-Planeten gewesen sei, dahin: die Chondrite als Tr"ummer eines und desselben Weltk"orpers anzusehen, Tr"ummer, welche nach der Zerst"orung des Planeten kreisten, bis sie gl"ucklicherweise in den Fallkreis unserer Erde kamen. Auch die Formen der Meteorite\index{Meteorite} selbst sprechen endlich f"ur Tr"ummer.

Wir haben also nur eine hypothetische Gewissheit: n"amlich die Wahrscheinlichkeit der urspr"unglichen Zusammengeh"origkeit der zu uns gelangten Tr"ummer.

Sollten sie aber auch von unserer Erde gekommen, Teile derselben gewesen sein: ihre Zusammensetzung aus Organismen ist immerhin noch eine Tatsache, welche wichtig genug w"are f"ur unsere Erdgeschichte. Stammen sie aber nicht von der Erde, so geben sie uns die Erkl"arung zweier Tatsachen: die Entstehung eines Planeten und die Wahrscheinlichkeit f"ur die Art und Weise der Entstehung unserer Erde. Waren sie aber jeder ein Planet f"ur sich, so bezeugen sie eine Sch"opfungskraft, welche wirklich unsere Begriffe von der Entstehung organischer Formen und deren Verlauf weit hinter sich lie"se.
\clearpage
\subsection{Die Erdbildung}
\paragraph{}
Anschlie"send an die bisherigen Resultate lie"sen sich auch f"ur die Erdbildung einige Schl"usse ziehen. H"ochst wahrscheinlich zeigt der Erddurchschnitt dieselbe Gesteins-Reihenfolge, wie die Meteorite\index{Meteorite}, welche vom Eisen zum Pallasite (Olivin mit Eisen), von da zu Olivin-, Enstatit-, (Feldspat)-Gestein "ubergehen.

Auf der Erde folgt dem Olivin der Granit, ein Feldspatgestein: diese Reihenfolge entspricht auch dem spezifischen Gewicht der Minerale.

Es haben Hornblende 3-3,40, Olivin 3,35, Enstatit 3,10-3,29, Orthoklas 2,53-3,10, Quarz 2-2,80 spezifisches Gewicht. Das hohe spezifische Gewicht der Hornblende r"uhrt offenbar noch von dem Eisengehalte her. Diese Aufeinanderfolge im Gewicht, wie in der Lagerung spricht ebenfalls entschieden f"ur Bildung im Wasser, in w"asseriger L"osung. Hier muss ich wiederholen, was ich schon in der \emph{Urzelle} sagte: die Sch"opfung, d. h. die organische Bildung kann nicht mit den Krebsen (Trilobiten) angefangen haben. Wir finden ja "uberall in den sp"ateren Schichten eine stete Entwicklungsreihe der Formen, warum sollte blo"s im Anfang dieses Gesetz nicht gewaltet haben?

Schon dieses w"urde zu der Annahme des organischen Ursprungs der unmittelbaren Vorl"aufer des Silur, des Gneises und des Granits f"uhren.

Mit dem Beweise der organischen Zusammensetzung der Chondrite ist das Hauptargument gefallen, welches bis daher im Wege stunde, den Granit f"ur ein Wassergebilde anzusehen: beide Gesteine enthalten vorzugsweise Feldspat. --- Was den Granit betrifft, so habe ich Formen darin gefunden, welche denen der Chondrite "ahnlich sind.

Ich will hier zum Beweis des Ursprungs des Granits nicht nur aus Wasser, sondern aus Organismen, einige Punkte nachtragen. Feldspat und Quarz kristallisieren, ich m"ochte sagen, leidenschaftlich. Im Granit finden sich aber beide Minerale regelm"a"sig nicht kristallisiert; der Feldspat zeigt blo"s einen Bl"atterbruch. Einen solchen zeigt aber auch jede in Kalk verwandelte Versteinerung, z. B. ein Crinoidenstiel. Warum kommt der Feldspat im Granit nicht kristallisiert vor? Weil er durch eine st"arkere formbildende Kraft gebunden war. Der Feldspat des Granits (wo letzterer wirklich erhalten ist) zeigt ferner stets bestimmte, stets wiederkehrende Formen, nicht Konglomerat- oder Roll-, auch, wie ich bemerkte, keine Kristall-Formen. --- Auch hier w"achst immer eine Form aus der andern heraus. Diese Formen sind Schwammformen. Der Quarz f"ullt die Hohlr"aume.

Auch auf die Gebirgsbildung m"ochte ich hinweisen. Dr. [Friedrich Moritz] Stapff, welcher den Gebirgsbau im Gotthard-Tunnel gewiss zur Gen"uge beobachtet hat, erkl"art (Neues Jahrbuch f"ur Mineralogie 1869, S. 792), dass er keine Spur einer Massen-Hebung oder Zertr"ummerung im Gotthard-Tunnel, dem gr"o"sten Aufschluss des Erdinnern den man kennt, beobachtet habe. Dieses "`Urgebirge"' ist nach seiner Feststellung ein Sedimentgebirge. Ja! es ist sogar denkbar, dass es sich gebildet hat, als unsere Atmosph"are noch den gr"o"sten Teil des Wassers in sich gefasst hielt, eine Atmosph"are, welche nicht durch Feuer im Erdinnern, wohl aber durch die chemische W"arme mehr erw"armt war als sie es heute ist. Ist dem aber so, so bleibt f"ur die Entstehung der Urgesteine, wie Urgebirge kein Erkl"arungsgrund als das organische Leben.

Heute noch k"onnen niedere Tiere und Pflanzen einen Hitzegrad ertragen, welcher f"ur andere Wesen absolut t"odlich wirkt, somit steht auch der Annahme organischen Lebens bei erh"ohtem W"armegrad nichts im Wege. Apatit und Graphit k"onnen ebenfalls als Zeugen organischer T"atigkeit gelten. Mit dem Niederschlag der Kieselerde (Kiesels"aure) war das Erdgerippe fertig: es bestand aus den Knochen der abgestorbenen Tiere; Ton, Kalk, Salz nebst Gasen und Wasser bildeten nun die Baustoffe f"ur die fernere T"atigkeit auf der Erd-Oberfl"ache. Weil dieser (nicht Erstarrungs-, sondern Niederschlags-) Prozess in der Hauptsache abgeschlossen war, erhielt nun der Organismus Raum und Zeit zu einer h"oheren Entwicklung, welche bis dahin unm"oglich war, denn jede neue Bildung begrub die kaum entstandene. Jetzt erst, nachdem eine schwer l"osliche Verbindung als Mantel um die Erde gelegt war, konnte die Formen-Entwicklung in ihre Rechte eintreten. Die Erdperioden wurden jetzt l"anger; mit dem Vorrat an feineren Baustoffen kam das Gesetz der Symmetrie in Geltung. Aber noch eine weitere Ursache trat hinzu: die niedersten Organismen sind Kinder der Nacht; ein Pilz erstirbt im Licht der Sonne. Die ganze bisherige Sch"opfung, bis zum Niederschlag der dichteren Baustoffe, war eine Nachtsch"opfung: die fortw"ahrenden chemischen Verbindungen mussten eine W"arme erzeugen, welche dem Wasser nicht gestattete, in dem Grade zum Meere zu werden wie heute. Endlich waren die chemischen Verbindungen in der Hauptsache abgeschlossen und es war dadurch eine Oberfl"ache, eine Art Schale geschaffen. Jetzt aber trat der Licht- und W"armestrahl der Sonne in Wirkung, welchem bis dahin der Weg bis zur festeren Oberfl"ache durch eine hohe und dichte Atmosph"are verschlossen war. Es beginnt die Lichtsch"opfung; das K"onigreich der Sonne hat das Reich der Nacht auf unserem Erdball "uberwunden, hat die Nacht in die Tiefen der Erde gebannt.

So, durch das Licht, erkl"art sich nun auch das mit dem Silur pl"otzlich und m"achtig hervortretende h"ohere Leben: es war der erste Ruhepunkt der Sch"opfung. Unter dem Einfluss des Lichtes sehen wir nun eine Entwicklung beginnen, welche so weit von der fr"uheren Abstand, als heute das Leben am Pol absteht von dem am "aquator. So erkl"art sich auch die pl"otzliche "anderung. H"atte es sich blo"s um Abk"uhlung gehandelt, so m"usste die Sch"opfung einen viel langsameren "Ubergang aufweisen. Was nach dem Niederschlag des Magnesium, Silicium, Kalium, Natrium noch im Wasser gel"ost blieb, war verh"altnism"a"sig wenig; hier konnte nun das Licht anfangen zu wirken. Durch diese Annahme erkl"art sich allein, dass das Leben auf der ganzen Erde, dass auch auf ihrer ganzen Oberfl"ache Wasser war, sowie dass Wassertiere noch Gebirge aufbauen konnten, welche weit "uber den jetzigen Spiegel des Meeres reichen. Diese Gebirge sind nicht gehoben, auch nicht nach mechanischem Gesetze (durch Schwungkraft) hinaufgetrieben, ebensowenig durch Erkaltung der Oberfl"ache herausgepresst worden; denn als Letztere erkaltete (richtiger "`vertrocknete"'), konnten h"ochstens Spr"unge und Kl"ufte entstehen und unter der Oberfl"ache war kein Brei, sondern feste Masse. Was ist nun nach meinen jetzigen Feststellungen Oberfl"ache, jetzt nachdem die Grenze des Urgebirgs und der folgenden Schichten aufgehoben ist?\footnote{Man hat bei der Hebungstheorie vergessen, dass eine Gewalt, welche n"otig w"are, um Gebirge zu heben, diese zugleich zermalmt h"atte: bei der Pressungstheorie ist man nicht im Stande zu sagen, wo denn eigentlich das Gebirge geblieben ist, durch welches "`der Brei"' gepresst worden w"are! Die ganze Oberfl"ache kann doch nicht herausgepresst worden sein.} Was diese Schichte hinsichtlich ihrer Sch"opfung von dem Urgebirge scheidet, ist nur die Wirkung des Lichts, welche um so st"arker werden musste, je mehr sich die Wasserd"ampfe verdichteten und das Wasser die Kl"ufte des Erdballs ausf"ullte.

Nun aber w"aren die Tage der Erde doch gez"ahlt gewesen, wenn nicht eben durch das Licht gesorgt worden w"are, dass der Niederschlagsprocess sich nicht rasch vollendet, dass die einzigen noch "ubrigen chemischen Verbindungen sich nicht rasch vollzogen h"atten und damit das Leben der Erde und auf der Erde f"ur ewig zum Stillstand gebracht gewesen w"are. Die Sch"opfungen des Lichts waren neue, h"ohere Organismen. Diese Organismen bauten sich auf aus den noch nicht in organische Verbindungen getretenen Abfallstoffen der bisherigen Sch"opfung und dadurch wurde dem Tode Halt geboten. Dieser w"are eingetreten und die Erde w"are zur W"uste geworden, wenn nicht eben die durch das Licht geschaffenen Organismen mit ihrer Nahrung und durch ihre Einatmung Verbindungen eingingen und solche wieder l"osten und so einen Kreislauf, Leben genannt, bewirkten. Es ist also das Licht, welches unsere Erde vor dem Tode sch"utzt, der auf ihrem Satelliten schon eingetreten zu sein scheint. Das Licht aber wirkt durch das Wasser. Das Wasser verbindet den Stein und den "ather, und dies er"offnet uns den Blick in die Zukunft unseres Planeten.
\clearpage
\subsection{Die Zukunft Unseres Planeten}
\paragraph{}
Der Fall von Planeten-Tr"ummern auf unsere Erde, (f"ur diesen Ursprung der Meteorite\index{Meteorite} sprechen die meisten Gr"unde) lie"se ein mechanisches Enden, einen gewaltsamen Tod auch f"ur unsere Erde f"urchten. Geschah es jenem oder jenen Planeten, von welchen die Meteorite\index{Meteorite} herr"uhren, dass sie zertr"ummert wurden, und zwar wurden sie es wohl nicht durch eine Kraft von innen, sondern durch Ansto"s von au"sen: so m"ussten wir darauf gefasst sein, dass auch unserer Erde einmal dieses Schicksal widerfahren werde, wenigstens drohte es uns. Ich muss es den Astronomen "uberlassen, sich und ihre Zeitgenossen dar"uber zu tr"osten.

Aber auch auf das andere, oben schon angedeutete Aufh"oren des Lebens auf der Oberfl"ache m"ussten wir gefasst sein, allerdings ein weniger blutiges, aber darum nicht tr"ostlicheres Ende, n"amlich auf das Schicksal des allm"ahlichen Absterbens, des Erl"oschens der Lebenskraft durch die Verbindung der Baustoffe zu unl"oslichen Verbindungen: w"ur m"ussten f"urchten, es werde unsere Atmosph"are in der Bildung unl"oslicher Verbindungen aus den noch "ubrigen Baustoffen fortfahren und es werde mit dem Verlust an verf"ugbarem Baustoff der Kreislauf ein stets schw"acherer und langsamerer werden und endlich --- aufh"oren.

Vor diesem sonst fast vorausberechenbaren Verlaufe bewahrt uns einzig und allein --- das Wasser; das Wasser, welches unsere Erde in ihrer Bildung sich anzueignen und festzuhalten vermochte.

Dadurch, dass die geschaffenen Wesen selbst die Verbindungen wieder l"osen, welche sich in ihren K"orpern bilden --- dass also insbesondere die Pflanze das was sie aufsaugt, selbst wieder zerlegt, w"ahrend das Thier diese Ausscheidungen in sich aufnimmt, um sie dann alsbald wieder auszuscheiden und der Pflanze (nicht dem Boden) zur"uckzugeben: durch all diess ist ein Kreislauf geschaffen, dessen Ende nicht abzusehen ist.

Dieser Vorgang und nicht die Abk"uhlung der Erdrinde, von welcher so viel geredet worden ist, bildet die wahre Geschichte unserer Erdoberfl"ache. Allerdings scheinen wir an unserem Trabanten, dem Mond, ein schreckendes Beispiel zu haben: Dort, glaube ich, ist das Leben erloschen. Nicht Meere sind dort, wie man glaubte und nicht Vulkane waren es, sondern der Mangel oder der Verlust des Wassers wird es gewesen sein, was diesem Planeten einen vorzeitigen Tod bereitete, was das Leben bald nach der Geburt wieder verl"oschen lie"s.\footnote{Nicht die Abnahme der Erdw"arme oder der von der Sonne ausgestrahlten W"arme w"are das n"achst drohende Schreckgespenst, sondern das Verschwinden unserer Atmosph"are.}

Die W"arme auf unserer Erdoberfl"ache scheint mir mehr von der Erhaltung der die K"alte des Weltraums abwehrenden Atmosph"are abzuh"angen. Die gr"o"sere H"ohe der Erdatmosph"are am "aquator in Folge der Drehung der Erde und nicht der Ausfallswinkel der Sonnenstrahlen allein ist die Ursache der dort h"oheren und konstanten W"arme: sonst w"are unter dem "aquator 500m "uber dem Meere nicht schon eine Abk"uhlung von mehreren Graden Durchschnittsw"arme; sonst m"usste die Schneemasse des Chimborasso sofort schmelzen.

Mag nun auch die W"arme in Folge der vom Wasser vermittelten chemischen Prozesse mit der Zeit abnehmen, soviel ist gewiss, dass unsere Erdoberfl"ache ohne den sch"utzenden Mantel der Atmosph"are, trotzdem sie tagt"aglich neue Sonnenw"arme aufnimmt, doch schon bei Nacht einer so niederen Temperatur verfiele, dass sie das Leben nicht erhalten k"onnte, wie dies neuerer Zeit als Ursache des Erl"oschens alles Lebens auf dem Monde behauptet wird.

Die W"arme str"omt uns von der Sonne zu und wird durch die Atmosph"are zur"uckgehalten, so dass sie nicht sofort, wie sie da ist, wieder in den Weltraum ausstr"omen kann. So sind wir von einem doppelten, sch"utzenden Mantel umgeben: der Erdrinde, welche die W"arme aufsaugt, und der Luft, welche sie zur"uckh"alt, (sie ist das Kleid der Erde), und zwischen beiden leben wir, lebt die ganze Sch"opfung im steten Austausch der Stoffe. Hier lebt der Mensch, hier entstehen dieselben Wesen, welche einst den ersten Grundstein zum gro"sen Bau der Erde gelegt haben. Und gerade diese niederen Wesen bezeugen heute noch durch ihre riesenhafte Vermehrung, durch ihre Erhaltung in einer Temperatur, in welcher h"ohere Wesen sofort sterben, dass sie f"ahig waren die ersten Bildner der Erde selbst zu sein.

Also nur, wenn die Quelle des Lichts und der W"arme selbst versiegte, m"usste das Leben auf der Erde erstarren; vom Erl"oschen des fraglich feurigen Erdkerns haben wir nichts zu f"urchten. F"ur die Erhaltung des Lebens sorgt der Stoffwechsel unter dem Einfluss der licht- und w"armestrahlenden Sonne. Licht und W"arme sind also Vater und Mutter alles Lebendigen; sie verhindern, dass das Organische vor der Zeit zum Unorganischen werde, indem sie letzteres stets wieder zu neuen Verbindungen f"uhren. M"ochte aber auch noch so viel Licht und W"arme der Erde zustr"omen, ohne die fortw"ahrende T"atigkeit, ohne die Umbildung durch die organische Zelle w"are doch das Leben unseres Planeten nach Jahren zu z"ahlen.

Der Anfang des Planeten war die Zelle, sie erh"alt ihn, so lange noch ein Lichtstrahl die Erde trifft.

M"oglich ist es dass mit der Zeit doch "anderungen in der chemischen Zusammensetzung der Erdoberfl"ache und der Atmosph"are durch Niederschl"age und feste Verbindungen eintreten, wodurch Baustoffe aus dem Kreislauf ausgeschieden werden. Sicher aber werden unter solchen ver"anderten Lebensbedingungen auch andere, "ahnliche und (nach der bisherigen Erfahrung) h"oher organisierte Wesen entstehen. Ja es l"asst sich denken, dass hier auf der Erde eine Verfeinerung der Organismen eintreten werde in demselben Verh"altnis, wie sie nach der Olivin-Granitzeit eingetreten ist, dass Gesch"opfe entstehen, welchen zu ihrer Erhaltung in h"oherem Masse Wasser und Gase gen"ugen, was ja bei vielen Pflanzen jetzt schon nahezu der Fall ist.
\clearpage
\section{Erkl"arung der Tafeln}
\subsection{Vorbemerkung}
\paragraph{}
Die Steine, von welchen ich meine D"unnschliffe nahm, sind durchaus beglaubigte.

Die D"unnschliffe selbst sind von mir unter der unerm"udlichen Beihilfe meiner Schw"agerin, Fr"aulein Pauline Schloz, hergestellt. So bel"auft sich meine Sammlung auf 560 Nummern (worunter 360 Knyahinya), wohl die gr"o"ste Sammlung, welche es "uberhaupt gibt.

Bez"uglich der Herstellung der D"unnschliffe muss ich eines Umstands erw"ahnen, welcher auf die Darstellung von Einfluss war.

Jeder, welcher Versteinerungen geschliffen hat, weis, dass nur ganz wenige einen d"unnen Schliff gestatten. Nicht allein wegen des h"aufig opaken oder gar undurchsichtigen Materials (Kalk, Ton), sondern deshalb, weil die Struktur mit einem Male verschwindet, wenn sie bis zur (vermuteten) Durchsichtigkeit geschliffen werden.

Es h"angt das mit der Art und Weise der jedem Versteinerungsprozess zu Grunde liegenden Umbildung zusammen.

So ist man vor die Wahl gestellt, entweder einen ziemlich dunklen Schliff vor sich zu haben, worin man wenig sieht, oder --- von dem Wunsch nach sch"arferen Umrissen getrieben, wobei man stets vergeblich nach h"oheren Objektiven greift --- einen Schliff zu bekommen, welcher nichts mehr zeigt.

Diese beiden Klippen konnten bei dem Meteoriten-Material (welches, beil"aufig gesagt, wegen des Eisens dem Schliff ziemliche Schwierigkeiten entgegensetzt) nur dadurch vermieden werden, dass abwechslungsweise d"unnere und dickere Schliffe gefertigt wurden.

Was die Auswahl der Formen betrifft, so werden k"unftige Forscher mich wohl entschuldigen, wenn ich diese und jene Form "ubersah. Meine Absicht freilich war, s"amtliche Formen, welche in meinem Material enthalten sind, abzubilden. Die Abbildungen sollten nicht nur Bilder, sondern auch ein Gesamtbild geben: gerade darauf lege ich ja in der Schlussfrage "uber die Natur des Gesteins das gr"o"ste Gewicht.

Was die Anordnung der Tafeln betrifft, so h"angt diese mit der Anordnung des Stoffs zusammen. Da ich mir aber doch bewusst war, das ganze Material bei weitem noch nicht ersch"opft zu haben: so gab ich mir auch nicht die M"uhe, die einzelnen Formen zu bestimmen, oder Ansichten "uber den genetischen Zusammenhang derselben auszusprechen, zu begr"unden und hiernach die Anordnung zu treffen: es gen"ugte wohl eine vorl"aufige Orientierung in dieser Richtung. Heute handelt es sich doch vorerst nur um den Beweis, dass das Gestein organisch und nicht darum, was alles darin ist.

Namen zu geben, vermied ich nicht aus Furcht damit der Kritik in die H"ande zu fallen, sondern weil ich zur Einsicht kam, dass durch Namensgebung vorerst nichts oder nicht viel gewonnen ist.

Lange stand ich vor der Wahl, ob ich wirklich den Weg der photographischen Darstellung einschlagen solle. Ich kam aber mehr aus "au"seren R"ucksichten zu dem betreffenden Entschl"usse.

Es ist bei der Kritik meiner \emph{Urzelle} viel von Phantasie die Rede gewesen. Dass die Abbildungen nicht auf der H"ohe der Zeit stehen --- wusste ich: dass sie aber doch richtig sind, das mag z. B. die photographische Abbildung der Objekte meiner \emph{Urzelle} Tafel 32. Figur 5 verglichen mit Tafel 4. und 5. der \emph{Urzelle} ergeben.

Ich m"ochte hiebe noch Herrn Dr. Kuntze in Leipzig fragen, ob er solche k"unstliche Algen etwa beizubringen weis --- zutreffendenfalls w"are ich sehr dankbar f"ur "Uberlassung eines solchen Pr"aparats um mich von einem Irrtum zu "uberzeugen.\footnote{"Ahnlich ist Dr. Kuntze mit der Flora Columbiae von Dr. H. Karsten verfahren. Ehe sich derselbe gegen die Anschuldigung, welche Dr. W. Joos auf diese Kritiken hin gegen ihn erhoben, reinigt, hat er kein Recht mehr, in der Wissenschaft geh"ort zu werden.} Meines Wissens sind die Dendriten und "`k"unstlichen Algen"', welche mir so ohne alle Pr"ufung und Kenntnis entgegengehalten wurden, blo"s Streifen ohne Gliederung und Absonderung. Ihrer Entstehung entsprechend ist es eine meist gleichm"a"sig verteilte, zusammenh"angende Farbstoffmasse, welche zwischen zwei Stein-Platten liegt, also in einer vollkommenen Fl"ache, und so Pflanzenschatten gleicht.

Ich gebe zu, dass "`k"unstliche Algen"' nach den Begriffen gewisser Forscher von Algen gemacht werden k"onnen. Aber ich muss auch darauf hinweisen, dass alle Gebilde, welche fadenoder bandartig sind, ohne viel Besinnens bisher f"ur Algen erkl"art wurden. Um zu wissen, dass man eine Alge vor sich habe, geh"ort noch etwas mehr dazu. So sind Dinge f"ur Pflanzen erkl"art worden, welche sicher nicht halb so viel Form und Struktur zeigen, als meine Bilder in der \emph{Urzelle}. Nicht alle Faden- oder B"undelformen in Gesteinen oder anderen Massen w"urde ich, auf dieses Merkmal allein hin, f"ur Algen erkl"aren.

Meine Abbildungen in der \emph{Urzelle} zeigen deutliche Zellenw"ande und Zellen; w"aren diese Dinge k"unstliche Algen oder Dendriten, so k"onnten sie keine Querw"ande haben.

Hiermit kehre ich zu meinem Gegenstand zur"uck.

Die Photographie hat gro"se Nachteile f"ur die wissenschaftliche Darstellung, das weis jeder Forscher. Bei dem vorliegenden Gegenstand aber musste ich diesen Weg gehen, einfach weil mir sonst wieder von "`Phantasie"' h"atte gesprochen werden k"onnen. Die Sonne und das Kollodium zusammen t"auschen nicht und m"ussen jeden derartigen Vorwurf von vornherein von mir abwenden. --- Wohl aber enth"alt das photographische Bild weniger als der Gegenstand. Das wurde besonders bei meinen besten Objekten f"uhlbar. Es konnte ferner besonders bei h"oheren Vergr"o"serungen nur ein Teil des Schliffs zur Darstellung gebracht werden, aber auch dieser nicht scharf, weil das dar"uber- und darunterliegende Gestein das eingestellte Bild verwischte. Zu hohe Vergr"o"serungen (das bemerke ich etwaigen Mitarbeitern an der Sache) taugen deshalb durchaus nicht f"ur Gesteinsd"unnschliffe. Ein weiterer hindernder Umstand ist, dass die Gesteine aus stark lichtbrechenden und das Licht verschieden brechenden Mineralien bestehen; dadurch entstehen Lichtreflexe der unangenehmsten Art, welche ein Unge"ubter leicht f"ur Formen ansehen kann. Um dies zu vermeiden, habe ich mich stets der schw"achsten Vergr"o"serungen bedient und habe unvollkommene Strukturbilder zur"uckgelegt.

Die photographischen Bilder stehen also eher unter dem Objekt. Allein, wie gesagt, ich musste wegen der Glaubw"urdigkeit der Darstellung diesen Weg gehen.

Eine Ursache weiterer besonders empfindlicher M"angel der photographischen Darstellung besteht in der Wirkung der Farben auf das Bild. Unter den schlimmen ist Gelb die schlimmste.

Wo Gelb im Pr"aparat ist, erscheint statt aller Struktur ein schwarzer Fleck. Mit keinem Mittel war diesem "Ubel abzuhelfen. Und gerade das Gelb des Olivins ist dasjenige, welches absolut keinen Lichtstrahl durchl"asst. Das macht sich am meisten geltend bei der Koralle, Tafel 1. Figur 6: der schwarze Ring auf dem Bilde ist ein lichtes Gelb (Eisen). --- Dem Gelb folgt Braun, welches ebenfalls sehr dunkelt. Blau hat den entgegengesetzten Fehler, es wird zu licht, doch zeigt es noch Strukturen.

Dass der hohe Preis des Materials gewisse Sparsamkeit in den Pr"aparaten auferlegt, ist selbstredend. Es ist dadurch die Auswahl beschr"ankt. Gerade dieser Umstand ist der Grund, dass die Schliffe von dem Forscher selbst hergestellt werden m"ussen. Das ist eine Aufgabe. Aber es ist auch nur so ein gr"undliches, freilich durch gro"sen Zeitaufwand erschwertes Studium der Sache m"oglich.

Zur Vergr"o"serung und photographischen Darstellung habe ich mich des mittleren mikrophotographischen Apparats von Seibert \& Krafft in Wetzlar bedient und kann denselben nur r"uhmlich empfehlen. Die Bilder wurden unter meiner Leitung im photographischen Atelier der Herren Otto Lauer \& Carl Bossler hier gefertigt. Da wir alle noch keine "Ubung in dieser Art Aufnahmen hatten, so war die Beihilfe des Herrn Dr. Schreiner , Assistenten am chemischen Laboratorium in T"ubingen eine "au"serst erw"unschte. Weitere Hilfe habe ich nicht zu verzeichnen, wohl aber glaube ich nicht unerw"ahnt lassen zu d"urfen, die v"ollige Teilnahmslosigkeit aller derjenigen Gelehrten, welche die Sache am meisten ber"uhrt.

Bei der Anordnung des Stoffs habe ich die Schw"amme vorangestellt, diesen die Korallen und dann die Crinoiden\index{Crinoid} folgen lassen.

Entsprechend der H"aufigkeit des Vorkommens habe ich auch die einzelnen Gattungen in der Zahl sich vertreten lassen. Leider musste ich manches bessere Objekt wegen der gelben F"arbung zur"ucklegen. Wenn es sich bew"ahrt, was G"umbel in seiner trefflichen Abhandlung "uber die bayerischen Meteoriten sagt, dass es ihm gelungen sei, die gelbe Farbe durch S"auren zu entfernen, so w"are viel gewonnen.

Was die Vergr"o"serungen betrifft, oder richtiger das Format der Vergr"o"serungen, so kam in Betracht, dass eben die Einrichtung der Kamera die Einhaltung eines bestimmten Formats auferlegt. Das f"uhrt zu dem Missstand, dass die Formen zuletzt alle gleich gro"s erscheinen.

Die Angabe der Vergr"o"serung, d. h. das Verh"altnis der wahren Gr"o"se zum Durchmesser des dargestellten Bildes ist also ein sehr wenig bezeichnendes.

Ich habe daher vorgezogen mit der Angabe des Durchmessers jeder Form die wirkliche Gr"o"se des Objekts unmittelbar zu bezeichnen.
\clearpage
\subsection{Tafelverzeichniss}
\begin{enumerate}
    \item Die Nummerierung der Abbildungen geschieht von links oben nach rechts unten.
    \item Abk"urzungen: V. hei"st Vergr"o"serung, D. hei"st wirklicher Durchmesser, mm hei"st Millimeter.
\end{enumerate}
\clearpage
\pagestyle{fancy}
\fancyhf{}
\rhead{Tafel 1: Zusammenstellung von Mineral-Strukturen mit organischen aus Chondriten\index{chondrite}}
\cfoot{\thepage}
\begin{figure}[b]
\includegraphics[width=\textwidth,height=\textheight,keepaspectratio]{figures/meteorite_1-1_edit-b2.jpg}
\caption{Tafel 1: Figur 1 --- Enstatit (-Bronzit) vom Kupferberg V.}
\centering
\end{figure}
\clearpage
\begin{figure}[t]
\includegraphics[width=\textwidth,height=\textheight,keepaspectratio]{figures/meteorite_1-2_edit-b.jpg}
\caption{Tafel 1: Figur 2 --- Enstatit von Texas V.}
\centering
\end{figure}
\clearpage
\begin{figure}[t]
\includegraphics[width=\textwidth,height=\textheight,keepaspectratio]{figures/meteorite_1-3_edit-b2.jpg}
\caption{Tafel 1: Figur 3 --- Spherulite-Liparite\index{spherulite} from Lipari M.}
\centering
\end{figure}
\clearpage
\begin{figure}[t]
\includegraphics[width=\textwidth,height=\textheight,keepaspectratio]{figures/meteorite_1-4_edit-b.jpg}
\caption{Tafel 1: Figur 4 --- ein Theil der Coralle Taf. 8, 9 und 10}
\centering
\end{figure}
\clearpage
\begin{figure}[t]
\includegraphics[width=\textwidth,height=\textheight,keepaspectratio]{figures/meteorite_1-5_edit-b2.jpg}
\caption{Tafel 1: Figur 5 --- Kettenkoralle D. 0,90 mm.}
\centering
\end{figure}
\clearpage
\begin{figure}[t]
\includegraphics[width=\textwidth,height=\textheight,keepaspectratio]{figures/meteorite_1-6_edit-b2.jpg}
\caption{Tafel 1: Figur 6 --- Crinoid\index{Crinoid} D. 1,20 mm.}
\centering
\end{figure}
\clearpage
\rhead{Tafel 2: \emph{Urania}\index{Urania}}
\begin{figure}[t]
\includegraphics[width=\textwidth,height=\textheight,keepaspectratio]{figures/meteorite_2-1_edit-b2.jpg}
\caption{Tafel 2: Figur 1 --- aus Knyahinya. Dieselbe Tafel 5. Fig. 1.}
\centering
\end{figure}
\clearpage
\rhead{Tafel 3: \emph{Urania}\index{Urania}}
\begin{figure}[t]
\includegraphics[width=\textwidth,height=\textheight,keepaspectratio]{figures/meteorite_3-1_edit-b.jpg}
\caption{Tafel 3: Figur 1 --- aus Knyahinya D. 0,60 mm.}
\centering
\end{figure}
\clearpage
\begin{figure}[t]
\includegraphics[width=\textwidth,height=\textheight,keepaspectratio]{figures/meteorite_3-2_edit-b.jpg}
\caption{Tafel 3: Figur 2 --- aus Knyahinya D. 1,30 mm. (man "ubersehe nicht die prachtvollen Crinoidenglieder links oben!)}
\centering
\end{figure}
\clearpage
\begin{figure}[t]
\includegraphics[width=\textwidth,height=\textheight,keepaspectratio]{figures/meteorite_3-3_edit-b.jpg}
\caption{Tafel 3: Figur 3 --- aus Knyahinya D. 1 mm.}
\centering
\end{figure}
\clearpage
\begin{figure}[t]
\includegraphics[width=\textwidth,height=\textheight,keepaspectratio]{figures/meteorite_3-4_edit-b.jpg}
\caption{Tafel 3: Figur 4 --- aus Knyahinya D. 1 mm.}
\centering
\end{figure}
\clearpage
\begin{figure}[t]
\includegraphics[width=\textwidth,height=\textheight,keepaspectratio]{figures/meteorite_3-5_edit-b2.jpg}
\caption{Tafel 3: Figur 5 --- aus Knyahinya D. 1 mm. (zu beachten die Schichtung oben)}
\centering
\end{figure}
\clearpage
\begin{figure}[t]
\includegraphics[width=\textwidth,height=\textheight,keepaspectratio]{figures/meteorite_3-6_edit-b2.jpg}
\caption{Tafel 3: Figur 6 --- aus Knyahinya D. 1 mm. (Schichtung wie 5, doch im Bilde nicht wiedergegeben, 5 und 6 aus einem D"unnschliff.)}
\centering
\end{figure}
\clearpage
\rhead{Tafel 4: \emph{Urania}\index{Urania}}
\begin{figure}[t]
\includegraphics[width=\textwidth,height=\textheight,keepaspectratio]{figures/meteorite_4-1_edit-b.jpg}
\caption{Tafel 4: Figur 1 --- aus Knyahinya D. 0,90 mm.}
\centering
\end{figure}
\clearpage
\begin{figure}[t]
\includegraphics[width=\textwidth,height=\textheight,keepaspectratio]{figures/meteorite_4-2_edit-b.jpg}
\caption{Tafel 4: Figur 2 --- aus Siena D. 3 mm. (der dunkle Strich r"uhrt von gelber F"arbung des Pr"aparats)}
\centering
\end{figure}
\clearpage
\begin{figure}[t]
\includegraphics[width=\textwidth,height=\textheight,keepaspectratio]{figures/meteorite_4-3_edit-b.jpg}
\caption{Tafel 4: Figur 3 --- aus Knyahinya D. 0,60 mm.}
\centering
\end{figure}
\clearpage
\begin{figure}[t]
\includegraphics[width=\textwidth,height=\textheight,keepaspectratio]{figures/meteorite_4-4_edit-b.jpg}
\caption{Tafel 4: Figur 4 --- aus Knyahinya D. 0,90 mm. (Luftblase)}
\centering
\end{figure}
\clearpage
\begin{figure}[t]
\includegraphics[width=\textwidth,height=\textheight,keepaspectratio]{figures/meteorite_4-5_edit-b.jpg}
\caption{Tafel 4: Figur 5 --- aus Knyahinya D. 1,60 mm.}
\centering
\end{figure}
\clearpage
\begin{figure}[t]
\includegraphics[width=\textwidth,height=\textheight,keepaspectratio]{figures/meteorite_4-6_edit-b.jpg}
\caption{Tafel 4: Figur 6 --- aus Knyahinya D. 1,00 mm. (Luftblase)}
\centering
\end{figure}
\clearpage
\rhead{Tafel 5: \emph{Urania}\index{Urania}}
\begin{figure}[t]
\includegraphics[width=\textwidth,height=\textheight,keepaspectratio]{figures/meteorite_5-1_edit-b.jpg}
\caption{Tafel 5: Figur 1 --- aus Knyahinya D. 1,40 mm. (siehe Tafel 2. Rings Crinoidendurchschnitte. Form unten links, vergl. Tafel 1. Fig. 6 und Tafel 25. 1, 2)}
\centering
\end{figure}
\clearpage
\begin{figure}[t]
\includegraphics[width=\textwidth,height=\textheight,keepaspectratio]{figures/meteorite_5-2_edit-b2.jpg}
\caption{Tafel 5: Figur 2 --- aus Knyahinya D. 1,80 mm.}
\centering
\end{figure}
\clearpage
\begin{figure}[t]
\includegraphics[width=\textwidth,height=\textheight,keepaspectratio]{figures/meteorite_5-3_edit-b.jpg}
\caption{Tafel 5: Figur 3 --- aus Knyahinya D. 1,80 mm.}
\centering
\end{figure}
\clearpage
\begin{figure}[t]
\includegraphics[width=\textwidth,height=\textheight,keepaspectratio]{figures/meteorite_5-4_edit-b.jpg}
\caption{Tafel 5: Figur 4 --- aus Knyahinya D. 1,30 mm. (undeutliches Bild)}
\centering
\end{figure}
\clearpage
\begin{figure}[t]
\includegraphics[width=\textwidth,height=\textheight,keepaspectratio]{figures/meteorite_5-5_edit-b2.jpg}
\caption{Tafel 5: Figur 5 --- aus Knyahinya D. 1,40 mm. (Luftblase)}
\centering
\end{figure}
\clearpage
\begin{figure}[t]
\includegraphics[width=\textwidth,height=\textheight,keepaspectratio]{figures/meteorite_5-6_edit-b2.jpg}
\caption{Tafel 5: Figur 6 --- aus Knyahinya D. 0,60 mm. (mangelhaftes Bild. Der wei"se Ring ist der Durchschnitt)}
\centering
\end{figure}
\clearpage
\rhead{Tafel 6: \emph{Urania}\index{Urania}}
\begin{figure}[t]
\includegraphics[width=\textwidth,height=\textheight,keepaspectratio]{figures/meteorite_6-1_edit-b2.jpg}
\caption{Tafel 6: Figur 1 --- aus Siena D. 4,00 mm.}
\centering
\end{figure}
\clearpage
\begin{figure}[t]
\includegraphics[width=\textwidth,height=\textheight,keepaspectratio]{figures/meteorite_6-2_edit-b.jpg}
\caption{Tafel 6: Figur 2 --- aus Knyahinya D. 0,80 mm.}
\centering
\end{figure}
\clearpage
\begin{figure}[t]
\includegraphics[width=\textwidth,height=\textheight,keepaspectratio]{figures/meteorite_6-3_edit-b.jpg}
\caption{Tafel 6: Figur 3 --- aus Siena D. 1,20 mm.}
\centering
\end{figure}
\clearpage
\begin{figure}[t]
\includegraphics[width=\textwidth,height=\textheight,keepaspectratio]{figures/meteorite_6-4_edit-b.jpg}
\caption{Tafel 6: Figur 4 --- aus Knyahinya D. 0,70 mm. (in der Mitte zu stark beleuchtet)}
\centering
\end{figure}
\clearpage
\begin{figure}[t]
\includegraphics[width=\textwidth,height=\textheight,keepaspectratio]{figures/meteorite_6-5_edit-b.jpg}
\caption{Tafel 6: Figur 5 --- aus Knyahinya D. 0,30 mm.}
\centering
\end{figure}
\clearpage
\begin{figure}[t]
\includegraphics[width=\textwidth,height=\textheight,keepaspectratio]{figures/meteorite_6-6_edit-b2.jpg}
\caption{Tafel 6: Figur 6 --- aus Knyahinya D. 0,90 mm. (Luftblase)}
\centering
\end{figure}
\clearpage
\rhead{Tafel 7: Schw"amme}
\begin{figure}[t]
\includegraphics[width=\textwidth,height=\textheight,keepaspectratio]{figures/meteorite_7-1_edit-b.jpg}
\caption{Tafel 7: Figur 1 --- aus Knyahinya D. 2,30 mm.}
\centering
\end{figure}
\clearpage
\begin{figure}[t]
\includegraphics[width=\textwidth,height=\textheight,keepaspectratio]{figures/meteorite_7-2_edit-b.jpg}
\caption{Tafel 7: Figur 2 --- aus Knyahinya D. 1,80 mm. (ein Riss im Pr"aparat. Die Nadeln)}
\centering
\end{figure}
\clearpage
\begin{figure}[t]
\includegraphics[width=\textwidth,height=\textheight,keepaspectratio]{figures/meteorite_7-3_edit-b.jpg}
\caption{Tafel 7: Figur 3 --- aus Knyahinya D. 2,10 mm.}
\centering
\end{figure}
\clearpage
\begin{figure}[t]
\includegraphics[width=\textwidth,height=\textheight,keepaspectratio]{figures/meteorite_7-4_edit-b.jpg}
\caption{Tafel 7: Figur 4 --- (Crinoid\index{Crinoid}-Querschnitt?) aus Knyahinya D. 3,00 mm.}
\centering
\end{figure}
\clearpage
\begin{figure}[t]
\includegraphics[width=\textwidth,height=\textheight,keepaspectratio]{figures/meteorite_7-5_edit-b.jpg}
\caption{Tafel 7: Figur 5 --- Schwamm? D. 1,00 mm.}
\centering
\end{figure}
\clearpage
\begin{figure}[t]
\includegraphics[width=\textwidth,height=\textheight,keepaspectratio]{figures/meteorite_7-6_edit-b.jpg}
\caption{Tafel 7: Figur 6 --- Schwamm? D. 2,40 mm.}
\centering
\end{figure}
\clearpage
\rhead{Tafel 8: Korallen}
\begin{figure}[t]
\includegraphics[width=\textwidth,height=\textheight,keepaspectratio]{figures/meteorite_8-1_edit-b2.jpg}
\caption{Tafel 8: Figur 1 --- (\emph{Favosites}) aus Knyahinya (vergl. Tafel 1: Figur 4)}
\centering
\end{figure}
\clearpage
\rhead{Tafel 9: Korallen}
\begin{figure}[t]
\includegraphics[width=\textwidth,height=\textheight,keepaspectratio]{figures/meteorite_9-1_edit-b3.jpg}
\caption{Tafel 9: Figur 1 --- Strukturbild aus links oben Tafel 8.}
\centering
\end{figure}
\clearpage
\rhead{Tafel 10: Korallen}
\begin{figure}[t]
\includegraphics[width=\textwidth,height=\textheight,keepaspectratio]{figures/meteorite_10-1_edit-b.jpg}
\caption{Tafel 10: Figur 1 --- aus Knyahinya Querschnitt D. 0,40 mm.}
\centering
\end{figure}
\clearpage
\begin{figure}[t]
\includegraphics[width=\textwidth,height=\textheight,keepaspectratio]{figures/meteorite_10-2_edit-b.jpg}
\caption{Tafel 10: Figur 2 --- L"angsschnitt 0,50 mm.}
\centering
\end{figure}
\clearpage
\begin{figure}[t]
\includegraphics[width=\textwidth,height=\textheight,keepaspectratio]{figures/meteorite_10-3_edit-b2.jpg}
\caption{Tafel 10: Figur 3 --- aus Knyahinya D. 1,80 mm.}
\centering
\end{figure}
\clearpage
\begin{figure}[t]
%this figure has the same figure as Tafel 1: Figur 5, which has better quality
\includegraphics[width=\textwidth,height=\textheight,keepaspectratio]{figures/meteorite_1-5_edit-b2.jpg}
\caption{Tafel 10: Figur 4 --- aus Knyahinya D. 0,90 mm. (siehe Tafel 8. 9.)}
\centering
\end{figure}
\clearpage
\begin{figure}[t]
\includegraphics[width=\textwidth,height=\textheight,keepaspectratio]{figures/meteorite_10-5_edit-b.jpg}
\caption{Tafel 10: Figur 5 --- aus Knyahinya D. 0,30 mm.}
\centering
\end{figure}
\clearpage
\begin{figure}[t]
\includegraphics[width=\textwidth,height=\textheight,keepaspectratio]{figures/meteorite_10-6_edit-b.jpg}
\caption{Tafel 10: Figur 6 --- aus Knyahinya D. 0,80 mm.}
\centering
\end{figure}
\clearpage
\rhead{Tafel 11: Korallen}
\begin{figure}[t]
\includegraphics[width=\textwidth,height=\textheight,keepaspectratio]{figures/meteorite_11-1_edit-b.jpg}
\caption{Tafel 11: Figur 1 --- aus Knyahinya D. 1,20 mm.}
\centering
\end{figure}
\clearpage
\begin{figure}[t]
\includegraphics[width=\textwidth,height=\textheight,keepaspectratio]{figures/meteorite_11-2_edit-b.jpg}
\caption{Tafel 11: Figur 2 --- aus Knyahinya D. 1,00 mm.}
\centering
\end{figure}
\clearpage
\begin{figure}[t]
\includegraphics[width=\textwidth,height=\textheight,keepaspectratio]{figures/meteorite_11-3_edit-b.jpg}
\caption{Tafel 11: Figur 3 --- aus Knyahinya D. 1,80 mm.}
\centering
\end{figure}
\clearpage
\begin{figure}[t]
\includegraphics[width=\textwidth,height=\textheight,keepaspectratio]{figures/meteorite_11-4_edit-b.jpg}
\caption{Tafel 11: Figur 4 --- aus Knyahinya D. 1,20 mm.}
\centering
\end{figure}
\clearpage
\begin{figure}[t]
\includegraphics[width=\textwidth,height=\textheight,keepaspectratio]{figures/meteorite_11-5_edit-b.jpg}
\caption{Tafel 11: Figur 5 --- aus Parnallee D. 0,80 mm.}
\centering
\end{figure}
\clearpage
\begin{figure}[t]
\includegraphics[width=\textwidth,height=\textheight,keepaspectratio]{figures/meteorite_11-6_edit.jpg}
\caption{Tafel 11: Figur 6 --- aus Moung County D. 0,60 mm.}
\centering
\end{figure}
\clearpage
\rhead{Tafel 12: Korallen}
\begin{figure}[t]
\includegraphics[width=\textwidth,height=\textheight,keepaspectratio]{figures/meteorite_12-1_edit-b.jpg}
\caption{Tafel 12: Figur 1 --- aus Knyahinya D. 0,80 mm.}
\centering
\end{figure}
\clearpage
\begin{figure}[t]
\includegraphics[width=\textwidth,height=\textheight,keepaspectratio]{figures/meteorite_12-2_edit-b.jpg}
\caption{Tafel 12: Figur 2 --- aus Knyahinya D. 1,20 mm.}
\centering
\end{figure}
\clearpage
\begin{figure}[t]
\includegraphics[width=\textwidth,height=\textheight,keepaspectratio]{figures/meteorite_12-3_edit-b.jpg}
\caption{Tafel 12: Figur 3 --- aus Knyahinya D. 1,30 mm.}
\centering
\end{figure}
\clearpage
\begin{figure}[t]
\includegraphics[width=\textwidth,height=\textheight,keepaspectratio]{figures/meteorite_12-4_edit-b.jpg}
\caption{Tafel 12: Figur 4 --- aus Knyahinya D. 1,40 mm.}
\centering
\end{figure}
\clearpage
\begin{figure}[t]
\includegraphics[width=\textwidth,height=\textheight,keepaspectratio]{figures/meteorite_12-5_edit-b.jpg}
\caption{Tafel 12: Figur 5 --- aus Knyahinya D. 2,00 mm.}
\centering
\end{figure}
\clearpage
\begin{figure}[t]
\includegraphics[width=\textwidth,height=\textheight,keepaspectratio]{figures/meteorite_12-6_edit-b.jpg}
\caption{Tafel 12: Figur 6 --- aus Knyahinya D. 3,20 mm.}
\centering
\end{figure}
\clearpage
\rhead{Tafel 13: Korallen}
\begin{figure}[t]
\includegraphics[width=\textwidth,height=\textheight,keepaspectratio]{figures/meteorite_13-1_edit-b.jpg}
\caption{Tafel 13: Figur 1 --- aus Parnallee D. 0,20 mm.}
\centering
\end{figure}
\clearpage
\begin{figure}[t]
\includegraphics[width=\textwidth,height=\textheight,keepaspectratio]{figures/meteorite_13-2_edit-b.jpg}
\caption{Tafel 13: Figur 2 --- aus Knyahinya D. 0,80 mm.}
\centering
\end{figure}
\clearpage
\begin{figure}[t]
\includegraphics[width=\textwidth,height=\textheight,keepaspectratio]{figures/meteorite_13-3_edit-b.jpg}
\caption{Tafel 13: Figur 3 --- aus Siena D. 0,20 mm.}
\centering
\end{figure}
\clearpage
\begin{figure}[t]
\includegraphics[width=\textwidth,height=\textheight,keepaspectratio]{figures/meteorite_13-4_edit-b.jpg}
\caption{Tafel 13: Figur 4 --- aus Knyahinya D. 1,80 mm.}
\centering
\end{figure}
\clearpage
\begin{figure}[t]
\includegraphics[width=\textwidth,height=\textheight,keepaspectratio]{figures/meteorite_13-5_edit-b.jpg}
\caption{Tafel 13: Figur 5 --- aus Knyahinya D. 1,70 mm.}
\centering
\end{figure}
\clearpage
\begin{figure}[t]
\includegraphics[width=\textwidth,height=\textheight,keepaspectratio]{figures/meteorite_13-6_edit-b.jpg}
\caption{Tafel 13: Figur 6 --- aus Cabarras D. 0,30 mm.}
\centering
\end{figure}
\clearpage
\rhead{Tafel 14: Korallen}
\begin{figure}[t]
\includegraphics[width=\textwidth,height=\textheight,keepaspectratio]{figures/meteorite_14-1_edit-b2.jpg}
\caption{Tafel 14: Figur 1 --- Koralle D. 0,90 mm.}
\centering
\end{figure}
\clearpage
\rhead{Tafel 15: Korallen}
\begin{figure}[t]
\includegraphics[width=\textwidth,height=\textheight,keepaspectratio]{figures/meteorite_15-1_edit-b3.jpg}
\caption{Tafel 15: Figur 1 --- Koralle. Strukturbild von 14. Der linke obere Teil des Pr"aparats, Vergr"o"serung 300, zeigt die Knospen-Kan"ale.}
\centering
\end{figure}
\clearpage
\rhead{Tafel 16: Crinoiden\index{Crinoid}}
\begin{figure}[t]
\includegraphics[width=\textwidth,height=\textheight,keepaspectratio]{figures/meteorite_16-1_edit-b2.jpg}
\caption{Tafel 16: Figur 1 --- aus Knyahinya D. 0,40 mm.}
\centering
\end{figure}
\clearpage
\rhead{Tafel 17: Crinoiden\index{Crinoid}}
\begin{figure}[t]
\includegraphics[width=\textwidth,height=\textheight,keepaspectratio]{figures/meteorite_17-1_edit-b2.jpg}
\caption{Tafel 17: Figur 1 --- aus Knyahinya D. 2,00 mm.}
\centering
\end{figure}
\clearpage
\rhead{Tafel 18: Crinoiden\index{Crinoid}}
\begin{figure}[t]
\includegraphics[width=\textwidth,height=\textheight,keepaspectratio]{figures/meteorite_18-1_edit-b2.jpg}
\caption{Tafel 18: Figur 1 --- aus Knyahinya, 4 Hauptarme durchschnitten, D. 2,20 mm.}
\centering
\end{figure}
\clearpage
\rhead{Tafel 19: Crinoiden\index{Crinoid}}
\begin{figure}[t]
\includegraphics[width=\textwidth,height=\textheight,keepaspectratio]{figures/meteorite_19-1_edit-b2.jpg}
\caption{Tafel 19: Figur 1 --- Crinoid, vergl. Tafel 25. 1 und 2.}
\centering
\end{figure}
\clearpage
\rhead{Tafel 20: Crinoiden\index{Crinoid}}
\begin{figure}[t]
\includegraphics[width=\textwidth,height=\textheight,keepaspectratio]{figures/meteorite_20-1_edit-b2.jpg}
\caption{Tafel 20: Figur 1 --- Crinoid und Koralle durchschnitten aus Knyahinya D. 1,20 mm.}
\centering
\end{figure}
\clearpage
\rhead{Tafel 21: Crinoiden\index{Crinoid}}
\begin{figure}[t]
\includegraphics[width=\textwidth,height=\textheight,keepaspectratio]{figures/meteorite_21-1_edit-b.jpg}
\caption{Tafel 21: Figur 1 --- aus Knyahinya D. 0,80 mm.}
\centering
\end{figure}
\clearpage
\begin{figure}[t]
\includegraphics[width=\textwidth,height=\textheight,keepaspectratio]{figures/meteorite_21-2_edit-b.jpg}
\caption{Tafel 21: Figur 2 --- vergr"o"sertes Bild von Figur 1}
\centering
\end{figure}
\clearpage
\begin{figure}[t]
\includegraphics[width=\textwidth,height=\textheight,keepaspectratio]{figures/meteorite_21-3_edit-b.jpg}
\caption{Tafel 21: Figur 3 --- aus Knyahinya D. 1,20 mm.}
\centering
\end{figure}
\clearpage
\begin{figure}[t]
\includegraphics[width=\textwidth,height=\textheight,keepaspectratio]{figures/meteorite_21-4_edit-b.jpg}
\caption{Tafel 21: Figur 4 --- vergr"o"sertes Bild von Figur 3}
\centering
\end{figure}
\clearpage
\begin{figure}[t]
\includegraphics[width=\textwidth,height=\textheight,keepaspectratio]{figures/meteorite_21-5_edit-b.jpg}
\caption{Tafel 21: Figur 5 --- aus Knyahinya D. 1,80 mm. (ich bemerke die "ahnlichkeit mit Figur 1)}
\centering
\end{figure}
\clearpage
\begin{figure}[t]
\includegraphics[width=\textwidth,height=\textheight,keepaspectratio]{figures/meteorite_21-6_edit-b.jpg}
\caption{Tafel 21: Figur 6 --- aus Knyahinya D. 0,30 mm. (die Mund"offnung zwischen den Armen sichtbar)}
\centering
\end{figure}
\clearpage
\rhead{Tafel 22: Crinoiden\index{Crinoid}}
\begin{figure}[t]
\includegraphics[width=\textwidth,height=\textheight,keepaspectratio]{figures/meteorite_22-1_edit-b.jpg}
\caption{Tafel 22: Figur 1 --- aus Knyahinya D. 0,50 mm.}
\centering
\end{figure}
\clearpage
\begin{figure}[t]
\includegraphics[width=\textwidth,height=\textheight,keepaspectratio]{figures/meteorite_22-2_edit-b.jpg}
\caption{Tafel 22: Figur 2 --- aus Knyahinya D. 0,60 mm.}
\centering
\end{figure}
\clearpage
\begin{figure}[t]
\includegraphics[width=\textwidth,height=\textheight,keepaspectratio]{figures/meteorite_22-3_edit-b.jpg}
\caption{Tafel 22: Figur 3 --- aus Knyahinya (Titelbild) D. 1,50 mm.}
\centering
\end{figure}
\clearpage
\begin{figure}[t]
\includegraphics[width=\textwidth,height=\textheight,keepaspectratio]{figures/meteorite_22-4_edit-b.jpg}
\caption{Tafel 22: Figur 4 --- aus Knyahinya D. 0,70 mm.}
\centering
\end{figure}
\clearpage
\begin{figure}[t]
\includegraphics[width=\textwidth,height=\textheight,keepaspectratio]{figures/meteorite_22-5_edit-b.jpg}
\caption{Tafel 22: Figur 5 --- aus Knyahinya D. 0,60 mm.}
\centering
\end{figure}
\clearpage
\begin{figure}[t]
\includegraphics[width=\textwidth,height=\textheight,keepaspectratio]{figures/meteorite_22-6_edit-b.jpg}
\caption{Tafel 22: Figur 6 --- aus Knyahinya D. 1,20 mm.}
\centering
\end{figure}
\clearpage
\rhead{Tafel 23: Crinoiden\index{Crinoid}}
\begin{figure}[t]
\includegraphics[width=\textwidth,height=\textheight,keepaspectratio]{figures/meteorite_23-1_edit-b.jpg}
\caption{Tafel 23: Figur 1 --- aus Knyahinya D. 0,90 mm.}
\centering
\end{figure}
\clearpage
\begin{figure}[t]
\includegraphics[width=\textwidth,height=\textheight,keepaspectratio]{figures/meteorite_23-2_edit-b.jpg}
\caption{Tafel 23: Figur 2 --- aus Knyahinya D. 1,60 mm.}
\centering
\end{figure}
\clearpage
\begin{figure}[t]
\includegraphics[width=\textwidth,height=\textheight,keepaspectratio]{figures/meteorite_23-3_edit-b.jpg}
\caption{Tafel 23: Figur 3 --- aus Knyahinya D. 1,00 mm.}
\centering
\end{figure}
\clearpage
\begin{figure}[t]
\includegraphics[width=\textwidth,height=\textheight,keepaspectratio]{figures/meteorite_23-4_edit-b.jpg}
\caption{Tafel 23: Figur 4 --- aus Knyahinya D. 1,40 mm.}
\centering
\end{figure}
\clearpage
\begin{figure}[t]
\includegraphics[width=\textwidth,height=\textheight,keepaspectratio]{figures/meteorite_23-5_edit-b.jpg}
\caption{Tafel 23: Figur 5 --- aus Knyahinya D. 1,30 mm.}
\centering
\end{figure}
\clearpage
\begin{figure}[t]
\includegraphics[width=\textwidth,height=\textheight,keepaspectratio]{figures/meteorite_23-6_edit-b.jpg}
\caption{Tafel 23: Figur 6 --- aus Knyahinya D. 0,60 mm.}
\centering
\end{figure}
\clearpage
\rhead{Tafel 24: Crinoiden\index{Crinoid}}
\begin{figure}[t]
\includegraphics[width=\textwidth,height=\textheight,keepaspectratio]{figures/meteorite_24-1_edit-b.jpg}
\caption{Tafel 24: Figur 1 --- aus Siena D. 0,80 mm.}
\centering
\end{figure}
\clearpage
\begin{figure}[t]
\includegraphics[width=\textwidth,height=\textheight,keepaspectratio]{figures/meteorite_24-2_edit-b.jpg}
\caption{Tafel 24: Figur 2 --- aus Knyahinya D. 2,80 mm.}
\centering
\end{figure}
\clearpage
\begin{figure}[t]
\includegraphics[width=\textwidth,height=\textheight,keepaspectratio]{figures/meteorite_24-3_edit-b.jpg}
\caption{Tafel 24: Figur 3 --- aus Knyahinya D. 1,00 mm.}
\centering
\end{figure}
\clearpage
\begin{figure}[t]
\includegraphics[width=\textwidth,height=\textheight,keepaspectratio]{figures/meteorite_24-4_edit-b.jpg}
\caption{Tafel 24: Figur 4 --- aus Knyahinya D. 2,00 mm.}
\centering
\end{figure}
\clearpage
\begin{figure}[t]
\includegraphics[width=\textwidth,height=\textheight,keepaspectratio]{figures/meteorite_24-5_edit-b.jpg}
\caption{Tafel 24: Figur 5 --- aus Knyahinya D. 1,50 mm.}
\centering
\end{figure}
\clearpage
\begin{figure}[t]
\includegraphics[width=\textwidth,height=\textheight,keepaspectratio]{figures/meteorite_24-6_edit-b.jpg}
\caption{Tafel 24: Figur 6 --- aus Cabarras D. 0,80 mm.}
\centering
\end{figure}
\clearpage
\rhead{Tafel 25: Crinoiden\index{Crinoid}}
\begin{figure}[t]
\includegraphics[width=\textwidth,height=\textheight,keepaspectratio]{figures/meteorite_25-1_edit-b.jpg}
\caption{Tafel 25: Figur 1 --- aus Knyahinya D. 1,20 mm.}
\centering
\end{figure}
\clearpage
\begin{figure}[t]
\includegraphics[width=\textwidth,height=\textheight,keepaspectratio]{figures/meteorite_25-2_edit-b.jpg}
\caption{Tafel 25: Figur 2 --- aus Knyahinya D. 1,20 mm.}
\centering
\end{figure}
\clearpage
\begin{figure}[t]
\includegraphics[width=\textwidth,height=\textheight,keepaspectratio]{figures/meteorite_25-3_edit-b.jpg}
\caption{Tafel 25: Figur 3 --- aus Knyahinya D. 1,80 mm.}
\centering
\end{figure}
\clearpage
\begin{figure}[t]
\includegraphics[width=\textwidth,height=\textheight,keepaspectratio]{figures/meteorite_25-4_edit-b.jpg}
\caption{Tafel 25: Figur 4 --- aus Knyahinya D. 0,60 mm.}
\centering
\end{figure}
\clearpage
\begin{figure}[t]
\includegraphics[width=\textwidth,height=\textheight,keepaspectratio]{figures/meteorite_25-5_edit-b.jpg}
\caption{Tafel 25: Figur 5 --- aus Siena D. 1,80 mm.}
\centering
\end{figure}
\clearpage
\begin{figure}[t]
\includegraphics[width=\textwidth,height=\textheight,keepaspectratio]{figures/meteorite_25-6_edit-b.jpg}
\caption{Tafel 25: Figur 6 --- aus Knyahinya D. 1,40 mm. (Beide letztere Querschnitte von Crinoiden)}
\centering
\end{figure}
\clearpage
\rhead{Tafel 26: Crinoiden}
\begin{figure}[t]
\includegraphics[width=\textwidth,height=\textheight,keepaspectratio]{figures/meteorite_26-1_edit-b.jpg}
\caption{Tafel 26: Figur 1 --- aus Knyahinya\index{Meteorite!Knyahinya} D. 0,20 mm.}
\centering
\end{figure}
\clearpage
\begin{figure}[t]
\includegraphics[width=\textwidth,height=\textheight,keepaspectratio]{figures/meteorite_26-2_edit-b.jpg}
\caption{Tafel 26: Figur 2 --- aus Knyahinya\index{Meteorite!Knyahinya} D. 2,00 mm.}
\centering
\end{figure}
\clearpage
\begin{figure}[t]
\includegraphics[width=\textwidth,height=\textheight,keepaspectratio]{figures/meteorite_26-3_edit-b.jpg}
\caption{Tafel 26: Figur 3 --- aus Knyahinya\index{Meteorite!Knyahinya} D. 1,20 mm.}
\centering
\end{figure}
\clearpage
\begin{figure}[t]
\includegraphics[width=\textwidth,height=\textheight,keepaspectratio]{figures/meteorite_26-4_edit-b.jpg}
\caption{Tafel 26: Figur 4 --- aus Knyahinya\index{Meteorite!Knyahinya} D. 1,20 mm. (bis hierher gewundene Crinoiden\index{Crinoid})}
\centering
\end{figure}
\clearpage
\begin{figure}[t]
\includegraphics[width=\textwidth,height=\textheight,keepaspectratio]{figures/meteorite_26-5_edit-b.jpg}
\caption{Tafel 26: Figur 5 --- aus Knyahinya\index{Meteorite!Knyahinya} D. 2,00 mm.}
\centering
\end{figure}
\clearpage
\begin{figure}[t]
\includegraphics[width=\textwidth,height=\textheight,keepaspectratio]{figures/meteorite_26-6_edit-b.jpg}
\caption{Tafel 26: Figur 6 --- aus Knyahinya\index{Meteorite!Knyahinya} D. 2,20 mm. (die dunkle Linie in 5 und 6 ist der Nahrungskanal)}
\centering
\end{figure}
\clearpage
\rhead{Tafel 27: Crinoiden}
\begin{figure}[t]
\includegraphics[width=\textwidth,height=\textheight,keepaspectratio]{figures/meteorite_27-1_edit-b.jpg}
\caption{Tafel 27: Figur 1 --- aus Knyahinya\index{Meteorite!Knyahinya} D. 0,80 mm.}
\centering
\end{figure}
\clearpage
\begin{figure}[t]
\includegraphics[width=\textwidth,height=\textheight,keepaspectratio]{figures/meteorite_27-2_edit-b.jpg}
\caption{Tafel 27: Figur 2 --- aus Knyahinya\index{Meteorite!Knyahinya} D. 1,50 mm.}
\centering
\end{figure}
\clearpage
\begin{figure}[t]
\includegraphics[width=\textwidth,height=\textheight,keepaspectratio]{figures/meteorite_27-3_edit-b.jpg}
\caption{Tafel 27: Figur 3 --- aus Knyahinya\index{Meteorite!Knyahinya} D. 1,40 mm.}
\centering
\end{figure}
\clearpage
\begin{figure}[t]
\includegraphics[width=\textwidth,height=\textheight,keepaspectratio]{figures/meteorite_27-4_edit-b.jpg}
\caption{Tafel 27: Figur 4 --- aus Knyahinya\index{Meteorite!Knyahinya} D. 1,40 mm.}
\centering
\end{figure}
\clearpage
\begin{figure}[t]
\includegraphics[width=\textwidth,height=\textheight,keepaspectratio]{figures/meteorite_27-5_edit-b.jpg}
\caption{Tafel 27: Figur 5 --- aus Knyahinya\index{Meteorite!Knyahinya} D. 1,20 mm.}
\centering
\end{figure}
\clearpage
\begin{figure}[t]
\includegraphics[width=\textwidth,height=\textheight,keepaspectratio]{figures/meteorite_27-6_edit-b.jpg}
\caption{Tafel 27: Figur 6 --- aus Knyahinya\index{Meteorite!Knyahinya} D. 1,00 mm.}
\centering
\end{figure}
\clearpage
\rhead{Tafel 28: Crinoiden}
\begin{figure}[t]
\includegraphics[width=\textwidth,height=\textheight,keepaspectratio]{figures/meteorite_28-1_edit-b.jpg}
\caption{Tafel 28: Figur 1 --- aus Knyahinya\index{Meteorite!Knyahinya} (Coralle?) D. 3,00 mm. aus demselben D"unnschl. wie Tafel 18.}
\centering
\end{figure}
\clearpage
\begin{figure}[t]
\includegraphics[width=\textwidth,height=\textheight,keepaspectratio]{figures/meteorite_28-2_edit-b.jpg}
\caption{Tafel 28: Figur 2 --- aus Knyahinya\index{Meteorite!Knyahinya} D. 1,20 mm.}
\centering
\end{figure}
\clearpage
\begin{figure}[t]
\includegraphics[width=\textwidth,height=\textheight,keepaspectratio]{figures/meteorite_28-3_edit-b.jpg}
\caption{Tafel 28: Figur 3 --- aus Knyahinya\index{Meteorite!Knyahinya} D. 2,30 mm.}
\centering
\end{figure}
\clearpage
\begin{figure}[t]
\includegraphics[width=\textwidth,height=\textheight,keepaspectratio]{figures/meteorite_28-4_edit-b.jpg}
\caption{Tafel 28: Figur 4 --- aus Knyahinya\index{Meteorite!Knyahinya} D. 0,90 mm.}
\centering
\end{figure}
\clearpage
\begin{figure}[t]
\includegraphics[width=\textwidth,height=\textheight,keepaspectratio]{figures/meteorite_28-5_edit-b.jpg}
\caption{Tafel 28: Figur 5 --- aus Knyahinya\index{Meteorite!Knyahinya} D. 1,50 mm.}
\centering
\end{figure}
\clearpage
\begin{figure}[t]
\includegraphics[width=\textwidth,height=\textheight,keepaspectratio]{figures/meteorite_28-6_edit-b.jpg}
\caption{Tafel 28: Figur 6 --- aus Knyahinya\index{Meteorite!Knyahinya} D. 1,40 mm.}
\centering
\end{figure}
\clearpage
\rhead{Tafel 29: Crinoiden (1-3 von oben gesehen, 4 von unten.)}
\begin{figure}[t]
\includegraphics[width=\textwidth,height=\textheight,keepaspectratio]{figures/meteorite_29-1_edit-b.jpg}
\caption{Tafel 29: Figur 1 --- aus Knyahinya\index{Meteorite!Knyahinya} D. 0,20 mm.}
\centering
\end{figure}
\clearpage
\begin{figure}[t]
\includegraphics[width=\textwidth,height=\textheight,keepaspectratio]{figures/meteorite_29-2_edit-b.jpg}
\caption{Tafel 29: Figur 2 --- aus Knyahinya\index{Meteorite!Knyahinya} D. 0,90 mm.}
\centering
\end{figure}
\clearpage
\begin{figure}[t]
\includegraphics[width=\textwidth,height=\textheight,keepaspectratio]{figures/meteorite_29-3_edit-b.jpg}
\caption{Tafel 29: Figur 3 --- aus Tabor D. 2,10 mm.}
\centering
\end{figure}
\clearpage
\begin{figure}[t]
\includegraphics[width=\textwidth,height=\textheight,keepaspectratio]{figures/meteorite_29-4_edit-b.jpg}
\caption{Tafel 29: Figur 4 --- aus Knyahinya\index{Meteorite!Knyahinya} D. 1,10 mm.}
\centering
\end{figure}
\clearpage
\begin{figure}[t]
\includegraphics[width=\textwidth,height=\textheight,keepaspectratio]{figures/meteorite_29-5_edit-b.jpg}
\caption{Tafel 29: Figur 5 --- aus Borkut D. 1,50 mm.}
\centering
\end{figure}
\clearpage
\begin{figure}[t]
\includegraphics[width=\textwidth,height=\textheight,keepaspectratio]{figures/meteorite_29-6_edit-b.jpg}
\caption{Tafel 29: Figur 6 --- aus Knyahinya\index{Meteorite!Knyahinya} D. 1,30 mm. (zweifelhaft)}
\centering
\end{figure}
\clearpage
\rhead{Tafel 30: Crinoiden}
\begin{figure}[t]
\includegraphics[width=\textwidth,height=\textheight,keepaspectratio]{figures/meteorite_30-1_edit-b.jpg}
\caption{Tafel 30: Figur 1 --- aus Knyahinya\index{Meteorite!Knyahinya} D. 1,10 mm. (Koralle?)}
\centering
\end{figure}
\clearpage
\begin{figure}[t]
\includegraphics[width=\textwidth,height=\textheight,keepaspectratio]{figures/meteorite_30-2_edit-b.jpg}
\caption{Tafel 30: Figur 2 --- aus Knyahinya\index{Meteorite!Knyahinya} D. 1,40 mm. (Koralle und Crinoid\index{Crinoid}, vergl. Tafel 20.)}
\centering
\end{figure}
\clearpage
\begin{figure}[t]
\includegraphics[width=\textwidth,height=\textheight,keepaspectratio]{figures/meteorite_30-3_edit-b.jpg}
\caption{Tafel 30: Figur 3 --- aus Knyahinya\index{Meteorite!Knyahinya} D. 0,30 mm. (die Arme nezf"ormig verschlungen)}
\centering
\end{figure}
\clearpage
\begin{figure}[t]
\includegraphics[width=\textwidth,height=\textheight,keepaspectratio]{figures/meteorite_30-4_edit-b.jpg}
\caption{Tafel 30: Figur 4 --- aus Knyahinya\index{Meteorite!Knyahinya} D. 1,85 mm. (Anschnitt)}
\centering
\end{figure}
\clearpage
\begin{figure}[t]
\includegraphics[width=\textwidth,height=\textheight,keepaspectratio]{figures/meteorite_30-5_edit-b.jpg}
\caption{Tafel 30: Figur 5 --- aus Knyahinya\index{Meteorite!Knyahinya} D. 0,70 mm. (Anschnitt)}
\centering
\end{figure}
\clearpage
\begin{figure}[t]
\includegraphics[width=\textwidth,height=\textheight,keepaspectratio]{figures/meteorite_30-6_edit-b.jpg}
\caption{Tafel 30: Figur 6 --- aus Knyahinya\index{Meteorite!Knyahinya} D. 0,40 mm. (Struktur dem des Schreibersits im Meteoreisen gleich)}
\centering
\end{figure}
\clearpage
\rhead{Tafel 31: \emph{Problematica}}
\begin{figure}[t]
\includegraphics[width=\textwidth,height=\textheight,keepaspectratio]{figures/meteorite_31-1_edit-b.jpg}
\caption{Tafel 31: Figur 1 --- aus Knyahinya\index{Meteorite!Knyahinya} D. 1,20 mm. (nicht ganz vollst"andiges Bild)}
\centering
\end{figure}
\clearpage
\begin{figure}[t]
\includegraphics[width=\textwidth,height=\textheight,keepaspectratio]{figures/meteorite_31-2_edit-b.jpg}
\caption{Tafel 31: Figur 2 --- aus Knyahinya\index{Meteorite!Knyahinya} D. 0,50 mm.}
\centering
\end{figure}
\clearpage
\begin{figure}[t]
\includegraphics[width=\textwidth,height=\textheight,keepaspectratio]{figures/meteorite_31-3_edit-b.jpg}
\caption{Tafel 31: Figur 3 --- aus Knyahinya\index{Meteorite!Knyahinya} D. 1,20 mm. (Drei "ubereinstimmende Formen aus 3 D"unnschliffen, in 1 und 2 beidemale der horizontale Ausschnitt)}
\centering
\end{figure}
\clearpage
\begin{figure}[t]
\includegraphics[width=\textwidth,height=\textheight,keepaspectratio]{figures/meteorite_31-4_edit-b.jpg}
\caption{Tafel 31: Figur 4 --- aus Knyahinya\index{Meteorite!Knyahinya} (ob Schwamm oder Koralle?) D. 0,90 mm.}
\centering
\end{figure}
\clearpage
\begin{figure}[t]
\includegraphics[width=\textwidth,height=\textheight,keepaspectratio]{figures/meteorite_31-5_edit-b.jpg}
\caption{Tafel 31: Figur 5 --- aus Knyahinya\index{Meteorite!Knyahinya} D. 1,50 mm.}
\centering
\end{figure}
\clearpage
\begin{figure}[t]
\includegraphics[width=\textwidth,height=\textheight,keepaspectratio]{figures/meteorite_31-6_edit-b.jpg}
\caption{Tafel 31: Figur 6 --- aus Knyahinya\index{Meteorite!Knyahinya} D. 1,40 mm.}
\centering
\end{figure}
\clearpage
\rhead{Tafel 32: Verschieden}
\begin{figure}[t]
\includegraphics[width=\textwidth,height=\textheight,keepaspectratio]{figures/meteorite_32-1_edit-b.jpg}
\caption{Tafel 32: Figur 1 --- aus Knyahinya\index{Meteorite!Knyahinya} (Einschluss) D. 1,50 mm.}
\centering
\end{figure}
\clearpage
\begin{figure}[t]
\includegraphics[width=\textwidth,height=\textheight,keepaspectratio]{figures/meteorite_32-2_edit-b.jpg}
\caption{Tafel 32: Figur 2 --- Borkutkugel D. 1,00 mm.}
\centering
\end{figure}
\clearpage
\begin{figure}[t]
\includegraphics[width=\textwidth,height=\textheight,keepaspectratio]{figures/meteorite_32-3_edit-b.jpg}
\caption{Tafel 32: Figur 3 --- Nummulit von Kempten. Die Kan"ale sind (mit der Lupe) scharf zu erkennen}
\centering
\end{figure}
\clearpage
\begin{figure}[t]
\includegraphics[width=\textwidth,height=\textheight,keepaspectratio]{figures/meteorite_32-4_edit-b.jpg}
\caption{Tafel 32: Figur 4 --- D"unnschliff von Lias $\gamma\delta$. Dieser D"unnschliff ist der von mir zusammengestellten Sammlung von 30 D"unnschliffen von Sedimentgesteinen entnommen, gefertigt von Geognost Hildebrand in Ohmenhausen bei Reutlingen, welche ich zum Studium der mikroskopischen Beschaffenheit der Sedimentgesteine nebst Einschl"ussen dringend empfehle.}
\centering
\end{figure}
\clearpage
\begin{figure}[t]
\includegraphics[width=\textwidth,height=\textheight,keepaspectratio]{figures/meteorite_32-5_edit-b.jpg}
\caption{Tafel 32: Figur 5 --- \emph{Eozo"on canadense}\index{Eozo"on}, angebliches Kanalsystem des \emph{Eozo"on}\index{Eozo"on}.}
\centering
\end{figure}
\clearpage
\begin{figure}[t]
\includegraphics[width=\textwidth,height=\textheight,keepaspectratio]{figures/meteorite_32-6_edit-b.jpg}
\caption{Tafel 32: Figur 6 --- desgl. Beide Gesteine, denen die Schliffe entnommen sind, von mir in Little Nation gesammelt. Man vergleiche Kanalsystem des Nummuliten Fig. 3 mit diesem angeblichen Kanalsystem! Bild 3 und 5 sollen dasselbe Ding sein. Zu Fig. 5 vergleiche \emph{Urzelle} Tafel 4. 5.}
\centering
\end{figure}
\clearpage
\pagestyle{plain}
\printindex
\clearpage
\end{document}
