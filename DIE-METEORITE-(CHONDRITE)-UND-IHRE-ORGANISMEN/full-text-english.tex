\documentclass[a4paper, 12pt, oneside]{article}
\usepackage[utf8]{inputenc}
\usepackage{fouriernc}
\usepackage{booktabs}
\usepackage{url}
\usepackage{graphicx}
\setlength{\emergencystretch}{15pt}
\graphicspath{ {./figures/} }
\usepackage[figurename=]{caption}
\usepackage{fancyhdr}
\usepackage{imakeidx}
\makeindex[columns=2, title=Alphabetical Index, intoc]
\begin{document}
\begin{titlepage} % Suppresses headers and footers on the title page
	\centering % Centre everything on the title page
	\scshape % Use small caps for all text on the title page

	%------------------------------------------------
	%	Title
	%------------------------------------------------
	
	\rule{\textwidth}{1.6pt}\vspace*{-\baselineskip}\vspace*{2pt} % Thick horizontal rule
	\rule{\textwidth}{0.4pt} % Thin horizontal rule
	
	\vspace{0.75\baselineskip} % Whitespace above the title
	
	{\LARGE THE METEORITE (CHONDRITE)\\ AND\\ ITS ORGANISMS\\} % Title
	
	\vspace{0.75\baselineskip} % Whitespace below the title
	
	\rule{\textwidth}{0.4pt}\vspace*{-\baselineskip}\vspace{3.2pt} % Thin horizontal rule
	\rule{\textwidth}{1.6pt} % Thick horizontal rule
	
	\vspace{1\baselineskip} % Whitespace after the title block
	
	%------------------------------------------------
	%	Subtitle
	%------------------------------------------------
	
	{Presented and Described\\ by\\ \scshape\Large Dr. Otto Hahn\\} % Subtitle or further description
	
	\vspace*{1\baselineskip} % Whitespace under the subtitle
	
    {\small 32 Tables with 142 Photographs} % Subtitle or further description
    
	%------------------------------------------------
	%	Editor(s)
	%------------------------------------------------
	
	\vspace{1\baselineskip} % Whitespace before the editors

	Translated\\ by\\
	
	{\scshape Github Contributors \\ } % Editor list
			
	\vspace{1\baselineskip} % Whitespace below the editor list
	
    %------------------------------------------------
	%	Cover photo
	%------------------------------------------------
	
	\includegraphics[scale=0.95]{cover}
	
	%------------------------------------------------
	%	Publisher
	%------------------------------------------------
		
	\vspace{1\baselineskip} % Whitespace under the publisher logo
	
	German 1$^{st}$ Edition, Tübingen 1880 % Publication year
	
	{\small H. Laupp'schen Buchhandlung } % Publisher

	\vspace{1\baselineskip} % Whitespace under the publisher logo

    English 1$^{st}$ Edition, Internet Archive 2020 % Publication year
	
	{\small Attribution NonCommercial ShareAlike 4.0 International } % Publisher
\end{titlepage}
\setlength{\parskip}{1mm plus1mm minus1mm}
\setcounter{tocdepth}{2}
\setcounter{secnumdepth}{3}
\tableofcontents
\clearpage
\listoffigures
\clearpage
\section{Introduction}
\subsection{Introduction}
\paragraph{}
It was not the inconsequential attacks on \emph{Primordial Cell}\index{Primordial Cell} that gave me stamina to establish certain new geological facts, but rather it was the untenability of all previous views regarding that undisputedly most important part of the geological sciences, the part through which it relates to the cosmos --- that is, in the doctrine of the so-called plutonic rocks.

If, in the first part of \emph{Primordial Cell}\index{Primordial Cell} I had tolerated the doctrine and with resignation accepted that the core of our Earth\index{Earth}, and with it the knowledge pertaining to its real genesis, will always remain hidden from us, then, at the end of this book there is yet a possibility: the meteorite\index{meteorite} indicates a passage from far away, although not yet actively pursued by researchers.

With this as a guide, I would like to continue.

I did it, accompanied on the one hand by sharply pronounced ridicule from the specialists, and on the other with joy from my earlier results and the now daily support and counsel from the few friends whom I have succeeded in convincing.

The results yielded from this strenuous endeavor of almost super-human effort over the previous year are laid down in the following pages.

It is a world of animals in a rock that arrived on Earth\index{Earth} to bring us tidings from the smallest beings of a most distant place --- a life-world which a mortal eye could hardly hope to behold: a world of beings showing us the creative power that made our Earth\index{Earth} out of a nebula and has worked universally and evenly in the universe.

Admittedly, the meteorites, namely the chondrites\index{chondrite}, for these are the ones which are preferentially subject to my investigations, contain no life of higher construction; rather, all are lower life forms --- the same ones which prevail in the Silurian strata --- sponges, corals, and crinoids --- it is with these species that similar characteristics are found.

The chondrites that I have studied are olivine\index{olivine} enstatite\index{enstatite} rock. They have undergone alterations, although not considerable, since the time of their formation as the remains of life up until landing on Earth\index{Earth}. They have been permeated with a silicate solution, in a similar manner to how the Jura\index{Jura} deposits are with a solution of lime. While it was part of the parent body it probably underwent planetary cycles, just as new layers follow old ones down to the lowest strata on Earth\index{Earth}, under the influence of which the former have undergone a certain, though not as considerable as one assumes, transformation.

Only the surface of the meteorites\index{meteorite} has changed considerably, indeed, only at the last moment of their planetesimal existence and mostly due to the influence of frictional heating created, in this case, by the Earth's atmosphere\index{Earth!atmosphere}. But the original meteorite\index{meteorite} itself essentially remains. We now see that before us is a piece of a planet as it was in the process of becoming, and thus the history of our Earth's\index{Earth} body is now open to us, provided that we are correct that the meteorites\index{meteorite}, in their formation, are homogeneous in their chemical composition with the world matter that formed the Earth\index{Earth} and vice versa. At the same time, by sending me the ``Meteorite of Ovifak\index{Ovifak}''\index{meteorite} (I owe it to the kindness of Professor Dr. von Nordenskjöld) I was offered the opportunity of bringing this rock into the investigation.

I consider it to be terrestrial --- as part of the deepest layer of Earth\index{Earth}, i.e. the olivine\index{olivine} layer, which belongs under the granite. I call this original layer the Olivine\index{olivine} Formation. Since the rock is very similar to a meteorite\index{meteorite}, it is natural to declare it to be the same. The reasons why I do not consider it to be meteoritic but true to the Earth's core\index{Earth!core} are laid down in this book.

Thus, we have gained two solid points by which a lever can be set.

The chondrites\index{chondrite}, an olivine\index{olivine} feldspar (enstatite)\index{enstatite} rock, consist of an animal world; they are not part of a sedimentary rock layer nor a conglomerate, but a felt of animals, a fabric whose meshes were all once living beings and life of the lowest kind, the beginnings of creation.

However, I could not make a systematic enumeration of the life which is preserved in the meteorites\index{meteorite}: I just wanted to prove that it is so --- that is all. I therefore only depicted the organic beings that I was able to assure myself as determining undoubtedly: on the one hand the genera which coincide with terrestrial forms and, on the other, separating out the specifically meteoritic forms, while leaving both to future investigation.

It is to be expected that my enumeration will be, through further research and with the help of richer material than I have available, multiplied and supplemented. Subdivisions had to be avoided: since every newly discovered being would overturn any divisions and make the effort arduous with any work in vain.

This is the reason why I only made large divisions, and these only to the extent that this contributed to the understanding of the forms. I repeat, the work in this direction should not be considered exhaustive and complete.

In other ways I have also made an indulgence, such as in the demarcation of the main divisions themselves.

Anyone who even superficially surveys the forms will soon find that they provide an actual historical development. All the transitions from the sponge to the coral, from the coral to the crinoid are present, so that it becomes doubtful if one should assign new species to these transitions.

In such beginnings mistakes are inevitable, so it is only a small demand in asking to forgive them. Nor did I want to delay the publication of this work too long and therefore have it just as it is now.
\clearpage
\subsection{History and Overview}
$\Delta$o$\sigma$ $\mu$o$\iota$ $\chi\epsilon\nu\tau\rho$o$\nu$%Δός μοι χέντρον
\paragraph{}
Last year, when I wrote-down in my diary certain new observations about the composition of the rocks of our Earth\index{Earth}, and also of the meteorites\index{meteorite}, the importance of the latter to geology was not fully clear to me.

It was only when I was forced by the attacks of opponents to take the investigation again into my own hands that I clearly realized the importance that a careful study of the meteorites\index{meteorite} could be to the history of our Earth\index{Earth}. Lastly, I came to the conclusion that in the present state of geology the meteorites\index{meteorite} --- and only the meteorites\index{meteorite} --- give the point from which the history of Earth\index{Earth} could at last be explored with near certainty.

If in \emph{Primordial Cell}\index{Primordial Cell} I thought that I had reached the limit of research with granite, I soon learned better. I contemplated that by virtue of its specific gravity, the Earth's core\index{Earth!core} must also consist of at least solid iron, especially considering the very probable order of the meteorites\index{meteorite}, which go from the pure iron to the feldspar rocks of Earth\index{Earth}. I further believed that a conclusion for Earth\index{Earth} based upon the meteorites\index{meteorite} could be ventured, the conclusion that in the other planets and in those, or the one, whose debris we have in the hundreds of thousands of orbiting meteorites before us we have a sequence of stratification from heavy to light, a stratigraphy which we probably pass through in the series from the pure-irons through the half-irons (Pallasite\index{meteorite!Pallasite}, Hainwood\index{meteorite!Hainwood}) to the chondrites\index{chondrite} and the eucrites\index{eucrite}, then to the coal meteorites\index{meteorite!coal} (Cold Bokkeveld)\index{meteorite!Cold Bokkeveld}.

Once this likelihood had been understood, it was obvious that the meteorites\index{meteorite} should be subjected to a thorough examination of their morphological characteristics. This was also highly necessary because so far almost nothing has happened in this direction: one can convince oneself of this by comparing my photomicrographs with the roughly twenty meager illustrations, which taken together form the material of the science today. The academic writings of Berlin, Vienna, and Munich have only a few panels each, the drawings are small, and it immediately shows, are taken from the least suitable meteorites\index{meteorite} for this direction of investigation and, moreover, probably not from the best part, the interior.

So if, I thought, my earlier assertion that the Knyahinya Meteorite\index{meteorite!Knyahinya} consisted entirely of life was not confirmed by my new investigations, then science would still have been served if I were to show the true nature of this meteorite\index{meteorite}. Fortunately, however, I was spared this retreat: on the contrary, the results of my new research were far beyond expectations --- a new world emerged.

But, of course --- science is skeptical --- it rightly demands more stringent evidence than I offered in \emph{Primordial Cell}\index{Primordial Cell}; a book written more at the stage of, I would say, intuition. Today I present the evidence.

As one examines the tables of this work, it immediately becomes clear that these are not mineral forms, but organic ones; that we have before us the images of life, images of life of the lowest order, a creation which in greater part finds some of its closest relatives here on Earth\index{Earth} --- regarding the corals\index{coral} and crinoids\index{crinoid}, this is determined with absolute certainty; however, the sponges\index{sponge} have only some similarity with those forms of the terrestrial genera.

Thus, the genesis is determined in terms of the parts. However, in my study of twenty chondrites\index{chondrite} (and 360 thin sections of them) the assertion made in \emph{Primordial Cell}\index{Primordial Cell} was confirmed --- that the rock of the chondrites\index{chondrite} is not a type of sedimentary rock as on Earth\index{Earth}, in which fossils\index{fossil} are embedded, that it is not a conglomerate formation; but rather, its whole mass is entirely formed of organic beings, like our coral\index{coral} rocks. So not a plant, as I had assumed earlier, but plant-animals! The whole stone is life: --- I think science will forgive me the first mistake.

Needless to say, the iron meteorites\index{meteorite!iron} were now subject to additional testing. Here it rests as only a first observation.

However, time and circumstances, especially the lack of available materials, prevented me from concluding the investigation prior to this publication. But if I repeat today the first assertion, that meteoritic iron is nothing but a mat of plants, then I may now regard myself as more legitimate than at the time when I wrote \emph{Primordial Cell}\index{Primordial Cell} and asserted the prior statement. I have to add that I also found life-forms in the irons. The researchers who avoid the forms of the chondrites that I depict may overlook the fact that the so-called Widmanstätten's\index{Widmanstätten} figures are, for the most part, plant cells and not crystals.

The investigations up till now, in the whole field, with the exception of [Karl Wilhelm von] Gümbel's\index{Gümbel} work in the Munich Academy, are of little use, both regarding the accuracy of their observations and even more their interpretations based upon those observations, i.e. on unproven hypotheses and weak assumptions --- not suitable for scientific findings as such. And due to this the field was still wide open to me, although my only regret is that I cannot make a draft in time regarding the irons.

I now come to the conclusions for geology. If the chondrites, an olivine and enstatite\index{enstatite} rock, are really what I assert: that is, only pieces of sponge-coral-crinoid rock, then a fact of immeasurable consequence has been discovered for the science of Earth\index{Earth}.

The feldspar minerals are a purely water production --- they are the petrifying matter of millions of organisms! Thus, all hypotheses about the metamorphic and plutonic rocks of Earth\index{Earth} fall, and with them the theory of the fire-liquid Earth\index{Earth} interior --- or at least no conclusion can be drawn from the rocks any more.

I now have to justify this. A comparison of terrestrial rocks with the meteorites shows that the chondrites, at least according to their chemical nature, have their closest relatives on Earth\index{Earth}.

The olivine\index{olivine} rock of Earth\index{Earth} is, as a lherzolite, a bedrock layer such as we see with basalt breaking through granite; I arrive at the results that [Gabriel Auguste] Daubrée has shown.

The deeper granite is definitely younger than the olivine\index{olivine}. But if the olivine\index{olivine} of the meteorites\index{meteorite}, by virtue of its composition, is a water-rock, it will probably be likewise for the granite of Earth\index{Earth}; if the olivine of the meteorites\index{meteorite} is the remains of life, then the same will be the case with the olivine\index{olivine} of the Earth\index{Earth}: it could probably be concluded then, that the rock of our Earth\index{Earth} is also composed in its original deposits of the same life as that of the chondrites. And for the same reason the granite, as younger rock, will probably have a similar origin. We only have to look at our Swabian basalts leaching through the original olivine\index{olivine} to see that the lherzolite bedrock is found under the granite. And even if this rock appeared as a liquid deposit without distinguishable forms, the iron of Ovifak\index{Ovifak} has such; but this is highly connected with the basalt, so intimately, and not only mechanically, that both must be regarded as one rock. So, this is the original olivine\index{olivine} bedrock. But because of this, scientific reasoning thus removes the presumption of the origin of the Earth\index{Earth} by way of fire.

If the surface of the planet, or of the planets, consists of layers of olivine\index{olivine} from life, then the same layers of our Earth\index{Earth} were probably not formed by fire, or at least there is not the slightest reason for this supposition; on the contrary, it should be assumed that the same layer of the Earth\index{Earth} was a water formation. Here I encounter the Kant-Laplace theory.

I imagine that the planetary materials (including water, which is usually overlooked) during the first mass formation were not, as [Immanuel] Kant and [Pierre-Simon] Laplace say, a glowing haze, but rather a vapor and mass as cold as space. Here, however, one has overlooked a great logical error in the above mentioned theory.

That the attraction of mass should form mass! That the effect should simultaneously be the cause! The mass is to be formed by mass attraction, that is, by the fact that it was already there! It is to be regretted that this error of thought has not been discovered earlier. A mass, when it is present, can increase through attraction, but not from it: it is as if someone should be his own father!

So another force had to have formed the mass; but this could only be either the crystallization force or the organic formative force.

The former does not suffice to explain the formation of the planets, and no crystals are found: consequently only the second force remains --- the organic one. Here I recall my observations on the structure of the meteorite\index{meteorite} and so today, for me at least, it is clear that the first beginnings of Earth\index{Earth}, and the rest of the planets, had an organic cause.

If the sentence appears a bit deafening, one need only resort to the familiar.

First, the mass of building materials available at the beginning of planetary formation is completely sufficient to explain the formation of the planetary mass in an organic way.

Secondly, the experience of today shows how, in short time, the lowest plants and animals multiply their number, including their mass, in a way that is conditioned only by the mass of the building materials, while their organization itself makes it possible to expand into infinity as long as building materials are present.

What seems to contradict this explanation is only the geothermal heat and the associated appearance of the volcanoes still active today. With regard to these two facts, one has long been led back to a different explanation, that of a liquid-fire Earth\index{Earth} interior. Water has a dissolving effect on feldspar. In this dissolving process, heat is released. The volcanoes follow the sea because water helps form the gases, which are ignited from above to melt the forthcoming rock.

How could a fiery Earth core\index{Earth!core} ultimately survive without oxygen! And does not the very existence of combustible gases (for these are the cause of volcanic phenomena), especially that of sulphur, indicate the presence of organic substances in the Earth's\index{Earth} interior? There really is no need for new evidence here, but only the abandonment of certain ideas, which have taken possession of the imagination excited by some obvious phenomena.

These are the conclusions from the study of the meteorites\index{meteorite} for our Earth's\index{Earth} formation. But the facts that astronomy can derive from them are no less significant.

The twenty meteorite\index{meteorite} (chondrite)\index{chondrite} thin sections that I have studied, some from falls which are more than a century apart, show the same forms, much as fossil\index{fossil} shells occur everywhere in the same formation; Gümbel\index{Gümbel}, if he did not correctly interpret the forms of the chondrites\index{chondrite}, has excellently expressed this.

So these chondrites\index{chondrite} are probably from one and the same world body, a planet. Or else how could evolution coincide on different planets?

This planet carries water life, so life has arisen in water and lives by water; this planet has not passed through fire, because the traces of fire do not show in these rocks. The meteorite\index{meteorite}, having been shattered, only receives a 1 mm thick enamel fusion crust in its short path through our atmosphere as a result frictional heating.

The life of the chondrite\index{chondrite} is almost entirely a microscopic one, it ranges from 0.20 to a maximum of 3 mm in diameter; often it takes a magnification of 1000 to clearly see the delicate structures, while at such magnification our terrestrial fossils\index{fossil} dissolve into a shapeless surface.

Thus, through the observations first laid down in \emph{Primordial Cell}\index{Primordial Cell}, the path was wide open for me to cover the distances that science must cross.

But it really doesn't take a titanium effort to destroy an old building. It has already been much worn, only ignored: it requires only one striking proof and the work will have been done. Traditions, based on insufficient observations, dissolve into what they are, allowing science to once again proceed freely on its course.
\clearpage
\subsection{Previous Views on the Meteorites\index{meteorite}}
\paragraph{}
The following is a brief presentation on the previous views regarding the origin and nature of the meteorites\index{meteorite}.

Only the morphological work on individual meteorites\index{meteorite}, from the time when the microscope began to be used in geology, should be enumerated. What the microscope has so far provided for the interpretation of the meteorites\index{meteorite} is, apart from the enlarged olivine\index{olivine} crystals in [Nikolai Ivanovich] Koksharov's\index{Koksharov} \emph{Minerals of Russia VI}, contained in the following writings:
\begin{enumerate}
    \item Gustav Tschermak von Seysenegg: ``The Fragmentary Structure of the Orvinio\index{meteorite!Orvinio} and Chantonnay\index{meteorite!Chantonnay} Meteorites,'' presented at the meeting of The Royal Academy of Sciences (Vienna) on November 12, 1874. (20$^{th}$ volume of \emph{The Proceedings of the Royal Academy of Sciences}, Section 1, November issue 1874. With 2 tables.)
    \item Alexander Makowsky\index{Makowsky} and Gustav Tschermak\index{Tschermak} von Seysenegg: ``Report on the Fall of a Meteorite near Tieschitz in Moravia.'' With 5 plates and 2 woodcuts, presented at the meeting of the Mathematics and Natural Science Class (The Royal Academy of Sciences in Vienna) on November 21, 1878. Volume 29 of memoranda of the mentioned class.
    \item Johann Gottfried Galle\index{Galle} and Arnold Constantin Peter Franz von Lasaulx\index{Lasaulx}, submitted by Christian Friedrich Martin Websky: ``Report on the Meteorite Fall at Gnadenfrei\index{meteorite!Gnadenfrei} on May 17, 1879.'' Session of July 31, 1879. Monthly reports of The Royal Prussian Academy of Berlin\index{Berlin}.
\end{enumerate}
\paragraph{}
The previous descriptions are limited to examinations with the naked eye and magnifying glass, as well as chemical analysis.

They all agree that the chondrites\index{chondrite!composition} consist of a matrix of spheres of enstatite\index{enstatite} (bronzite), olivine\index{olivine}, iron, nickel and chromium-iron.

Gümbel\index{Gümbel}: ``On the Stone Meteorites Found in Bavaria'', \emph{Meeting Reports of the Mathematics-Physics Class of the Munich Royal Academy of Sciences} 1878. Issue 1, p. 14 et seq. In the description of the meteorites of Eichstädt\index{meteorite!Eichstädt} and Schöneberg\index{meteorite!Schöneberg}, he mentions ``mesh-structure'' (p. 27. 46). However, he also speaks of ``descendants of larger broken chondrules'' (p. 28). The important section of his observation is on page 58, which follows here:

``If one examines the results of the investigation of this, albeit limited, group of stone meteorites\index{meteorite}, then the perception that comes to the fore is that, in spite of some differences in the nature of their conglomeration, they are nevertheless governed by completely identical structural relations. All are undoubtedly debris, composed of small and large mineral grains, from the well-known roundish chondrules: which are usually completely preserved, but often appear as broken pieces, to the globs of metallic meteoritic substances, sulfur-iron, and chromium-iron. All these fragments are glued together, not cemented by an intermediate substance or a binder, as there are no amorphous, glassy, or lava admixtures at all. Only the fusion crust and black constrictions, which often appear on clefts and are similar to the crust, consist of amorphous glass, which, however, originated after falling within our atmosphere. In this melted crust, the denser melt-able and larger mineral grains are usually still embedded un-melted. The mineral splinters do not bear any traces of rounding or tumbling, they are sharp-edged and pointed. As for the chondrules, their surface is not smooth, as it would have been if they were the product of tumbling, rather it is always uneven, mulberry-like and warty, or multifaceted with a projection of crystalline surfaces. Many of them are elongated with a distinct tapering or sharpening in one direction, as is the case with hailstones. Often you encounter pieces which apparently must be regarded as parts of shattered chondrules. As an exception are twin-like connected beads, most common in those which meteoritic iron beads have grown. In numerous thin sections they are composed differently. Most often there is an eccentric, radiating fibrous structure which spreads from a point far from the center after tapering or slightly tattered lines spread like rays toward the outside. Since cuts made at various angles always reveal a columnar or needle-shaped arrangement, never leaves or lamellas in the substance forming these tufts, it seems to be columnar fibers from which such chondrules are built. With certain cuts, according to this assumption, in the cross-sections of the fibers that are perpendicular to the length direction, only irregularly angular minute fields are observed, as if the whole was composed of small polyhedral granules. Sometimes they appear as if there were several systems radiating in different directions in a sphere, as if the point of radiation were altered during its formation, so that a constant and seemingly confused elongated structure emerges. Towards the outside, against which the junction point of the radiating bundle is shifted unilaterally, the fiber structure normally becomes indistinct or replaced by a more granular aggregate formation. In none of the numerous ground-up chondrules could I observe that the tufts ran directly to the edge, as if the point of emission were outside the sphere, provided that it was completely preserved and not a mere shattered piece. The delicate transversely dividing fibers usually do not run along the entire length of the tuft, but rather they gradually sharpen, branch or end to allow others to take their place, so that in the cross-sections, a manifold, mesh-like or netted image is created. These fibrils consist, as has often been described above, of a mostly lighter core with a darker envelope that is dissolved by acids, while the latter resists. Highly curious are the bowl-shaped constructions, which seem to be meteoritic iron, that are generally only spread over a small part of the globules. The same unilateral striations, visible on the average as crescent bowed streaks, also appear inside the chondrules and provide strong evidence contrary to their being formed by a tumbling of some material, the entire arrangement of the tufted structure speaks to a resolution against their origin by tumbling. However, not all chondrules are the eccentric fibrous type; many, especially the smaller ones, have a fine-grained composition, as if they are composed of a mass of aggregated dust. Here too, the one-sided formation of the spheres is sometimes noticeable by an intensely greater compression of the dust pieces.''

And further, p. 61:

``The most common type of stony meteorite\index{meteorite} is predominately that of the so-called chondrites\index{chondrite}, the composition and structure of which coincide so much that we do not see how a common origin and the initial cohesion of these chondrites\index{chondrite} --- if not all meteorites\index{meteorite} --- could be in doubt.''

``The fact is that they enter our atmosphere as highly irregular pieces --- apart from the shattering within into several fragments, which is common, but cannot be assumed in all cases, especially if, by direct observation the falling of only a single piece is confirmed; it can be further concluded that they make their orbits in the heavenly space as demolished pieces of a single larger celestial body and in their absent-mindedness occasionally fall to Earth\index{Earth} when they enter into the region of Earth's\index{Earth} attraction. The lack of original lava-like amorphous constituents in connection with the external irregular form is likely to exclude from the geo- or cosmological points of view the assumption that these meteorites\index{meteorite} are ejections of lunar volcanoes, as is often claimed.''

Gümbel\index{Gümbel}, having placed the meteorites\index{meteorite} as related to the olivine\index{olivine} rocks of Earth\index{Earth}, summarizes his view on their origin (p. 64) in the sentence: ``Therefore, the meteorites\index{meteorite} appear to be a kind of first process of encasing the celestial bodies, but since they contain metallic iron --- to have been produced in the absence of oxygen and water.''

``So ingenious,'' he continues (p. 68), ``...are these hypotheses of Daubrée's and Tschermak's (origins from shattered volcanic rock), however, I cannot agree with their view on the formation of the globules (chondrules)\index{chondrite!chondrule} on the basis of my latest research. Contrary to Tschermak's assumption, I sought to prove that the internal structure of the chondrules is not out of context with their spherical shape and that these globules cannot be regarded as pieces of a mineral crystal or solid rock. Their unsmooth, unpolished surface stands out, which, if they were formed by abrasion or tumbling, should be mirror-smooth due to the similar hardness of the material, while instead it appears rough, bumpy, often facially striated, against the theory of friction, and there is no reason at all by which to understand why the other mineral fragments are rounded like grains of sand, and why, in particular, the meteorite\index{meteorite}, the iron, and the very hard chromium-iron, as I have been convinced in the meteorite\index{meteorite} of L'Aigle\index{meteorite!L'Aigle}, are always not rounded, with often extremely finely sliced forms. How is it conceivable that, as if often observed, there would be a concentric accumulation of meteoritic iron within the globules? Also, the eccentric fibrous structures of most globules in their one-sided radiating do not appear to be random in relation to the surface, but rather like the nature of the structure of hailstones. This inner structure is closely related to the act of its formation, which can only be explained as a growth of mineral forming substances with simultaneous rotation in gaseous vapors that provided the material for further support, whereby more material began in the direction of movement.''

Gümbel\index{Gümbel} goes on to say that the material constituting the chondrites\index{chondrite} was formed by a disturbed crystallization and fragmentation resulting from explosive processes within a space filled with vapor and hydrogen gases supplying the minerals. He closes p. 72 with a discussion of the Kaba meteorite\index{meteorite!Kaba}:

``Perhaps, however, it is still possible to prove the presence of organic beings on extraterrestrial bodies.'' I hope this is successful. From his illustrations one can see that the investigation was based on bad material. After all, more thin sections should have been made and the magnification is far from enough. What I refer to is the upcoming description of my tables.

What I value so highly in Gümbel's\index{Gümbel} report is the scrupulous prejudice-free, let's say impartial, observations. I have allowed myself to quote the work of Gümbel\index{Gümbel} literally because it is indeed difficult for me to summarize such representations and to separate fact from interpretation.

Proper observations and incorrect explanations are so closely intertwined that it is impossible to do both. I thought while I read Gümbel's\index{Gümbel} paper (after completing my own investigations and manuscript) that I was coming to step on my conclusions at every moment. But, just as the surge of the surf seizes and throws back the man who wants to reach the shore, while with each attempt he thinks he has made it, so also here: the old dogma always pulls the honored researcher from the saving cliff into the sea and into the bottomless whirlpool of traditions.

Daubrée's commendable work \emph{Experimental Geology} was obtained only in translation and after completion of my work. No one will find that it refutes my conclusions. Daubrée himself depicted Knyahinya\index{meteorite!Knyahinya}: pressed, melted, dissolved, calculated, only not --- seen.
\clearpage
\subsection{Meteorites\index{meteorite} and their Mineralogical Properties}
\paragraph{}
The literature on meteorites\index{meteorite} is very extensive. However, it is so well known in terms of the type and number of chemical compositions, that I do not need to dwell on this part of it, in particular the earlier works.

The meteorites\index{meteorite!classification} are divided into iron and stone, but there is still a class between the two: ``half-iron'', i.e. a combination of solid iron and stone --- the pallasites. While the irons show many similarities, both in their chemical composition and in the form of their structure, the pallasites are very different (depending on the predominance of iron). But there are other differences among them. Hainholz\index{meteorite!Hainholz} [mesosiderite]\index{meteorite!mesosiderite}, for example, has a blue mineral (enstatite)\index{enstatite} in addition to iron and olivine\index{olivine}, and in this a great richness of life-forms. The stones are divided into chondrites\index{chondrite}, stannerites [Stannern meteorite\index{meteorite!Stannern} --- eucrites], luotolaxers [Luotolax meteorite\index{meteorite!Luotolax} --- howardites], bokkefelders [Cold Bokkeveld meteorite\index{meteorite!Cold Bokkeveld} --- carbonaceous chondrites]\index{chondrite!carbonaceous}, bishopvilles [Bishopville meteorite\index{meteorite!Bishopville} --- aubrites], (Quenstedt, Klar and Wahr p.280 follow).

I prefer to study the chondrites\index{chondrite}, and where I speak of meteorites\index{meteorite}, I am referring to this class of stone meteorites\index{meteorite!stone}, which is also the most abundant.
\paragraph{}
I have examined:
\begin{center}
\begin{tabular}{ l r }
 Tabor, Böhmen [Czech Republic] & July 3, 1753\index{meteorite!Tabor} \\
 Siena, Toskana [Italy] & June 16, 1794\index{meteorite!Siena} \\
 L'Aigle, Normandy [France] & April 26, 1803\index{meteorite!L'Aigle} \\
 Weston, Connecticut [USA] & December 14, 1807\index{meteorite!Weston} \\
 Tipperary, Ireland & November 23, 1810\index{meteorite!Tipperary} \\
 Blansko, Brünn [Czech Republic] & November 25, 1833\index{meteorite!Blansko} \\
 Château-Renard, Loiret [France] & July 12, 1841\index{meteorite!Château-Renard} \\
 Linn [Marion] County, Iowa [USA] & February 25, 1847\index{meteorite!Marion County}\index{meteorite!Linn} \\
 Cabarras [Monroe] County, North Carolina [USA] & October 31, 1849\index{meteorite!Monroe County}\index{meteorite!Cabarras} \\
 Mezö-Madaras [Romania] & September 4, 1852\index{meteorite!Mezö-Madaras} \\
 Borkut, Hungary & October 13, 1852\index{meteorite!Borkut} \\
 Bremervörde, Hanover [Germany] & May 13, 1855\index{meteorite!Bremervörde} \\
 Parnallee, East India [Tamil Nadu] & February 28, 1857\index{meteorite!Parnallee} \\
 Heredia, Costa Rica & April 1, 1857\index{meteorite!Heredia} \\
 New Concord, Ohio [USA] & May 1, 1860\index{meteorite!New Concord} \\
 Knyahinya, Hungary & June 9, 1866\index{meteorite!Knyahinya} \\
 Pultusk, Warsaw [Poland] & January 30, 1868\index{meteorite!Pultusk} \\
 Orvinio [Italy] & August 31, 1872\index{meteorite!Olvinio} \\
 Simbirsk [Russia] & [1838]\index{meteorite!Simbirsk} \\
\end{tabular}
\end{center}
\clearpage
\paragraph{}
All rocks are thoroughly certified. Above all, I have the kindness of my revered teacher, Professor Dr. [Friedrich August] von Quenstedt\index{Quenstedt}, who thanked me with the excellent Tübingen\index{Tübingen} University Collection (which, as is well known, originates for the most part from [Karl Ludwig] Baron von Reichenbach\index{Reichenbach} in Vienna\index{Vienna}).

Of Knyahinya\index{meteorite!Knyahinya} I own 360 thin sections, of L'Aigle\index{meteorite!L'Aigle} 6, of Pultusk\index{meteorite!Pultusk} 6, of the remaining 1-3 each. I will name all stones after their place of fall. While making the thin sections, I made cuts in two directions. After several attempts on Knyahinya\index{meteorite!Knyahinya}, it turned out that it breaks in certain directions.

This was deduced from the inclusions that, once their positions had been found, regularly resulted in certain forms, to which these forms corresponded in sections made perpendicular to this position.

The forms of the stone were situated in such a way that the same position in the remaining stones would have been obtained, provided, of course, that the material had been available. For some, this happened by chance, while not in others; but for the reasons stated above further determination is required in this direction.

Also, I deliberately made the thin sections in three different thicknesses: thickly translucent, in order to see whole inclusions as completely as possible; very thin, in order to clarify the structural relationships; and for the majority, in such a way that both are still visible.

I would like to make a comment here, which will be confirmed by anyone who has dealt with thin sections of fossiliferous\index{fossil} material.

Only in rare cases of total transparency, i.e. cut very thin, is the structure visible. Anyone who looks at a thin section, if cut in this manner, with the microscope will be delighted at the beautiful shapes and lines. At the joy of this, one will want to make things even better and one expects with continued grinding a perfect picture. But when one puts the thin section under the microscope after this second try --- there is nothing left but an almost structureless surface, with hardly hinted, even blurred shapes, and those which you previously perceived with the magnifying glass can no longer be seen, not even with the microscope. However, this phenomenon is related to the type of metamorphosis of the rock and the forms within it. The matter is well-known and therefore does not require further explanation. I only mentioned this matter so that those who want to make such observations will not be surprised and will improve their own manner of observation.

The fact that the chondrites\index{chondrite} consist for the most part of bronzite\index{bronzite}, enstatite\index{enstatite} (augite), and olivine\index{olivine}, as well as being magnetic throughout, is an accepted fact in the science. Quenstedt, \emph{Handbook of Mineralogy} p. 722.

However, the inclusions that I claim are coral\index{coral} have been addressed as enstatite\index{enstatite}. This was believed to be able to explain their structures. Others went further and explained the inclusions as a type of glass (Tschermak)\index{Tschermak}.

So, before getting to the justification of my view, the microscopic appearance of the primary mineral, enstatite\index{enstatite}, must be clearly identified.

Allow me to give a brief outline of what [Karl Heinrich Ferdinand] Rosenbusch\index{Rosenbusch} says in his book: \emph{Microscopic Physiography of Petrographically Important Minerals}, Stuttgart 1873, p. 252, about enstatite\index{enstatite} (and bronzite):

``As is known, since the optical investigations of [Alfred] Des Cloizeaux\index{Des Cloizeaux}, enstatite\index{enstatite}, bronzite, and hypersthene have been treated as rhombically crystallizing separated from pyroxene and compiled into their own group. In addition to the cleavage above the prism of 87°, the same shows further divisions above the vertical pinacoid, the relative perfection of which the data of various researchers do not exactly match. Chemically, these three minerals form an uninterrupted series, at the beginning of which stands the almost iron-free enstatite\index{enstatite}, and at the end of which stands the very iron-rich hypersthene. Additionally, enstatite\index{enstatite} and bronzite are so similar in all physical properties that it is difficult to separate them into two species. Hypersthene, on the other hand, shows a different optical orientation and therefore forms its own species. It is interesting to note that Tschermak's\index{Tschermak} arrangement of the negative angles of the optical axes and the iron content of the three minerals mentioned makes it clear that the angle of the optical axes decreases steadily as the [iron oxide] FeO content increases. The microstructure of all the minerals of the enstatite\index{enstatite} group is generally so similar that, in special cases, a safe decision can only be made by chemical and precise optical analysis.''

``Enstatite\index{enstatite} and bronzite\index{bronzite} are not found in the rocks as crystals, but almost always in irregularly limited crystalline grains, which usually show a very dense striation, which is more straightforward in the case of enstatite\index{enstatite}, more gently winding and wave-like in the latter. But this difference is not a pervasive one. The same striation is also shown by the monoclinic diopside and rhombic bastite, which cannot easily be separated from bronzite by other, later to be discussed, visual phenomena. If the cut meets the enstatite\index{enstatite} or bronzite at a strong incline to the main cleavage surface, then the surface will not be equally fine-grained, but rather like a rough stairway. Transverse surfaces and fractures are not uncommon.''

``Both are relatively poor in extraneous deposits; they are missing, for example, in the enstatite\index{enstatite} from pseudophites of the Aloysthals in Mähren and in some enstatites\index{enstatite} or bronzites of the lherzolites and olivines\index{olivine}. The former is traversed only by frequent veins of pseudophite, from which fine-grained decomposition products penetrate the enstatite\index{enstatite} in a vertical direction. Other occurrences and even other individuals of the same hand specimen often contain mass inclusions of green or brown lamallae, splints, and grains (depending on the position of the grinding plane) which, without exception, are invariably parallel to the most perfect cleavage direction. This suggests the idea that various indications about the relative perfection of the pinacoid ($\infty$ P $\infty$) cleavage compared to the prismatic one may be due to the more or less mass presence of these interpositions, which undoubtedly also determine the metallic shimmer on the brachypinakoid. Then, however, the ease of separation in this direction would be more a separation than true fissility.''

``The enstatite\index{enstatite} without, and the bronzite with metallic shimmer on the brachypinacoid cleavage surface, can be found in the serpentines of Aloysthal in Mähren (enstatite)\index{enstatite} and Mont Bresouars in the Vosges, in the lherzolites and olivine\index{olivine} rocks, in some olivine\index{olivine} gabbros, in Streng's Enstatitfels from Radauthal near Harzburg\index{Harzburg} and in the olivinite bombs of the Dreiser Weiher [Daun area of Germany], as well as in some meteorites\index{meteorite}; so always in the company of olivines\index{olivine} and altered olivines\index{olivine}.''

For those who have command of the book, I provide two illustrations, one of bronzite\index{bronzite} from Kupferberg\index{Kupferberg} (Table 1: Figure 1), the other of enstatite\index{enstatite} from Texas\index{Texas} (Table 1: Figure 2), which are quite similar to Rosenbuschite.

As far as olivine\index{olivine} is concerned, there is no need for a picture, since the forms of this mineral are completely encompassed with circles. Suffice it to say that pure olivine\index{olivine} does not show any structure. Olivine\index{olivine} only shows structure if one wants to call its inclusions, or growth sites of the crystal, or decomposition phenomena (serpentine formation), structures. However, there is certainly no crystal that looks similar to my forms. As for the claim that the spheres are glass, it is not even made clear what chemical composition these glass spheres should have, compared with enstatite\index{enstatite}, bronzite, and olivine\index{olivine}. Apparently, all forms are thrown together and declared as glass, although enstatite\index{enstatite}, according to Quenstedt\index{Quenstedt} (Mineralogy p. 318), is infusible, and according to Naumann-Zirkel p. 585 it is, at least, difficult to melt. It is even claimed that these glass spheres were first created while falling. But the effects of fire are found only in the fusion crust. The fusion crust of most meteorites\index{meteorite!fusion crust} is barely 2 mm thick.

To the assertion that the chondrules\index{chondrite!chondrule} are glass, which is countered by the message sent by my thin sections, comes the reply that there are similarities of the meteoritic form with glass in the rocks of Earth\index{Earth}. Thus, I was referred by [Ferdinand] Zirkel to a spherulite liparite of which I give in Table 1: Figure 3. This form should show that my \emph{Urania} is a deception. I think the form in the liparite is a crystallite formation (probably zeolite). Now look at the structures on Table 1: Figure 4, 5, 6!

Our researchers, apart from Gümbel\index{Gümbel}, speak of the meteorites\index{meteorite} as volcanic bombs, declaring the rock as identical to the volcanic rocks of Earth\index{Earth} and so counting the meteorite\index{meteorite} without hesitation with the volcanic rocks. The evidence to the contrary is the subject of this book.

Rightly, Quenstedt\index{Quenstedt} alone has declared the question an open one when he said: ``...it is reserved for the microscope to solve the riddle of the composition of the meteorites!\index{meteorite}'' Handbook of Mineralogy p. 722.
\clearpage
\section{The Organic Nature of the Chondrite\index{chondrite!organic}}
\subsection{Organic or Inorganic?}
\paragraph{}
In order to prove that a plant or animal organism is present, I consider it necessary to prove:
\begin{enumerate}
    \item a closed form,
    \item a recurring form,
    \item recurring in developmental stages,
    \item structure, either cells or vessels,
    \item similarity to known forms.
\end{enumerate}
\paragraph{}
If these requirements are valid, it remains only to decide whether plant or animal? Now ask yourself, do my forms fulfill these requirements?

I believe, before I go to the positive proof, that the negative proof ought to lead.

You see, the proof that I claim for the existence of organic beings is twofold: a negative one, by showing that the meteoritic forms do not belong to the mineral kingdom, and a positive one, by showing the similarity of the meteoritic forms with those of Earth\index{Earth}, whether living or extinct. The first thing to prove, therefore, is the following sentence:

The inclusions in the meteorites\index{meteorite} are not mineral formations.
\begin{enumerate}
    \item Our mineralogists explain the inclusions of chondrites\index{chondrite} as enstatite\index{enstatite}, bronzite, and olivine\index{olivine}. 
    
    Olivine\index{olivine} has no visible sheet breakage, but in enstatite\index{enstatite} and bronzite it is obvious. I depict a bronzite from Kupferberg, Table 1: Figure 1, an enstatite\index{enstatite} from Texas\index{Texas}, Table 1: Figure 2 (thin section at 75 times magnification). Figure 2 shows a good sheet fracture. Now compare this with Table 1: Figure 4, one of my favorites from the Knyahinya Meteorite\index{meteorite!Knyahinya} (about 250 times enlarged) and you will probably no longer speak about the fact that sheet breakage is the cause of the structural phenomena of the chondrites\index{chondrite}. Now look at all the tables and this explanation will be put aside once and for all.
    \item The inclusions of the chondrites\index{chondrite} consist of enstatite\index{enstatite} or olivine\index{olivine}; or they are glass: if this is the case, I ask, how is it possible that the same mineral, or glass, as a whole occur in such different forms (outlines and structures), and different minerals occur in such acutely coinciding forms? Look at hypersthene\index{hypersthene}, hornblende\index{hornblende}, augite\index{augite}! Apart from some visible, easy-to-explain inclusions --- (and this is not the case here) it is always the same picture! Here we have at most three minerals with a hundred different images!
    
    A mineral is simple, it must be simple in its expression and therefore always gives the picture of a homogeneous mass (field), with some inclusions at most. How could the same mineral be present in such different structures, in such coherent forms that differ from crystal forms?
    \item Minerals are either crystallized or not crystallized. In the first order they have a certain regular and recurrent form: they move along surfaces which, on average, project themselves as straight lines. These forms (lines and angles) are recurring, varying only in size, not ratio. Such forms are not found among the forms I have addressed as organic. In the organic forms there is no form with a surface or an angle; all are spheres or ellipses with deviations from a mathematical form, deviations that are nevertheless constant. It is these other forms which give rise to the need to foresee just what these matching structures are, showing themselves with constant outlines, these forms which are different from the crystal form of the enstatites\index{enstatite} and olivines\index{olivine}.
    
    Though they are rare, small sections are true crystals, but in a way, they are probative values that do not impinge on the facts. See below and Table 32: Figure 2.
    \item If the minerals were originally crystallized but happened to lose their crystalline form due to some mechanical force, the only form that could be repeated is the sphere or one approaching it, such as an ellipse. Here a repetition would be possible without a conclusion being drawn about the form. In these spheres, surface cuts of the body would immediately show the influence of such mechanical forces as the inclusions would be hit arbitrarily.
    
    However, the structure in the meteorite\index{meteorite} inclusions is always, I would like to say: symmetric, in harmony with its outlines.
    \item When crystals are weathered the layers change from the outside to the inside --- concentrically --- but there is no trace of weathering in the inclusions of the chondrites\index{chondrite} and their structures are always eccentric.
    \item Regarding the mineral inclusions, they provide different sights depending on their nature. The deposits have quite arbitrary forms, such as glass-liquid-inclusions and crystallites\index{crystallite}.
    
    But where crystal laws appear in the inclusions, they always depend on the crystalline form. This is not the case with meteoritic forms. No trace of inclusions in accordance with crystalline forms!
    \item A sheet breakage is only visible if a mechanical force creates a surface for light refraction phenomena. Without this, it is imperceptible. If cleavage surfaces are not present, light refraction phenomena do not reveal the meteorite\index{meteorite} inclusions, just ``dust material''.
    
    One finds in terrestrial minerals that there are interpositions parallel to the sheet cleavage: these do not show in the meteorites\index{meteorite}.
    
    I believe that the sight of my forms will make further discussion about the diversity of mineral and crystal images unnecessary.
    \item But so much has been said of crystallites\index{crystallite}, of crystallization.
    
    It has been previously held that such crystallization will turn into the enstatite-bronzite-olivine\index{olivine}\index{enstatite} spheres. Gümbel\index{Gümbel} pointed out that all spheres have eccentric centers!
    
    Here the idea about the basic difference between meteoritic forms and crystallites\index{crystallite} is made quite clear.
    
    Crystallites always grow around one point (concentric). The forms in the meteorites\index{meteorite} are all elliptical and pear-shaped: if the outer form is also spherical, the alleged inclusions are eccentrically arranged and the center lies on the periphery (even beyond it, namely, it is ground away, which Gümbel\index{Gümbel} overlooked) --- a phenomenon that never occurs in the mineral kingdom. It is precisely the condition of crystallites, i.e. sphere formation, that crystals unite with a crystal of equal mass, which then create the concentric forms.
    
    Therefore, if the spheres in the meteorites\index{meteorite} were crystallites\index{crystallite}, then, at least according to the laws of Earth\index{Earth}, concentric formations should show.
    \item Finally, I must point out a contradiction that science came up with in order to explain the structure of the chondrites\index{chondrite} as being a mineral property. This is the optical behavior of these inclusions.
    
    If they are crystals and have broken sheets (of course olivine\index{olivine} has none, yet there are structures in the alleged olivine\index{olivine} sphere structures, i.e. sheet fractures!) as the source of their structure, the mineral should by necessity refract light. In most of the inclusions, however, there is no refraction of light, not even aggregate-polarization\index{polarization}! So, they can neither be simple minerals nor crystals, nor, least of all, be sheet fracture structures. This matter, that of the optical behavior, should have already led to the correct interpretation.
\end{enumerate}
\paragraph{}
All this evidence is of course unknown to the botanist and zoologist, while every mineralogist knows it. Therefore, I must ask the botanist and zoologist colleagues to confirm what my photographs show. These forms are not mineral forms. With this the mineralogist has done his part, and now it passes into the hands of the paleontologist, or rather the zoologist, and now begins the positive proof.
\clearpage
\subsection{The Individual Forms: Sponges --- \emph{Urania}\index{Urania}}
\paragraph{}
Rounded, lobed bodies with an obvious place of growth. Table 2 gives a larger image of an \emph{Urania}\index{Urania} (compared with Table 5: Fig. 1, the same picture). One sees here: the acute general form, the outermost lobed edge (white, on the left), the folds, which developed while contracting, the place of growth. Even more clearly is the latter as a chalice, Table 4: Figure 3.

Consolidated spiral-form \emph{Urania}\index{Urania} Table 3: Figure 5 and 6.

In comprehending the threads of Table 4: Figure 1 the structure consists of an outer membrane enclosing lamellar layers, Table 3: Figure 4. Table 4: Figure 6 (the latter can be seen with a magnifying glass). Median diameter of \emph{Urania}\index{Urania} 1 mm, color slate gray.

This structure was maintained to be a breakage of the bronzite\index{bronzite} sheet! Whether Table 4: Figure 4 belongs to the \emph{Urania}\index{Urania} is doubtful. The form and color suggest as much. The trim cuts on both sides show clear structure.

Table 5: Figure 5 shows entirely winding lobes. Either it is a hoisted spiral-form body, or it is several lobes, of which the outer one surrounds the inner.

Table 4: Figure 6 is a cross section, which does not show much. In the object itself you can see an average uncolored outer thin shell.

Table 5: Figure 2 shows such clear stratification, that if the outer form did not exist, one might attempt to place the form as coral\index{coral}.

Table 4: Figure 5 shows cross sections through both vanes of the lobes.

Table 6: Figure 3 lamellar structure. Figure 5 and 6 may also contain the simplest crinoids\index{crinoid}, whose arms have been laid out, on each other. Regarding the transitions of forms, I must refer to the chapter on that question.

The most incredible is Table 6: Figure 1 and 2. In Figure 1, the dull spot in the specimen is yellow, the striped blue. I have situated Figure 2 next, which clearly shows two lobes, connected like two shells in one place and at first sight also makes the impression of a double shell. (It is not a mere cut.) If you think a shell, the dull spot of Figure 1 would be the stone piece. But the structure is \emph{Urania}-like\index{Urania}.

Table 5: Figure 3: Two individuals show the structure most clearly, as well as the growth points. In Figure 4 (which is a bad photo), several individuals lie together in a fan-like manner.

In Table 3: Figure 3 and Table 4: Figures 1 and 2, it is believed to be seen the round mouth opening as implied from above.

After all this, I think \emph{Urania}\index{Urania} is a sessile sponge that contracts in a spiral form, absorbing and expelling water like our living sponges.

\emph{Urania} composes three twentieths of the rock mass.
\clearpage
\subsection{The Individual Forms: Sponges --- Needle Sponges\index{sponge}}
\paragraph{}
In Table 7 the forms of Figures 1, 2, 3, 5, and 6 show a spicule framework. Figure 1 points to \emph{Astrospongia}\index{Astrospongia}. The needles are regularly crossed. Figure 6 is an irregularly massive spicule framework with a cavity, which from the picture suggests is very delicate. These two forms seem unquestionable to me.

Almost certain are Figures 2 and 5 (in Figure 2, the white line is a rock crack).

The shape of Figure 4 I kept in the arrangement of tables as a sponge. After changing the arrangement was no longer possible, I realized this form was the skewed average of a crinoid and what I initially considered to be needles --- are fine crinoid arms. I note that the determination is very difficult because of the exceptionally plain meteoritic crinoid forms, which means a decision must be avoided pending further investigation. The cavity of the needle sponge can be confused for the food channel of the crinoid\index{crinoid} arms, when the latter are stretched straight and the limbs are no longer clearly preserved. This fact of the matter, however unpleasant for the investigator of individual forms, is more rewarding for the one who pursues the development of the forms --- for proving the development of one form to another. It is always enough one to the other. This puts us in a more favorable position.
\clearpage
\subsection{The Individual Forms: Corals\index{coral}}
\paragraph{}
Here we have such well-preserved terrestrial forms that not a doubt is left remaining.

Table 8 shows a sample image, Table 9 its channel structure: obvious bud channels that are tubular connections (for there are such). In addition, there is the curvature of the channels, which absolutely cannot be mistaken for a sheet breakage, plus there is the very clear tube openings and finally an equally clear growth site. (Table 1: Figure 4 shows an even sharper picture of the same object.) Regrettably, staining of the specimen gives the structure pictured in Table 9, such appalling shadows. The bud channels are 0.003 mm apart. Of course, everything you can ask for from a \emph{Favosites}\index{Favosites} structure.

Table 10: Figures 3 and 4 shows the image of \emph{Favosites multiformis}\index{Favosites} from the Silurian\index{Silurian}, in this one cannot even separate the species.

In Table 11: Figures 1, 2, and 3 (where 2 also shows growth points), any researcher will easily recognize the image of living coral\index{coral} forms, the more so as the cup shape (cavity) is indicated in Figure 1 above. The same object also shows the cross partitions of the tubes, which clearly emerge. Unfortunately, part of the picture is obscured by black in the photograph due to the yellow coloring of the specimen.

Table 10: Figures 1 and 2 show less well-preserved cross-wise and longitudinal sections, though the exact same repetition of both in several sections raises doubts that they are organic forms, and if they are such, then they can only be corals\index{coral}. Figure 3 seems to be a cup coral, Figure 4 has grown the same. The fact that Figure 6 has a coral\index{coral} structure does not require proof. This form recurs several times.

Table 11: Figure 4: This form also recurs several times. Peculiar coral\index{coral} forms are shown in Figures 5 and 6. Figure 5 is formed of tubular rings and most likely also Figure 6. I note that this shape appears hundreds of times.

At high magnification, partitions show: Table 11: Figures 1, 2, 3, and 6.

Table 12: Figures 1, 2, and 3 show clear lamellar structure. The transverse groove in Figure 4 is reminiscent of \emph{Fungia}\index{Fungia}. Table 30: Figures 1 and 2 and Table 20 probably also belong here.

The coincidence of the structure in Table 20 with that in Table 30: Figure 1 (in two different cut preparations) would alone suffice to exclude any possible thought of inorganic formation. Moreover, the form occurs about twenty times in 350 cuts.

Table 12: Figure 5 I found only once. In the original there are clear lamellae, which in the picture appear only in the lower part. Figure 6 is a milky white object, hence indistinct. I believe I recognize the star shape and have therefore placed the form here as a star coral\index{coral}.

Table 13: Figures 1, 2, 3, and 4 are corals which undoubtedly belong with the tubular corals. In the original, one can clearly distinguish: glassy like intermediate masses, black tube walls, yellow tubular filling material, occasionally the latter is also black. This form occurs a hundredfold in all the chondrites\index{chondrite}. Figure 5 is composed of lamellas showing clear cavities and Figure 6 has tubes with partitions. These forms belong with the largest of forms: they have diameters of up to 3 mm.

In Table 25: Figures 1 and 2 the form is here so well-preserved that the existence of an organism cannot be doubted, the more so because it occurs in two cuts and otherwise recurs frequently. See Table 2, lower left, Table 5: Figure 6 has the form, Table 1: Figure 6 and Table 25: Figures 1 and 2 are posed in sequence with the crinoids; the channels are unquestionable, the cross lines can also be interpreted as crinoid links. You can see incisions, furthermore the arms are broken, which can only be associated with crinoids\index{crinoid}.

Broken or kinked arms also appear in Table 25: Figure 4, with this form there are multiple examples which give precisely the same image.

All coral\index{coral} forms throughout make up about a twentieth the total volume of the chondrite\index{chondrite} rock, but constituting the remaining sixteen twentieths, that which is by far the greatest part of the whole mass, is the:
\clearpage
\subsection{The Individual Forms: Crinoids\index{crinoid}}
\paragraph{}
They are found in the simplest form, from their articulately divided arms to the developed crinoid\index{crinoid} with stem, crown, main and auxiliary arms. Their preservation is good for the most part. The difficulty lies only in the thousands of possible directions of cutting, which always give different perspectives of the same object. The pear-shaped bodies, which are regarded as glass are crinoids --- their crowns.

I present four crinoids\index{crinoid} in an upright position and in high quality in Tables 16, 17, 18, and 19 and in profile in Table 20.

Table 21: Figures 1, 2, 3, 4, and 5 show average vertical sections of more developed crinoids. These are the main arms with auxiliary arms and distinct joint surfaces.

Table 21: Figure 3 shows stem and crown. (Figures 2 and 4 have double the magnification of 1 and 3.) Figure 5, from another thin section, is shown to display the conformity of the forms. In Figure 6 I believe one can perceive the mouth opening in the cusp between the arms.

Table 22: Figures 1, 3, 4, and 5, and Table 23: Figures 1 and 2, show five as the number of arms, as well as with the auxiliary arms.

In Table 23: Figures 2 and 3 shows the kinking of arms due to pressure from above.

Table 22: Figures 2 and 4 call to mind Comatulida\index{Comatulida}.

There are particular species of crinoids, which consist only of a number of arms. These are seen in Table 23: Figures 4 and 5, Table 24: Figures 4, 5, and 6 and Table 26 (The picture on Table 24: Figure 6 is a smaller scale of the coral\index{coral} from Cabarras\index{meteorite!Cabarras}, Table 13: Figure 6.)

Table 29: Figures 1, 2, 3, 4, 5, and 6 and Table 27: Figure 3 show pictures of crinoids\index{crinoid} as seen from above.

Table 27: Figure 2 and Table 29: Figure 4 show crinoids\index{crinoid} from below: here the base of the stem emerges as a bright spot. The cross-sectional cuts give dozens of cases showing a consistent form. (See also Table 3: Figure 2, top left. Finer results could probably not have been asked for: the muscle layers are clearly visible here.)

Peculiar entanglements are shown in Table 26: Figures 1, 2, 3, and 4.

The clearest profiles are given in Table 25: Figures 5 and 6. Table 27: Figure 3 is a longitudinal profile with broken arms.

Table 24: Figures 1 and 2 are forms which I first viewed as coral\index{coral}.

Table 28: Figure 1 could, nevertheless, be added to the latter. (The structure should be more clearly preserved for a final decision to be made).

A little clearer is Table 27: Figure 1: an apparent outer casing, which is nothing but regular closed main arms.

An exceptionally nice picture is given in Table 30: Figure 3; whether crinoid?\index{crinoid} this is doubtful. I only take notice, the two parts are symmetrical, and the arms are not placed beside each other, rather they cross.

Table 30: Figure 5 with a cut, I had at first placed as \emph{Urania}. It shall be added to the crinoids.

Table 31: Figures 1, 2, and 3 appear to be similar forms. In Figures 1 and 3 one can perceive a distinct furrow, perhaps this is the place where two crinoid arms lie against one another. With the polarization device, the furrow appears even more clearly. In Figure 4 two individuals are merged, leaving it open to interpretation as either sponge or coral. Figure 5 has a structure in the middle part, some structural tissue, showing the upper arms as distinct structures. Do these belong together? Since the form only occurs once, I dare not make a final decision. The resemblance of the central image with the structure of the schreibersite\index{schreibersite} in meteorites\index{meteorite} is striking. Figure 6 is found twice, so that I consider both parts as related.

The same mesh structure is shown in Table 30: Figure 6 at increased magnification. The structure of both agrees, as suggested before, with the structure of the schreibersite\index{schreibersite} in the meteorites\index{meteorite} and makes an appearance several times.

As I already noted at the beginning, I do not consider my task here to enumerate species. My task is only to establish the existence of organisms by proving unified recurring forms with undoubtedly organic structures. I think that I have done this, and I think that no one should have even the slightest doubt (especially after viewing the originals in thin section) that these do not act as minerals. Even if only five organic forms were verified without a doubt, the other less well-preserved forms would also be organic.

The final determination of the genera and even the species requires more material and years of investigation. (I will be grateful for the former.) Above all, I should have more time than the current night hours and more strength than my current strenuous profession leaves me to finish my work. I think I have given the required points asked for, on which one can stand. In conclusion, I refer to the table commentary.

Thus, the forms are presented. I have been pursuing a plan, of making a statistical study on the occurrence of the forms, to count out something such as the occurrence of same forms that one finds in 500 thin sections. I bring this up, because I felt I had to say, that I did not think such would have great value. Each multiplication of my collection by twelve new ones would change the ratio. I therefore preferred to give an approximate numerical ratio for the individual forms.
\clearpage
\subsection{All Life}
\paragraph{}
The individual forms were brought to view in the previous sections. All these forms are not buried upon death, but one grows upon another and, in truth, they are buried alive by life. Here of course only our vision can provide conviction. To this purpose one should look at all the pictures with the individual forms within their surroundings!

What at first glance appears as a bright spot, upon closer examination shows on the average a sponge, a coral, or a crinoid part. Nowhere are there, as Gümbel\index{Gümbel} has quite rightly observed, disassembled tumbled forms and fragments --- also there is not a binder between them. Only soft tissues are missing, everything else is preserved, just as it was when the life was in water. The crinoid forms show this clearly. For these are, at most, curved on a side, winding, and seldom broken; one sees also that there was only a weak mechanical resistance against neighboring heads. But everything together, grown apart --- nothing laid down, nothing buried. There is also no mass available that could have constituted a grave.

The fact, that there is nothing inorganic in the chondrite\index{chondrite} rocks and not a single place without life in them, I consider to be as important as the existence of the organisms themselves. First, this fact casts full light on the emergence of planets. If one adds to this, that the rock that includes these formations consists of minerals belonging to the purported primary mountains [Urgestein]\index{Urgestein}, yes ``volcanism'' associated with the mountains: then our geology must take a different path in the explanation of the facts. My belief is by no means that the sponges, corals, and crinoids are from minerals we have here, that constitute forms today. The original organisms must have been composed differently; they must have endured a transformation.

It is so much, I think, beyond all doubt that what is nowadays hornblende, augite, and olivine\index{olivine} are what filled the referred-to forms, formerly these minerals must have been in a different condition, namely a liquid water one, a water solution.

Now we find these minerals in our primary mountains as forms, which are not crystals, but are like the meteoritic ones. We find mountain masses composed of such forms. So here too it is highly probable that organic forms, subsequently transformed, are what we now call rocks. These rocks, however, point to a layer that is undoubtedly close to the meteoritic (chondritic)\index{chondrite}, indeed they are closely related. Under this must lie the iron. This testifies to the specific weight of the Earth\index{Earth}. Again, the identical situation appears in the fallen iron meteorites\index{meteorite!iron}: here, as in the Ovifak\index{Ovifak} rock, we find transitions, compositions of iron and olivine\index{olivine}.

This gives us the greatest baseline for geology --- we have the chronological development of the body of the Earth\index{Earth}. The development of form --- the reason for the growth of the forms themselves is at the same time open. If the organism in the lowest layer, that we know of, was the source of mass creation then it could also have been the initial cause for the beginning of the planet itself. The assumption of mere mass-attraction, the mechanical formation of the Earth\index{Earth} and the heavenly bodies would in general be thereby refuted.

Admittedly organisms in iron, in the Earth's core\index{Earth!core}, and in the meteoritic iron must also be detected. It is this task which I set for myself in what follows next. The previous results allow for a hopeful solution.
\clearpage
\subsection{Stone in the Stone}
\paragraph{}
When I said that the chondrite\index{chondrite} is nothing but an animal-fabric, an animal-felt, a qualification must be sustained.

There are, however, very small, sharply outlined places in this animal-bone stone which could probably (but not necessarily) be from the first rocks. These are slate-blue, uncommon inclusions with 3-5 mm. diameters lacking definite recurring forms which include distinct crystals in their grayish mass, these are on average either squares or rhombuses while in other places it includes hexagons. This mineral can be either augite or olivine\index{olivine}. Here the crystalline form is pronounced in favor of a mineral. The sole existence of this speaks for my views. Why have the crystals not grown themselves identically everywhere? And why should there not be hollow cavities remaining in the organisms? It is known that fillers in organic forms later crystallize. And in the final-filled organic forms, cavities are found in which their outlines look like surfaces recessing at an angle.

The reason why I acknowledge that these inclusions are inorganic parts of the chondrites\index{chondrite}, as distinct from actual meteoritic stone (stone in the stone), is because the outlines do not give the indication, that is, their form does not address itself as being organic. These inclusions may be deposits of an already developed rocky mass or they may have only developed in the cavities.

This situation is possible, even probable, that it was a falling-in of pieces of already deposited rock that were fully developed and does not need to be denied: it does not knock on the fact that in the olivine\index{olivine} strata formations exist and that these are the cause of the construction of the planet bodies, their self-constructed development and complex composition.

In all cases, however, the ratio in the chondritic rock is the opposite as that in the sedimentary layers of Earth\index{Earth}. In the latter the organisms are interred and the rock strata enclose them; in the first there are only organisms and the rock strata are masses of such. I put an image of an actual rock-piece from Borkut [Ukraine], Table 32: Figure 2, next to that (Figure 1) I have depicted a form, slate-blue like \emph{Urania}, however, without a set structure its outlines are inconsistent which could be from the lack of filler. If it were an organic form, it would be of the lowest nature. For comparison I show in Table 32: Figure 4 a thin section of Lias $\gamma\delta$\index{Lias} [Early Jurassic] (Zwischenkalk), here shells are located in limestone but most parts are merely pieces of shells; the parts are crushed into all sizes and, regarding their origin, they are tumbled beyond any recognition. In the chondrite\index{chondrite} there is no place remaining that can leave a doubt as to their composition.
\clearpage
\subsection{Reproduction}
\paragraph{}
In the stone there are found a multitude of round and pear-shaped forms with 0.10-0.50 mm. diameters, which barely indicate structure. I hold these forms to be the first developmental forms. Among the many forms, the most outstanding are the transparent spherical forms of rock in the center of which are channel openings. Here one finds these channels within spheres, with two further below and a larger above, and so forth on up to the forms of Table 13: Figures 1, 2, 3, and 4. The case is here, I believe, secure. Not only is this form evident in all the chondrites\index{chondrite}, but in each of them one also finds full developmental stages with up to twenty or more channels: they are common and at the same time certain because of their self-evident channel structure. They have been preserved in those chondrites\index{chondrite} which hardly show the forms on the left. The development suggested here is that the channels reproduce.

Of course, there are many faint spherical and pear-shapes which indicate structure. They appear to have been made of sarcode when they were suddenly interred. I would not dare to bring these forms up if they did not indicate a definite structure. They consist of two, three, four, and five lobed-form branches and are probably the beginnings of crinoids. That the observation of developmental forms is difficult is well-known. Hence, I do not allow myself to act prematurely here. What I say here should only be considered as a pointer towards future research.

Good preservation is an impossibility. This is because meteoritic forms face the same destiny as living animals: it is always the ultimate labor to find that first beginning of development, the embryo.

I will refer to a single fact here, which is a considerable point of proof for the organic nature of the forms: the ever occurring association of the individual forms. Many forms that one finds collectively resemble each other: a few stand individually and at the same time as a unit. I hold this as highly significant. If several individuals of the same species come together, it goes to follow from this that there exists mother or sibling relationships. The same phenomenon is known to occur in the terrestrial types. This would seem to signify, as minerals often do, to which form it belongs, as undoubtedly the same applies to other species' mineral fillings, so that a mineralogical ground from which the different derivatives of structure could be inferred.
\clearpage
\subsection{Development}
\paragraph{}
After having depicted the individual forms, I must now discuss their relations to each other, the development of the unfolding of forms.

That \emph{Urania} is the simplest form, this is certain. However, it establishes the inception of what follows.

These layers in the hemispherical lobes, these tubular layers, they part themselves crosswise --- that which today would constitute an arm connects a channel. It develops a crown between the arms and the growth point and the simplest crinoid is there. If this seems like a twisted chain of events, the forms involuntarily demand it. But just as we always find somewhere in living forms a line of development so should we also not find that the same changes have taken place here? Certainly. Only, I believe, they are found with more quantity and with much greater visibility of transitions in the meteoritic forms. One can find the ancestor of the \emph{Pentacrinus briareus} nowhere else on Earth\index{Earth} except with the corals, and one can see the origin of the coral in the sponge form: it is decidedly a lower form than that of the coral.

What this meteorite-creation\index{meteorite} gives of such great importance to the evolutionary theory is not only the occurrence of animal forms in the deepest strata, but also consistent types for all meteoritic organisms. This becomes clear after viewing hundreds of thin sections one after the other.

The scale of the organisms is uniform, at least one thousand times smaller than the ones of Earth\index{Earth}: the development of the individual forms attains an approximately equal high level. The construction of the forms corresponds perfectly with the circumstances under which they grew, namely an extremely shortened lifespan, which was an experience it had: it is a hasty, relatively incomplete creation. The crinoid is the highest representative of this animal world. I hold that the most advanced is the form in Table 22: Figures 1, 3, 5, and 6, because it really embodies the number five.

One will not want to go so far, however, as to derive the crinoids through the corals, thus the form of \emph{Urania} must offer some clue. I show some forms which have the loose branches. They are indicated in their descriptions. I find at high magnification overlying arms.

Even here an adequate observation of a single is not enough for a complete conclusion.
\clearpage
\section{The Iron Meteorites\index{meteorite!iron}}
\paragraph{}
As I have already indicated in \emph{Primordial Cell}\index{Primordial Cell}, the structure of the iron meteorites\index{meteorite} is nothing other than a single mat of unicellular plants. The so-called Widmanstätten figures are, for the most part nothing other than these unicellular plants.

A piece of the Toluca iron meteorite\index{meteorite!Toluca}\index{meteorite!iron} lies in front of me in which the cylindrical cells alternately emerge from each other, the two are often copulated. The individual cells show a double cell wall (iron band), show cross partitioning, show clear round root points; in some there is a marrow substance (which it is really called), indeed, in the inside of the cell there is yet more structure. All of the cells lie in a mat of filler (iron-filler).

Compare these figures with the forms of the Lias slate, especially \emph{Algacites} [\emph{Fucoides}] \emph{granulatus} and ask yourself, of the two, which one shows a plant structure clearest, the Toluca iron or the \emph{Algacites} from Lias-Epsilon?

These forms are cylindrical, from time to time one sees (on average) approximately polyhedral surfaces: they have walls. What especially distinguishes them from crystals (which can be foreseen from the round forms) are the growth sites.

Crystals, which grow together, set themselves against one crystal surface as well along surfaces, (dendrites of silver, copper); they place themselves along the surfaces of another, without entering them, but in the meteoritic iron one finds penetration instead. The cross section is not a straight line (crystal surface), but a curve.

Here end all similarities with crystals, unless one assumes that there could be cylindrical crystals, which grow out of each other. The claim, that these figures have fixed mathematical positions, may be correct here and there by chance; all researchers accept this fact, that nowhere are the angles constant, which with dendrites is always the case. If one finds a place, out of which an octahedron, a cube, or a different regular crystal form derive their location, even a rhombohedron: immediately the order compared with another is quite different. And how can one speak of crystal laws, when from identical minerals not once has this fixed crystal system been repeated? Because one finds, as I have said, rhombohedral slices next to regular ones.

I find just two objections that seem to be justified:
\begin{enumerate}
    \item The objection, that the figures are occasional sheets:
    
    Against this I want to object that, once a cylindrical form is verified, the forms are just not crystalline and now the conclusion is not that they are cylindrical crystals, but on the contrary, that the plates, which bear the same structure, are not crystals.
    \item The second objection is this: How is it supposed to be that plants transform themselves into iron? This objection is not difficult to refute. One has only to think of our many petrifacts, especially the fossilized stems in the Lias; one recalls the so-called Mansfeld [buds] ears in the Zechstein (Cupressites ulmanni), where cypresses are transformed thru silver-bearing copper. One should think that such an objection could be made.
\end{enumerate}
\paragraph{}
But now I am well by uniting with a revered friend, Professor Dr. H. [Gustav Karl Wilhelm Hermann] Karsten in Schaffhausen, who presently is in a position to furnish evidence for the transformation of plants into iron. Karsten has already proven in the year 1869 that our lowest plants absorb iron through entirely outstanding means; I owe the iron plants of today to his kindness. With his permission I include an excerpt from his excellent work, \emph{The Chemistry of Plant Cells}, Vienna 1869, p. 53 which here follows:

``Bring \emph{Oidium lactis} or yeast in heavy moist air (not under liquid) that has for some time been in contact with lactose together with metallic iron by scattering iron filings on the vegetated milk yeast via a glass objective, at first some of the iron touches the cells, later many are vaguely situated then more or less a rapid intense red color soon comes to a surprising size.''

``One would be constrained to suppose that the cause of this strange and exceptional, often very accelerated enlargement, which alone should cause one to search for a mechanical swell up of the cell membranes if one did not also witness simultaneously, within the layered part of the thickened mother cell under the above indicated cultivation ratios, that the available daughter cells multiply at a modest rate and fill up the mother cell completely.''

``The membranes of the daughter cells also produce an acid, as seen in the iron reaction; their shape is according to the connection of their skin with that of the iron, which is very similar to the previously described protein-crystalloid; such as those located on the surface, 3-4-5-sided, though with fewer sharp edges and angular plates; irregularly juxtaposed, they completely fill the size of the cell cavity, but decrease when the skin of the mother cell breaks, as they fall out more or less together.''

``Similar metamorphoses are experienced by the \emph{Oidium} mycelia, especially the dissecting branches rising in the air, which will, when they are brought under similar conditions and indeed this type often expand unequally from the dissimilar member cells, for the most part primarily the upper more than the lower, and usually a round stem remains, with some stretched, whereby these branches with their head-shaped swollen end-cells Mucor- then fruit- or flower-like will, when the top ones enlarge at the well-defined parting top, or from above to below starts to tear open. The membrane of the primary and secondary cells tears apart, each in its own peculiar manner.''

``Even in regard to the organization of plant cells in general, these vegetations of are of great interest.''

``Those namely, which the above described crystalloid cells contain, are also on the inner surface of each of both the nested cell membranes, which the wall forms, with one minor layer occupied that is either laid and flattened closely together or vaguely with some of each other, and gives to the entire cell system the view and small reticular structure, of a tubercular or porous thickened parenchym cell. \emph{De Cella Vitali} 1843, supplement page 37 and 437. These cells, equivalent morphologically to the secretion cells of the composite plant, grow simultaneously with their mother cell close by, they lie between the primary and secondary and form an epidermis. The whole cell system is highly similar to the envelope of many Pollen- and Diatomaceae- (\emph{Gallionella}, \emph{Biddulphia}, \emph{Coscinodiscus}, \emph{Triceratium}, \emph{Amphitetras} etc.) cells.''

``If one records such a cell system colored red by iron and places it into a new mixture made from the above-mentioned nutrient solution without iron, it will quickly decompose into its elements. The cells, which are similarly assembled, with both the crystalloid cell content and also with the epidermis start to round themselves and enlarge; new generations are originated in them and, finally becoming free as their special mother cell liquefies, one sees through months of continued observation the way that the bottom yeast microsporum, through the development of suitor daughter cells, multiplies.''

``The warty thickened \emph{Oidium} cells permeated with lactic acid iron were the ones which grew forth highly long-shaped contents, from or next to the cells which display a reticular warty epidermis, which one would notice, is in the manner of \emph{Micrococcus}, the \emph{Vibrio} spores.''

``Hyphomycetes, particularly \emph{Penicillium} and \emph{Botrytis}, as well as \emph{Rhizopus}, also give, once they have been vegetated and nourished with lactose for some time and brought into contact with metallic iron, a very interesting preparation, partly like those of \emph{Oidium} with swollen gonidium chains or hyphaloid cells. The gonidia chains of \emph{Penicillium} have a rule in which the gonidium original ancestors at first swell up followed in succession by others down to the youngest. The \emph{Penicillium} gonidia, saturated with nutrient salts in a lactose solution after contact with iron soon slowly swell and develop numerous cells on the inner surface of their progressively enlarged and thickened outer skin, giving it a reticulated or porous appearance, so that forms are similar to those described above with \emph{Oidium}, porous and thick-walled. In other cases, the daughter cells fill the cavity more and become like a mucor-head filled with gonidia.''

``Very often are found, as in the case of \emph{Oidium} when it is poorly cultivated, empty cells with very smooth walls. Quite often the inner cell, impregnated with lactic acid iron, breaks through the outer cellular-warty-etc. thickened membrane, which peels or splits as it grows out.''

``The culture used for this purpose should not be kept moist, because undertaken in humid air these vegetations, which are permeated with acidic iron salts, are very susceptible to decay. Even without such a precaution for the culture, I have seen the member cells and gonidia of mold, as well as \emph{Micrococcus} cells and vibrion germs contained in dust, swell as described when brought into contact with polished metallic iron, no doubt because these cells contain acids or acidic salts.''

``It becomes apparent from the phenomena of the growth of these fungal cells that the cause of their abnormal enlargement is to be found in the subsequent association of this acid with the neutral lactic acid iron to an acidic salt, so that the whole phenomenon of peculiar malformation is based on a purely chemical process that changes those cells vegetating under normal conditions in such a way that normal development becomes pathological and causes the ultimate destruction of the organism.''

``Against the idea that the acid here in the fungi as well as the resin, wax, etc. is produced by the assimilation activity of the cell membrane, could be raised the concern that it may be the secretion cells (microgonidia, vibrion germs) alone that are between these membranes of the cellular system (the cells nested in each other in the 1$^{st}$, 2$^{nd}$, 3$^{rd}$, etc. degrees), as noted above these organic acids produce by their vegetative activity, especially since, without doubt, the vibrions that develop from them, even in the total absence of more developed cell forms, are very energetic producers of acids, e.g. milk, butter, and acetic acid. However, those cells enlarged by the absorption of iron in the same way, whose walls are quite structure-less, i.e. without recognizable cellular organizations between the two composing membranes of the cells nested in each other and without enclosed free cells in their cavity; furthermore, the fact that \emph{Oidium} mycelium and its yeast cells, if they are submerged, first have their membranes blackened followed by the liquid contents of the nucleus and are blackened by iron and sulfur ammonium. Against other metals, like aluminum, magnesium, zinc, cobalt, nickel, even against copper, these lactic acid cells behave similarly as with the iron, but with the same colorless or only slightly colored, partly (especially with copper) fragile organizations. Therefore, these metals are less favorable to experiments with this acid yeast.''

I think that if iron plants can be produced before our eyes, then we should not raise concerns against the assumption of the same process at work in an earlier time, at a time when all the materials of organic formation were available. We have mass formations before us here in the atolls of the calm seas, we have in the chondrites\index{chondrite} a composition of similar animals: what stands in the way of assuming such previous plant-mass formations?

At last, through yeast production, we have a process that is completely analogous, once the fiery heat idea goes away.

Here I come back to the Kant-Laplace hypothesis about mass formation. I have already proven their logical error. How do you seek to bring forth a glowing ball from a vapor mass that also surely included water? Or shall the Earth\index{Earth} only come to embers after it has been formed? By what? Experience speaks only for mass formation through organic means. Apparently, only the sight of the volcanoes has led to the assumption of a liquid fire interior of the Earth\index{Earth}, and this notion led to the assumption that the whole Earth\index{Earth} had once been in this state and that the plutonic rocks were the products of this period. Also, it is by no means certain that the thermal radiation of the Sun comes from a liquid fire body. However, the fact of free water on our Earth\index{Earth}, and also the fact of the Moon (without atmosphere!), indicates that from the beginning mass could not have been in a liquid fire state.

In any case, it is certain that meteoritic iron is not a smelting product, for what should have put the meteorite\index{meteorite} into blaze? I also found crinoid and sponge forms in the meteoritic iron. There is no doubt that Hainholz\index{meteorite!Hainholz} shows such.

As already the Pallasites\index{meteorite!Pallasite} show organic and even animal forms, rocks that form the transition from the pure iron to the chondrite\index{chondrite}, there is thus no reason to assume the pure iron is an inorganic formation and much less as being formerly liquid.

Once the iron is assumed to be the nucleus of planets, I believe it then becomes most probable that the first beginnings of our planet, and therefore of all planets, was an organic formation.
\clearpage
\section{The Iron of Ovifak\index{Ovifak}}
\paragraph{}
Through the kindness of Professor Dr. von Nordenskjöld, I was given six pieces of the iron of Ovifak\index{Ovifak} and a basalt, in which the same was found, for examination.

[Friedrich] Wöhler (New Yearbook for Mineralogy 1869, p. 32) does not consider it to be meteoritic because of its chemical composition. The occurrence of an item in a cleft in one of my pieces does not speak for a meteoritic origin either. Iron parts with Widmannstätten's figures are also found in the basalt and olivine\index{olivine}, and yet both are not addressed as meteoritic. Finally, there are transitions from stone to iron, indicating that the iron did not fall into the basalt by chance. It would be a great miracle if this iron had fallen into it just at the time when the basalt was liquid, quite apart from the fact that this iron would hardly be preserved for more than a few years. And yet this iron is said to be meteoritic because of its structure.

We know, however, that Earth's core\index{Earth!core} is at least the density of this metal, and it probably consists of iron of the same nature, thus the likelihood of us seeing the iron core of the Earth\index{Earth} in Ovifak's\index{Ovifak} iron would be obvious.

That would have won us infinitely more than a new meteorite\index{meteorite}.

On the surface of this iron, which, of course, I do not yet have the permission to assail, I find structures very similar to those of the crinoids\index{crinoid} in the chondrites\index{chondrite}.

However, I must save a thin section investigation until the time when the material is made available to me.
\clearpage
\section{Conclusions}
\subsection{The Origin of Meteorites\index{meteorite!origin}}
\paragraph{}
It is quite certain that small planets, weighing half of the Earth's\index{Earth} kilograms, fall and therefore revolve. One can now think of the following options:
\begin{itemize}
    \item The meteorites\index{meteorite} revolve outside the solar system (one such might have been observed by [Frédéric] Petit in Toulouse)
    \item The meteorites\index{meteorite} revolve within the solar system: by themselves around the sun --- around the Sun with the planets (perhaps even individuals with the Earth\index{Earth}) --- around the sun, the planets, and their satellites.
    \item The meteorites\index{meteorite} revolve in all these paths.
\end{itemize}
\paragraph{}
It is known, from many years of observation, that at certain periods (August 10th, November 13th) swarms of meteorites\index{meteorite} approach our planet and intersect with its orbit; it is known that these swarms are more numerous in certain years than others and that also single meteorites\index{meteorite} fall upon the Earth\index{Earth}, both facts have their cause in the attraction of the Earth\index{Earth}. The orbits of the meteorites\index{meteorite}, however, are not known, neither those of the swarms nor of the individuals; neither those which have fallen nor of those which have merely passed the Earth\index{Earth}. Thus, nothing for the formation of the meteorites can be derived from their orbits.

We now come to wonder what follows from the composition of the meteorites\index{meteorite}. Their chemical elements are the same as those of our Earth\index{Earth}. This fact points to a common origin, that is, the mass of the Earth\index{Earth} formed together with the meteorites\index{meteorite} and the formation and development of all planets was the same. The mere fact of chemical equality leads to various conclusions. I have demonstrated, however, Earthly organisms in the meteorites\index{meteorite} and it cannot be assumed as certain that the dissimilar ones do not occur on Earth\index{Earth}. To my regret, I must admit that the number of doubts has been increased by my discovery.

These questions now arise: did the meteorites\index{meteorite} arise with the Earth\index{Earth}? Are they from the Earth\index{Earth}? Thus, from the beginning, were they a mass along with the Earth\index{Earth} and then separated from it, so that they might be or still are a kind of invisible satellite of the Earth\index{Earth}?

First, I only raise these questions because they are the most important for geology. The specific gravity of the Earth\index{Earth} and the rock of Ovifak\index{Ovifak} make it likely that the Earth\index{Earth} is entirely composed of the same rocks as the meteorites\index{meteorite}, provided that the iron and the stone meteorites\index{meteorite} belong together. It could be concluded that the meteorites\index{meteorite} had originally been part of the Earth\index{Earth} at the time that its formation had progressed to the olivine\index{olivine} layers, and that they had then become detached from it. The latter would have happened as a result of an impact between a world body with the Earth\index{Earth}, for without such, a separation could not be explained unless the gravity of the Earth\index{Earth} suddenly stopped or diminished to such a degree that part of its mass could have been thrown out from its circle of attraction. It is difficult to believe in a shattering from the inside, from gas power or the like, although this too cannot be completely ruled out.

So, for chemical and morphological reasons, it is not possible to draw conclusions from the rock as to whether the meteorites\index{meteorite} are children or brothers of the Earth\index{Earth}, and one must rely on the pronouncement of the astronomer.

But if the latter confirms, by virtue of their orbits, that the meteorites\index{meteorite} could not have been part of the Earth's\index{Earth} mass, then a second question arises: how do the individual cases relate to one another? Are the stones and the irons originally related, or do the stones and the irons have different origins? And a third question: do the chemically and morphologically identical stones belong to a planet which was destroyed by some cause?

The latter, at first sight, could be deduced from the chemical and morphological similarities, and in fact, the matter seems quite simple and clear. But there is another possibility, the possibility that under the same conditions a myriad of small planets could form and perhaps still form today. The pieces would then not be rubble but their own world bodies.

The irons and the stones would now be their own world bodies --- size alone would not stand in the way of the hypothesis. But if the small masses consist of water creatures and they being a mere microscopic creation, then it is natural to wonder: did they live in water or water vapor? Provided they had a continuous source of water, which we can easily imagine since today we have areas on Earth\index{Earth} where rain is always falling and others where there is none. The question must be countered by the fact that the necessary building materials for the microscopic creation must be sought not under but above the creatures, because only aqueous solutions could have built up this microscopic animal world.

This animal world is already at least partially organized. A unicellular plant, a yeast fungus, may have been the beginning of a planet: it could not have been crinoids that organized it because we have to think of the long periods of time, and therefore the much greater mass that this stage of development must have required.

These facts, in connection with the likelihood that the irons were the core of the chondrite\index{chondrite!origin} planet, lead us to regard the chondrites\index{chondrite} as the debris of one and the same world body, debris that has been orbiting, following the destruction of this planet, until it fortunately falls into the path of our Earth\index{Earth}. The forms of the meteorites\index{meteorite} suggest themselves as being rubble.

So, we have only one hypothetical certainty: the likelihood of the original unity of the debris that reaches us.

But if they came from Earth\index{Earth}, then they have been parts of it: the composition of organisms is still a fact that is important for our geological history. However, if they do not come from Earth\index{Earth} they illustrate two facts: the origin of a planet and the probability of the way in which our Earth\index{Earth} was born. But if they were each a planet they testify to a creative power that leaves our concepts about the origin of organic forms and their development far behind.
\clearpage
\subsection{The Formation of the Earth\index{Earth}}
\paragraph{}
Going off the results so far, some conclusions could also be drawn regarding the formation of the Earth\index{Earth}. It is most likely, on average, that the Earth\index{Earth} shows the same sequence of rocks as the meteorites\index{meteorite}, which pass from the iron to the pallasite (olivine\index{olivine} with iron) and from there to the olivine\index{olivine}, enstatite\index{enstatite}, and (feldspar) rocks.

On the Earth\index{Earth}, olivine\index{olivine} is followed by granite, a feldspar rock: this order also corresponds with the specific gravity of the mineral.

The specific gravity of hornblende is 3-3.40, olivine\index{olivine} 3.35, enstatite\index{enstatite} 3.10-3.29, orthoclase 2.53-3.10, and quartz 2-2.80. The high specific gravity of hornblende seems to stem from its iron content. This sequence of specific gravity, just as in their stratification, strongly suggests mineral formation in water, i.e. in an aqueous solution. Here I must repeat what I have already said in \emph{Primordial Cell}\index{Primordial Cell}: that creation, i.e. organic formation, could not have started with crabs (Trilobites)\index{Trilobite}. We find a constant series of forms everywhere in the later strata, so why should this law not continue all the way down to the very beginning?

This alone should lead one to the assumption that the immediate precursors to the Silurian, gneiss, and granite have an organic origin.

With the evidence for the organic composition of the chondrites\index{chondrite} no argument stands in the way for considering the granite as a water structure: both rocks contain mainly feldspar. As concerns the granite, I have found forms in it which are like those of the chondrites\index{chondrite}.

I would like to add some points here to prove that the origin of the granite was not only from water, but from organisms. Feldspar and quartz crystallize, I would say, fervently. In the granite, however, both minerals are regularly not crystallized; feldspar merely shows sheet fractures. This is also seen in lime petrification\index{petrifact}, e.g. a crinoid stalk. Why does feldspar in granite not appear crystallized? Because it is bound by a stronger formative force. The feldspar in granite (where the latter is truly preserved) always shows definite recurring forms, not conglomerated or tumbled, nor, as I have noticed, crystal forms. Here also one form always grows out of an another. These forms are sponge shapes. The quartz fills the cavities.

I would also like to point out the formation of the mountains. Dr. [Friedrich Moritz] Stapff\index{Stapff}, who has sufficiently observed mountain structure from the Gotthard Tunnel\index{Gotthard Tunnel}, explains (New Yearbook of Mineralogy 1869, p. 792) that there is no sign of mass uplift or fragmentation in the Gotthard Tunnel\index{Gotthard Tunnel}, the greatest insight into the Earth's interior\index{Earth!interior} that is known. This ``primordial mountain'' is, according to the findings, a sedimentary mountain. Yes! It is even conceivable that it was formed when our atmosphere still held most of the water, an atmosphere that was not heated by fire in the Earth's interior\index{Earth!interior}, but rather by chemical heat, as it is today. But if this is the case then there remains no reason against explaining the origin of the primitive rocks, and the primordial mountains, by organic life.

Even today lower animals and plants can endure a degree of heat which is fatal for other beings, so there is nothing standing in the way of accepting organic life with an increased degree of heat. Apatite and graphite can also be considered a witness of organic activity. With the precipitation of silica the Earth's\index{Earth} body was finished: it consisted of the bones of dead animals; clay, lime, and salt together with gases and water formed the building materials for further activity on the Earth's\index{Earth} surface. Because this (not solidification, but precipitation) process was mostly completed, the organism obtained space and time for higher development, which was until then impossible, for every new formation buried the barely formed one. Only after a sparingly soluble compound was laid as a coat around the Earth\index{Earth} could the development of forms enter their own right. The Earth's\index{Earth} periods grew longer; with the supply of finer building materials the law of symmetry came into effect. But another cause helped: the lowest organisms are children of the night; a fungus dies in the light of the Sun. The whole of the previous creation, up until the precipitation of the denser building materials, was a nighttime creation: the continuous chemical coupling had to have produced a heat that prevented water from becoming the ocean that it is today. Finally, the chemical coupling was essentially completed, creating a surface, a kind of shell. But now, the light and heat rays of the Sun came into effect, which, until then, had been hitherto blocked by the tall and dense atmosphere. The light creation begins; the kingdom of the Sun overcame the kingdom of the night on our planet, capturing the night into the depths of the Earth\index{Earth}.

Thus, through light, the higher life that suddenly and powerfully emerges with the Silurian is explained: it was the first resting place of creation. Under the influence of light, we now see a development begin, which is so far removed from the earlier forms as life today at the pole differs from that at the equator. This explains the sudden change. If it had merely been a matter of cooling, creation would show a much slower transition. What remained dissolved in the water after the precipitation of magnesium, silicon, potassium, and sodium was relatively little; light could now begin to work. This assumption explains how life arose on the whole Earth\index{Earth}, that there was water on its entire surface, and that aquatic animals could build mountains that would extend far above the current level of the sea. These mountains have not been lifted, nor driven upwards through mechanical force (by momentum), nor squeezed out by the cooling of the surface; because as the latter cooled (more correctly ``dried up''), at most only cracks and clefts could have arisen, for under the surface there was no slurry, but solid mass. According to my current findings, what is the surface, now that the boundary of the ``primordial mountains'' and the succeeding strata has been abolished?\footnote{It has been forgotten in the theory of uplifting that a force which would be necessary to lift mountains would at the same time have crushed them: in the theory of pressing one is unable to say where the mountain has actually remained, through which the semi-solid would have been pressed! The whole surface could not have been squeezed out.} What separates this layer from the ``primordial mountains'' is only the effect of light, which became stronger as the water vapors condensed and filled the fissures of the globe.

But the days of the Earth\index{Earth} would have been numbered if the light had not ended the process of precipitation quickly enough, because the dwindling chemical coupling would have not have taken place quickly enough and life on Earth\index{Earth} and the Earth\index{Earth} itself would have been brought to a standstill forever. These creations of light were new, higher organisms. These organisms were built from the waste materials of the previous creation, which had not yet ceased their organic coupling, and thus halting death. This would have occurred and the Earth\index{Earth} become a desert had it not been for the very reason that the organisms created by the light, with their nourishment and through their respiration, entered into a coupling and once again dissolved the waste, thus creating a cycle called life. So it is light that protects our Earth\index{Earth} from a death that had already occurred on its satellite. But the light works through the water. The water connects the stone and the air and this opens for us a glimpse into the future of our planet.
\clearpage
\subsection{The Future of Our Planet}
\paragraph{}
The fall of planetary fragments upon our Earth\index{Earth} (for this is what the existence of meteorites\index{meteorite} suggests) could cause a physical destruction, a violent death for Earth\index{Earth!destruction} to fear. If it happened to this planet or that planet from which the meteorites\index{meteorite!origin} originate, that it was pulverized, and probably not due to a force from the inside but by an impetus from the outside: so we should be prepared for this fate on Earth\index{Earth}, at least it does threaten us. I will leave it to the astronomers to comfort themselves and their contemporaries.

But we should also be prepared for the previously mentioned cessation of life on the surface, a less bloody but no less comforting end, namely the fate of gradual death, the termination of the coupling of insoluble compounds with the life force and the building materials: we have to worry that our atmosphere will continue to form insoluble compounds from the remaining building materials and thus the cycle will become weaker and slower, and finally --- stopping.

The only thing saving us from this almost certain fate is water; the water that our Earth\index{Earth} was able to acquire and retain in its formation.

The fact that these created beings release the compounds that formed their bodies and that the plant in particular decomposes what it absorbs, while the animal absorbs these excretions within itself and then excretes them immediately again and again, then returning them to the plant (not the soil): through all this, a cycle is created whose end cannot be foreseen.

This process, not the cooling of the Earth's crust\index{Earth!crust}, of which so much has been spoken, constitutes the true story of the Earth's\index{Earth} surface. However, we seem to have a frightening example in the Moon: there, I think, life is extinct. There are neither seas, as it was believed, nor volcanoes; the lack or loss of water was what caused this planet's premature death, which made life extinct soon after its birth.

The heat on our surface seems to depend mostly on the preservation of the atmosphere, which defends against the cold of space. The greater height of the Earth's atmosphere\index{Earth!atmosphere} at the equator, due to the rotation of the Earth\index{Earth} and not just the angle of the Sun's rays, causes a higher and more constant heat: or else, 500 meters above sea level at the equator would experience a cooling of several degrees from the average heat; and otherwise the glacial mass of Chimborazo would melt immediately.

Although heat, as a result of the chemical processes mediated by water, may decrease with time, it is certain that without the protective coat of the atmosphere the Earth's\index{Earth} surface, although it absorbs new solar heat each day, will succumb to such low temperatures at night that it could not sustain life, as has recently been claimed as the cause for the extinction of all life on the Moon.

Heat flows to us from the Sun and is trapped by the atmosphere so that it cannot immediately emanate back into space. Thus, we are surrounded by a double protective mantle: the crust which absorbs heat and the air that holds it back (it is the jacket of the Earth\index{Earth}), and between the two we live, the whole of creation lives in a constant exchange of substances. Here man lives, here the same beings arose which once laid the first foundation stone for the great construction of the Earth\index{Earth!formation}. These lower beings even today testify, by their enormous multiplication and preservation in a temperature in which higher beings would immediately die, to their being the first sculptors of the Earth\index{Earth} itself.

Thus, only if the source of light and heat itself were destroyed would life on Earth\index{Earth} freeze; we have nothing to fear from the extinction of the fiery core of the Earth\index{Earth!core}. For the preservation and metabolism of life, the Sun provides radiating light and heat. Light and heat are therefore mother and father to all living things; from before time they have prevented the organic from becoming inorganic, constantly forming new compounds. But even if so much light and heat should flow to the Earth\index{Earth}, without the continuous activity and transformation of the organic cell life on our planet would be numbered in years.\footnote{The loss of geothermal heat or heat radiated by the Sun would not be the next threatening nightmare, but the disappearance of our atmosphere.}

The origin of the planets is the cell, it is maintained so long as light rays hit the Earth\index{Earth}.

It is possible that over time changes in the chemical composition of the Earth's\index{Earth} surface and atmosphere will occur due to the precipitation of solid compounds, whereby building materials are removed from the cycle. Certainly, under such modified living conditions, other similar, and (according to previous experience) higher organized beings will emerge. Indeed, it can be imagined that there will be a refinement of organisms here on Earth\index{Earth}, in the same proportion as occurred after the olivine-granite\index{olivine} period, and that creatures will arise that consume high amounts of water and gas for their preservation, as is almost the case with many plants.
\clearpage
\section{Explanation of the Tables}
\subsection{Preliminary Note}
\paragraph{}
The stones from which I made my thin sections were thoroughly certified.

The thin sections themselves were made by me with the untiring support of my sister-in-law, Miss Pauline Schloz\index{Schloz}. My collection numbers at 560 (including 360 of Knyahinya\index{meteorite!Knyahinya}), probably the largest collection that is available.

Regarding the manufacture of thin sections, I must mention the circumstances which influenced their appearance.

Anyone who has polished petrifacts\index{petrifact} knows that very few allow a thin slice. Not only because of the often opaque or difficult material (lime, clay), but because structures disappear when ground to (presumed) transparency.

It depends on the way in which the process of petrification occurs in each.

Thus, one is faced with the choice of either having a rather dim cut, in which one sees little, or, driven by the desire for sharper outlines, getting a cut that no longer shows anything, resorting to higher objectives in vain.

Both obstacles can be avoided in the meteorite\index{meteorite} material (which, incidentally, because of the iron, is difficult to grind) only by alternately making thinner and thicker cuts.

Regarding the choice of forms, future researchers will excuse me if I overlooked this or that form. My intention, of course, was to depict all the forms contained in my material. The figures should not only give pictures but also an overall view: I placed the greatest weight on concluding the matter of the nature of the rock.

As far as the order of the tables is concerned, it is related to the order of the material. Since I was aware that I had not yet exhausted the entire material, I did not bother to determine individual forms or to express views on their genetic links to justify them and their order: it was sufficient only to make a preliminary orientation in this direction. And for the present time, it is only a proof of organic rock, not about what everything is.

I avoided giving names not for fear of falling into the hands of critics, but because I came to the realization that by naming, nothing, or not much, is gained.

For a long time, I was faced with the choice of whether I should really take the path of photographic representation. However, I arrived at the decision in question more so out of thoughtfulness for the outsider.

There was a lot of talk regarding imagination in the criticism of \emph{Primordial Cell}\index{Primordial Cell}. I realize that the illustrations were not exact, that might be, but they are correct. For example, see the photographic depiction of the objects in \emph{Primordial Cell}\index{Primordial Cell} on Table 32: Figure 5 compared to Table 4 and 5 in \emph{Primordial Cell}\index{Primordial Cell}.

I would like to ask Dr. Kuntze\index{Kuntze} in Leipzig\index{Leipzig} whether he teaches of such synthetic algae\index{algae} --- if so, I would be very grateful for the provisioning of such a preparation to convince me of an error.\footnote{A similar treatment of Dr. Kuntze\index{Kuntze} with Dr. H. Karsten's\index{Karsten} \emph{Flora Columbiae}. Until he cleanses himself of the accusation Dr. W. Joos\index{Joos} raised against him on these criticisms, he has no right to be heard in science.} As far as I know, the dendrites and ``synthetic algae'', which were thus held against me without any examination or knowledge, are merely stripes not structures and secretions. In accordance with its formation it is usually a uniformly distributed continuously stained bulk, which lies between two stone slabs, i.e. as a perfect surface and so resembles plant shadows.

I admit that ``synthetic algae'' can be made from algae, as some researchers have said. But I must also point out that all structures that are thread or band-like have been explained as algae without much thought. To know that you have an alga in front of you, something more is needed. Things have been explained as plants that certainly do not show half as much form or structure as my pictures in \emph{Primordial Cell}\index{Primordial Cell}. Not all thread or sheaf shapes in rocks or other masses would I explain, using only these features, as algae.

My illustrations in \emph{Primordial Cell}\index{Primordial Cell} clearly show cell walls and cells; if these things were artificial algae or dendrites, they would not have any transverse walls.

With this I return to my subject.

Photography has significant drawbacks for scientific representation, as every researcher knows. For the present subject I had to follow this path simply because I would otherwise have been accused of ``imagination'' again. The Sun and collodion\index{collodion} together do not fool and must ward off any such accusation from the start. But the photographic image incorporates the object to a lesser extent. This was especially felt with my best subjects. In addition, especially at the higher magnifications, only a part of the thin section could be displayed and it was not sharp because of higher and lower rocks blurring the focus of the image. Too high of a magnification (I note this matter for any colleagues) is therefore not suitable in rock thin sections. Another obstacle is that the rocks consist of highly refractive material and the light of mineral fractures must be overcome; this creates light reflections of the most unpleasant kind that an untrained person could easily mistake for forms. To avoid this, I always work with the weakest magnifications to put aside the imperfect structural images.

The photographic focus is more likely to be below the object. The credibility of representation, as I have said, was the only reason for taking this path.

One particularly sensitive cause of additional shortcoming in the photographic representation is the effect of colors on the image. Of all the bad ones, yellow is the worst.

Where yellow is present in the preparation a black stain appears instead of structure. There was no means to rectify this evil. And it is the yellow of the olivine\index{olivine} that does not allow any ray of light through. This is most pronounced in the coral\index{coral} in Table 1: Figure 6: the black ring in the picture is a light yellow (iron). Brown follows yellow, which is also very dark. Blue has the opposite shortcoming, it becomes too light, but it still shows structures.

It goes without saying that the high price of the material imposes a certain economy in the preparations. This limits the selection. It is precisely for this reason why the thin sections must be made by the researcher himself. It is his duty. Admittedly this complicates things by the great amount of time required but it is the only possible way to thoroughly study the subjects.

For magnification and photographic representation, I have the intermediate microphotographic apparatus of Seibert\index{Seibert} \& Krafft\index{Krafft} from Wetzlar\index{Wetzlar} and can commend it as praiseworthy. The pictures were produced under my direction here in the photographic studio of Messrs. Otto Lauer\index{Lauer} \& Carl Bossler\index{Bossler}. Since we all had no practice in this sort of shooting, the contribution of Dr. Schreiner\index{Schreiner}, assistant at the chemical laboratory in Tübingen, was highly welcomed. I did not have additional help, but I think it should not go without mentioning the complete lack of participation from all those scholars to whom this matter most concerns.

In the ordering of the material, I put the sponges\index{sponge} first, followed by the corals\index{coral} and then the crinoids\index{crinoid}.

I have also represented the individual genera numerically in accordance with their frequency of occurrence. Unfortunately, I had to put aside some of the better objects because of their yellow coloring. If Gümbel\index{Gümbel}, as he says in his excellent essay on the Bavarian meteorites\index{meteorite}, proves correct in removing the yellow color by acids, much would be gained.

As for the magnifications, or more correctly the exact size of the magnifications, it came into consideration that the camera imposes a certain observance size. This leads to the bad state of affairs in which all the forms seem equally large.

The magnification specification, i.e. the ratio of the true size to the diameter of the displayed image is thus of very little significance.

I therefore preferred to denote the real size of the object by directly stating the diameter of each shape.
\clearpage
\subsection{Table Index}
\begin{enumerate}
    \item Pictures are numbered from top left to bottom right.
    \item Abbreviations: M. indicates magnification, D. indicates real diameter, mm indicates millimeter.
\end{enumerate}
\clearpage
\pagestyle{fancy}
\fancyhf{}
\rhead{Table 1: Mineral structures along with organic ones from the chondrites\index{chondrite}}
\cfoot{\thepage}
\begin{figure}[b]
\includegraphics[width=\textwidth,height=\textheight,keepaspectratio]{figures/meteorite_1-1_edit-b2.jpg}
\caption{Table 1: Figure 1 --- Enstatite\index{enstatite} (-Bronzite)\index{bronzite} from Kupferberg\index{Kupferberg} M.}
\centering
\end{figure}
\clearpage
\begin{figure}[t]
\includegraphics[width=\textwidth,height=\textheight,keepaspectratio]{figures/meteorite_1-2_edit-b.jpg}
\caption{Table 1: Figure 2 --- Enstatite\index{enstatite} from Texas\index{Texas} M.}
\centering
\end{figure}
\clearpage
\begin{figure}[t]
\includegraphics[width=\textwidth,height=\textheight,keepaspectratio]{figures/meteorite_1-3_edit-b2.jpg}
\caption{Table 1: Figure 3 --- Spherulite-Liparite\index{spherulite} from Lipari M.}
\centering
\end{figure}
\clearpage
\begin{figure}[t]
\includegraphics[width=\textwidth,height=\textheight,keepaspectratio]{figures/meteorite_1-4_edit-b.jpg}
\caption{Table 1: Figure 4 --- A part of the coral\index{coral} from Table 8, 9, and 10}
\centering
\end{figure}
\clearpage
\begin{figure}[t]
\includegraphics[width=\textwidth,height=\textheight,keepaspectratio]{figures/meteorite_1-5_edit-b2.jpg}
\caption{Table 1: Figure 5 --- Chain coral\index{coral} D. 0.90 mm.}
\centering
\end{figure}
\clearpage
\begin{figure}[t]
\includegraphics[width=\textwidth,height=\textheight,keepaspectratio]{figures/meteorite_1-6_edit-b2.jpg}
\caption{Table 1: Figure 6 --- Crinoid\index{crinoid} D. 1.20 mm.}
\centering
\end{figure}
\clearpage
\rhead{Table 2: \emph{Urania}}
\begin{figure}[t]
\includegraphics[width=\textwidth,height=\textheight,keepaspectratio]{figures/meteorite_2-1_edit-b2.jpg}
\caption{Table 2: Figure 1 --- Knyahinya\index{meteorite!Knyahinya}, same as Table 5: Figure 1.}
\centering
\end{figure}
\clearpage
\rhead{Table 3: \emph{Urania}}
\begin{figure}[t]
\includegraphics[width=\textwidth,height=\textheight,keepaspectratio]{figures/meteorite_3-1_edit-b.jpg}
\caption{Table 3: Figure 1 --- Knyahinya\index{meteorite!Knyahinya} D. 0.60 mm.}
\centering
\end{figure}
\clearpage
\begin{figure}[t]
\includegraphics[width=\textwidth,height=\textheight,keepaspectratio]{figures/meteorite_3-2_edit-b.jpg}
\caption{Table 3: Figure 2 --- Knyahinya\index{meteorite!Knyahinya} D. 1.30 mm. (do not overlook the magnificent crinoid\index{crinoid} limbs on the top left!)}
\centering
\end{figure}
\clearpage
\begin{figure}[t]
\includegraphics[width=\textwidth,height=\textheight,keepaspectratio]{figures/meteorite_3-3_edit-b.jpg}
\caption{Table 3: Figure 3 --- Knyahinya\index{meteorite!Knyahinya} D. 1 mm.}
\centering
\end{figure}
\clearpage
\begin{figure}[t]
\includegraphics[width=\textwidth,height=\textheight,keepaspectratio]{figures/meteorite_3-4_edit-b.jpg}
\caption{Table 3: Figure 4 --- Knyahinya\index{meteorite!Knyahinya} D. 1 mm.}
\centering
\end{figure}
\clearpage
\begin{figure}[t]
\includegraphics[width=\textwidth,height=\textheight,keepaspectratio]{figures/meteorite_3-5_edit-b2.jpg}
\caption{Table 3: Figure 5 --- Knyahinya\index{meteorite!Knyahinya} D. 1 mm. (notice the stratification at the top)}
\centering
\end{figure}
\clearpage
\begin{figure}[t]
\includegraphics[width=\textwidth,height=\textheight,keepaspectratio]{figures/meteorite_3-6_edit-b2.jpg}
\caption{Table 3: Figure 6 --- Knyahinya\index{meteorite!Knyahinya} D. 1 mm. (Stratification like 5, but not reproduced in the image, 5 and 6 of a thin section)}
\centering
\end{figure}
\clearpage
\rhead{Table 4: \emph{Urania}}
\begin{figure}[t]
\includegraphics[width=\textwidth,height=\textheight,keepaspectratio]{figures/meteorite_4-1_edit-b.jpg}
\caption{Table 4: Figure 1 --- Knyahinya\index{meteorite!Knyahinya} D. 0.90 mm.}
\centering
\end{figure}
\clearpage
\begin{figure}[t]
\includegraphics[width=\textwidth,height=\textheight,keepaspectratio]{figures/meteorite_4-2_edit-b.jpg}
\caption{Table 4: Figure 2 --- Siena\index{meteorite!Siena} D. 3 mm. (the dark line is due to the yellow color of the preparation)}
\centering
\end{figure}
\clearpage
\begin{figure}[t]
\includegraphics[width=\textwidth,height=\textheight,keepaspectratio]{figures/meteorite_4-3_edit-b.jpg}
\caption{Table 4: Figure 3 --- Knyahinya\index{meteorite!Knyahinya} D. 0.60 mm.}
\centering
\end{figure}
\clearpage
\begin{figure}[t]
\includegraphics[width=\textwidth,height=\textheight,keepaspectratio]{figures/meteorite_4-4_edit-b.jpg}
\caption{Table 4: Figure 4 --- Knyahinya\index{meteorite!Knyahinya} D. 0.90 mm. (air bubble)}
\centering
\end{figure}
\clearpage
\begin{figure}[t]
\includegraphics[width=\textwidth,height=\textheight,keepaspectratio]{figures/meteorite_4-5_edit-b.jpg}
\caption{Table 4: Figure 5 --- Knyahinya\index{meteorite!Knyahinya} D. 1.60 mm.}
\centering
\end{figure}
\clearpage
\begin{figure}[t]
\includegraphics[width=\textwidth,height=\textheight,keepaspectratio]{figures/meteorite_4-6_edit-b.jpg}
\caption{Table 4: Figure 6 --- Knyahinya\index{meteorite!Knyahinya} D. 1.00 mm. (air bubble)}
\centering
\end{figure}
\clearpage
\rhead{Table 5: \emph{Urania}}
\begin{figure}[t]
\includegraphics[width=\textwidth,height=\textheight,keepaspectratio]{figures/meteorite_5-1_edit-b.jpg}
\caption{Table 5: Figure 1 --- Knyahinya\index{meteorite!Knyahinya} D. 1.40 mm. (see Table 2. All around average crinoid\index{crinoid}. Form bottom left, magnification. Table 1: Figure 6 and Table 25: Figures 1 and 2)}
\centering
\end{figure}
\clearpage
\begin{figure}[t]
\includegraphics[width=\textwidth,height=\textheight,keepaspectratio]{figures/meteorite_5-2_edit-b2.jpg}
\caption{Table 5: Figure 2 --- Knyahinya\index{meteorite!Knyahinya} D. 1.80 mm.}
\centering
\end{figure}
\clearpage
\begin{figure}[t]
\includegraphics[width=\textwidth,height=\textheight,keepaspectratio]{figures/meteorite_5-3_edit-b.jpg}
\caption{Table 5: Figure 3 --- Knyahinya\index{meteorite!Knyahinya} D. 1.80 mm.}
\centering
\end{figure}
\clearpage
\begin{figure}[t]
\includegraphics[width=\textwidth,height=\textheight,keepaspectratio]{figures/meteorite_5-4_edit-b.jpg}
\caption{Table 5: Figure 4 --- Knyahinya\index{meteorite!Knyahinya} D. 1.30 mm. (blurred picture)}
\centering
\end{figure}
\clearpage
\begin{figure}[t]
\includegraphics[width=\textwidth,height=\textheight,keepaspectratio]{figures/meteorite_5-5_edit-b2.jpg}
\caption{Table 5: Figure 5 --- Knyahinya\index{meteorite!Knyahinya} D. 1.40 mm. (air bubble)}
\centering
\end{figure}
\clearpage
\begin{figure}[t]
\includegraphics[width=\textwidth,height=\textheight,keepaspectratio]{figures/meteorite_5-6_edit-b2.jpg}
\caption{Table 5: Figure 6 --- Knyahinya\index{meteorite!Knyahinya} D. 0.60 mm. (poor picture. The white circle is the average)}
\centering
\end{figure}
\clearpage
\rhead{Table 6: \emph{Urania}}
\begin{figure}[t]
\includegraphics[width=\textwidth,height=\textheight,keepaspectratio]{figures/meteorite_6-1_edit-b2.jpg}
\caption{Table 6: Figure 1 --- Siena\index{meteorite!Siena} D. 4.00 mm.}
\centering
\end{figure}
\clearpage
\begin{figure}[t]
\includegraphics[width=\textwidth,height=\textheight,keepaspectratio]{figures/meteorite_6-2_edit-b.jpg}
\caption{Table 6: Figure 2 --- Knyahinya\index{meteorite!Knyahinya} D. 0.80 mm.}
\centering
\end{figure}
\clearpage
\begin{figure}[t]
\includegraphics[width=\textwidth,height=\textheight,keepaspectratio]{figures/meteorite_6-3_edit-b.jpg}
\caption{Table 6: Figure 3 --- Siena\index{meteorite!Siena} D. 1.20 mm.}
\centering
\end{figure}
\clearpage
\begin{figure}[t]
\includegraphics[width=\textwidth,height=\textheight,keepaspectratio]{figures/meteorite_6-4_edit-b.jpg}
\caption{Table 6: Figure 4 --- Knyahinya\index{meteorite!Knyahinya} D. 0.70 mm. (the center is heavily illuminated)}
\centering
\end{figure}
\clearpage
\begin{figure}[t]
\includegraphics[width=\textwidth,height=\textheight,keepaspectratio]{figures/meteorite_6-5_edit-b.jpg}
\caption{Table 6: Figure 5 --- Knyahinya\index{meteorite!Knyahinya} D. 0.30 mm.}
\centering
\end{figure}
\clearpage
\begin{figure}[t]
\includegraphics[width=\textwidth,height=\textheight,keepaspectratio]{figures/meteorite_6-6_edit-b2.jpg}
\caption{Table 6: Figure 6 --- Knyahinya\index{meteorite!Knyahinya} D. 0.90 mm. (air bubble)}
\centering
\end{figure}
\clearpage
\rhead{Table 7: Sponges}
\begin{figure}[t]
\includegraphics[width=\textwidth,height=\textheight,keepaspectratio]{figures/meteorite_7-1_edit-b.jpg}
\caption{Table 7: Figure 1 --- Knyahinya\index{meteorite!Knyahinya} D. 2.30 mm.}
\centering
\end{figure}
\clearpage
\begin{figure}[t]
\includegraphics[width=\textwidth,height=\textheight,keepaspectratio]{figures/meteorite_7-2_edit-b.jpg}
\caption{Table 7: Figure 2 --- Knyahinya\index{meteorite!Knyahinya} D. 1.80 mm. (a crack in the preparation. Needle)}
\centering
\end{figure}
\clearpage
\begin{figure}[t]
\includegraphics[width=\textwidth,height=\textheight,keepaspectratio]{figures/meteorite_7-3_edit-b.jpg}
\caption{Table 7: Figure 3 --- Knyahinya\index{meteorite!Knyahinya} D. 2.10 mm.}
\centering
\end{figure}
\clearpage
\begin{figure}[t]
\includegraphics[width=\textwidth,height=\textheight,keepaspectratio]{figures/meteorite_7-4_edit-b.jpg}
\caption{Table 7: Figure 4 --- (Crinoid\index{crinoid} cross section?) of Knyahinya\index{meteorite!Knyahinya} D. 3.00 mm.}
\centering
\end{figure}
\clearpage
\begin{figure}[t]
\includegraphics[width=\textwidth,height=\textheight,keepaspectratio]{figures/meteorite_7-5_edit-b.jpg}
\caption{Table 7: Figure 5 --- Sponge? D. 1.00 mm.\index{sponge}}
\centering
\end{figure}
\clearpage
\begin{figure}[t]
\includegraphics[width=\textwidth,height=\textheight,keepaspectratio]{figures/meteorite_7-6_edit-b.jpg}
\caption{Table 7: Figure 6 --- Sponge? D. 2.40 mm.\index{sponge}}
\centering
\end{figure}
\clearpage
\rhead{Table 8: Corals}
\begin{figure}[t]
\includegraphics[width=\textwidth,height=\textheight,keepaspectratio]{figures/meteorite_8-1_edit-b2.jpg}
\caption{Table 8: Figure 1 --- (\emph{Favosites}\index{Favosites}) of Knyahinya\index{meteorite!Knyahinya} (see Table 1: Figure 4)}
\centering
\end{figure}
\clearpage
\rhead{Table 9: Corals}
\begin{figure}[t]
\includegraphics[width=\textwidth,height=\textheight,keepaspectratio]{figures/meteorite_9-1_edit-b3.jpg}
\caption{Table 9: Figure 1 --- Structure picture from top left of Table 8.}
\centering
\end{figure}
\clearpage
\rhead{Table 10: Corals}
\begin{figure}[t]
\includegraphics[width=\textwidth,height=\textheight,keepaspectratio]{figures/meteorite_10-1_edit-b.jpg}
\caption{Table 10: Figure 1 --- Knyahinya\index{meteorite!Knyahinya} cross section D. 0.40 mm.}
\centering
\end{figure}
\clearpage
\begin{figure}[t]
\includegraphics[width=\textwidth,height=\textheight,keepaspectratio]{figures/meteorite_10-2_edit-b.jpg}
\caption{Table 10: Figure 2 --- Longitudinal section 0.50 mm.}
\centering
\end{figure}
\clearpage
\begin{figure}[t]
\includegraphics[width=\textwidth,height=\textheight,keepaspectratio]{figures/meteorite_10-3_edit-b2.jpg}
\caption{Table 10: Figure 3 --- Knyahinya\index{meteorite!Knyahinya} D. 1.80 mm.}
\centering
\end{figure}
\clearpage
\begin{figure}[t]
%this figure has the same figure as Table 1: Figure 5, which has better quality
\includegraphics[width=\textwidth,height=\textheight,keepaspectratio]{figures/meteorite_1-5_edit-b2.jpg}
\caption{Table 10: Figure 4 --- Knyahinya\index{meteorite!Knyahinya} D. 0.90 mm. (see Table 8 and 9.)}
\centering
\end{figure}
\clearpage
\begin{figure}[t]
\includegraphics[width=\textwidth,height=\textheight,keepaspectratio]{figures/meteorite_10-5_edit-b.jpg}
\caption{Table 10: Figure 5 --- Knyahinya\index{meteorite!Knyahinya} D. 0.30 mm.}
\centering
\end{figure}
\clearpage
\begin{figure}[t]
\includegraphics[width=\textwidth,height=\textheight,keepaspectratio]{figures/meteorite_10-6_edit-b.jpg}
\caption{Table 10: Figure 6 --- Knyahinya\index{meteorite!Knyahinya} D. 0.80 mm.}
\centering
\end{figure}
\clearpage
\rhead{Table 11: Corals}
\begin{figure}[t]
\includegraphics[width=\textwidth,height=\textheight,keepaspectratio]{figures/meteorite_11-1_edit-b.jpg}
\caption{Table 11: Figure 1 --- Knyahinya\index{meteorite!Knyahinya} D. 1.20 mm.}
\centering
\end{figure}
\clearpage
\begin{figure}[t]
\includegraphics[width=\textwidth,height=\textheight,keepaspectratio]{figures/meteorite_11-2_edit-b.jpg}
\caption{Table 11: Figure 2 --- Knyahinya\index{meteorite!Knyahinya} D. 1.00 mm.}
\centering
\end{figure}
\clearpage
\begin{figure}[t]
\includegraphics[width=\textwidth,height=\textheight,keepaspectratio]{figures/meteorite_11-3_edit-b.jpg}
\caption{Table 11: Figure 3 --- Knyahinya\index{meteorite!Knyahinya} D. 1.80 mm.}
\centering
\end{figure}
\clearpage
\begin{figure}[t]
\includegraphics[width=\textwidth,height=\textheight,keepaspectratio]{figures/meteorite_11-4_edit-b.jpg}
\caption{Table 11: Figure 4 --- Knyahinya\index{meteorite!Knyahinya} D. 1.20 mm.}
\centering
\end{figure}
\clearpage
\begin{figure}[t]
\includegraphics[width=\textwidth,height=\textheight,keepaspectratio]{figures/meteorite_11-5_edit-b.jpg}
\caption{Table 11: Figure 5 --- Parnallee\index{meteorite!Parnallee} D. 0.80 mm.}
\centering
\end{figure}
\clearpage
\begin{figure}[t]
\includegraphics[width=\textwidth,height=\textheight,keepaspectratio]{figures/meteorite_11-6_edit.jpg}
\caption{Table 11: Figure 6 --- Moung County\index{meteorite!Moung County} D. 0.60 mm.}
\centering
\end{figure}
\clearpage
\rhead{Table 12: Corals}
\begin{figure}[t]
\includegraphics[width=\textwidth,height=\textheight,keepaspectratio]{figures/meteorite_12-1_edit-b.jpg}
\caption{Table 12: Figure 1 --- Knyahinya\index{meteorite!Knyahinya} D. 0.80 mm.}
\centering
\end{figure}
\clearpage
\begin{figure}[t]
\includegraphics[width=\textwidth,height=\textheight,keepaspectratio]{figures/meteorite_12-2_edit-b.jpg}
\caption{Table 12: Figure 2 --- Knyahinya\index{meteorite!Knyahinya} D. 1.20 mm.}
\centering
\end{figure}
\clearpage
\begin{figure}[t]
\includegraphics[width=\textwidth,height=\textheight,keepaspectratio]{figures/meteorite_12-3_edit-b.jpg}
\caption{Table 12: Figure 3 --- Knyahinya\index{meteorite!Knyahinya} D. 1.30 mm.}
\centering
\end{figure}
\clearpage
\begin{figure}[t]
\includegraphics[width=\textwidth,height=\textheight,keepaspectratio]{figures/meteorite_12-4_edit-b.jpg}
\caption{Table 12: Figure 4 --- Knyahinya\index{meteorite!Knyahinya} D. 1.40 mm.}
\centering
\end{figure}
\clearpage
\begin{figure}[t]
\includegraphics[width=\textwidth,height=\textheight,keepaspectratio]{figures/meteorite_12-5_edit-b.jpg}
\caption{Table 12: Figure 5 --- Knyahinya\index{meteorite!Knyahinya} D. 2.00 mm.}
\centering
\end{figure}
\clearpage
\begin{figure}[t]
\includegraphics[width=\textwidth,height=\textheight,keepaspectratio]{figures/meteorite_12-6_edit-b.jpg}
\caption{Table 12: Figure 6 --- Knyahinya\index{meteorite!Knyahinya} D. 3.20 mm.}
\centering
\end{figure}
\clearpage
\rhead{Table 13: Corals}
\begin{figure}[t]
\includegraphics[width=\textwidth,height=\textheight,keepaspectratio]{figures/meteorite_13-1_edit-b.jpg}
\caption{Table 13: Figure 1 --- Parnallee\index{meteorite!Parnallee} D. 0.20 mm.}
\centering
\end{figure}
\clearpage
\begin{figure}[t]
\includegraphics[width=\textwidth,height=\textheight,keepaspectratio]{figures/meteorite_13-2_edit-b.jpg}
\caption{Table 13: Figure 2 --- Knyahinya\index{meteorite!Knyahinya} D. 0.80 mm.}
\centering
\end{figure}
\clearpage
\begin{figure}[t]
\includegraphics[width=\textwidth,height=\textheight,keepaspectratio]{figures/meteorite_13-3_edit-b.jpg}
\caption{Table 13: Figure 3 --- Siena\index{meteorite!Siena} D. 0.20 mm.}
\centering
\end{figure}
\clearpage
\begin{figure}[t]
\includegraphics[width=\textwidth,height=\textheight,keepaspectratio]{figures/meteorite_13-4_edit-b.jpg}
\caption{Table 13: Figure 4 --- Knyahinya\index{meteorite!Knyahinya} D. 1.80 mm.}
\centering
\end{figure}
\clearpage
\begin{figure}[t]
\includegraphics[width=\textwidth,height=\textheight,keepaspectratio]{figures/meteorite_13-5_edit-b.jpg}
\caption{Table 13: Figure 5 --- Knyahinya\index{meteorite!Knyahinya} D. 1.70 mm.}
\centering
\end{figure}
\clearpage
\begin{figure}[t]
\includegraphics[width=\textwidth,height=\textheight,keepaspectratio]{figures/meteorite_13-6_edit-b.jpg}
\caption{Table 13: Figure 6 --- Cabarras\index{meteorite!Cabarras} D. 0.30 mm.}
\centering
\end{figure}
\clearpage
\rhead{Table 14: Corals}
\begin{figure}[t]
\includegraphics[width=\textwidth,height=\textheight,keepaspectratio]{figures/meteorite_14-1_edit-b2.jpg}
\caption{Table 14: Figure 1 --- Coral\index{coral} D. 0.90 mm.}
\centering
\end{figure}
\clearpage
\rhead{Table 15: Corals}
\begin{figure}[t]
\includegraphics[width=\textwidth,height=\textheight,keepaspectratio]{figures/meteorite_15-1_edit-b3.jpg}
\caption{Table 15: Figure 1 --- Coral\index{coral}. Structure picture from 14. The upper left part of the preparation, magnification 300, shows the bud canals.}
\centering
\end{figure}
\clearpage
\rhead{Table 16: Crinoids}
\begin{figure}[t]
\includegraphics[width=\textwidth,height=\textheight,keepaspectratio]{figures/meteorite_16-1_edit-b2.jpg}
\caption{Table 16: Figure 1 --- Knyahinya\index{meteorite!Knyahinya} D. 0.40 mm.}
\centering
\end{figure}
\clearpage
\rhead{Table 17: Crinoids}
\begin{figure}[t]
\includegraphics[width=\textwidth,height=\textheight,keepaspectratio]{figures/meteorite_17-1_edit-b2.jpg}
\caption{Table 17: Figure 1 --- Knyahinya\index{meteorite!Knyahinya} D. 2.00 mm.}
\centering
\end{figure}
\clearpage
\rhead{Table 18: Crinoids}
\begin{figure}[t]
\includegraphics[width=\textwidth,height=\textheight,keepaspectratio]{figures/meteorite_18-1_edit-b2.jpg}
\caption{Table 18: Figure 1 --- Knyahinya\index{meteorite!Knyahinya}, cut through four main arms, D. 2.20 mm.}
\centering
\end{figure}
\clearpage
\rhead{Table 19: Crinoids}
\begin{figure}[t]
\includegraphics[width=\textwidth,height=\textheight,keepaspectratio]{figures/meteorite_19-1_edit-b2.jpg}
\caption{Table 19: Figure 1 --- Crinoid\index{crinoid}, see Table 25: Figures 1 and 2.}
\centering
\end{figure}
\clearpage
\rhead{Table 20: Crinoids}
\begin{figure}[t]
\includegraphics[width=\textwidth,height=\textheight,keepaspectratio]{figures/meteorite_20-1_edit-b2.jpg}
\caption{Table 20: Figure 1 --- Cut through crinoid\index{crinoid} and coral\index{coral} in Knyahinya\index{meteorite!Knyahinya} D. 1.20 mm.}
\centering
\end{figure}
\clearpage
\rhead{Table 21: Crinoids}
\begin{figure}[t]
\includegraphics[width=\textwidth,height=\textheight,keepaspectratio]{figures/meteorite_21-1_edit-b.jpg}
\caption{Table 21: Figure 1 --- Knyahinya\index{meteorite!Knyahinya} D. 0.80 mm.}
\centering
\end{figure}
\clearpage
\begin{figure}[t]
\includegraphics[width=\textwidth,height=\textheight,keepaspectratio]{figures/meteorite_21-2_edit-b.jpg}
\caption{Table 21: Figure 2 --- magnified image from Figure 1}
\centering
\end{figure}
\clearpage
\begin{figure}[t]
\includegraphics[width=\textwidth,height=\textheight,keepaspectratio]{figures/meteorite_21-3_edit-b.jpg}
\caption{Table 21: Figure 3 --- Knyahinya\index{meteorite!Knyahinya} D. 1.20 mm.}
\centering
\end{figure}
\clearpage
\begin{figure}[t]
\includegraphics[width=\textwidth,height=\textheight,keepaspectratio]{figures/meteorite_21-4_edit-b.jpg}
\caption{Table 21: Figure 4 --- magnified image from Figure 3}
\centering
\end{figure}
\clearpage
\begin{figure}[t]
\includegraphics[width=\textwidth,height=\textheight,keepaspectratio]{figures/meteorite_21-5_edit-b.jpg}
\caption{Table 21: Figure 5 --- Knyahinya\index{meteorite!Knyahinya} D. 1.80 mm. (I notice resemblance with Figure 1)}
\centering
\end{figure}
\clearpage
\begin{figure}[t]
\includegraphics[width=\textwidth,height=\textheight,keepaspectratio]{figures/meteorite_21-6_edit-b.jpg}
\caption{Table 21: Figure 6 --- Knyahinya\index{meteorite!Knyahinya} D. 0.30 mm. (the mouth opening between the arms is visible)}
\centering
\end{figure}
\clearpage
\rhead{Table 22: Crinoids}
\begin{figure}[t]
\includegraphics[width=\textwidth,height=\textheight,keepaspectratio]{figures/meteorite_22-1_edit-b.jpg}
\caption{Table 22: Figure 1 --- Knyahinya\index{meteorite!Knyahinya} D. 0.50 mm.}
\centering
\end{figure}
\clearpage
\begin{figure}[t]
\includegraphics[width=\textwidth,height=\textheight,keepaspectratio]{figures/meteorite_22-2_edit-b.jpg}
\caption{Table 22: Figure 2 --- Knyahinya\index{meteorite!Knyahinya} D. 0.60 mm.}
\centering
\end{figure}
\clearpage
\begin{figure}[t]
\includegraphics[width=\textwidth,height=\textheight,keepaspectratio]{figures/meteorite_22-3_edit-b.jpg}
\caption{Table 22: Figure 3 --- Knyahinya\index{meteorite!Knyahinya} (Cover picture) D. 1.50 mm.}
\centering
\end{figure}
\clearpage
\begin{figure}[t]
\includegraphics[width=\textwidth,height=\textheight,keepaspectratio]{figures/meteorite_22-4_edit-b.jpg}
\caption{Table 22: Figure 4 --- Knyahinya\index{meteorite!Knyahinya} D. 0.70 mm.}
\centering
\end{figure}
\clearpage
\begin{figure}[t]
\includegraphics[width=\textwidth,height=\textheight,keepaspectratio]{figures/meteorite_22-5_edit-b.jpg}
\caption{Table 22: Figure 5 --- Knyahinya\index{meteorite!Knyahinya} D. 0.60 mm.}
\centering
\end{figure}
\clearpage
\begin{figure}[t]
\includegraphics[width=\textwidth,height=\textheight,keepaspectratio]{figures/meteorite_22-6_edit-b.jpg}
\caption{Table 22: Figure 6 --- Knyahinya\index{meteorite!Knyahinya} D. 1.20 mm.}
\centering
\end{figure}
\clearpage
\rhead{Table 23: Crinoids}
\begin{figure}[t]
\includegraphics[width=\textwidth,height=\textheight,keepaspectratio]{figures/meteorite_23-1_edit-b.jpg}
\caption{Table 23: Figure 1 --- Knyahinya\index{meteorite!Knyahinya} D. 0.90 mm.}
\centering
\end{figure}
\clearpage
\begin{figure}[t]
\includegraphics[width=\textwidth,height=\textheight,keepaspectratio]{figures/meteorite_23-2_edit-b.jpg}
\caption{Table 23: Figure 2 --- Knyahinya\index{meteorite!Knyahinya} D. 1.60 mm.}
\centering
\end{figure}
\clearpage
\begin{figure}[t]
\includegraphics[width=\textwidth,height=\textheight,keepaspectratio]{figures/meteorite_23-3_edit-b.jpg}
\caption{Table 23: Figure 3 --- Knyahinya\index{meteorite!Knyahinya} D. 1.00 mm.}
\centering
\end{figure}
\clearpage
\begin{figure}[t]
\includegraphics[width=\textwidth,height=\textheight,keepaspectratio]{figures/meteorite_23-4_edit-b.jpg}
\caption{Table 23: Figure 4 --- Knyahinya\index{meteorite!Knyahinya} D. 1.40 mm.}
\centering
\end{figure}
\clearpage
\begin{figure}[t]
\includegraphics[width=\textwidth,height=\textheight,keepaspectratio]{figures/meteorite_23-5_edit-b.jpg}
\caption{Table 23: Figure 5 --- Knyahinya\index{meteorite!Knyahinya} D. 1.30 mm.}
\centering
\end{figure}
\clearpage
\begin{figure}[t]
\includegraphics[width=\textwidth,height=\textheight,keepaspectratio]{figures/meteorite_23-6_edit-b.jpg}
\caption{Table 23: Figure 6 --- Knyahinya\index{meteorite!Knyahinya} D. 0.60 mm.}
\centering
\end{figure}
\clearpage
\rhead{Table 24: Crinoids}
\begin{figure}[t]
\includegraphics[width=\textwidth,height=\textheight,keepaspectratio]{figures/meteorite_24-1_edit-b.jpg}
\caption{Table 24: Figure 1 --- Siena\index{meteorite!Siena} D. 0.80 mm.}
\centering
\end{figure}
\clearpage
\begin{figure}[t]
\includegraphics[width=\textwidth,height=\textheight,keepaspectratio]{figures/meteorite_24-2_edit-b.jpg}
\caption{Table 24: Figure 2 --- Knyahinya\index{meteorite!Knyahinya} D. 2.80 mm.}
\centering
\end{figure}
\clearpage
\begin{figure}[t]
\includegraphics[width=\textwidth,height=\textheight,keepaspectratio]{figures/meteorite_24-3_edit-b.jpg}
\caption{Table 24: Figure 3 --- Knyahinya\index{meteorite!Knyahinya} D. 1.00 mm.}
\centering
\end{figure}
\clearpage
\begin{figure}[t]
\includegraphics[width=\textwidth,height=\textheight,keepaspectratio]{figures/meteorite_24-4_edit-b.jpg}
\caption{Table 24: Figure 4 --- Knyahinya\index{meteorite!Knyahinya} D. 2.00 mm.}
\centering
\end{figure}
\clearpage
\begin{figure}[t]
\includegraphics[width=\textwidth,height=\textheight,keepaspectratio]{figures/meteorite_24-5_edit-b.jpg}
\caption{Table 24: Figure 5 --- Knyahinya\index{meteorite!Knyahinya} D. 1.50 mm.}
\centering
\end{figure}
\clearpage
\begin{figure}[t]
\includegraphics[width=\textwidth,height=\textheight,keepaspectratio]{figures/meteorite_24-6_edit-b.jpg}
\caption{Table 24: Figure 6 --- Cabarras\index{meteorite!Cabarras} D. 0.80 mm.}
\centering
\end{figure}
\clearpage
\rhead{Table 25: Crinoids}
\begin{figure}[t]
\includegraphics[width=\textwidth,height=\textheight,keepaspectratio]{figures/meteorite_25-1_edit-b.jpg}
\caption{Table 25: Figure 1 --- Knyahinya\index{meteorite!Knyahinya} D. 1.20 mm.}
\centering
\end{figure}
\clearpage
\begin{figure}[t]
\includegraphics[width=\textwidth,height=\textheight,keepaspectratio]{figures/meteorite_25-2_edit-b.jpg}
\caption{Table 25: Figure 2 --- Knyahinya\index{meteorite!Knyahinya} D. 1.20 mm.}
\centering
\end{figure}
\clearpage
\begin{figure}[t]
\includegraphics[width=\textwidth,height=\textheight,keepaspectratio]{figures/meteorite_25-3_edit-b.jpg}
\caption{Table 25: Figure 3 --- Knyahinya\index{meteorite!Knyahinya} D. 1.80 mm.}
\centering
\end{figure}
\clearpage
\begin{figure}[t]
\includegraphics[width=\textwidth,height=\textheight,keepaspectratio]{figures/meteorite_25-4_edit-b.jpg}
\caption{Table 25: Figure 4 --- Knyahinya\index{meteorite!Knyahinya} D. 0.60 mm.}
\centering
\end{figure}
\clearpage
\begin{figure}[t]
\includegraphics[width=\textwidth,height=\textheight,keepaspectratio]{figures/meteorite_25-5_edit-b.jpg}
\caption{Table 25: Figure 5 --- Siena\index{meteorite!Siena} D. 1.80 mm.}
\centering
\end{figure}
\clearpage
\begin{figure}[t]
\includegraphics[width=\textwidth,height=\textheight,keepaspectratio]{figures/meteorite_25-6_edit-b.jpg}
\caption{Table 25: Figure 6 --- Knyahinya\index{meteorite!Knyahinya} D. 1.40 mm. (Both latter are cross sections of crinoids)}
\centering
\end{figure}
\clearpage
\rhead{Table 26: Crinoids}
\begin{figure}[t]
\includegraphics[width=\textwidth,height=\textheight,keepaspectratio]{figures/meteorite_26-1_edit-b.jpg}
\caption{Table 26: Figure 1 --- Knyahinya\index{meteorite!Knyahinya} D. 0.20 mm.}
\centering
\end{figure}
\clearpage
\begin{figure}[t]
\includegraphics[width=\textwidth,height=\textheight,keepaspectratio]{figures/meteorite_26-2_edit-b.jpg}
\caption{Table 26: Figure 2 --- Knyahinya\index{meteorite!Knyahinya} D. 2.00 mm.}
\centering
\end{figure}
\clearpage
\begin{figure}[t]
\includegraphics[width=\textwidth,height=\textheight,keepaspectratio]{figures/meteorite_26-3_edit-b.jpg}
\caption{Table 26: Figure 3 --- Knyahinya\index{meteorite!Knyahinya} D. 1.20 mm.}
\centering
\end{figure}
\clearpage
\begin{figure}[t]
\includegraphics[width=\textwidth,height=\textheight,keepaspectratio]{figures/meteorite_26-4_edit-b.jpg}
\caption{Table 26: Figure 4 --- Knyahinya\index{meteorite!Knyahinya} D. 1.20 mm. (here twisted crinoids)}
\centering
\end{figure}
\clearpage
\begin{figure}[t]
\includegraphics[width=\textwidth,height=\textheight,keepaspectratio]{figures/meteorite_26-5_edit-b.jpg}
\caption{Table 26: Figure 5 --- Knyahinya\index{meteorite!Knyahinya} D. 2.00 mm.}
\centering
\end{figure}
\clearpage
\begin{figure}[t]
\includegraphics[width=\textwidth,height=\textheight,keepaspectratio]{figures/meteorite_26-6_edit-b.jpg}
\caption{Table 26: Figure 6 --- Knyahinya\index{meteorite!Knyahinya} D. 2.20 mm. (the dark line in 5 and 6 is the food channel)}
\centering
\end{figure}
\clearpage
\rhead{Table 27: Crinoids}
\begin{figure}[t]
\includegraphics[width=\textwidth,height=\textheight,keepaspectratio]{figures/meteorite_27-1_edit-b.jpg}
\caption{Table 27: Figure 1 --- Knyahinya\index{meteorite!Knyahinya} D. 0.80 mm.}
\centering
\end{figure}
\clearpage
\begin{figure}[t]
\includegraphics[width=\textwidth,height=\textheight,keepaspectratio]{figures/meteorite_27-2_edit-b.jpg}
\caption{Table 27: Figure 2 --- Knyahinya\index{meteorite!Knyahinya} D. 1.50 mm.}
\centering
\end{figure}
\clearpage
\begin{figure}[t]
\includegraphics[width=\textwidth,height=\textheight,keepaspectratio]{figures/meteorite_27-3_edit-b.jpg}
\caption{Table 27: Figure 3 --- Knyahinya\index{meteorite!Knyahinya} D. 1.40 mm.}
\centering
\end{figure}
\clearpage
\begin{figure}[t]
\includegraphics[width=\textwidth,height=\textheight,keepaspectratio]{figures/meteorite_27-4_edit-b.jpg}
\caption{Table 27: Figure 4 --- Knyahinya\index{meteorite!Knyahinya} D. 1.40 mm.}
\centering
\end{figure}
\clearpage
\begin{figure}[t]
\includegraphics[width=\textwidth,height=\textheight,keepaspectratio]{figures/meteorite_27-5_edit-b.jpg}
\caption{Table 27: Figure 5 --- Knyahinya\index{meteorite!Knyahinya} D. 1.20 mm.}
\centering
\end{figure}
\clearpage
\begin{figure}[t]
\includegraphics[width=\textwidth,height=\textheight,keepaspectratio]{figures/meteorite_27-6_edit-b.jpg}
\caption{Table 27: Figure 6 --- Knyahinya\index{meteorite!Knyahinya} D. 1.00 mm.}
\centering
\end{figure}
\clearpage
\rhead{Table 28: Crinoids}
\begin{figure}[t]
\includegraphics[width=\textwidth,height=\textheight,keepaspectratio]{figures/meteorite_28-1_edit-b.jpg}
\caption{Table 28: Figure 1 --- Knyahinya\index{meteorite!Knyahinya} (Coral?) D. 3.00 mm. from the same thin section as Table 18.}
\centering
\end{figure}
\clearpage
\begin{figure}[t]
\includegraphics[width=\textwidth,height=\textheight,keepaspectratio]{figures/meteorite_28-2_edit-b.jpg}
\caption{Table 28: Figure 2 --- Knyahinya\index{meteorite!Knyahinya} D. 1.20 mm.}
\centering
\end{figure}
\clearpage
\begin{figure}[t]
\includegraphics[width=\textwidth,height=\textheight,keepaspectratio]{figures/meteorite_28-3_edit-b.jpg}
\caption{Table 28: Figure 3 --- Knyahinya\index{meteorite!Knyahinya} D. 2.30 mm.}
\centering
\end{figure}
\clearpage
\begin{figure}[t]
\includegraphics[width=\textwidth,height=\textheight,keepaspectratio]{figures/meteorite_28-4_edit-b.jpg}
\caption{Table 28: Figure 4 --- Knyahinya\index{meteorite!Knyahinya} D. 0.90 mm.}
\centering
\end{figure}
\clearpage
\begin{figure}[t]
\includegraphics[width=\textwidth,height=\textheight,keepaspectratio]{figures/meteorite_28-5_edit-b.jpg}
\caption{Table 28: Figure 5 --- Knyahinya\index{meteorite!Knyahinya} D. 1.50 mm.}
\centering
\end{figure}
\clearpage
\begin{figure}[t]
\includegraphics[width=\textwidth,height=\textheight,keepaspectratio]{figures/meteorite_28-6_edit-b.jpg}
\caption{Table 28: Figure 6 --- Knyahinya\index{meteorite!Knyahinya} D. 1.40 mm.}
\centering
\end{figure}
\clearpage
\rhead{Table 29: Crinoids (1-3 viewed from above, 4 from below)}
\begin{figure}[t]
\includegraphics[width=\textwidth,height=\textheight,keepaspectratio]{figures/meteorite_29-1_edit-b.jpg}
\caption{Table 29: Figure 1 --- Knyahinya\index{meteorite!Knyahinya} D. 0.20 mm.}
\centering
\end{figure}
\clearpage
\begin{figure}[t]
\includegraphics[width=\textwidth,height=\textheight,keepaspectratio]{figures/meteorite_29-2_edit-b.jpg}
\caption{Table 29: Figure 2 --- Knyahinya\index{meteorite!Knyahinya} D. 0.90 mm.}
\centering
\end{figure}
\clearpage
\begin{figure}[t]
\includegraphics[width=\textwidth,height=\textheight,keepaspectratio]{figures/meteorite_29-3_edit-b.jpg}
\caption{Table 29: Figure 3 --- Tabor\index{meteorite!Tabor} D. 2.10 mm.}
\centering
\end{figure}
\clearpage
\begin{figure}[t]
\includegraphics[width=\textwidth,height=\textheight,keepaspectratio]{figures/meteorite_29-4_edit-b.jpg}
\caption{Table 29: Figure 4 --- Knyahinya\index{meteorite!Knyahinya} D. 1.10 mm.}
\centering
\end{figure}
\clearpage
\begin{figure}[t]
\includegraphics[width=\textwidth,height=\textheight,keepaspectratio]{figures/meteorite_29-5_edit-b.jpg}
\caption{Table 29: Figure 5 --- Borkut\index{Borkut} D. 1.50 mm.}
\centering
\end{figure}
\clearpage
\begin{figure}[t]
\includegraphics[width=\textwidth,height=\textheight,keepaspectratio]{figures/meteorite_29-6_edit-b.jpg}
\caption{Table 29: Figure 6 --- Knyahinya\index{meteorite!Knyahinya} D. 1.30 mm. (questionable)}
\centering
\end{figure}
\clearpage
\rhead{Table 30: Crinoids}
\begin{figure}[t]
\includegraphics[width=\textwidth,height=\textheight,keepaspectratio]{figures/meteorite_30-1_edit-b.jpg}
\caption{Table 30: Figure 1 --- Knyahinya\index{meteorite!Knyahinya} D. 1.10 mm. (Coral?)}
\centering
\end{figure}
\clearpage
\begin{figure}[t]
\includegraphics[width=\textwidth,height=\textheight,keepaspectratio]{figures/meteorite_30-2_edit-b.jpg}
\caption{Table 30: Figure 2 --- Knyahinya\index{meteorite!Knyahinya} D. 1.40 mm. (Coral and Crinoid, see Table 20)}
\centering
\end{figure}
\clearpage
\begin{figure}[t]
\includegraphics[width=\textwidth,height=\textheight,keepaspectratio]{figures/meteorite_30-3_edit-b.jpg}
\caption{Table 30: Figure 3 --- Knyahinya\index{meteorite!Knyahinya} D. 0.30 mm. (the arms entwined like a mesh)}
\centering
\end{figure}
\clearpage
\begin{figure}[t]
\includegraphics[width=\textwidth,height=\textheight,keepaspectratio]{figures/meteorite_30-4_edit-b.jpg}
\caption{Table 30: Figure 4 --- Knyahinya\index{meteorite!Knyahinya} D. 1.85 mm. (first slice)}
\centering
\end{figure}
\clearpage
\begin{figure}[t]
\includegraphics[width=\textwidth,height=\textheight,keepaspectratio]{figures/meteorite_30-5_edit-b.jpg}
\caption{Table 30: Figure 5 --- Knyahinya\index{meteorite!Knyahinya} D. 0.70 mm. (first slice)}
\centering
\end{figure}
\clearpage
\begin{figure}[t]
\includegraphics[width=\textwidth,height=\textheight,keepaspectratio]{figures/meteorite_30-6_edit-b.jpg}
\caption{Table 30: Figure 6 --- Knyahinya\index{meteorite!Knyahinya} D. 0.40 mm. (Structure like the Schreibersite in the iron meteorites)}
\centering
\end{figure}
\clearpage
\rhead{Table 31: Problematic}
\begin{figure}[t]
\includegraphics[width=\textwidth,height=\textheight,keepaspectratio]{figures/meteorite_31-1_edit-b.jpg}
\caption{Table 31: Figure 1 --- Knyahinya\index{meteorite!Knyahinya} D. 1.20 mm. (not quite complete picture)}
\centering
\end{figure}
\clearpage
\begin{figure}[t]
\includegraphics[width=\textwidth,height=\textheight,keepaspectratio]{figures/meteorite_31-2_edit-b.jpg}
\caption{Table 31: Figure 2 --- Knyahinya\index{meteorite!Knyahinya} D. 0.50 mm.}
\centering
\end{figure}
\clearpage
\begin{figure}[t]
\includegraphics[width=\textwidth,height=\textheight,keepaspectratio]{figures/meteorite_31-3_edit-b.jpg}
\caption{Table 31: Figure 3 --- Knyahinya\index{meteorite!Knyahinya} D. 1.20 mm. (Three corresponding forms of three thin sections, in both 1 and 2 horizontal cuts)}
\centering
\end{figure}
\clearpage
\begin{figure}[t]
\includegraphics[width=\textwidth,height=\textheight,keepaspectratio]{figures/meteorite_31-4_edit-b.jpg}
\caption{Table 31: Figure 4 --- Knyahinya\index{meteorite!Knyahinya} (whether sponge or coral?) D. 0.90 mm.}
\centering
\end{figure}
\clearpage
\begin{figure}[t]
\includegraphics[width=\textwidth,height=\textheight,keepaspectratio]{figures/meteorite_31-5_edit-b.jpg}
\caption{Table 31: Figure 5 --- Knyahinya\index{meteorite!Knyahinya} D. 1.50 mm.}
\centering
\end{figure}
\clearpage
\begin{figure}[t]
\includegraphics[width=\textwidth,height=\textheight,keepaspectratio]{figures/meteorite_31-6_edit-b.jpg}
\caption{Table 31: Figure 6 --- Knyahinya\index{meteorite!Knyahinya} D. 1.40 mm.}
\centering
\end{figure}
\clearpage
\rhead{Table 32: Miscellaneous}
\begin{figure}[t]
\includegraphics[width=\textwidth,height=\textheight,keepaspectratio]{figures/meteorite_32-1_edit-b.jpg}
\caption{Table 32: Figure 1 --- Knyahinya\index{meteorite!Knyahinya} (inclusion) D. 1.50 mm.}
\centering
\end{figure}
\clearpage
\begin{figure}[t]
\includegraphics[width=\textwidth,height=\textheight,keepaspectratio]{figures/meteorite_32-2_edit-b.jpg}
\caption{Table 32: Figure 2 --- Borkut\index{Borkut} sphere D. 1.00 mm.}
\centering
\end{figure}
\clearpage
\begin{figure}[t]
\includegraphics[width=\textwidth,height=\textheight,keepaspectratio]{figures/meteorite_32-3_edit-b.jpg}
\caption{Table 32: Figure 3 --- Nummulite\index{nummulite} from Kempten\index{Kempten}. The channel is clearly visible (with the magnifying glass).}
\centering
\end{figure}
\clearpage
\begin{figure}[t]
\includegraphics[width=\textwidth,height=\textheight,keepaspectratio]{figures/meteorite_32-4_edit-b.jpg}
\caption{Table 32: Figure 4 --- Thin section from Lias\index{Lias} $\gamma\delta$. This thin section is taken from the assembled collection of 30 thin sections of sedimentary rocks, manufactured by geologist Hildebrand\index{Hildebrand} in Ohmenhausen\index{Ohmenhausen} near Reutlingen\index{Reutlingen}, which I strongly recommend for studying the microscopic nature of sedimentary rocks and inclusions.}
\centering
\end{figure}
\clearpage
\begin{figure}[t]
\includegraphics[width=\textwidth,height=\textheight,keepaspectratio]{figures/meteorite_32-5_edit-b.jpg}
\caption{Table 32: Figure 5 --- \emph{Eozoön canadense}\index{Eozoön}, so-called channel system of \emph{Eozoön}\index{Eozoön}.}
\centering
\end{figure}
\clearpage
\begin{figure}[t]
\includegraphics[width=\textwidth,height=\textheight,keepaspectratio]{figures/meteorite_32-6_edit-b.jpg}
\caption{Table 32: Figure 6 --- ditto. Both cuts taken from rocks collected by me in Little Nation. Compare the channel system of the numulites\index{nummulite} in Figure 3 with this alleged channel system! Picture 3 and 5 should be the same object. Compare to Figure 5 from \emph{Primordial Cell}\index{Primordial Cell} Table 4 and 5.}
\centering
\end{figure}
\clearpage
\pagestyle{plain}
\printindex
\clearpage
\end{document}
