\documentclass[a4paper, 11pt, oneside, german]{article}
\usepackage[utf8]{inputenc}
\usepackage[T1]{fontenc}
\usepackage[ngerman]{babel}
\usepackage{yfonts}
%\usepackage{fbb} %Derived from Cardo, provides a Bembo-like font family in otf and pfb format plus LaTeX font support files
\usepackage{booktabs}
\usepackage{url}
\usepackage{enumitem}
\usepackage{graphicx}
\setlength{\emergencystretch}{15pt}
\graphicspath{ {./figures/} }
\usepackage[figurename=]{caption}
\usepackage{fancyhdr}
\usepackage{imakeidx}
\usepackage{microtype}
\usepackage[titles]{tocloft}
\makeindex[columns=2, title=\swabfamily{Alphabetical Index}, intoc]
\usepackage{sectsty}

\sectionfont{\Huge}
\subsectionfont{\LARGE}
\subsubsectionfont{\LARGE}
\begin{document}
\swabfamily
\renewcommand{\contentsname}{
\swabfamily{Inhaltsverzeichnis}
}
\renewcommand{\listfigurename}{
\swabfamily{Abbildungsverzeichnis}
}
% fix toc page numbers
\let\origcftsecfont\cft
\let\origcftsecpagefont\cftsecpagefont
\let\origcftsecafterpnum\cftsecafterpnum
\renewcommand{\cftsecpagefont}{\swabfamily{\origcftsecpagefont}}
\renewcommand{\cftsecafterpnum}{\swabfamily{\origcftsecafterpnum}}
\let\origcftsubsecpagefont\cftsubsecpagefont
\let\origcftsubsecafterpnum\cftsubsecafterpnum
\renewcommand{\cftsubsecpagefont}{\swabfamily{\origcftsubsecpagefont}}
\renewcommand{\cftsubsecafterpnum}{\swabfamily{\origcftsubsecafterpnum}}
\begin{titlepage} % Suppresses headers and footers on the title page
	\centering % Centre everything on the title page
	\scshape % Use small caps for all text on the title page

	%------------------------------------------------
	%	Title
	%------------------------------------------------
	
	\rule{\textwidth}{1.6pt}\vspace*{-\baselineskip}\vspace*{2pt} % Thick horizontal rule
	\rule{\textwidth}{0.4pt} % Thin horizontal rule
	
	\vspace{0.75\baselineskip} % Whitespace above the title
	
	{\Huge Die Urzelle\\}
	\vspace{0.75\baselineskip}
	{\Large Nebst dem Beweis\\ dass\\ Granit, Gneis, Serpentin, Talk, Gewisse\\ Sandsteine, auch Basalt, endlich\\ Meteorstein und Meteoreisen aus Pflanzen\\ Bestehen:\\ die Entwicklungslehre durch\\ Tatsachen neu Begr"undet\\} % Title
	
	\vspace{0.75\baselineskip} % Whitespace below the title
	
	\rule{\textwidth}{0.4pt}\vspace*{-\baselineskip}\vspace{3.2pt} % Thin horizontal rule
	\rule{\textwidth}{1.6pt} % Thick horizontal rule
	
	\vspace{1\baselineskip} % Whitespace after the title block
	
	%------------------------------------------------
	%	Subtitle
	%------------------------------------------------
	
	{\Large von\\ \LARGE Dr. Otto Hahn\\} % Subtitle or further description
	
	\vspace*{1\baselineskip} % Whitespace under the subtitle
	
    {\small mit 30 Lithographirten Tafeln} % Subtitle or further description
    
	%------------------------------------------------
	%	Editor(s)
	%------------------------------------------------

    %------------------------------------------------
	%	Cover photo
	%------------------------------------------------
	
	%------------------------------------------------
	%	Publisher
	%------------------------------------------------
		
	\vspace*{\fill}% Whitespace under the publisher logo
	
	1$^{st}$ Edition, T"ubingen 1879 % Publication year
	
	{\small H. Laupp'schen Buchhandlung } % Publisher

	\vspace{1\baselineskip} % Whitespace under the publisher logo

    Internet Archive Online Edition  % Publication year
	
	{\small Attribution NonCommercial ShareAlike 4.0 International } % Publisher
\end{titlepage}
\setlength{\parskip}{1mm plus1mm minus1mm}
\setcounter{tocdepth}{2}
\setcounter{secnumdepth}{3}
\pagestyle{fancy}
\fancyhf{}
\cfoot{\swabfamily{\thepage}}
\tableofcontents
\clearpage
\listoffigures
\clearpage
\LARGE
\section*{\swabfamily{Vorrede}}
\paragraph{}
Es gibt wohl keine Wissenschaft, welche nicht und zwar gerade da, wo sie es am wenigsten w"unscht, wei"se Bl"atter in ihrem goldenen Buche aufzuweisen h"atte.

In der Geologie war die Frage "uber die Entstehung und Bildung unserer Erdoberfl"ache eine vielbesprochene und man redete sich manchmal ein, eine gel"oste.

Aber man musste sich dann, wenn man ehrlich sein wollte, doch gestehen, dass eben exakte Beweise f"ur den einen oder andern behaupteten Hergang fehlten.

Dies traf vorzugsweise beim Urgebirge zu --- kehrte aber auch beim Fl"oz- und am bedenklichsten bei den "alteren "`vulkanischen Gesteinen"' wieder. Wohl m"uhten sich M"anner der Wissenschaft ab, ihre Behauptungen "uber die "`Entstehung"' des Urgebirgs zur Geltung zu bringen. Beweise lagen vor, aber kein einziger, welcher durchschlug.

Einen solchen glaube ich jetzt geliefert zu haben.

Die "`Entstehung"' der ganzen heutigen Erdoberfl"ache mit Ausnahme der heute noch sich bildenden vulkanischen Gebirge und auch bei diesen hat das Wasser --- zum mindesten einen mechanischen, ich glaube aber auch chemischen Anteil, --- ich sage die "`Entstehung des Urgebirgs"' ohne alle Feuerwirkung ist durch die vorgelegten Tatsachen erwiesen, und zwar so voll als "uberhaupt etwas zu beweisen denkbar ist. M"oglich ist es freilich, auch diesen Beweis eine Weile anzuzweifeln. Allein ich bin sicher, der Widerspruch wird bald verstummen, denn wo man etwas millionenfach sehen kann, da kann ein Zweifel nicht mehr bestehen. Ein gew"ohnliches Mikroskop, eine gute Lupe gen"ugt vollst"andig, sich zu "uberzeugen.

So kann ich denn meine Arbeit ruhig in die Welt senden. Ihr Erfolg, die Anerkennung der Tatsachen und ihrer Deutung, macht mir keine Sorge.

Allerdings werden die Botaniker nicht ungestraft in ihre Wissenschaft mich haben einbrechen lassen.

M"ogen sie die Hauptsache ins Auge fassen.

Was die Bedeutung meiner Entdeckungen betrifft, so wird mir billig Niemand zumuten, sie nicht zu kennen und zu verstehen. Die bisherige Wissenschaft der Geologie ist mit Einem Schlage zum dritten Teile antiquiert. Ja ich darf es aussprechen, es ist nun erst ein Grund darin gelegt.

Durch meine Entdeckung der Pflanze im Meteorstein von Knyahinya war mir ein Blick in die Sph"aren des Himmels gestattet. Was dem stolzen Fernrohr nicht, das war dem stillen, bescheidenen Mikroskop verg"onnt.

Hier dr"angt es mich den Dank gegen Gott auszusprechen, der eine jahrelange Arbeit, f"ur welche ich die Zeit neben meinem anstrengenden Berufe als Rechtsanwalt nur in den Stunden der Erholung finden konnte, mit einem Erfolg gelohnt hat.

Alles hat seine Zeit, auch die Entdeckungen haben sie. Wer sie macht, ist zuletzt gleichgiltig.

Ferner habe ich hier noch "offentlich Dank zu sagen der hochgeehrten Naturwissenschaftlichen Fakult"at in T"ubingen, welche meine erste Arbeit "uber das \emph{Eozoon canadense} mit der Doktor-W"urde honoris causa belohnt hat. Sie hat mir durch diese Anerkennung den Weg der Arbeit erleichtert und mich aufgefordert, der hohen Ehre mich w"urdig zu zeigen.

Es m"oge mir gestattet sein, meinen Genossen in der Wissenschaft und auf dem Wege der Forschung noch dringend ans Herz zu legen, jetzt auf dem aufgeschlossenen Wege besonnen vorzugehen.

Ich muss die Arbeit in andere H"ande legen, das wei"s ich.

Wer sie aufnimmt, der setze sich vor, jahrelang zu beobachten und zu sammeln, ehe er die Welt mit einer Hypothese bel"astigt, welche doch meist nur eine Behauptung "uber das M"ogliche ist. Eine Tatsache festgestellt, festgestellt wie es ist, ist mehr wert, als zehn Hypothesen, wie es gegangen sein k"onnte. ---

Die Zeichnungen sind alle von mir entworfen, teilweise ausgef"uhrt und auf Stein gezeichnet.

Eine besondere Erkl"arung zu den Figuren hielt ich nicht f"ur notwendig, da der Text sehr kurz ist.

Reutlingen, August 1879

\rightline{Dr. Otto Hahn, Rechts-Anwalt.}
\clearpage
\section{}
\subsection[\swabfamily{Geschichte einer Entdeckung}]{\swabfamily{Geschichte einer Entdeckung\footnote{\swabfamily{Ich nenne das Kapitel "`Geschichte einer Entdeckung"': muss aber auf II. verweisen als Fortsetzung dieses Kapitels. Sp"atere Entdeckungen konnte ich nicht mehr hier auff"uhren.}}}}
\paragraph{}
Sir W. Logan in Montreal war es, welcher zuerst die Aufmerksamkeit auf einen im Laurentian-Gneis von Canada eingelagerten Serpentin-Kalk lenkte, indem er dabei die Vermutung aussprach, es enthalte dieser Kalk und zwar in eingebetteten "`Knollen"' die Reste der ersten organischen Sch"opfung.

Dr. W. Dawson in Montreal, Kanzler des M'Gill College dort, nahm sich der Sache weiter an und bestimmte in Verbindung mit Dr. Hunt in Montreal und Dr. W. Carpenter in London, dem ber"uhmtesten Mikroskopiker Englands, die Natur jener Knollen dahin, sie seien die versteinerten Reste einer Riesen-Foraminifere, die dem Genus \emph{Nummulina} "ahnlich sei. Dr. Dawson nannte dieselbe \emph{Eozoon canadense}. Unter diesem Namen wurde sie dann auch zur gro"sen Freude Aller auf das erste noch ganz wei"s gelassene Blatt der Naturgeschichte eingetragen. Allerdings wurde der Eintrag angefochten. Alsbald erhoben sich bedeutende Stimmen gegen die organische Natur des Gesteins. Die Schriften "uber \emph{Eozoon} schwollen derma"sen an, dass man sagen darf: das \emph{Eozoon} hat seine eigene Literatur.

Ich selbst beteiligte mich sp"at erst an dem Kampfe und glaubte in einem Aufs"atze "uber \emph{Eozoon canadense} in den W"urttembergischen naturwissenschaftlichen Jahresheften den Beweis gegen die organische Natur gef"uhrt zu haben. Carl M"obius erg"anzte denselben durch Abbildungen, insbesondere von lebenden Foraminiferen.

Ich werde am Schluss dieser Abhandlung noch "uber die Frage, warum das \emph{Eozoon canadense} kein Tier sein kann, mich verbreiten.

Bei diesem Streit war das Misslichste der Mangel an Material. Von Canada war so gut wie nichts zu bekommen, im Handel vollends nichts.

Die St"ucke, welche gelehrte Freunde mir schenkten, waren d"unn geschnittene Pl"attchen einer ganz bestimmten Form, n"amlich der "`lamellaren"' des Gesteins.

M"obius allerdings war gl"ucklicher.

Ihm stellten Dr. Dawson und Carpenter neben Handst"ucken ihre besten Schliffe zur Verf"ugung.

Freilich schlie"st diese besondere G"ute zugleich eine gro"se Gefahr in sich, weil man nur das zu sehen bekommt, was der Gegner als Beweisst"uck verwerten will. ---

So lag die Sache. Ich war von der unorganischen Natur des Eozoongesteins "uberzeugt, davon wenigstens, dass es ein Tier nicht sei.

Einzig die feinen "`Kanal-Systeme"', wie sie Dr. Carpenter nannte, lie"sen mir keine Ruhe. Die Unruhe wurde wesentlich versch"arft durch den Gedanken, welcher sich mir immer wieder aufdr"angte:

"`der Gniess ist ein durch Wasser gebildetes also Sedimentgestein: seine Kalklager m"ussen die ersten organischen Einschl"usse enthalten, denn mit dem Silur kann das Leben nicht anfangen."'

Letzteres war Hypothese, aber eine Hypothese wie viele, Gedanken, von welchen man sich nicht mehr losmachen kann: denn sie sind eben wahr.

Da f"ugte sich's, dass ich auf eine Einladung der Regierung von Canada eine Reise dorthin zu machen hatte. Dr. Dawson wurde besucht, mein n"achster Gang war nach Côte St. Pierre (Petite Nation). Dort sah ich die Schichtenlagerung und setzte mich in den Besitz einer gro"sen Anzahl von Eozoonkalk- und Eozoonst"ucken.

Nach meiner R"uckreise untersuchte ich das Material.

Das Ergebnis lege ich vor. Es besteht darin: dass der Kalk des Laurentian-Gneises von Canada, der "altesten Sedimentschichte unserer Erde, eine Pflanzensch"opfung enth"alt, angeh"orend der Familie der Algen.

Es sind bis jetzt wenige neue, von den lebenden verschiedene Arten festgestellt. Dass es aber bei fortgesetzter Untersuchung in kurzer Zeit mehr werden, davon bin ich "uberzeugt.

Die Pflanzen sind alle in dem wirklichen "`Eozoongestein"', welches ich von nun an Eophyllumkalk nenne, eingeschlossen von mir gefunden worden.

Ich bemerke n"amlich hier sofort, worauf auch mein verehrter Freund Dr. Dawson aufmerksam machte: Not all is \emph{Eozoon}!

Anfangs allerdings wurde zwischen Eozoongestein und den Knollen nicht unterschieden.

Manches Wort w"are nicht geschrieben worden, wenn man "uber das Gestein selbst einig gewesen w"are.
\clearpage
\subsection{\swabfamily{Die Lagerung des \emph{Eophyllum}-Kalks}}
\paragraph{}
Wir verdanken dem Geological Survey of Canada unter der Leitung (fr"uher Sir Willam Logans nun) des Herrn Dr. Selvyn die geologische Beschreibung und eine sehr sch"one geologische Karte von Canada, ferner seit 10 Jahren fortlaufende in englischer wie franz"osischer Sprache erscheinende j"ahrliche Berichte. Das Verdienst Sir Willams ist es auch, die Laurentian-Formation als solche auf dem Granit auflagernd festgestellt zu haben. Diese Formation, also die "alteste, soll nach Sir Logans Feststellung 30,000 Fu"s m"achtig sein.

Gro"se Z"uge des Laurentian-Gneises erstrecken sich von S"uden nach Norden streichend "uber ganz Canada.

Das einmal f"uhrt er Orthoklas- das andermal Anorthit-Feldspat.

In letzterem Gneis sind die Kalkschichten eingebettet, welche wieder die Eophyllumschichten enthalten.

Der Kalk wird in der "`Geologie von Canada"' Kristalline Limestone genannt.

Er mag bei Côte St. Pierre 100 M. m"achtig sein.

Das den Kalk umh"ullende Gestein ist Diorit und dieser ist in dem Gneis eingelagert, welcher hier vorzugsweise von Hornblende und Glimmer gebildet wird. Der Glimmer ist Biotit.

Ich f"uge hier bei, dass dieselben Kalkschichten eine Reihe von Mineralien und zwar in gro"ser Menge Apatit, Glimmer und Augit enthalten.

Sodann kommen vor Datolit, Scapolit, Salit, Spinell, und sehr sch"one Zirkon.

Im m"achtigen Kalke liegen Schichten von Dolomit und Augit, "uber diesen eine Lage von edlem Serpentin (Ophit) mit Chrysotiladern und nun folgen wechsellagernd die Serpentin- und Kalklamellen und Schichten, welche als "`\emph{Eozoon}"' angesprochen werden.

Die erste Kalklamelle ist im Durchschnitt etwa 50 mm., die zweite etwa 30 mm.; es folgen sich 5-6 immer kleiner und dann geht das Gestein in eine k"ornige (acervuline) Form "uber.

In diesen Serpentin-Kalkschichten, aber auch auf Gneis oder Dolomit aufgelagert, finden sich "`Knollen"', d. h. rundliche Gesteinsst"ucke, welche eine gewisse Regelm"a"sigkeit der Form zeigen.

Es sind Serpentinlamellen, welche von einem Mittelpunkt ausgehend sich dann verj"ungend nach der Seite verbreiten.

Die beste Abbildung, die ich kenne, ist in Dawson's \emph{Life's Dawn on Earth}, London 1875 zu S. 168 in Naturselbstdruck und daher unfehlbar treu.

Unbegreiflich ist, dass man beim Anblick dieser Form nicht sofort an eine Pflanze dachte.

Es erkl"art sich nur daraus, dass man von Anfang an, da man solche St"ucke nicht hatte, einmal in den Gedanken einer Foraminifere vertieft, davon nicht mehr loskam, w"ahrend die Gegner (worunter auch ich) im Kampf gegen die Foraminifere auch an nichts weiter dachten, als daran: kein Tier, --- also Mineral. Die Abbildungen des \emph{Eozoon}-Gesteins sind nach St"ucken gemacht, welche einfach in Platten geschnitten sind. So auch die Abbildungen von Professor Dr. Zittel in der Pal"aontologie: doch es soll hier nicht vorgegriffen werden.

Die Eophyllumschichte ist etwa 50 Cm. m"achtig: Aus den k"ornigen Formen geht das Gestein wieder "uber in die geschichtete. Es folgen sich Serpentin und Dolomit und dann lagert sich wieder m"achtiger Kalk auf.

Der Serpentin ist "uberall ein wirkliches Lagergestein. Er ist durchsichtig, gr"un, gelb, wo er der Luft ausgesetzt ist, wird er r"otlich.

Die Schichten von Serpentin, wie die von Dolomit um den \emph{Eophyllum}-Kalk und der Eophyllumkalk selbst sind je 30-50 Cm. m"achtig. Die Eophyllumschichte hat nicht ein horizontales Lager und eine solche Decke, sondern beide sind uneben, rundlich, zuweilen in allen Curvenlinien ausgezackt. Das \emph{Eophyllum}-Gestein --- weggedacht aus seinem Lager, w"urde eine unregelm"a"sige H"ohlung "ahnlich unseren Tropfsteinh"ohlen "ubrig bleiben.

Die Eophyllumschichte selbst schlie"st Brocken Dolomit ein, um welche sich das breite Serpentinband und dann erst der Eophyllumkalk wieder lagert.
\clearpage
\subsection{\swabfamily{Was ist \emph{Eophyllum}?}}
\paragraph{}
Ich fand die Form, welche ich zuerst \emph{Eophyllum} nannte, in einem "`Eozoonst"uck"' in dem ersten wei"sen Kalkbande, das auf der Serpentinschichte lagert, also zwischen zwei Serpentinb"andern.

Erst jetzt kam mir der Gedanke: Sind denn nicht die ganzen "`\emph{Eozoon}-Knollen"' auch Pflanzen?

Diese Frage musste ich bejahen, nachdem ich durch Behandlung des Kalks mit Salzs"aure gr"o"sere Lamellen, welche in Verbindung mit den Serpentinb"andern stehen, blo"sgelegt hatte. Die Formen sind so konstante, stets wiederkehrende, dass sie anders nicht erkl"art werden k"onnen.

Damit war nat"urlich auch der beste Gegenbeweis gegen ein Tier gewonnen. Denn die entdeckten Algenarten finden sich auch heute nie in Steinen oder Muschelschalen.

Die Pflanzen geh"oren alle der Familie der Algen an. Sie sind entweder unmittelbar auf Dolomit und Gneis aufgelagert oder in dem eigentlichen Eophyllumkalk d. h. der Schichte von Serpentinkalk zwischen den gro"sen Dolomit- und Serpentinschichten.

Aber nicht blo"s im Kalk dieser Schichten finden sie sich, sondern auch im Serpentin derselben.

Keine oder wenige Pflanzen finden sich in den gro"sen Serpentinschichten, welche den Eophyllumkalk umgeben, wenigstens nicht in der untersten.

Diese Serpentinlagen im Eophyllumkalk eignen sich verm"oge ihrer Durchsichtigkeit vorz"uglich zu Beobachtungen: die feinsten Struktur-Verh"altnisse lassen sich hier in Form der zartesten Linien wahrnehmen.

Allerdings geh"ort ein scharfes Auge dazu, um nicht die Linien, welche der Serpentin immer bildet, mit den Zellw"anden zu verwechseln. Diese Eigenschaft des Serpentins war wohl die Ursache, warum es 19 Jahre anstand, bis die "altesten aber beste erhaltenen Pflanzenreste entdeckt wurden.

Im Serpentin finden sich F"ullmassen zerst"orter Zellen in Dolomit verwandelt. An der Hand derselben braucht man nur senkrecht nach dem Kalkband hinabzusteigen und man wird die feinste Zeichnung einer Pflanze finden.

Die gefundenen Pflanzen sind teils mit blo"sen Augen, teils nur mikroskopisch sichtbar und zwar bis zu den kleinsten denkbaren Gr"o"sen. Sie sind in ein Silicat verwandelt, also durch Behandlung des Kalks mit S"aure blo"szulegen.

Da erscheinen die in Kalk eingelagerten Pflanzen als blendend weisse Stengel, Kelche, Bl"atter. In D"unnschliff freilich sehen sie bei durchfallendem Licht gelbbraun aus, was M"obius veranlasste ihre Farbe als hellbraun zu beschreiben. Dies ist blo"s Folge der Lichtberechnung in der opaken Masse. Welches Mineral die F"ullmasse bildet, ist zweifelhaft. Mann nannte es Flocculit.

Die F"ullmasse der Pflanzen im Serpentin ist wieder Serpentin, doch finden sich darin, wie oben gesagt, die Modelle von Pflanzen von Dolomit. Die F"ullmasse der Pflanzen im Kalk ist meist ebenfalls ein Silicat.

Meine Aufgabe ist nun die Darstellung.

Diese kann nur durch Abbildungen erreicht werden.

In der Unterscheidung der Arten habe ich die Fortpflanzungsorgane, soweit erkennbar, zu Grunde gelegt.

Obgleich es nun m"oglich w"are, f"ur die gefundenen Pflanzen analoge Bildungen unter den lebenden oder fossilen zu finden, so musste ich, wenn ich meine Entdeckung schnell zum Gemeingut der Wissenschaft und zur Grundlage fernerer Forschung machen wollte, doch den andern Weg einschlagen, und ohne R"ucksicht auf diese bestimmen.

Ein System l"asst sich mit dem vorhandenen Material doch nicht begr"unden. Ich bin "uberhaupt kein Freund von Systemen und halte die Zeit f"ur Aufstellung solcher auch noch nicht f"ur gekommen. Wir sind noch in der Zeit der Saat und nicht der Ernte in der Naturwissenschaft.

Ehe ich aber auf die Beschreibung der Pflanzen "ubergehe, will ich noch einmal einen Blick auf das \emph{Eozoon} werfen.
\clearpage
\subsection{\swabfamily{Das \emph{Eozoon}}}
\paragraph{}
Die Behauptung, dass die "`\emph{Eozoon}"'-Formen einer Foraminifere angeh"oren, f"allt in sich zusammen mit dem Beweis, dass die Formen Pflanzen sind. Foraminifere und Pflanze schlie"sen sich aus.

Meine neuesten Untersuchungen aber haben, wenn sie auch einzelne fr"uhere Behauptungen modifizieren, die Beweiss"atze der Gegner f"ur eine Foraminifere vollst"andig entkr"aftet.

Dazu geh"ort vor Allem der wahre Sachverhalt bez"uglich der "`Kanalsysteme"'. Auf diese hat man das gr"o"ste Gewicht gelegt. Sie bildeten das Schlussglied in der Beweiskette.

Hier aber verhalten sich die Tatsachen jetzt vollkommen anders, und es ist unbegreiflich, wie Dr. Carpenter diese Kanalsysteme in das Schema des \emph{Eozoon} aufnehmen konnte.

Diese Kanalsysteme sind einmal, wie ich Taf. III. zeige, sehr verschieden. W"are das \emph{Eozoon} eine Foraminifere, so w"are es gar nicht denkbar, dass in --- zugestandenerma"sen ein und derselben Foraminifere ein Kanal becherf"ormig, ein anderer fadenf"ormig, ein dritter perlschnurartig gereiht w"are.

Sodann finden sich, wie M"obius dargestellt hat, bei der lebenden Foraminifere blo"s fadenf"ormige Kanalsysteme.

Becher-, Kugel- und Perlschnur-Formen kommen gar nicht vor. Die Fadenformen habe ich in etwa 20 D"unnschliffen blo"s einmal (Taf. III. Fig. 3) gefunden, "uberall herrscht die Becher- form vor. Aber auch, wo die Fadenform vorkommt, gehen die "`Kan"ale"' von Zweigen aus, was verschwiegen wurde.

Was aber endlich den Ausschlag gibt, ist, dass diese Kanalsysteme blo"s an ganz vereinzelten Stellen vorkommen, ohne dass man irgend einen Grund h"atte, anzunehmen, dieses Fehlen habe seine Ursache in der Metamorphose des Gesteins oder in einer anderen solchen Ursache. Dies ist l"angst hervorgehoben worden.

All dies konnte nur solange angehen, als man nicht "uber Material genug verf"ugte. D"unnschliffe sind in vielen, sehr vielen Beziehungen unentbehrlich; aber sie ersetzen nicht die Gesamt-Anschauung. Und ferner ist notwendig eine Untersuchung der ganzen Lagerung, insbesondere des die Knollen umgebenden Gesteins.

Damit w"are es jedem halbwegs Ge"ubten leicht gewesen, sich sofort ein Urteil zu bilden und es w"are wohl nicht zu Gunsten der Foraminifere ausgefallen.

Dass die "`Schale"' gr"o"stenteils fehlt, kann ich aus meiner Anschauung versichern; dieser "`proper wall"' soweit er nicht auf optischer T"auschung beruht, findet sich an seltenen Stellen blo"s angedeutet: ebenfalls ohne dass man f"ur das Fehlen desselben die Schuld auf Zerst"orung schieben k"onnte: denn die feinsten "`Kanalsysteme"' unmittelbar daneben sind ja erhalten.

Dass die Tubuli "uberall Chrysotil sind, bedarf keiner weiteren Ausf"uhrung mehr.

So bleiben von dem "`Tier"' blo"s die Kammern "ubrig.

Diese aber sind wie aller Serpentin. Allerdings die reihenf"ormige Lage der K"orner w"are auffallend. Das Merkmal der Kammer nun w"are die einzige Beweis-Tatsache: sie ist f"ur sich allein nach dem Zugest"andnis der Gegner selbst nicht hinreichend, um eine Foraminifere festzustellen.

Ich gieng bei meiner Untersuchung von dem Bild Tafel IV. und V. aus.

Nachdem mir durch dieses Bild die Pflanze unwiderleglich festgestellt und so ein Bild der Pflanze des Laurentiangneises gegeben war: konnte es nicht schwer sein, die Kanalsysteme der "`Eozoonknollen"' "uberhaupt zu deuten. Sie l"osten sich alle in Pflanzen auf.

Sodann bildete dieselbe Form auch den Schl"ussel zur L"osung des ganzen Gesteins.

Sobald die Form des \emph{Eophyllum} einmal als Pflanze feststand, so wurde es wahrscheinlich, dass auch die gro"sen Serpentin-Kammern nichts als die F"ullmassen ehemaliger Pflanzenformen seien.

Der Beweis wurde erg"anzt mit Durchschnitten gr"o"serer St"ucke, wo eine Anzahl Pflanzen alle und immer in Einem Ausgangspunkt zusammenlaufen.

Man sehe doch einmal den Naturselbstdruck des \emph{Eozoon} in Dawson's "`Life's Dawn on Earth"'! Man denke sich das Bild nach der anderen Seite erg"anzt. Freilich in 4 eckig willk"urlich herausges"agten kleinen St"ucken war die Pflanzenform nicht leicht zu erkennen. Man nehme nun ferner ein 1/2 Meter gro"ses St"uck Lagergestein und man sieht dort von etwa 3 zu 3 Cmetern auf dem Serpentin die Formen der "`Kanalsysteme"', hier im Gro"sen von Serpentin nachgebildet: man wird ferner die Wurzelans"atze an der Gr"anze der Serpentin- und Dolomit-Schichte finden. Endlich sind nicht nur die Zellh"aute erhalten, sondern die Prothallien dem Hundert nach im Kalk und im Serpentin!

Wenn man das Gestein genau untersucht, so findet man die Formen schon mit blo"sem Auge.

Blos eine Erg"anzung dieses Beweises ist es, dass dieselben Pflanzenformen in den Dolomit- und Kalk-Lagern vorkommen, im Kalk in Serpentin, aber auch in Glimmer verwandelt sind. Diese Glimmerpflanzen sind das Sch"onste, was man sehen kann.

Nun muss ich aber noch auf einen Umstand aufmerksam machen, darauf n"amlich, dass der Serpentin in den St"ucken, welche \emph{Eozoon} sein sollten, also in den Knollen, F"ullmasse ist. Der Kalk ist das Lager.

Anders bei den Formen der Tafel I. 3., welche ich Coralloidea nenne. Hier sind die Algen in Dolomit verwandelt, der Serpentin ist das Lager. Hier also ist das Verh"altnis umgekehrt.

Man sieht hieraus, wie oberfl"achlich, einseitig "uberhaupt die bisherigen Untersuchungen waren, wie sie vollst"andig durch die Knappheit des Materials beherrscht waren.
\clearpage
\subsection{\swabfamily{Die Pflanze}}
\paragraph{}
Ein gr"o"seres R"atsel freilich als das "`\emph{Eozoon}"' ist wohl schwerlich der Naturforschung aufgegeben worden.

Als ich die erste Ank"undigung des \emph{Eophyllum} dem "`Ausland"' "ubergab, dachte ich noch nicht daran, dass die gro"sen Serpentinb"ander auch Pflanzen seien. Ich hatte diese Arbeit zur H"alfte nach dem ersten Plan fertig, als ich endlich ein schlechtes Gesteinsexemplar in die Hand bekam, bei welchem sich aber gerade deshalb die Serpentinteile eigent"umlich deutlich abhoben.

Ich sah und sah und mit einem Male war mir klar, dass die "`Sarcode-Kammer"' nichts als Pflanzenzellen seien. So geht es den Mikroskopikern.

Was man mit blo"sem Auge sehen kann, sehen sie nicht mehr.

Nun war es aber eine schwere Arbeit, die Sache zu pr"ufen.

Ich habe jetzt keinen Zweifel mehr.

Und nur so erkl"aren sich alle Tatsachen.

Die Serpentinb"ander, welche das, was man bisher \emph{Eozoon} nannte, zusammensetzen, geh"oren einer Alge mit breiten Bl"attern, wenn der Ausdruck erlaubt ist, an, welche von einem Punkte ausgehend sich in regelm"a"sigen Formen legt. Die Basalzelle sitzt auf dem Serpentin oder Dolomit auf. Wurzeln fand ich nur in Einem Falle, jedoch nicht sicher.

Der Kalk ist die Einlagerung. Deutliche Brutzellen sind darin sichtbar: denn im D"unnschliff stechen sie durch den Kalk noch durch.

Die Gegenprobe kann man durch Aufl"osung des Kalks mit S"aure machen. Hier sind die Bl"atter v"ollig mit Brutzellen besetzt: die "`Warzenans"atze"' G"umbel's.

Die Sache ist noch viel klarer, wo die Pflanze in Dolomit verwandelt ist. Hier kann man mit blo"sem Auge die Brutzellen erkennen.

Wasserklare Becherzellen sieht man auf dem Dolomit herausgewittert lagern.

Am sch"onsten aber sind die Kalke, in welchen die Pflanzen teils in Serpentin teils in Glimmer verwandelt sind.

Dieselben Zellen finden sich in einem Doppelspat in Kupfer und Malachit verwandelt mit blo"sem Auge sichtbar!

Die Kanalsysteme des "`Zwischen-Skelets"' sind also mikroskopische Pflanzen, welche teils einfach von Kalk oder auf den gro"sen Algen angewachsen oder dort tot abgelagert worden sind.

Wie ich im Eingang angedeutet, braucht man allerdings einen Schl"ussel zu dieser neuen Sch"opfung, --- denn sie ist unserer Vorstellung vollkommen neu. Dieser Schl"ussel liegt in den mikroskopischen Formen.

Nun aber l"asst sich von diesem sicheren Ausgangspunkte die Sache leicht verfolgen.

Nur muss ich auch hier vor der alleinigen Ben"utzung der D"unnschliffe warnen. Es ist Sache des Zufalls, ob man hier ein Bild bekommt. Hunderte lassen sich anfertigen, aber nur ein ganz ge"ubtes Auge vermag sie zu entziffern.
\clearpage
\subsection{\emph{Eophyllum canadense}}
\paragraph{}
Ich "ubertrage nun den ersten Namen:

\emph{Eophyllum} auf die mit blo"sem Auge sichtbare gr"o"ste Pflanzenform.

Sie bildet die Algenwiesen im Laurentian, aber freilich nicht in gar riesiger Form, nur durch ihre Menge ist sie riesig. Sie ist enthalten in den Serpentin-Knollen des Laurentiankalks, welche man bisher \emph{Eozoon canadense} nannte.

Aber hier ist sie nur in ihrer gr"o"sten Entwicklung: die acervuline Form des \emph{Eozoon} ist nichts, als kleines \emph{Eophyllum} und vorzugsweise Brutzellen desselben.

Die Pflanze ist eine Alge: in ihrer eigent"umlichen Form stimmt sie mit keiner lebenden vollst"andig, jedenfalls geh"ort sie schon zu den h"oher entwickelten Fucusarten.

Die Basalzelle ist eine halbrunde becherf"ormige Zelle, welche im Dolomit aufsetzt, einige kleinere Zellen und dann die n"achste in Becherform treibt, welche immer wieder neue Zellen ansetzt.

Die neuen Knospen, ich nenne sie Brutzellen, sitzen "uberall auf dem Rande der Zelle auf, und geben dem Blatte dadurch das h"ockerige Ansehen.

Die Form ist in Tafel I. 1. dargestellt.

Das von Carpenter abgebildete St"uck \emph{Eozoon}, welches nachher in so viele B"ucher "uberging, zeigt in dem Kugligen nur Brutzellen des \emph{Eophyllum}, und ich bin "uberzeugt, dass sie sich an dem abgebildeten St"uck noch mit blo"sem Auge erkennen lie"sen.

Die Durchschnitte der Brutzellen sind leicht verst"andlich, sobald man sich diese Zellen aus dem Kalk herausl"ost, indem man ihn in eine verd"unnte S"aure legt, oder auch, wenn man sie im Serpentin sucht, wo sie wie lebend gesehen werden k"onnen.

Ganze Pflanzen ziehen sich durch die Serpentinlager hindurch.

Ich habe die Basalzelle in Taf. I. f. 2. abgebildet, eine Brutzelle daselbst 4:5. zeigt eine solche im Serpentin.

Die regelm"a"sige Lage des Eophyllums r"uhrt wahrscheinlich daher, dass sie sich schon bei Lebzeiten in Kalk eingebettet hat.
\clearpage
\subsection{\swabfamily{Die Formen}}
\paragraph{}
Man nehme nun Taf. II. zur Hilfe, um sich eine Vorstellung von dem Leben zu machen, welches sich hier entwickelt.

Alle diese Formen sind teils aus dem Kalk gel"ost, teils nach den im Serpentin erhaltenen Pflanzen gezeichnet; meist bei 90-facher Vergr"o"serung.

Ich wage es nicht, bei dem mir vorliegenden Material hier irgend schon Arten zu bestimmen. Dazu ist die Zeit noch nicht gekommen.

1, 2, 4, 6, 8, 17, 18, 22, 23, sind Prothallien, 3 a u. b. 10. F"ullmassen (Steinkerne), 5. 23. zeigen eine Fadenzelle. --- Fig. 9. ist wahrscheinlich eine Sporangie, Fig. 16. wasserhell eine F"ullmasse, wahrscheinlich von einer gestreckten Zelle, wie sie bei den Algen h"aufig vorkommen.
\clearpage
\subsection{\swabfamily{Di Arten}}
\paragraph{}
Ich lasse nun di Arten folgen, welche durch vollst"andige Exemplare festgestellt sind.

Allein auch hier kann es sich blo"s darum handeln, nach der Gesamtform Namen zu geben.

Alle folgende sind mikroskopisch im Kalk.

Ich gebe in Taf. III die 4 von mir unterschiedenen Arten von mikroskopischen Algen des Laurentian, welche als "`Kanal-Systeme"' bezeichnet wurden, 1. 3. 4. in 80-, 2. in 20- maliger Vergr"o"serung.

Sie leben in Gesellschaft. Einzelne Formen werden nach diesen aufgef"uhrt.

\centerline{\emph{Kampyloklon}.}

Taf. III. 1. 1/80.

Kampyloklon: von der Biegung seines Stammes und der Zweige. Die Basalzelle ist rund, abw"arts gekehrt. An diese reiht sich eine gewundene Blattzelle, an diese eine Becherzelle an und diese treibt entweder wieder eine Blatt- oder eine Becherzelle.

Wahrscheinlich dieser Art geh"ort an Taf. VII. 4.

\centerline{\emph{Leucophyllum}.}

Taf. III. 2. 1/20.

Es sind dies die gr"o"sten mikroskopischen Pflanzen im Kalk. Ihre Anordnung ist leicht verst"andlich. Im Eck des Bildes ist eine Zelle 80-mal vergr"o"sert.

Die Bl"atter sind in ein milchwei"ses Silicat verwandelt.

\centerline{\emph{Pseudozoon}.}

Taf. III. 3. (1/80).

Denn diese Kan"ale wurden von Dr. Carpenter als die Kanalsysteme von Nummuliten gekennzeichnet; aber freilich nicht vollst"andig wurden sie gezeichnet, sondern blo"s die oberen geraden (Faden-) Zellen des Bildes. W"aren die unteren dazu gezeichnet worden, so h"atte wohl Niemand sich "uberreden lassen, dass es sich hier um Kanalsysteme einer Foraminifere handelte.

\centerline{\emph{Kilocarpon}.}

Taf. III. 4. 1/80. und Taf. IX. 1/140 Taf. X. 1/750. 1/450.

Ebenfalls ein "`System"'.

Diese Algen mussten eine Gallert um sich ansetzen, wie es die Nostocaceen tun. Nur so erkl"art sich das lappenartige Aussehen bei regelm"a"siger Lagerung des Ganzen.

Wahrscheinlich dieselbe Pflanze ist Tab. IX. und X.

Ich nenne sie Kilo-Karpon, denn die Zelle des deutlich sichtbaren Oogoniums (Tab. X. 2) zeigt eine sehr gro"se Anzahl von Prothallien, welche sich sodann auch in dem Kalk und Serpentin finden.

\centerline{\emph{Chairokerdos}.}

Der willkommene (erste) Fund einer Pflanze des Laurentian-Gneises. Tafel IV. und V.

Wie in allen bisherigen Algen bildet je ein Stengel eine Pflanze.

Ein schwacher Wurzelansatz in Form einer platt gew"olbten oder becherf"ormigen Zelle, dann eine gewundene l"angere, weiter eine becherartige Zelle.

Nun beginnt die Teilung erst mit einer gewundenen Zelle, setzt dann fort in gestreckten, diesen folgen perlschnurartige Zellen, welche mit einer Sporenzelle endigen. Die Sporen liegen geordnet. Taf. V. Fig. 2.

Der n"achste Stengel tr"agt eine leere Sporenzelle. (Taf. IV.)

V. 1. ist eine 300-malige Vergr"o"serung der dritten Pflanze von rechts, der auch die Sporenzelle angeh"ort.

Die erste Pflanze (von rechts) hat Bl"atterform, geh"ort aber ohne Zweifel zu derselben Art.

Eine Zellmembran ist deutlich wahrzunehmen.

Tafel VI. 4. 1/320 wahrscheinlich dieselbe Art.

\centerline{\emph{Poterion}.}

Tafel VI. Fig. 1. Die Becheralge. Fig. 2. Basalzelle.

\centerline{\emph{Margarodes}.}

Taf. VI. 3. (1/650).

Perlschnuralge entwickelt sich aus einer Wurzelzelle, welche auf einem geraden Zweige (einer anderen Art) aufsitzt.

\centerline{\emph{Lichnon}.}

(Die Lampe) Taf. VII. 1. 2. (Fig. 4. die Wurzelzelle) (1/90).

Die Zelle 2. geh"ort derselben Art an, ist jedoch ohne Stengel im D"unnschliff. Ich habe aus dem Kalk eine Zellhaut vollkommen gleich mit S"aure ausgewaschen.

Fig. 3. ist eine Becherzelle, welche nicht zu 1. geh"ort.

\centerline{\emph{Salpinx}.}

Trompeten-Alge (1/90).

\centerline{\emph{Kilikodendron}.}

Becherbaum. Tafel VII. 6. (1/90).

Fig. 7. 8. Die Pflanze in ihrer ersten Entwicklung. Eine wundervolle Form bildet \emph{Pleurophyllum}.

\centerline{\emph{Pleurophyllum}.}

Ribben-Blatt. Taf. VIII. Fig. 1. (1/750).

Die Kelche sitzen unmittelbar im Kalk.

\centerline{\emph{Phiala}.}

Die Schale. Taf. VIII. 2. (1/90).

Schalenf"ormige Zellen. Die Stengel daneben zeigen Kopulation.

Das Zierlichste aber ist Theochara.

\centerline{\emph{Theochara}.}

Gotteslust. Taf. XI. 1. 1/90.

in Dolomit verwandelt, glashell.

\centerline{\emph{Linophyton}.}

Tafel XI. 2. 3. 4.

Ich habe die Formen zusammengestellt, welche alle denselben Charakter an sich tragen.

Bei Fig. 2. ist zu beachten, dass der eine Zell-Knoten nach rechts, der andere nach links geht, was ich nur nach sorgf"altiger Beobachtung zeichnen lie"s.

In einem Bilde, welches leider nicht auf die Tafel kam, sitzt in dieser (wahrscheinlich Kopulations-) Zelle in der Mitte ein Kelch.

M"oglich also, dass an der Stelle Fig. 2. diese Kelch-Zelle abgebrochen ist.

Fig. 5. zeigt den Fadenansatz einer "ahnlichen Art.
\clearpage
\subsection{\swabfamily{"Uberblick}}
\paragraph{}
Der allgemeine Charakter dieser Pflanzenformen des Laurentian-Kalks ist (ich wage den Ausdruck) ein h"ochst roher, einfacher, anf"anglicher. Aber etwas haben sie, was sie den Pflanzen der Jetztzeit mindestens gleichstellt: sie sind verm"oge ihrer Anlage einer ungemessenen Vermehrung f"ahig.

Das Individuum selbst ist einfach, ist eine Zelle.

Eine Zelle setzt sich an (nach unten), die n"achste kehrt sich nach oben. Tafel VII. Fig. 7.

Ich habe F"alle beobachtet, wo mehrere Zellen nach unten sich "uber einander setzten.

Mir ist es wahrscheinlich, dass hier die Zelle von Kalk "uberlagert und so gen"otigt wurde, ihre Wurzelbildung fortzusetzen.

Eigentliche Wurzeln fand ich, jedoch nicht sicher, nur in einem D"unnschliffe.

Nun setzt sich an diese Zelle, meist seitlich an ihrem Rand, oder an einem beliebigen Teile ihrer Oberfl"ache, eine Brut-Zelle (Knospe) in Form eines Blattes oder eines Bechers an. Es teilen sich die Zellen: so entsteht ein Stamm.

Die Befruchtungsorgane scheinen erst nach einem gr"o"seren Wachstum sich zu bilden. Tafel V. und IX.

Was die Form der Zellen betrifft, so ist dieselbe unersch"opflich verschieden.

Immer aber bewahrt sie den Charakter der Urpflanze.

Merkw"urdig ist die Erhaltung.

Nicht blo"s ganze Pflanzen, sondern Membrane sind erhalten. Sehr h"aufig finden sich die Modelle der Zellh"ohlen. Diese sind gew"ohnlich mit Dolomit ausgef"ullt, wenigstens wo sie in Dolomit und Serpentin lagern.

So sind uns Modelle der Hohlr"aume der Pflanzen erhalten. Tafel II. Fig. 10.

Diese Form findet sich tausendmal im Serpentin wie im Gneis, auch im Dolomit selbst. Der offene Zwischenraum zeigt "uberall die Linie der Zellscheidew"ande.

Ebenso h"aufig finden sich ganze Brutzellen in Serpentin verwandelt.

Die einfache Behandlung mit Salzs"aure liefert diese Zelle in gro"ser Menge. Ja ich habe sogar Kieselskelette erhalten, wo der "ubrige Teil der Pflanze wahrscheinlich in Kalk verwandelt war.

Wenn man so die Form der Urpflanze gewonnen hat, so nehme man einen beliebigen D"unnschliff des "`\emph{Eozoon}"' und man wird die Linien derselben in jedem Teile finden. Die einen Exemplare freilich sind blo"s Abschnitte, andere Halbschnitte; am besten erkennt man deshalb die Pflanze an den seitlichen Einschl"ussen.

"Uberall zeigt die in Kiesel (wahrscheinlich Glimmer) erhaltene Zellmembrane die Form der Pflanze.

Jetzt erst ist mir auch klar, warum die Chrysotilfasern "uberall parallel sich lagern (dieselben, die Dr. Carpenter f"ur eine Haut mit Tubuli eines Nummuliten ansah).

Diese Chrysotilfasern haben sich um die Pflanze (den Serpentin) in der Art gelagert, dass sie die Oberfl"ache der Zelle in Parallel-Kreisen umgeben: (in einer Foraminifereschale st"unden sie senkrecht auf der Kammer!)

Wenn ich jetzt meine D"unnschliffe durchsehe und finde darin die deutlichsten Brutzellen an der Pflanze, in der Tiefe des Kelches die runde Öffnung, welche ganz konstant (also hundertf"altig) sich findet, ferner ganz junge Zellen noch in den alten, endlich im Kalk daneben diese Zellen: so muss ich glauben, dass alle, welche am \emph{Eozoon} arbeiteten (ich eingeschlossen), mit Blindheit geschlagen waren.

Vollends unbegreiflich ist dies hinsichtlich der Pflanzen im Serpentin des Eophyllumkalks. Hier liegen sie sogar f"ur das blo"se Auge. Aber weil dieser Serpentin nicht \emph{Eozoon} war, deshalb untersuchte ihn auch Niemand mehr. Jeder, der die Sache jetzt untersucht, wird sich an die Stirne schlagen und ausrufen: wie war das zu "ubersehen m"oglich!

Und vollends ist es unbegreiflich bei Gelehrten, welche "uber jede Menge von Gesteinen verf"ugten, um es ohne R"ucksicht auch vernichten zu k"onnen.

Insofern ist die Erfahrung mit dem \emph{Eozoon} ein bedeutender Pr"azedenzfall.

Nur in Einem Punkt muss ich meine fr"uhere Beweisf"uhrung berichtigen, indem ich sie erkl"are.

Ich habe sehr gro"ses Gewicht auf das urspr"ungliche, unzersetzte Gestein in den Serpentin-K"ornern gelegt und damit gegen die "`Kammer"' bewiesen.

Richtig ist, dass dies vorkommt.

Allein hier war die erste Frage: was ist \emph{Eozoon}?

Dies war von Anfang an nicht so festgestellt.

Wenige konnten sich nur dar"uber vergewissern, was die Entdecker als "`Eozoongestein"' anerkennen. Ich gebe zu, dass in dem, was mir nachher Dr. Dawson als "`\emph{Eozoon}"' vorlegte, wenig unzersetzte Gestein in den "`Kammern"' gefunden werden wird.

Der Serpentin ist allerdings in einem fl"ussigen Zustand gewesen, hat sogar ganze Lager gebildet: ein solches ist die Unterlage des ganzen Eophyllumgesteins, wie insbesondere das breite Serpentinband, unter dem ersten breiteren Kalkband.

Da sind, allerdings selten, Mineralteile, welche polarisieren und daher in der Regel auf das erhaltene urspr"ungliche Gestein zu deuten waren.

Dass ich den Serpentin erst im zweiten Stadium der Verwitterung begriffen ansah, der Irrtum r"uhrt daher, dass ich die Zellw"ande als Gesteinsteile ansah und dass ich zuviel Werth auf die Teile des Gesteins legte, wo diese noch wirkliche polarisierende Mineraleinschl"usse enthalten.

Diese Einschl"usse k"onnten allerdings auch Dolomit sein, welcher sich im Serpentin ausschied und so die Polarisationserscheinung bewirkte.
\clearpage
\subsection{\swabfamily{Ergebnisse}}
\paragraph{}
Die wichtigen Ergebnisse meiner Entdeckung sind:
\begin{enumerate}
\item die ersten organischen Formen sind die niedersten Pflanzen, woraus h"ochst wahrscheinlich
\item folgt, dass wirklich die Bildungen der Gegenwart die Erzeugnisse der Fortentwicklung sind; es ist also eine blo"s einmalige gleichzeitige "`Sch"opfung"' nicht --- dagegen
\item ist eben verm"oge des Gesetzes der Entwicklung anzunehmen, dass die niedersten Formen der Laurentianzeit heute l"angst in h"ohere sich verwandelt haben: die niedersten von heute also k"onnen nicht die Nachkommen von damals sein und hieraus schlie"se ich, dass heute noch derselbe "Ubergang von Unorganischem zum Organischen (Sch"opfung) stattfindet.
\end{enumerate}
M"oge nun die Frucht langj"ahriger Arbeit zu neuen Forschungen anregen.

Ein Grund ist gelegt. Die Eophyllumkalke von Canada bieten dem Forscher eine unermessliche Fundgrube der Erkenntnis, denn keine Schichte der Erde hat wie diese erste ihre Toten so wunderbar sch"on erhalten.

Der klarste Kalksteinniederschlag, noch mehr aber die Serpentinschichten, haben H"aute von kaum messbarer Dicke uns aufbewahrt.

Das Bild der Urzelle, welches nun naturgeschichtlich festgestellt ist, ist fast durch ein Wunder erhalten.

Freilich das Werden selbst sieht man doch nicht. Zum Werden kommt der Mensch immer zu sp"at: wenn er es sieht, ist schon ein Gewordenes da.

Was f"uhrt die einfachsten Stoffe der Erde in diese einfachsten Formen? Dieses R"atsel der Entwicklungslehre wird nie gel"ost werden. Aus den wissenschaftlich registrierten Stoffen allein l"asst es sich nicht erkl"aren. Diese Tatsache wird auch den Naturforscher und gerade den Naturforscher, als den Mann der Erfahrung, zuerst zur Erkenntnis eines Stoffes treiben, welcher selbst zwar mit den Sinnen nicht mehr wahrgenommen werden kann, dessen Wirkungen aber in die Sinne fallen und damit dem Verstande den Beweis seines Daseins aufn"otigt. Nenne man diesen Stoff, wie man will, er ist da. Das ist zun"achst die Hauptsache. Unsere Zeit streitet "uber die Frage, ob es denn au"ser der Materie auch einen Geist gebe? Die Zeit wird freilich auch noch kommen, wo man einsieht, dass es ebensowenig einen Geist, (Leben) ohne die Materie g"abe.

Das Entwicklungsgesetz aber ist es gerade, welches den Menschen zu der "Uberzeugung leitet, dass die Sch"opfung auch mit seinem Dasein, wie es jetzt ist, noch nicht abgeschlossen sein kann, dass es vielmehr noch h"ohere Daseinsformen f"ur ihn geben muss, welche zu erreichen, Naturgesetz f"ur ihn ist.

So allein erkl"aren sich hundert Tatsachen des geistigen und Seelenlebens, welche ohne dieses Gesetz unverst"andlich w"aren, kurz, so allein erkl"art sich der Mensch selbst.

Es ist daher die Entwicklungslehre, auf den Menschen "ubertragen, das Gesetz des Fortschritts ja der Unsterblichkeit. Nicht im Kampf um das niedere Dasein, sondern im Kampf um das h"ochste Dasein geht der Mensch seinen Lauf nach dem Gesetz der Linie, die sich nur im Unendlichen mit einer andern schneidet.

Ich konnte mir nicht versagen, diese Schlussbemerkungen zu machen. Ich glaube in meiner Arbeit gezeigt zu haben, dass ich die exakte Wissenschaft liebe. Aber es gibt eine exakte Wissenschaft auch des Geistigen und nicht blo"s des K"orperlichen. Zu einer wie der andern geh"ort, dass man sehen will. Will man sehen, die Tatsachen sind da.

Auf diesem Wege m"ochte ich nicht blo"s die Naturwissenschaft, sondern auch die Wissenschaft des Menschen, die h"ochste welche es geben kann, durch meine Arbeit um einen Schritt weiter gef"uhrt sehen.
\clearpage
\section{\swabfamily{Das erste Tier}}
\paragraph{}
Ich hatte meine Arbeit schon geschlossen, als ich in dem (mit Salzs"aure aufgel"ostem) Eophyllumkalk noch das erste Tier finden sollte, welches wir kennen.

Ich besitze es in 1 vollst"andigen Exemplar (in Kanadabalsam) und in Bruchst"ucken, wovon einige nahezu vollst"andig.

Vergebens sah ich mich nach einem Nachbild unter den lebenden, wie unter den ausgestorbenen um. Es gleicht einigerma"sen Serpula oder Vermetus.

Ich nenne es Titanus Bismarki Taf. XII. unserm Reichskanzler zu Ehren.

Ein Wurm mit einem Kieselpanzer: denn derselbe hat der Salzs"aure widerstanden.

Der Panzer muss durch H"aute verbunden gewesen sein. Daf"ur spricht Taf. XIII. Fig. 1.

Ein Exemplar ist 0,59 mm. lang, 0,06 mm. durchschnittlich breit.

Der Panzer besteht aus zahllosen kontraktilen Ringen; der Titanus vermag jede Stellung anzunehmen.

Ein St"uck (vorausgesetzt, dass es derselben Art angeh"ort) zeigt, dass eine partienweise Zusammenziehung der Glieder stattfand, denn der Leib innerhalb des Kieselpanzers ist in gleichm"a"sig von einander abstehenden Segmenten zusammengezogen, und daher bildet der Panzer scharfe Kanten.

In dem Panzer steckt ein Leib, welcher ebenfalls der S"aure widerstanden hat.

Der Panzer ist egelartig geformt, nicht rund, sondern im Durchschnitt flach-linsenf"ormig, wie der Durchschnitt eines Blutegels. Er nimmt von oben nach unten an Dicke ab. Das dickere Ende ist herzf"ormig, die scharfe Spitze nach oben gekehrt.

Es bildet eine Art Wohnkammer: der betreffende Teil des Panzers ist deutlich glasig.

Mund"offnung 0,12 mm. breit,

Schwanz 0,05 mm. breit.

Durch den Panzer hindurch sieht man einen K"orper mit einer H"ohlung in der Mitte.

Durchmesser desselben 0,03 mm.

Aus der Mund"offnung ragt eine Wulst hervor, der eigentliche Mund.

Wenn man bei Taf. XII. Fig. 3. das Mikroskop hoch einstellt, so sieht man nur die herzf"ormige Öffnung: wird der Tubus gesenkt, so erblickt man durch die Schale und hinter derselben die Mund"offnung. Leider lassen sich solche zarte Dinge in der Zeichnung nicht vollst"andig wiedergeben.

Figur 3 hat einen "ubrigens undeutlichen Lappen, der "uber die Mund"offnung hervorragt.

Ein gebrochenes St"uck Taf. XIII. 2 zeigt eine deutliche R"ohre.

Das andere Ende des K"orpers (der Schwanz) ist zweilappig, wie eine Klappe. Taf. XII. Fig. 1. Taf. XIII. Fig. 5. ist ein abgebrochenes St"uck, welches vollst"andig mit Taf. XII. Fig. 1. stimmt.

Dass die Form eine organische ist, l"asst sich wohl in keiner Weise bezweifeln. Dass es ein Tier und keine Pflanze ist, daf"ur spricht die Bewegung, welche aus den verschiedenen Stellungen des Gesch"opfs geschlossen werden kann.

Die Form ist Serpula "ahnlich.

Freilich suchte man bis jetzt wohl schwerlich ein Tier, das den Anneliden gleicht, in der untersten Schichte, Tieren, welche ihre Stellung im Tierreich fast schon "uber dem ersten Vierteil desselben einnehmen.

Es w"are aber auch m"oglich, dass wir hier nur die "au"sere Gestalt eines h"oheren Tieres vor uns haben, denn die innere Organisation l"asst sich nicht mehr feststellen.

Titanus unterscheidet sich von allem, was im Eophyllumkalk vorkommt, sofort durch die Textur.

Die Streifen (Glieder) des Panzers sind wie mit dem Grabstichel gezeichnet, die Beweglichkeit muss deshalb auch eine vollkommene gewesen sein.

Man k"onnte im Titanus hienach den Vorg"anger der Trilobiten sehen.

Die Bruchst"ucke Taf. XIII. zeigen alle gleiche Textur.

Gewisse hyaline Ans"atze auf Pflanzen von derselben Art, wie die Mund"offnungen, k"onnten vermuten lassen, dass man es hier mit jungen Tieren zu tun hat, die sich auf anderen Gegenst"anden ansaugen.

Doch die Sache muss erst in weiteren Exemplaren untersucht werden, wie denn dies Alles mehr eine Ank"undigung und Aufforderung zu weiterer Forschung sein soll, als ein wissenschaftlicher Abschluss.

Es sind nur einmal neue organische Formen und damit ist eine neue Sch"opfungsperiode festgestellt.
\clearpage
\section{\swabfamily{Serpentin}}
\paragraph{}
Als ich die Arbeit bis hierher vollendet hatte, fiel mein Blick auf einen Briefbeschwerer von Serpentin, der auf meinem Arbeitstisch lag. Ich sah hier bl"auliche Flecken und sofort erkannte ich die vom Laurentian-Serpentin beschriebenen Formen in dem Serpentin von Sachsen wieder.

Bei genauer Untersuchung fand ich die sch"onsten Linien und Formen von solcher Reinheit und Vollendung, als sie der Pal"aontologe nur w"unschen kann.

Sofort nahm ich alle meine Serpentine, sowohl D"unnschliffe als Handst"ucke vor. Wie erstaunte ich aber, als ich in dem gr"o"sten Teil derselben Pflanzen-Formen von 1-2 cm. fand!

Ich fand daneben eine solche Masse kleinerer Formen, dass ich die Ansicht ausspreche:

ein gro"ser Teil der Serpentine, insbesondere der b"ohmischen und s"achsischen, ist nichts als ein gro"ses Algen-Lager, "ahnlich den fossilen Diatomeen-Lagern.

Es lagert die verkieselte Pflanze in der urspr"unglich fl"ussigen Serpentinmasse.

Ich zeichne nun sofort die Form aus 3 Briefbeschwerern cf. Tafel XIV. und begn"uge mich vorerst damit, die Wissenschaft auf die Tatsache aufmerksam gemacht zu haben.

Tafel XIV. 4. ist eine Form von 1 1/2 cm., welche ich so bestimmt gesehen, als es nur sein kann.

Ich nenne sie meinem verehrten Freunde K. Hofrath Dr. F. v. Hochstetter in Wien zu Ehren und wegen ihrer knospenartigen Form

Opthalmia Hochstetteri.

Wir haben also ungeheure Schichten eines bis dahin f"ur azoisch gehaltenen Gesteins als reines organisches Lager.

Die Kelche sind mit Granat gef"ullt. Offenbar steht diese Bildung in Verbindung mit der T"atigkeit der Pflanze.

Wie war es m"oglich, diese Formen zu "ubersehen?

Aber freilich, ehe man den Schl"ussel zur Form aus dem kanadischen Laurentian-Gestein hatte, war es nicht so leicht, nur daran zu denken und deshalb sah man es auch nicht.

Jede Apotheker-Reibschale, jeder Briefbeschwerer zeigt uns in einem Kubikzentimeter eine gr"o"sere und vielleicht 1000 kleinere Pflanzen-Zellen.

Ich zeichne aus dem \emph{Eophyllum}-Kalk noch 2 Tafeln Zellenformen, um die Grundform derselben zur Anschauung zu bringen.

Der Gedanke an das Erlebte besch"aftigte mich den ganzen Tag.

Wie ganz anders erschien mir mit Einem Male das bisher so r"atselhafte Gestein!

Ich war schon im Begriff, mich zur Ruhe zu begeben, als mir einfiel, dass ich noch eine Kiste Gesteine von Canada verpackt im Hausgang stehen habe.

Ich wollte insbesondere noch ein St"uck wasserhellen Kalkspats wegen der kupfergef"arbten Pflanze darin suchen und damit meine ganze Arbeit schlie"sen.

Ich lie"s die Kiste "offnen.

Da fielen mir die Laurentian-Gneis in die Hand.
\clearpage
\section{\swabfamily{Was ist der Laurentian-Gneis?}}
\paragraph{}
Diese Frage stellte ich an mich.

Ich besah das Gestein, w"ahrend ich es ins Zimmer trug, und erkannte sofort,

dass auch der Laurentian-Gneis nur ein gro"ses Pflanzen-Lager sei.

Sofort wurden noch D"unnschliffe gefertigt und durch das Mikroskop vollkommen best"atigt, was das blo"se Auge gesehen hatte.

Ich untersuchte in der Folge alle meine Handst"ucke, etwa 30 St"ucke und fand eines wie das andere gr"o"stenteils aus Pflanzen (Algen) bestehend.

Die meisten sind schon mit blo"sem Auge zu erkennen.

Mit schwacher Vergr"o"serung l"ost sich die ganze Masse des Gneises in Pflanzen auf.

Also eine Formation, welche in ihrer M"achtigkeit auf 30,000 Fu"s gesch"atzt ist, ein Pflanzenlager!

Meine Laurentiangneis-Handst"ucke sind aus der Gegend von Ottawa und vom Lake Simcoe bis zum Nipissing-See gesammelt, letztere also von einer Fl"ache von 130 engl. Meilen und alle sind Eine gro"se Pflanze!

Ich fertige nun Tafeln, um eine Vorstellung zu geben.

Das Handst"uck I. ist von Templeton bei Ottawa (der Hauptstadt Canadas). Glimmerschiefer.

Es ist aus einer Apatitgrube. Es hat fleischroten Kalkspat, graugr"une Apatitkristalle mit 6 seitigen Tafeln von schwarzem Glimmer.

Seine Textur ist fein-bl"attrig. Auf der Bruchfl"ache erkennt man deutlich strahlenf"ormige Parthien. Sofort fallen auch wei"se rundliche Stellen in die Augen. Es sind die Kelch-Zellen, welche gegen den Beschauer gekehrt sind.

Bei genauer Besichtigung findet man die Zellkelche in den gr"o"seren Glimmer-Partien, sodann die Zellen, welche den Stamm bilden.

Schwarze Glimmer-Tafeln umh"ullen die sch"onsten Kelche. Die feinste Textur ist noch sichtbar.

Eine Pflanze kann schon mit blo"sem Auge auf 1 cm. Raum verfolgt werden.

Schon mit blo"sem Auge sieht man ferner die Verwandlung der Pflanze in Apatit und Glimmer.

Die Pflanzen nehmen etwa 2/3 der Gesteinsmasse ein.\footnote{\swabfamily{Vergleiche unten.}}

Die Apatit-Kristalle sind wahrscheinlich das Produkt des kleinen organischen Lebens.

Eine Mischung von phosphorsaurem und kohlensaurem Kalk wird von der Pflanze zersetzt.

Es trennen sich phosphorsaurer und kohlensaurer Kalk und so lagern die Apatit-Kristalle im kohlensauren Kalke.

Die Pflanze dieses Gesteins habe ich auf Tafel XVII. 10. abgebildet.

Ich nenne sie Dufferinia zu Ehren des Lords Dufferin, des Generalgouverneurs von Canada, als ich dort war.

Einzelzeichnungen Fig. 11. 12. 13.

Um einen D"unnschliff daraus zu fertigen, war das Gestein I. zu weich und zu wenig homogen. Der Glimmer bl"atterte ab, wie die Pflanzen darin lagen.

Fig. 14. eine Pflanze aus einem Apatit-Kristall.

Handst"uck 2. vom Muskoka-See. Sehr feldspatreich.

Taf. XVII. 4. Aus demselben St"ucke Fig. 5. 6. 7. 8. 9.

Ich konnte nur eine Pflanzen-Art feststellen.

Taf. XVII. Figur 10-13 sind schwache Vergr"o"serungen 1/5.

Ich nenne die Pflanze Fig. 4.

Victoria

zu Ehren der K"onigin Victoria.

Diese Pflanze scheint eine der "altesten und urspr"unglichsten zu sein.

Das zweite Handst"uck zeigt ferner die Form

Tafel XIX. Figur 3.

Ich nenne sie zu Ehren des Vorstandes des Geological Surveys of Canada Mr. Selvyn

Selvynia.
\clearpage
\section{\swabfamily{Glimmer}}
\paragraph{}
In einem kanadischen Glimmer aus dem Laurentian bilde ich Taf. XVII. Fig. 1. 2. 3. ab.

Die Zelle Fig. 3. ist in H"amatit verwandelt, ist durchsichtig und zeigt so den Kanal, der durch die Zelle bis zur Wurzel hinabgeht, aufs deutlichste. Fig. 2. ist wahrscheinlich ein Durchschnitt des Kelchrandes.
\clearpage
\section{\swabfamily{Das "`Urgebirge"'}}
\paragraph{}
Wieder war die Arbeit geschlossen.

War der Laurentian-Gneis von einer Masse Zellen-Pflanzen durchzogen, warum sollte es unser Gneis nicht auch sein?

Und wenn der Gneis es ist, wie verh"alt sich Granit und Porphyr?

Es ist oft unbegreiflich, wie nahe die Tatsache liegt, ohne dass man im Stand ist, sie zu sehen. Trotz des Mikroskops --- man sieht eben nichts.

Ich muss die Form meiner Abhandlung hier entschuldigen.

Meine Arbeit galt dem \emph{Eozoon}. Daraus wurde das \emph{Eophyllum}. Sie galt dem \emph{Eophyllum}. Daran schloss sich die Entdeckung der Pflanzenwelt des Serpentins. Sie sollte mit dem Serpentin geschlossen werden. Da fand ich, dass der Laurentian-Gneis ein Pflanzenlager enth"alt, nichts als Pflanze ist.

Aufs "au"serste erm"udet, war ich froh, endlich Feder und Zeichenstift niederlegen zu k"onnen; da dr"angt es mich wieder in die Forschung hinein: ich kann noch nicht ruhen.

Ich bemerke hier, dass die Entdeckung des Serpentins --- da ich dieses schreibe --- erst 2 Tage alt, und das was ich heute schreibe, die Arbeit von 12 Stunden ist. Gestern hatte ich das Manuskript f"ur den Druck schon geschlossen, als der Gedanke sich aufs Neue meldete und mich festhielt. Der Gedanke war es, welchen ich oben ausgesprochen.

Es war die Arbeit einer Nacht, die D"unnschliffe des Urgebirges zu durchgehen und nun steht es mir fest:

Der Granit ist nichts als Pflanzen-, keine Gesteinsmasse daneben, alles Pflanze!

Glimmer und Hornblende sind die Kelchzellen. Der Feldspat ist F"ullmasse der Pflanze, Baustoff der Pflanzenzellenh"aute. Der Quarz bildet meist Brutzellen.

Der Gneis enth"alt schon die vorgeschrittenen Formen der Pflanzen.

Insbesondere beim Knotengneis sind die Knoten nichts als gr"o"sere Kelchzellen.

Auch der Porphyr enth"alt lebende Pflanzen. Derselbe ist der Schlamm des ersten Urgebirges. (Felsit ebenso.)

Erst nach dem Suchen durch das Mikroskop sehe ich die Form mit blo"sem Auge --- Pflanzen von der vollen L"ange der Handst"ucke.

Ich gebe diese Arbeiten, wie sie entstanden sind. Es sind Entdeckungen von solcher Bedeutung, dass es wohl nicht als Anma"sung erscheinen kann, wenn ich zugleich einen Blick in meine Werkst"atte tun lasse, nebenbei ist es der individuelle Grund, dass ich zu erm"udet bin, um Alles umzuarbeiten.

Bei der nunmehrigen Beweisf"uhrung muss ich die Form der Urpflanze als bekannt annehmen.

Die Formen sind in Laurentian-Kalk unumst"o"slich festgestellt, und zwar schon in verh"altnism"a"sig gro"ser Verschiedenheit. Dort bestehen, wie ich vermute, hunderte von Arten.

Im Kalk sind schon die mikroskopischen Arten enthalten, vielleicht sind sie dort erst entstanden. Im Gneis sind es Pflanzen, welche mit blo"sem Auge zu erkennen, dabei schon sehr entwickelt in der Form sind.

Das Prototyp all dieser Pflanzen ist zumeist ein einfacher Trichter.

Die untere Spitze verzweigt sich zur Wurzelzelle.

Der Trichterhals teilt sich und streckt sich, Zelle reiht sich an Zelle durch Teilung.

Dadurch entsteht der Stamm.

Nun setzen sich sowohl am Rande des Trichters, als auch an der Zellkante neue Trichter an.

Aber auch in Form von Mycelium-F"aden strecken sich Zellen aus, welche wieder Trichter treiben. Diese Fadenzellen suchen Vereinigung und verzweigen sich.

Ich bemerke aber hier, dass auch in den Trichtern wirkliche Mineral- (Kristall-?) Einlagerungen vorkommen. Diese erscheinen mir als Produkte des organischen Vorgangs.

Ich bilde die Entwicklung der Urzelle ab Taf. XVIII. 1-7. jedoch nur, soweit ich das Bild im Serpentin gewann. Im Granit herrschen die Stammzellen vor, jedoch in der uranf"anglichen Form, welche sich denken l"asst --- der Kristallform.

Mit diesen Vorbegriffen und Vorstellungen, welche ich durch meine Beobachtungen als festgestellt annehmen darf, wollen wir an die Arbeit gehen.

Um die Kontrolle wenigstens zu erleichtern, nehme ich die "`Sammlung typischer Gesteine"' von R. Fuess in Berlin, Serie II., und beginne mit

Nr. 1. dem Granit von Brixen.

Ich werde, was ich im Schliffe finde, beschreiben und das Wesentliche abbilden, damit der Leser sich selbst orientieren kann. Zum Verst"andnis dieses Schliffes muss man schon das Bild aus dem Laurentian-Gneis Taf. XVII. 4. XIX. 3 zu Hilfe nehmen.

Auf den ersten Blick erscheint Alles als ein Gemisch von Feldspat und Quarz-K"orner mit Glimmer und etwas Hornblende. Sind es Spr"unge, welche den Schliff durchziehen? Sind es Kristallfl"achen? Es sind keine Spr"unge da, keine Kristalle.

Die Linien zeigen beim ersten Anblick eine gewisse Regel, aber sie ist unverst"andlich.

Pl"otzlich springt das Gesetz in die Augen!

Das ganze Gestein l"ost sich in Pflanzenzellen auf und diese selbst stehen in organischem Zusammenhang.

Taf. XVIII. Fig. 8. folg. mag als Anhaltspunkt f"ur die Aufl"osung eines Granits dienen.

Merkw"urdigeres noch bietet

Nr. 2. Granit von Altendamm.

Brutzellen siehe XVII. Fig. 16. 17. 18.

Eine Hornblende-Pflanze Taf. XVII. 15.

Nicht minder l"ost sich

3. Gneisgranit vom St. Gotthardt.

Ein Pflanzenbild daraus XX. Fig. 1.

4. Cordieritgneis von Mittweida

zeigt Taf. 20. Fig. 2. und 3.

5. Granulit von Hardtmannsdorf.

Taf. XIX. Fig. 14. 15. 16.

Granulit (und Granat).

Ich habe schon die Vermutung ausgesprochen, dass auch der Granulit Pflanzen enthalte.

Das best"atigt sich nun vollst"andig.

Allerdings sind die gr"o"seren mit blo"sem Auge nicht sofort erkennbar.

Allein wie oft, so muss auch hier die kleinere Form den Schl"ussel zur Erkenntnis der gro"sen liefern.

Die kleinen mit blo"sem Auge kaum unterscheidbaren r"otlichen Punkte im Granulit werden allgemein f"ur "`eingesprengte"' Granaten angesehen.

Sie sind die Ausf"ullung der mikroskopischen das Gestein horizontal durchwuchernden Pflanze, bezw. ihrer Kelche.

Taf. XIX. Fig. 14. 15. 16.

Ich zeichne einige solcher Pflanzen und bemerke nur, dass dieselben nach allen Seiten fortsetzen, Kelche haben, vorzugsweise aber in horizontaler Richtung liegen.

Die gro"sen Kristalle sind Ausf"ullmassen der gr"o"seren Pflanzenart. Ich nenne die kleinere Art zu Ehren des Herrn Prof. Oswald Heer

Granatina Heeri.

Auch hier muss man sich freilich durch das Prototyp der Urpflanze leiten lassen, dann bleibt kein Zweifel.

Dasselbe Verh"altnis findet hinsichtlich des Granats im b"ohmischen und s"achsischen Serpentin statt.

Auch hier sind die Granaten F"ullmassen der gr"o"sten Pflanzenkelche.

Die wei"sen Granaten von Auerbach sind nichts anders.

Sie tragen noch deutlich den Steinkern ihrer Kelchvertiefungen an sich.

Im umgebenden Gestein ist eine Menge mikroskopischer Pflanzen.

Ich habe vor mir einen Kalkgranat.

Granatoeder von 2 cm., voll von deutlich sichtbaren Pflanzen-Formen. Der Kern rundlich, leicht als Stein-Kern eines Pflanzen-Kelchs zu erkennen. Die Schichten in demselben tragen einzelne Pflanzen (in der hexagonalen Stellung des Dodekaeders).

Als ob die Algen Krystalle zu Fr"uchten gehabt h"atten. ---

6. Quarzporphyr von Schw"arz bei Halle.

Eine schlammige Grundmasse mit unz"ahligen Zellmembranen aber auch lebenden Pflanzen.

Taf. XX. Fig. 4. folg.

Nun f"uge ich noch bei

Darstellungen

eines Laurentian-Gneises von Canada

Taf. XIX. Fig. 1-8.,

eines Hornblendegesteins von Montreal

Taf. XIX. Fig. 9.,

eines Protogyns vom Montblanc

Taf. XIX. Fig. 17. 18. 19. 20.,

eines Gneises vom Montblanc

Taf. XX. Fig. 1.

7. Protogyn.

Es war seiner Zeit ein gro"ser L"arm, als man das "alteste Gestein der Erde gefunden zu haben glaubte. Man nannte es deshalb Protogyn.

Ich habe eine Zeichnung vom Protogyn des Montblanc gegeben.

Er ist wie der Granit eine Pflanze.

Da sind langgestreckte Pflanzen, ich nenne sie Protogynia.

Zelle an Zelle. Brutzellen nach oben und unten.

Die gelbe F"arbung ist von Eisen. Daneben sind wasserhelle Zellen, Hornblende. Es sind die feinen Nadeln des Chlorits und Talks, wahrscheinlich auch zersetzte Hornblende.

Die wasserhellen Zellen bestehen aus Quarz: sie nehmen den gr"o"sten Teil des Gesteins ein. Sie lassen sich aber leicht in die Form der Urpflanze aufl"osen.

Ich bilde eine Pflanze, mehrere Endzellen und Brutzellen ab.

Ebenso die Wurzel-Zelle. Taf. XIX. Fig. 17-20.
\clearpage
\section{\swabfamily{Eine Gesteinslehre}}
\paragraph{}
Ich versuche es nach diesen Ergebnissen, eine Gesteinslehre aufzustellen.

Die Gesteinsarten sind

entweder urspr"ungliche oder abgeleitete Zersetzungs-Produkte, Schlamm und Laven.

Urspr"unglich sind blo"s 1. die Pflanzenlager, 2. das Gestein aus der Masse des Erdinnern; welches dies ist, ist zweifelhaft.

Die meisten unserer "`Urgesteine"' sind Pflanzenlager und daraus abgeleitete Gesteine. Ich habe aber gar keine Gesteine ohne tote Pflanzen wenigstens gefunden, letztere in den Laven und Schlammgesteinen.

Die Gesteinsart ist also bedingt durch die Pflanzen, aus welchen das Gestein besteht:
\begin{enumerate}[label=\Alph*]
\item Die Urgesteine sind die Pflanzen-Lager: sie lassen sich auf die Pflanzen-Formen zur"uckf"uhren, welche darin leben und sich so zusammensetzen.
\item Es gibt Mittelstufen: Detritus von Pflanzen-Lagern mit eingelagerten lebenden Pflanzen.
\item Es gibt Gesteinsschlamm blo"s mit eingelagerten toten Pflanzen.
\item Laven.
\end{enumerate}
\paragraph{}
Das urspr"ungliche Mineral, aus welchem die Ur-Gesteine zusammengesetzt sind, ist ein Gemisch von Quarz, Feldspat und Hornblende (Glimmer).

Diese scheidet die Pflanze.

Die gebildete Pflanze scheidet die "ubrigen Minerale aus, insbesondere Granat, Leuzit u. s. w.

Die ersten Pflanzenzellen sind Kristall-Formen.

Feldspat zwei- und eingliederig, Quarz sechsgliedrig bildet die Hauptform. Die Hornblende bildet die Kelchzellen.

Doch ist das Verh"altnis der Zellen zu ihrem Inhalt noch n"aher festzustellen.

Man sehe durch den Polarisations-Apparat und man wird jede Pflanzenzelle im Granit leicht unterscheiden und jedes Individuum leicht herausfinden. Die Streifen f"ur Oligoklas angesehen, sind horizontale Anwachsungsstreifen der Zelle.

Schriftgranit ist Feldspatmasse mit wundervoll sch"onen Quarzpflanzen.

Der Granit ist das erste Gestein. Von Anfang an ist Hornblende in ihm, welcher sich in Glimmer verwandelt. Sp"ater "ubernimmt Hornblende die Bauarbeit.

Der Basalt enth"alt ganze (ich nenne sie lebende) Pflanzen. Was Zirkel Nephelinbasalt nennt, sind Stammzellen einer Pflanze, welche ich Mycelium Zirkeli nenne. Die "`Fluidalstruktur"' also Algenzellen.

Die Olivine im Basalt sind Pflanzen-Zellen (sei es urspr"ungliche, sei es, dass der Olivin-Crystall nachher die Pflanze umschreibt.

Die meisten Basalte sind daher Wasser-Produkte.

Ebenso enth"alt Gabbro, Talk-Pflanzen.

Man betrachte die Gestalt der Ur-Pflanze und man wird sie "uberall wieder finden.

Man pr"age sich die Form derselben Tafel XVIII. ein, nehme ein St"uck grobk"ornigen Granit und man wird die Form so scharf gezeichnet sehen, als die eines Ammoniten im Jura.

Im Hypersthen ist die urspr"unglichste Form der Pflanzenzelle (nach einer rein geometrischen Linie) und so findet man auch die Zelle teilweise im Granit.

Naher Zusammenhang der Kristallform mit der Form der Pflanzenzelle! Urspr"unglich dasselbe.

Leucit bildet Kelchzell-Ausf"ullungen mit Anwachsungs-Streifen. Die Punkte in den kleinen Kristallen sind wohl Sporen. Hieraus erkl"aren sich die Polarisations-Erscheinungen.

Die Schiefersteine sind teils urspr"ungliche, teils abgeleitete. Im ersteren Falle bilden sie sich aus einer Masse von Zellpflanzen, im letzteren Fall sind sie ein Magma mit einer Unzahl toter und weniger lebender Zellen.

Der Pechstein ist ein Pflanzengestein und nicht vulkanisch.

In den meisten Gesteinen lassen sich die Pflanzen mit blo"sem Auge leicht erkennen, wenn man die Grundform wei"s.

Ich habe 2 Basalte von Biberich mit 2 1/2 cm. langer Pflanze, das Gesteinsst"uck ist 3 cm. lang.

Unbegreiflich, dass man dies Alles bis jetzt "ubersah.

Man hatte eben den Schl"ussel, das Bild der Urpflanze, nicht.

Ich machte den zweiten D"unnschliff aus dem Laurentian-Gneis und sah erst beim Bedecken mit dem Glas die Pflanze mit blo"sem Auge.

Die k"unftige Gesteinslehre ist Pflanzen-Versteinerungslehre.

Sismondi fand ein Equisetum (wie ich erst heute aus Mohr sehe), im Gneis eingelagert.

Dass der ganze Gneis Pflanze sei, sah er nicht.

Die n"ahere Darstellung auf Grund genauer Untersuchung der einzelnen Gesteine wird folgen.

Die vorstehenden Thesen sollten nur die ersten Beobachtungen geben. In der Hauptsache wird die Sache aber so bleiben.

Insbesondere m"ogen die Mineralogen auf die Pflanze zur"uckgehen. Es ist wunderbar, in welch engem Zusammenhang Pflanze und Kristall stehen.

Es gibt sehr viele Pseudo-Krystalle durch blo"sen Druck, Guss-Formen aus Kelchzellen, welche mathematische Form haben.

Man h"atte schon stutzig dar"uber werden sollen, dass der Granit keine Kristalle enth"alt. Die Linien im D"unnschliff des Granits sind Pflanzen-Zell-Membranen.

Man nehme den Polarisationsapparat. Auch der Bl"atterbruch wird oft nur die Anwachsstelle in der Zelle sein.

Die Tschermak'sche Theorie mag chemisch richtig sein. Wichtiger aber ist die mechanische Erkl"arungsweise.

Der Orthoklas ist in Anwachsungsschichten abgesetzt, daher die verschiedenen Polarisationsstreifen. Die Zellw"ande spielen mit. Dies ist im Labrador leicht erkennbar, welcher Pflanzenst"amme von Zolldurchmesser enth"alt. Feinste Zelllagerung, daher die Lichtbrechung. Sehe es eben an einer Pflanze von 1 cm. Durchmesser.

Nun haben wir f"ur die Zukunft eine einfache Gesteinslehre. Alles ist Sedimentgesteinslehre.

Im Urgebirge ist die Pflanze, vom "Ubergangesgebirge mag das Tier Leitfossil sein.

Ich habe den Vorteil sofort genossen.

Es ist zu untersuchen, inwiefern manche Gesteine nicht Umwandlungs-Produkte des Stoffes (nicht der Granitpflanzen) enthalten. Diese bleiben.

In jedem Gestein finden sich Zellen und Steinkerne von Pflanzen.

Man setzt allen Pechstein zu den Vulkan-Gesteinen: der eine ist ein Lager von lebenden, der andere von toten Pflanzen.

Der erste Kalk stellt sich wirklich als Ausscheidungsmasse dar, so der Kalk des Laurentian-Gneises.

Nachher wird er erst Baustoff f"ur Pflanzen-Arten. Vor mir liegt ein Serpentin von Todtmoos mit einem Pflanzen-Kelch von 1 Ctm. L"ange im Durchschnitt, so fein gezeichnet, als irgend ein Zeichner es vermag. Wundervoll wie im Basalt der Olivin-Kristall Pflanze, die Pflanze Kristall ist.

Der Olivin ist Pflanzenzelle, die Feldspate, der Nephelin ebenso.

Die gro"sen Kristall-Kelche stechen in die Augen. Die Wurzelzelle, welche dazu geh"ort, ist ein kleiner Kristall, h"aufig ist auch die Brutzelle auf dem Kelche ein Kristall. Zuweilen ist die ganze Zelle von dem Kristalle umschrieben. Wie eine mathematische Aufgabe.

Trachyt ist ein Gemisch von lebenden und toten Zellen, ebenso Klingstein.

Ich w"are heute schon im Stande gewesen, die Leitpflanzen f"ur etwa 30 Gesteine zu zeichnen. Wollte es aber lieber Anderen "uberlassen, welche mehr Zeit haben.

Die beliebten Metamorphosen m"ussen nach meiner Ansicht stark zusammenschrumpfen.

Mechanische Zerreibung, L"osung, Ablagerung --- dies sind aber nicht die eigentlichen Metamorphosen.

Die Tr"aume von Verwandlung eines Gesteins ins andere l"osen sich auf.

Denn wenn des Gesteins-Art und Struktur von der Pflanze abh"angt, aus deren K"orper das Gestein wurde, so kann ein urspr"ungliches Gestein nicht --- ein anderes werden. Ist so wenig m"oglich als eine Eiche in eine Palme, und eine Palme in eine Eiche sich verwandelt.
\clearpage
\section{\swabfamily{Carrararischer Marmor}}
\paragraph{}
Ich bezog eine wei"se Platte mit grauen Flecken zu einem Buffet welches mir als Kasten f"ur meine Steinschliffe dient.

Ich versch"uttete etwas Fuchsin und musste die Platte reinigen.

Ich rieb sie.

--- Pl"otzlich was sah ich!

Kelche von 1 Meter L"ange. Taf. XXI.

Ich nenne sie

\emph{Marmora darwini}.

Mein Schreibtisch hat dieselbe Platte, aber nur mit etwas weniger deutlichen Pflanzen.

Ich schrieb also eine ganze Abhandlung "uber die Urzelle auf der Urzelle, denn der ganze Marmor ist nichts als Pflanze. Wirklich --- nicht blo"s Ironie des Schicksals.
\clearpage
\section{\swabfamily{Nachtrag}}
\paragraph{}
Ich habe den Ophicalit von Krummau in B"ohmen, ferner den von Passau untersucht: dieselben, welche G"umbel und v. Hochstetter als \emph{Eozoon} erkl"arten. Beide enthalten Pflanzen, jedoch nicht so viele, als das canadische Gestein. Der "`Warzenans"aze"' G"umbels habe ich schon gedacht.
\clearpage
\section{\swabfamily{Basalt}}
\paragraph{}
Ich wollte den Basalt blos gelegentlich besprechen. Ich konnte mir aber doch nicht versagen, dieses von den Vulkanisten am eifrigsten umworbene Gestein einer genaueren Untersuchung zu unterziehen.

Ich habe 2 Basalte von Biberich.

Beide haben 3 cm. lange Pflanzen (ohne alle Unterbrechung) Zelle an Zelle gef"ugt.

Der D"unnschliff zeigt ein vollst"andiges zusammenh"angendes Pflanzen-Netz. Es kann sich daher bei beiden nicht darum handeln, dass sie etwa blos eingelagerte Pflanzen (Leichen) enthalten. Die Pflanzen sind durchaus erhalten, also, wie ich sie nenne, lebende. Dasselbe Verh"altniss ist bei den Alb-(Nephelin)Basalten. Einzelzeichnungen Taf. XXVIII. 1-7.

In jedem Basaltst"uck, besonders den leicht angewitterten, lassen sich die Pflanzen erkennen. Die Frage ist jetzt nur noch: was ist Basalt, was ist Lava?

Der Basalt ist ein Eruptivgestein: er r"uhrt aber von Schlammausbr"uchen, ebenso der Basalttuff.

Es w"are von grossem Interesse (es ist meines Wissens noch nicht geschehen), dass man die neueren Schlammausbr"uche in der Tiefe, wo sie rein erhalten sind, untersucht.
\clearpage
\section{\swabfamily{Feldspath}}
\paragraph{}
Auch "uber dieses Mineral muss ich noch einige besondere Bemerkungen machen.

Am sch"onsten ist der Labrador.

Es sind Riesenpflanzen. Die blaue F"arbung folgt, wie ich mich jetzt an einem grossen St"uck der T"ubinger Sammlung "uberzeugt habe, vollst"andig der Structur der Pflanze, "ahnlich der von mir im Silur von Quebec gefundenen Oldhamia antiqua. Die Streifungen und die microscopischen Einlagerungen sind auf Rechnung der Pflanze und des Pflanzen-Wachsthums zu schreiben.

Die gemeinen Feldsp"athe sind von Pflanzen ganz durchzogen. Mir scheint die 2- und 1 gliedrige Form an die Form der Pflanze und Pflanzenzelle sich anzuschliessen. Die Grundform der Stammzelle im Granit ist 2- und 1 gliedrig.

Es liegt mir ein Feldspath-Crystall aus der T"ubinger Sammlung mit einem Kelch vor, dessen Spitzen ich gerade auf Fl"ache M einschrieb.

Der Crystall ist aufgewachsen.

Das Muttergestein ist Schriftgranit, in welchem die Algen in Quarz verwandelt sind.

Man sieht nun, dass es eine Pflanze ist, ein grosser Kelch, welcher die Crystalle bildet.

In einem Hornblende-Gestein von Montreal sind v"ollig entwickelte Pflanzenkelche von einem Hornblende-Crystall umschrieben.

Es wird Aufgabe der Zukunft, das Verh"altniss von Crystall-Form zur Grundform der Pflanze und umgekehrt zu untersuchen.

Perthit von Granville Canada sind die sch"onsten macroscopischen Pflanzen. Ebenso gut h"atte es ein Schriftgranit werden k"onnen.

Ein Bild eines Hauyngesteins bietet Tafel XXII. Hornblende schr"ag, Hauyn horizontal schraffirt, das schwarze ist Magnetit, das "ubrige Gestein (Feldspat)-Pflanze. Man sieht an dem Kelche links den Kelchstil als Crystall: einen Kelch in einen Hornblendecrystall verwandelt, zwei Brutzellen in Hauyn: ebenso an der Wurzelzelle und in diese Crystalle eingef"ugt das Bild der Pflanze!
\clearpage
\section{\swabfamily{Silur}}
\paragraph{}
Die Algen des Urgebirgs setzen im Silur fort. In Quebec und Montreal habe ich eine ansehnliche Sammlung von Handst"ucken zusammengebracht. Pflanzenbilder daraus Taf. XXII. XXIII.

Von Quebec die Kalk- und Thonschichten, in Montreal Kalk und Hornblendegesteine.

Im Silur-Kalk von Quebec sind "uberall die Algen noch die Gesteinsbildner, dann folgen die Lagen von Detritus alias Thonschiefer.

Der Mont real bei Montreal ist ein H"ugel von Hornblende.

Ich gebe einige Pflanzenbilder aus demselben in Tafel XXIV. Aber auch hier sind nicht die Pflanzen im Gestein, sondern das Gestein ist noch die Pflanze.

In einer Kelchzelle, aber auch nur dort, fand ich die Bilder

Tafel XXIV. Figur 5. 6. 7.

Ich hielt sie f"ur eine Diatomacee und nenne sie

Diatoma Montreali

der Stadt Montreal zu Ehren, deren Gastfreundschaft ich genoss.

Die Pflanzen-Form Taf. XXIV. Fig. 2. nenne ich

Parthenon Munderlohi

(deutscher Consul in Montreal), der mich auf der Excursion begleitete.

Merkw"urdig sind Kieselknollen, die im Silur bei Montreal gefunden werden, am"obenartige Formen bis zur Gr"osse eines Kopfes, mit welchem die Anlagen des M'Gill-Colleges garnirt sind. Sie sind nichts als Stammzellen von Riesenalgen.

Ich nenne sie Herrn Dawson zu Ehren, der dort wohnt,

Photophoba Dawsoni.

M"ogen wir uns "uber diese Photophoba hin"uber die Hand der Vers"ohnung reichen.

Im Silur, schon in dem Hornblendegestein, sind riesige Formen. Eine Kelchzelle von 3 cm. Breite! und zwar dieselbe Form, aus welcher das Labrodorgestein gebildet ist.

Taf. XXIII. Fig. 7.

Ich nenne sie Beyrichia zu Ehren des Herrn Prof. Beyrich. Eine Art Oldhamia.

Im Silurkalk Kelche von 5-6 cm. Man sieht die Wirkung des jetzt ausgeschiedenen Kalkes als Pflanzenbildners.

In den Anlagen von Montreal sind sog. "`Fancy"'-Steine, Kalk mit Protuberanzen von 30-40 cm. Durchmesser. Es sind riesige Algen-Zellen. Ich habe am Lagerplatz einen abgeschlagen, der die deutlichste Structur zeigt.

Aber auch kleinere Formen gibt es, ja microscopische --- das Silur ist noch vorzugsweise Pflanze.

Hier beginnt erst das grosse Tierleben.
\clearpage
\section{\swabfamily{R"uckblick}}
\paragraph{}
Im Silur findet die gr"osste Entwicklung der Individuen statt, dagegen nehmen sie in der Zahl ab.

Durch den Kalkniederschlag schliesst sich die Urpflanze von ihrem Lebenselement, der Kieselerde, ab und nun beginnt die Aera des Thons, des Kalkes. Die Formen werden vielf"altiger, feiner.

Beweis f"ur die Entwicklung neuer Arten bei ver"anderter Lebensbedingung!

In der Tat finde ich in dem chemischen Vorgang den Schl"ussel zum morphologischen.

Jeder Niederschlag ist eine Aenderung der Mischung: das Zur"uckbleibende ist immer ein Teil, wie das Ausgeschiedene. Mit dem Laurentiankalk war der Anfang zur Entkalkung der Erdoberfl"ache gemacht und damit zugleich der Aufl"osung der Kieselerde durch das Wasser ein mechanischer Abschluss gegeben. Die alten Pflanzen, welche durch ihre Bez"uge, wie ihre Ausscheidungen die Oberfl"ache der Erde ge"andert hatten, welche zuletzt in mechanischen Decken auf ihrem Lebenselemente lagern, finden ihr Gedeihen nicht mehr wie fr"uher. Nur einzelne Individuen "uberdauern die Katastrophe, entwickeln sich im Laurentian (Labrador und Hornblende), zulezt in der Kohlenperiode zu Riesen, um dann einer j"ungeren Generation zu weichen.

Im Silur finden sich noch die alten Arten von Algen. Ich glaubte mein Werk nicht besser schliessen zu k"onnen, als mit Abbildungen einer Anzahl Pflanzen aus dem Silur. Allerdings nicht derer, welche man dort gew"ohnlich aufgef"uhrt findet. Die bekannten Algen aus dem Silur sind ganz andere, als die Urpflanze, welche ich nachwies; h"atte man sie im Silur erkannt, so h"atte sie auch im Granit und Gneis nicht "ubersehen werden k"onnen.

Der Granit und Gneis ist die eigentliche Pflanzenzeit unserer Erde. Im Silur schwindet die Masse der Pflanzen. Die Formen sind noch da, aber der Schlamm herrscht schon vor, welcher die Pflanze die Kieselerde zu l"osen verhindert.

(Jedes Gesch"opf gr"abt sich selbst sein Grab!)

Nachdem auch der Kalk niedergeschlagen ist, wird die Luft das Lebenselement der Pflanze, vorher war es das Wasser.

Damit brach die Kohlenzeit an.

Die Urpflanze verkriecht sich auf die Tiefe des Meeres und verschwindet auf der Erde, setzt aber ihr Leben unter der erh"arteten Oberfl"ache fort.

Zwar habe ich noch in einem lithographischen Kalk, auf welchen ich meine Abbildungen zeichnete, einzelne wenige Sporen eingebettet gesehen. Was soll das gegen fr"uher?

Eine wichtige Frage bleibt "ubrig.

Ich habe oben festgestellt, dass der Basalt Pflanze ist. Aber viele Basalte geh"oren der Terti"arzeit unserer Erde an.

Ich habe mir die Untersuchung auf sp"atere Zeiten aufsparen wollen, nehme sie aber doch lieber sogleich vor.

Ich untersuche

Basalt von Biberich.

Basalt von Neuhausen, Urach,

Basalt von Kirchheim,

ersterer Feldspat-, beide letztere Nephelin-Basalte.

Anfangs glaubte ich, letztere enthalten todte Pflanzen oder Pflanzen-Reste, bald "uberzeugte ich mich vom Gegenteil --- auch der sog. Nephelin-Basalt ist ein Pflanzen-Gestein mit lebenden Pflanzen (wenig todte Reste.) ---

Dies fordert zu neuen Gedanken auf.

Die Pflanze zur Terti"arzeit musste einen langen Weg machen und kam lebend zur Oberfl"ache!

Die Urpflanze lebte also in der Nacht, sie lebt heute noch dort und bedarf zu ihrem Wachsthum nur des Wassers.

Dies fliesst ihr von der Oberfl"ache zu.

Wir leben auf einer grossen Sargasso-Decke.

Dass dies so ist, beweisen die wellenf"ormigen Erdbeben, beweisen die Vulkane.

Die Vulkane kommen mit ihrem Meere, gehen mit ihrem Meere.

Alle unsere t"atigen Vulkane liegen nicht zu entfernt vom Meere.

Beweis, dass sie dem Wasser ihre Entstehung verdanken.

Die T"atigkeit des Erdinnern muss eine organische sein, eine unorganische w"are nicht denkbar. Denn alle m"oglichen blos chemischen Verbindungen h"atten l"angst vollzogen sein m"ussen. Ohne Hinzutreten eines neuen Stoffes w"are eine T"atigkeit nicht m"oglich. Dieser Stoff kann kein anderer sein, als Luft oder Wasser. Die Luft ist es nicht, wenigstens blos an offenen Stellen, an schon entstandenen Vulcanen wird sie von geringer Bedeutung.

Auch die fr"uheren wirklichen Revolutionen der Erde hatten ihren Grund in der organischen T"atigkeit der Urpflanze.

Heute noch wuchert sie unter der festen Erdoberfl"ache, bildet Gase, und so entstehen die Erdbeben, die Vulcane.

Die Vulcane sind nicht Feuerbr"ande im Innern der Erde. Vielmehr sind es blos Gase, welche beim Austritt aus der Erde sich r"uckw"arts entz"unden, mechanisch wirken und dabei das anstehende Gestein schmelzen. Lava.

Daher die periodische T"atigkeit, der Anfang derselben blosse Gasausstr"omungen. Erdbeben.

Es k"onnen sich leichtbrennliche Gase bilden, welche wie die schlagenden Wetter sich entz"unden und dann bei fortw"ahrendem Zustr"omen fortbrennen.

Eben diese Art der T"atigkeit in Verbindung mit Erdschwankungen spricht daf"ur, dass die von der Pflanze ausgeschiedenen Gase dem Ventile zustr"omen und dort verbrennen, bis der Vorrath ersch"opft ist.

Alles dies ist blos mittelst des Wassers m"oglich.

Der Basalt ist einer der letzten Schlammausbr"uche. Ausbruch der damals wenigstens sicher noch lebenden Urpflanze.

Die Steinkohlen kommen nicht von den Equiseten der damaligen Zeit, sondern verdanken ihr Dasein den Kohlen-Gasen aus den unermesslichen Lagern der Urpflanze. Daher auch das Fehlen der Kohle im Granit: Einschl"usse von Kohlens"aure, Anthracit- und Graphitlager sind noch davon "ubrig.

Jedes Meer der Erdzeiten hatte seine eigene Vulcane; sie sind aber erloschen, sobald das Wasser seinen Ort wechselte.

J"ungerer Granit? "alterer Granit? Geburtstag --- Zeugungstag. Verwandtschaft des Granits mit dem Trachyt.

Gibt es Crystalle ohne Pflanzen?

Der Diamant (daf"ur spricht seine Form) ist F"allmasse von Pflanzenkelchen.

Die Erdw"arme r"uhrt nicht von unterirdischem Feuer her, sondern von chemischen Processen; daher keine Angst vor der Erkaltung der Erde, solange Luft und Wasser sie umgiebt.

Ich untersuche einige »Vulkan-Gesteine« und finde

Roca Pichincha von Quito --- lebende Pflanzen.

Hauyngestein von Tomlada-Lazio --- lebende Pflanzen.

Abbildung Tafel XXII.

Schr"ag schraffirt Hornblende, gerade schraffirt Hauyn, dunkel Eisen. Hell gezeichnet weisse Glas- oder Feldspatmasse.

Wunderbare Zusammenf"ugung von Pflanze und Crystall!

Ich nenne diese Pflanze zu Ehren Sr. K. Hoheit des Kronprinzen Rudolph von Oestreich, des eifrigen Naturforschers,

Stygia Rudolphi.

Die Laven sind wirklich todte Gesteine, sie enthalten blos Leichen von Pflanzen.

Abgerissene St"ucke.

Sphaerulit hat macroscopische Pflanzen, todte eingebettet.

Perlit von Antisana --- lebende Pflanzen. Phonolit lebende Pflanzen.
\clearpage
\section{\swabfamily{Die Ergebnisse dieser Entdeckungen}}
\paragraph{}
Perlit von Antisana --- lebende Pflanzen. Phonolit lebende Pflanzen.

Die Ergebnisse dieser Entdeckungen sind bedeutend.

Wir d"urfen uns nun die Erdoberfl"ache beim Beginn des organischen Lebens jedenfalls nicht mehr als von einem heissen oder gar feuerfl"ussigen Granit- oder Gneisbrei bedeckt denken. Denn dieser Brei, welcher so lange die Phantasie der Geologen besch"aftigte, bez"uglich dessen sie sich nur dar"uber stritten, ob derselbe feuerfl"ussig oder wasserfl"ussig gewesen sei, (was man denn auch in die ganz unrichtige Frageform fasste: ob durch Wasser oder Feuer entstanden?) --- dieser Brei muss jetzt einer anderen Vorstellung weichen.

Da war von den ersten Atmosph"areniederschl"agen Eine grosse See "uber die ganze Erde, in dieser wuchsen Pflanzen in undenkbarer Menge, und zwar die niedersten Arten einer einzelligen Alge oder Flechte.

Diese Pflanzen, die Erzeugnisse der Verbindung des ersten Wassers mit der Erde, bauten sich in Kieselzellen auf, setzten neue Zellen an, starben ab, und das wurde die erste Erdschichte!

Leider haben sie die Erde selbst dem Geologen auf ewig bedeckt. Damit, dass der Basalt als Pflanze nachgewiesen ist, ist jede Hoffnung verloren, jemals das Erdinnere kennen zu lernen.

Diese Pflanzen des Granits m"ussen ein grosses Volum eingenommen haben, sie bildeten schon im Anfang ihres erstaunlichen Wachsthums Berge und H"ugel in der See und diese Berge und H"ugel wurden zu unseren heutigen Granit- und Gneis-Gebirgen. Daher die F"acher-, die Kuppel-Formen.

Die Gneisbildung war einzig und allein die Folge davon, dass die Schr"unde, Risse und Tiefen der urspr"unglichen Erdoberfl"ache durch Tange, Schlamm ausgef"ullt waren. Nun erst konnte sich ein eigentliches regelm"assiges Sediment bilden.

Es war m"oglich, wenn eine ebene Grundlage geschaffen war.

Indessen bildeten sich die Pflanzen in der Form weiter.

Wo die Ablagerung eine urspr"ungliche, das Wachsthum der Pflanze ein ungehindertes war, finden wir Gneis in seinen verschiedenen Formen.

Aber es gab auch Stellen, wo Detritus sich ablagerte. Hier entstand aus den Leichen und dem Schlamm des Granit- und Gneismeeres der Porphyr.

Allerdings nicht ohne eigenes Leben. Aber das Leben darin ist viel kleiner. Die Formen sind meist microscopische, eine Masse abgerissener Leichen (ohne Zusammenhang mit dem Pflanzenstock) finden sich. Dagegen haben sich auch Crystalle auszuscheiden angefangen.

Man hat sich gewiss schon oft im Stillen wenigstens dar"uber gewundert, warum im Granit und Gneis der Quarz nicht crystallisirt, sondern blos in (wie man glaubte) ganz willk"urlicher Form vorkommt.

Der Grund ist, weil der Quarz und Feldspat des Granits und Gneises in einen Pflanzenleib gebannt ist. (Hier sind blos die Anf"ange einer Crystallisation.) Er ist Produkt der ersten organischen T"atigkeit der Erde. Im Porphyr war er frei und so sehen wir aus der Pflanzenth"atigkeit den Stoff der k"unftigen Crystalle hervorgehen (wie den Apatit, Kalk aus den Gneis-pflanzen).

Es w"are sehr wichtig, jetzt die Entstehung bedeutender Mineralien (einschliesslich der Erze) in diesem Zusammenhang zu erforschen.

Die meisten Minerale sind wol Produkte organischer T"atigkeit, sofern durch letztere der chemische Process wenigstens eingeleitet wurde, der zur Entstehung der Minerale f"uhrte.

Diese Tatsachen erkl"aren auch die "`Einschl"usse im Quarz und Feldspat"' einzig gen"ugend, besonders was die Regelm"assigkeit der Lagerung betrifft, zerst"oren aber sofort die Hypothese vom ungeheuren Druck, welcher die Kohlens"aure darin zusammengepresst h"atte.

Wenn ich also jetzt sage: Granit, Gneis, Porphyr sind kein Urgebirge, so heisst dies richtiger gesagt:

Die Bestandteile dieser Gebebirge Quarz, Feldspat, Glimmer sind in ihrer jezigen Form in Folge der organischen Arbeit der Erde ausgeschieden und wieder zusammengelagert. Das f"uhrt auf die Frage: was war vorher da? Da sind wir allerdings durch meine Entdeckungen nur in soferne besser daran, als wir jetzt wissen, dass Etwas, nemlich das Urgebirge, "`im Anfang"' nicht da war, also auch entstand wie das "ubrige Sedimentgebirge. War es meteorsteinartige Masse?

Wir m"ussen also unsere Zuflucht wieder zur Beobachtung nehmen.

Was die Erde heute noch auswirft, Augit, Feldspat in der Lava ist der Kern der Erde, allerdings ist es aber auch zweifelhaft, ob ganz in dieser Form, ob die Masse nicht auf ihrem Wege zum Lichte selbst schon Beimischungen erhalten hat, ob nicht die Metalle, insbesondere das Eisen (nach dem Gesetz der Schwere) wenigstens teilweise davon zur"uckgeblieben sind.

Meine Entdeckungen n"othigen nun zu einer besseren Erforschung und Erkl"arung der Tatsachen bez"uglich des Erdinnern.

Ein Wunsch, den ich daran ankn"upfen m"ochte, ist der: nicht ohne Erforschung, ohne ganz gr"undliche Versuche Hypothesen auf Hypothesen zu h"aufen. Ich gestehe, ich schlich mich jedesmal aus der Gesellschaft fort, wenn die Rede auf die Vorg"ange in der Granitzeit kam und von diesen nat"urlich f"urchterlichen Revolutionen mit einer Sicherheit gesprochen ward wie von einem Theaterst"uck, das man von der 1. Gallerie aus mit ansah. Es gibt in der Tat nichts wohlfeileres als solch wichtige Besprechungen, aber auch keinen endloseren Streit als diesen.

"Uberhaupt m"ochte ich dringend rathen, den Weg der Beobachtung, des Versuchs noch lange nicht zu verlassen.

Seit der Aufstellung der Darwin'schen S"atze, welche aber durch tausend Beobachtungen, also Tatsachen, begr"undet waren, glaubt Jeder S"atze aufstellen zu d"urfen --- ohne Beobachtung. Damit ist die Wissenschaft selbst und zwar nicht unverschuldet in starken Misscredit gekommen, sie h"orte eben auf, die exacte zu sein und fiel damit in einen Fehler, welchen sie an einer andern Wissenschaft r"ugte.

Freilich, man braucht auf das Denken nicht zu verzichten. Man kann auch im K"ammerlein sich seine Hypothesen machen. Aber man h"ute sich, aus 2, 3 Tatsachen das Weltganze erkl"aren zu wollen.

Auch die Freude an dem grossen Naturleben (am Ganzen) will ich nicht verderben. Im Gegenteil, ich m"ochte sie erst erweitern. Dieses ist nur m"oglich nach redlicher Arbeit am Ganzen. Dies ist die Besch"aftigung mit dem Einzelnen, eine Besch"aftigung, die nie ermattet und die grosse Schuld der Zeiten sandkornweise tilgt, diese Arbeit gibt uns allein einen wahren wissenschaftlichen Einblick ins Ganze.
\clearpage
\section{\swabfamily{Sandsteine}}
\paragraph{}
Die Huronian-Formation ist, wie ich vermuthe, ebenfalls eine Pflanzenschichte, wie die meisten Sandsteine.

Sicher, dass es Pflanzenschichten sind, weiss ich vom Buntensandstein, Lettenkohlen- und Keuper-Sandstein.

Als ich nemlich in T"ubingen zur Druckerei ging, fand ich ein St"uck Stubensandstein, etwa 6 cm. Durchmesser mit den deutlichsten Kelchen. Die Staffel des Hauses von Kneuper-Sandstein zeigte Pflanzen von 1/2 Mtr. L"ange.

Die Rampe des Bahnhofs von Lettenkohlen-Sandstein ist durchaus Pflanze, wie auch daneben das Pflaster von Bonebed-Sandstein.

Wie hundertmal bin ich schon "uber diese Platten und Andere sind noch "ofter dar"uber gegangen.

Zu Hause untersuchte ich D"unnschliffe von Buntemsandstein und Keuper. In ersterem fand ich Kelch an Kelch (von l"anglicher Form), in letzterem ebenso kleine rundliche Kelche, beide microscopisch. Die Kelchzellen im Bunten-Sandstein von Glimmer! Prachtvoll.

Hier ist also die Pflanze "uberall mit Quarz ausgef"ullt. Daher Sandstein!

Wie einfach nun Alles.

Statt Str"omen von Meeren ruhige Pflanzen-Seen.

So wird noch aus mancher haarstr"aubend geschilderten, ja sogar abgebildeten Revolution auf der Erde, wie sie unsere Geologen mit Vorliebe vorf"uhren, ein stagnirendes Wasser.

Wo Platten und regelm"assige Schichten-Ablagerungen sind, muss immer an Pflanzen gedacht werden. Die Pflanze ist ein zerbrechlicher Stoff, daher der schieferige Abgang.
\clearpage
\section{\swabfamily{Meteorstein}}
\paragraph{}
Meteorstein von Knyahinya vom 6. Juni 1858.

Wieder war meine Arbeit geschlossen, als ich nochmals meine Schliffe vornahm, um einzelne Gesteine wie Gabbro zu untersuchen.

Ich war mit der Durchsicht am Schluss und hatte mich schon herzlich gefreut, auch in dem Gabbro-Gestein dieselben Pflanzen gefunden zu haben, wie im Granit, --- da waren noch 4 Meteorsteine "ubrig.

Der Zweck, der mich besch"aftigte, war, organische Einschl"usse zu finden. Sollte ich die Meteorsteine ansehen? Ich f"urchtete mich fast, so t"oricht zu sein und in der Tat vor einigen Tagen hatte ich die Schliffe f"ur jetzt als werthlos zur"uckgelegt.

Endlich entschloss ich mich doch.

Der erste Schliff von einem Stein der T"ubinger Sammlung (ganz klein) mit vielen Olivinen zeigte bedenkliche Formen.

Der zweite ebenfalls mit ziemlich viel Olivin zeigte Pflanzen und zwar Pflanzen mit solcher Sicherheit, dass nicht der kleinste Zweifel blieb.

Der Stein ist von Knyahinya 6. Juli 1858.

Die Schnittfl"ache ist etwa 70 mm.

Das Aussehen grau-gr"unlich, man sieht kleine metallische Punkte, ferner zwei durchsichtige Stellen: im Microscop zeigt er gelb und grauliche F"arbung mit schwarzen Einlagerungen.

Die Formen, welche ich feststelle, sind folgende:

\emph{Urania guilielmi}

zu Ehren unseres Kaisers. Taf. XVII.

Die Pflanze steht zwischen Alge und Farn.

Ein kreisrundes gew"olbtes Blatt.

Von dem Stammzelle-Ansatz gehen 12 Samenzellen aus, welche am Rande des Blattes mit einer runden Oeffnung endigen.

Am oberen linken Teile steht eine neue Zelle an.

Der Stamm w"achst aus einem runden Kelche und endigt dann mit einer Wurzelzelle. An demselben Stamm sind mehrere junge Zellen ausgewachsen.

Die Farbe des Gesteins ist ein Smalte-Grau.

Das Bild ist bei 90 facher Vergr"osserung gezeichnet.

Auf beiden Seiten unten sind Vergr"osserungen der Zellen bei 250 und 750. Die Zellen sind von unmessbaren schwarzen Punkten eingefasst.

In der N"ahe der Pflanzen ist ein Prothallium von derselben Farbe wie das Blatt, offenbar Urania angeh"orend.

Aber der ganze Schliff ist nichts als Pflanze.

Urania ist in 8 sehr sch"onen Exemplaren darin.

Am sch"onsten aber ist ein Durchschnitt der ganzen Pflanze in der Mitte des Schliffs.

Nun die Frage, wie verh"alt sich die Form der Urania zu denen der Erde?

Dieselbe Pflanze findet sich nicht. Aber die ganze Structur unserer lebenden Algen oder Farne ist da.

Der D"unnschliff enth"alt hunderte von Pflanzen.

Ich nehme nun meine D"unnschliffe vor.

Zun"achst untersuche ich einen Olivin vom Allenthal (von Neumann in Freiberg). Dieselbe ist durchzogen von schlingpflanzartigen Algen mit einigen deutlichen Fruchtorganen.

Merkw"urdigerweise zeigt Nr. 2. Meteor-D"unnschliff T"ubinger Sammlung (Fundort unbestimmt) "ahnliche Pflanzen.

Doch halt ---

Der ganze Schliff ist eine Kelchzelle (das "ubrige Material ging beim Schleifen zu Grunde). Taf. XXVI. Fig. 2. in nat"urlicher Gr"osse. Fig. 4. (Lupe).

Dieser D"unnschliff ist v"ollig durchzogen mit einer schlingpflanzartigen Alge (wie ich sie in den Algen des Laurentian-Serpentins beobachtet habe).

Die Masse ist Olivin.

Am Rande des Kelches sitzen graue Kelche. Taf. XXVI. Fig. 5. 6. 7. 8.

Die Schlingalge ist dunkler Olivin, metallisches (gediegenes?) Eisen ist "uberall ausgeschieden und bildet nun die Prothallien.

Vielfache Copulation.

Die Erhaltung ist ausgezeichnet.

Ich nenne sie

Heliola Caroli

unserem K"onig zur Ehre.

    Ein weiterer Schliff (Fundort unbekannt) Taf. XXVI. Fig. 9. nat"urliche Gr"osse. Fig. 1. u. 3. Teile 1/90.

    Meteorstein von Pultusk

Taf. XXVII. 3. 1/2. Der ganze Schliff.

    Desgl. 1. u. 2. 1/90.

Vergleiche Taf. XXVIII. Fig. 1. Basalt.

    Dhurmsala 1/90. Taf. XXVII. Fig. 7.

    Hainholz. Taf. XXVII. 9. u. 10. 1/2.

Diese Pflanze nenne ich dem mit Unrecht vergessenen schwedischen Naturforscher Imanuel von Swedenborg zu Ehren:

Sphaerothallium Swedenborgi.

    Ensisheim. Taf. XXVII. 4. 1/2.

    Schliff aus der T"ubinger Sammlung.
\clearpage
\section{\swabfamily{Meteoreisen}}
\paragraph{}
Nachdem der Meteorstein von Knyahinya und andere untersucht waren, nahm ich Meteoreisen vor.

Als sicherstes Kennzeichen des "achten Meteor-Eisens gelten die Widmannst"attischen Figuren.

Ihre wahre Gestalt geht aus Tafel XXIX. Fig. 1-9. (Meteoreisen von Toluca) hervor. Es sind Pflanzen und nichts als Pflanzen.
Pallaseisen.

Besonders merkw"urdig!

Die Olivine sind die Kelchzellen.

Das Schwefeleisen die F"ullmasse (Mark), das dunkle Eisen das Holz (wenn der Ausdruck erlaubt ist) , dass weisse Eisen die Zell-Membrane der Meteoreisen-Pflanze.

Die Crystallform ist ganz sicher hier zugleich Pflanzen-Form. (Siehe v. Kokscharow, Materialien zur Mineralogie Russlands Taf. 75. 76.

Es stimmt das specifische Gewicht der Erde mit dem Gestein, oder vielmehr Metall der Meteoriten.

Die Pflanzen gleichen wenigstens den tellurischen: insbesondere der "`acervulinen Form"' des \emph{Eozoon}.

Ob diese Meteoreisen und Steine nicht Teile der Erde sind, welche in fr"uhester Zeit losgel"ost und ihren eigenen Weg eine Zeit lang genommen, sp"ater aber sich wieder mit der Erde vereinigt haben?

Wie aber k"onnen Pflanzen Eisen werden?

K"onnen sie Eisen, Olivin sein?

Die Erhaltung der Formen, wie sie im Leben sind, eine solche Erhaltung der Organe ist wirklich wunderbar, und k"onnte zu letzterer Meinung f"uhren.

Im Hornblende-Gestein von Montreal sind Teile der Pflanze metallisches Eisen.

Ich nenne die Pflanze des Meteoreisens von Toluca

Astrosideron Quenstedti

zu Ehren meines Lehrers, die des Pallasits zu Ehren des Kaisers von Russland:

Alexandrea.

Gibt es nicht Protisten von Mineral und Pflanzen, wie von Pflanze und Tier?

Es ist die "Ubereinstimmung der siderischen und tellurischen Pflanzen zu untersuchen.

Nun ist ein sicheres Erkennungszeichen f"ur Meteoreisen gegeben.

Das Eisen aus dem Basalt von Gr"onland ist nichts als Pflanze, ist also Meteoreisen.
\clearpage
\section{\swabfamily{Die Urzelle --- Crystall-Pflanze --- das Tier}}
\paragraph{}
Bis jetzt hatte man keine Kenntniss von der wirklichen Form der Urzelle. Erst im Laufe meiner Untersuchung wurde sie mir klar und klar, dass sie auch die Grundform des thierischen K"orpers sei. Dies geschah aber erst, nachdem ich Alles Bisherige gesehen und geschrieben hatte.

Man wusste nur, dass die Zelle eigentlich die Einheit des thierischen K"orpers sei, dieser aus jener zusammengesetzt wird.

Man kannte eine Unzahl Formen dieser Zelle, man kannte Verbindungen dieser Zellen zu Gef"assen und Organen. Aber ein Ding, welches Urzelle und Urbild auch des menschlichen K"orpers sein k"onnte, davon hatte man keine Ahnung.

Dieses Urbild glaube ich in der Urzelle entdeckt zu haben, deren Wachsthum ich auf Tafel XVIII. dargestellt habe. Die Weiterentwicklung derselben ist die Figur Taf. I. und Taf. XVIII. 8. XIX. 3. Die n"ahere Begr"undung sp"ater.

Die Entstehung der Urzelle, das bemerke ich hier, rauss mit dem Stoff im Zusammenhang stehen. Sie entstand aus dem Stoff der Erdoberfl"ache in Verbindung mit dem der Atmosph"are, des Aethers. Die Stoffe sind verbraucht, sind in andere teilweise unl"osliche Formen "ubergef"uhrt und so ist eine Zwischenwand zwischen Atmosph"are und dem Erdinnern geschaffen, eine Absperrung, welche eine fortdauernde Erzeugung derselben im fr"uheren Maase unm"oglich macht.

Die Bau-Stoffe sind durch das Festwerden der Erde andere geworden und das Wachsen in der Menge ist abgeschnitten, andere Arten entstanden aus anderen Stoffen.

Die Urzelle hatte (solange ihr der Baustoff zu Gebot stand) die F"ahigkeit, sich ins Unglaubliche zu vermehren, aber auch die F"ahigkeit sich zu entwickeln.

Doch vor Allem ihre Grundform.

Was ich als Anfang gefunden, ist eine halbkugelige Schaale, deren Baustoff nichts anders sein kann, als ein Silicat.

Ich m"ochte "ubrigens die Frage hier offen lassen, ob die erste Form nicht blos ein Pl"attchen ist, das sich zur Halbkugel erst entwickelt. Da diese ganz einfache Form ihrer Natur nach sehr schwer nachzuweisen ist, so m"ochte ich diese Frage der Zukunft "uberlassen. Soviel ist aber gewiss, die erste Zelle tritt bald als eine Halbkugel auf.

Ist sie nach oben, oder nach unten gerichtet?

Ich habe auf Tafel XVIII. die Entwicklung der Zelle in ihrem ersten Stadium abgebildet.

Ich m"ochte auch die Frage offen lassen, ob das, was ich als den Anfang bezeichnet habe, wirklich das erste Stadium der Entwicklung ist: ob also nicht die erste Zelle nach abw"arts gerichtet ist? und dann erst der zweite Abschnitt der Zell-Bildung, nach oben, eintritt.

Ich habe einige Tatsachen beobachtet, welche daf"ur sprechen.

Im Kalk stehen die Zellen, so lange die Pflanze im Kalk w"achst, nach unten: erst wenn sie aus dem Kalk hervortreten, entstehen die Zellen, welche sich nach oben richten. Das l"asst sich bei gr"osseren Arten beobachten, welche den Serpentin durchsezen, also in die Luft- oder Wasserschichte eingetreten sind.

Ob die Zellen im Licht oder Wasser gelebt haben? Im Laurentian waren es Wasserpflanzen.

Nun wir haben als Form eine Halbkugel.

An dieser bildet sich ein Ansatz, Wurzelzelle. Die Zelle wird zum Trichter. Der Trichter schn"urt sich ab: dadurch entstehen Stamm-Zellen: die letzte ist die Kelchzelle, welche der eigentlichen Fortpflanzung dient. Das hindert aber nicht, dass sich "uberall an der Zelle Ans"aze zu neuen Zellen und neue Zellen bilden.

Die Quer-Durchschnitts-Form der Zelle ist die runde.

Ob ein Gesetz ihre L"ange beherrscht, ist zweifelhaft, eine Grenze hat das Wachsthum, aber nach welchem Gesetz ist nicht sicher.

(Anwendung auf die S"aule im Crystall).

Ich glaube annehmen zu d"urfen, dass ein solches Gesetz besteht. Das Pflanzen-Individuum hat im Kelch seinen Abschluss gefunden.

Ich muss solche Fragen offen lassen.

Es schn"uren sich aber auch weitere Zellen innerhalb der Zelle ab. Es erfolgt eine Teilung in zwei Reihen und sofort.

Die Teilungszelle ist ebenso gut Kelch als Stammzelle.

Wir haben also die Wurzel-, Stamm- und Kelch-(End-)Zellen der Urpflanze.

Die einzelnen Zellen sind zugleich Crystalle und bleiben es nach dem Tode; dadurch geht die Pflanze in das Mineral "uber.

Die eigentliche Pflanze ist da, sobald eine Zelle ein selbst"andiges Ganze aus der Baustoffmasse sich abscheidet.

Aus dieser entsteht dann, indem sie sich vom Boden losl"ost und ihre Nahrung nicht durch die Wurzel vom Boden oder durch den Kelch aus der Luft, dem Wasser zieht, was nach oben oder aussen offene Nahrungsorgane voraussetzt --- das Tier. Die Bewegung ist eine Folge davon.

Titanus Bismarki Tafel XII. XIII. ist ein Schlauch, der sich festsetzen, aber auch frei bewegen kann, der seine Nahrung nicht mehr aus dem Stein, der Erde, zieht, sondern frei w"ahlt, sich nach der Nahrung bewegt.

Damit glaube ich auch den Begriff des Tiers einigermassen festgestellt zu haben.

Die Urzelle besteht aus Wurzel-, Stamm- und Kelch-Zelle.

Es sind die 3 Hohlr"aume des organischen K"orpers, welche auf dem Weg der Abschn"urung entstehen.

Was freilich diese Abschn"urungen, diese Entwicklung bewirkt, wissen wir nicht.

Der Crystall ist die mit dem Entstehen stillstehende, abgeschlossene Zelle: die Pflanze --- w"achst --- bleibt aber mit der Erde in Verbindung, das Tier wird, w"achst und l"ost sich zu selbst"andiger freier Bewegung los.

Auch dies sind blos Entw"urfe, Grundlinien. M"oge die Zukunft sie best"atigen und verbessern.
\clearpage
\section{\swabfamily{Der Krystall}}
\paragraph{}
Ich habe schon oben zur Untersuchung dar"uber aufgefordert, ob der Crystall wenigstens mit der Ur-Pflanzen-Zelle nicht in einem gewissen Zusammenhang stehe.

Unsere Abbildungen Tab. XXII. XXVII. Fig. 5. 6. 8. legen die Frage nahe, ob der Crystall aus der Pflanze, oder die Pflanze aus dem Crystall entstanden sei.

Soviel aber ist mir sofort klar: wir m"ussen unsere Crystalle einer neuen Pr"ufung unterwerfen, nemlich in der Richtung, ob wir es wirklich mit Crystallen zu thun haben.

Ich habe gezeigt, dass gewisse Crystalle (Granit, Leuzit etc.) F"ullmassen von Pflanzen-Zellen sind. Daher also auch wohl das Gesetz der Symmetrie. Dieses ist nicht ein Gesetz der Molec"ule, sondern der Pflanzen-Symmetrie, der Pflanzenzelle.

Vor Allem den Augenschein.

Ich nehme eine Reihe Crystalle vor.

Das Ergebniss ist:

Die Crystallform ist eine Umschreibung der Pflanzenform in der Art, dass sich "uberall eine Pflanze in dem Crystall eingezeichnet findet, oder der Crystall um die Pflanze liegt.

Quarz. Tafel XXX. 9. 10.

Ich finde in zwei d"unnen Quarzen die Figuren der Urpflanzenform ein-, aber auch auf der Seite aufgezeichnet. Die Einschl"usse und gewisse dunkle und helle Stellen bezeichnen die Gestalt der eingelagerten Pflanze.

Die matten Fl"achen des Quarzes auf dem Dihexaeder, welche als Zwillingsbildung gedeutet wurden, sind Pflanzenformen. Fig. 14. Die gewundenen Quarze verdanken ihre Gestalt der spiraligen Zellstellung der Pflanzen.

Aragonit, Herrengrund. Zwillinge, Drillinge. Tafel XXX. 5.

Auf der geraden Endfl"ache sind soviele Pflanzen als Crystalle.

Das Merkw"urdigste ist die S"aule.

Der Mittelpunkt der S"aulenfl"ache f"allt mit dem Kelchmittelpunkt der Pflanze (gegen den Beschauer gekehrt) zusammen. Fig. 3.

Ein Pflanzen-Individuum und zwar das erste, streckt die Kelchspitze in den Kelchhals des zweiten und so fort. Dieser Pflanzenkelch liegt also wie ein Kranz in der Mitte des Crystalls horizontal und ber"uhrt dessen Peripherie.

Ein Wunder, aber wahr.

Corund. Saphir. Fig. 11. 12.

Der Corund ist die Ausf"ullung eines Pflanzenkelchs. Ein Saphir aus der T"ubinger Sammlung zeigt dies deutlich.

Er ist in der Mitte horizontal gebrochen. Die Bruchfl"ache zeigt den Pflanzenkelchdurchschnitt. Die Ecken legen sich in die Spitzen des Kelches.

Die Kelchbl"atter selbst sind in Glimmer verwandelt.

Pinit. Taf. XXX. Fig. 15. 16.

Soviel Pflanzen, soviel S"aulen. (Wahrscheinlich dasselbe Gesetz beim Turmalin, "uberhaupt bei der S"aulenform.)

Auf der unteren Endfl"ache der S"aule ein Z"apfchen (Wurzelzelle), auf der oberen Endfl"ache ebenso.

Auf jeder S"aulenfl"ache eine Pflanze.

Smaragd, Beryll. Ebenso. Tafel XXX. Fig. 13.

Kupferglanz. Fig. 2.

Mehrere Crystalle an einander gewachsen nach dem Gesetz der Zellbildung.

Die oberste gr"osste ein Kelch mit eingeschriebenen Linien.

Die "`dendritische"' Form der Metalle wahrscheinlich Pflanzen-Grundform.

In einem Meteorstein eine Pflanze in Nickel verwandelt.

Diamant.

Kelchausf"ullung: schliesst Brutzellen ein.
\clearpage
\section{\swabfamily{Schluss}}
\paragraph{}
Ich glaube, man kann drei Zeitalter der Erde annehmen, das Alter des Siliciums (der Kieselerde), das des Aluminiums (des Thons), das des Calciums (des Kalks).

Solange der gr"osste Teil des Lebens unter Wasser war, mussten nat"urlich die Lebensvorg"ange auch andere sein, als heute, wo die meisten Pflanzen-Arten wenigstens Luftpflanzen sind, und aus einer Erde sich n"ahren, welche in keiner Weise ebenso gedacht werden kann, wie die Erdoberfl"ache vor der ersten Pflanzenentwicklung.

In den Granitpflanzen scheint die Umwandlung die gewesen zu sein, dass aus der todten Pflanze die Stoffe ausschieden, welche in Gas, oder in Wasser-L"osung sich "uberf"uhren liessen und dass dann (vielleicht erst zur Gneiszeit) das Wasser seine in L"osung befindlichen Silicate etc. in den Zellen absetzte.

Mit R"ucksicht hierauf habe ich auch die schwarze Farbe der darauf folgenden Thonschiefer, welcher meist von Kohlen herr"uhrt, ferner die m"achtigen Graphit- und Anthrazitlager im Granit und Gneis und die Steinkohlenschichten selbst erkl"art.

Nach F. v. Hochstetter hat der Urkalk von B"ohmen bitumin"osen Geruch!

Es w"are in der Tat merkw"urdig, dass von den Tangen, welche nach anderer Meinung 50 M. m"achtige Kohlenschichten hinterlassen konnten, nicht Eine Pflanze erhalten ist.

Sind die Tange im Silur erhalten, so m"ussen sie noch viel mehr im Kohlengebirge sich auffinden lassen, vorausgesetzt, dass sie da sind.

Man findet aber dort eben keine Tange mehr, sondern blos den geschichteten Kohlenstoff.

Wenn man genau untersucht, ist der Kohlensandstein ebenfalls ein Pflanzen-Lager, das aber nicht hinreicht, um die Kohlenfl"oze zu erkl"aren und desshalb erkl"are ich letztere aus dem von der Granitpflanze ausgeschiedenen Kohlenstoff. Diese Pflanzen sind richtige Tange. Kommen ja solche Kohlenfl"oze nach Mohr schon im Porphyr vor, der nach meiner Deutung nichts ist, als der erste wahre Schlamm dieser Erde mit eingebetteten lebenden Pflanzen.

Wenn Meteorsteine als Pflanzen, oder wenigstens die Pflanzen im Meteorstein nachgewiesen sind, so k"onnte das erste Leben unserer Erde auch von einem anderen Planeten auf die Erde "ubertragen worden sein.

Sch"oner als im Graphit von Piemont (die Etikette hat in Klammer Passau?) habe ich die Pflanzen-Zellen nicht gesehen. Sie sind wahrscheinlich in Feldspat erhalten. Der Graphit selbst aber ist auch eine Pflanze, und nicht etwa Magma, in welchem die Feldspat-Pflanze liegt.

Ist dies vielleicht nicht der Anfang von Pflanzen der heutigen chemischen Zusammensetzung?

Diese Frage soll eine offene sein.

Man lese F. Zirkel, Lehrbuch der Petrographie, I. Band, "uber Graphit Seite 352, "uber Anthracit Seite 355 und vergleiche das dort Gesagte mit dem Ergebniss meiner Untersuchung, so wird sich das Geheimniss der Anthracitfl"oze im Granit und Gneis l"osen. Zirkel ist, wie ich, der Ansicht, dass sich die F"arbung der j"ungeren Gesteine durch Anthracit erkl"are.

Woher sollen diese Anthracit-Lager kommen, wenn der Granit nicht Pflanzen enth"alt, oder Pflanze war?

Ich halte wenigstens daf"ur, dass es keine einfachere Erkl"arung f"ur ihr Entstehen giebt.

Ich muss noch einmal auf die Verschiedenheit der Gesteine, welche man Serpentin nennt, hier zur"uckkommen. In dem canadischen Gneis-Serpentin ist das urspr"ungliche Mineral wahrscheinlich Olivin oder Augit. Die Lager von Serpentin sind sicher Auslaugungen, Niederschl"age aus w"asseriger L"osung. Pikrolith etc. sind ebenso unzweifelhaft w"asserige Niederschl"age, w"ahrend in dem Serpentin von Snarum noch die Gestalt des Olivins sich erhalten hat, was Quenstedt so "uberzeugend nachwies. Freilich mag jener Olivin schon Pflanzen eingeschlossen haben und daher die eigenth"umliche Textur.

Der Umstand, dass unkundige Leute alle gr"unen durchsichtigen talkerde- und kieselerdehaltigen Minerale Serpentin nannten, kann nat"urlich zu nichts weiterem f"uhren, als dass man die Serpentine nunmehr genauer untersucht und unterscheidet. Vergl. "uber Serpentin v. Draschke in Tschermak Mineralogische Mitteilungen I. 1871. S. 1. Kennt man die Form der Urpflanze, so wird es leicht sein, die urspr"unglichen Serpentine und die abgeleiteten zu unterscheiden.

Man hat den Pflanzen "uberhaupt noch sehr wenig Aufmerksamkeit geschenkt. Noch in den 50 ger Jahren sprach man kaum von Pflanzen (Algen) in dem obern "Ubergangsgebirge. Erst im Kohlengebirge waren sie wirklich anerkannt.

Wer sich mit dem Gegenstand, welchen ich vorgelegt habe, besch"aftigen. will, muss zuerst das Silur genauer ansehen und sich hier davon "uberzeugen, dass die Form, welche ich schon im Laurentian nachgewiesen habe, im Silur millionenfach vorkommt. Der Riesenform im Marmor von Carrara habe ich schon erw"ahnt. Man muss mit Einem Wort sich mit dem Gegenstand erst vertraut machen. Oldhamia antiqua ist schon ein gutes Object f"ur die Studien.

Hat man aber die Form dieser Urpflanze einmal festgestellt, so wird es ein Leichtes sein, sie in den fr"uheren Formationen zu erkennen.

Ich m"ochte die Forscher dort haupts"achlich auf die besser erhaltenen Brutzellen und darauf aufmerksam machen, dass die Zellscheidew"ande stets durch schwarze Punkte, n"amlich zur"uckgebliebene Kohlen, manchmal sogar durch Kohlens"aure-Einschl"usse angedeutet sind. Nimmt man den Polarisationsapparat zu Hilfe, so wird jede Zelle, welche so abgegr"anzt ist, auch verschieden polarisiren, weil die Lage des Gesteins in jeder Zelle eine verschiedene ist.

Freilich das gebe ich zu, dass ein Teil des Pflanzen-Gesteins in Folge sp"aterer Vorg"ange auch unkenntlich wird. Ein zersetzter Granit kann nie als Untersuchungs-Material verwendet werden.

Der gelbbraune "`Alabaster"', welcher von den Italienern in Vasen, Statuetten in den Handel gebracht wird, ist eine Pflanze. Unser Keuper, Muschelkalkgyps ist es.

Ich erstaunte, als ich den Einwurf h"oren musste: "`Wie aus den Bestandteilen einer Pflanze: Kohlenstoff, Sauerstoff und Wasserstoff, in den Kelchen Kieselerde, Thonerde, Kalk, Bittererde, Eisenoxyd entstehe?"' Als ob dies die Frage w"are.

Einmal sind die Bestandteile keiner Pflanze blos Kohlenstoff, Sauerstoff und Wasserstoff: das Knochenger"uste von Kiesel, Kalk, die Aschenbestandteile sind vergessen, vergessen sind die Kiesel-Skelette der Diatomaceen, die Kieselzellen der Equiseten etc. Hier wird sicher nicht aus Kohlenstoff, Sauerstoff und Wasserstoff Kieselerde, sondern die Kieselerde ist durch die Pflanze hindurchgegangen, von der Pflanze abgelagert.

Man vergleiche N"ageli und Schwendener, das Microscop (1877) S. 488.

Nun glaube ich aber eben mit der Serpentinpflanze nachgewiesen zu haben, dass sogar an die Stelle der Urpflanze, mag sie bestanden haben, aus was sie will, ein Bittererde-Silicat treten kann. Dass aber Pflanzen aus der Granitzeit anders organisirt waren, als die der Jetztzeit, versteht sich von selbst und waren sie anders, so waren nat"urlich auch ihre Ausscheidungen anders als heute.

"Ubrigens scheiden viele Pflanzen heute noch vorzugsweise Kieselerde in den Fruchtschalen, Samenh"ulsen aus.

Was die "`Ausscheidungen"' betrifft, so erinnere ich an die verschiedensten Minerale in den Ammonitenkammern, welche dort und nur dort vorkommen. Sind sie von den Ammoniten ausgeschieden? Gewiss nicht. Es ist also das Wort "`Ausscheidung"' cum grano salis zu verstehen. Das Mineral liegt da: Welches der Zusammenhang mit der Pflanze ist, ist noch nicht --- aber dass ein Zusammenhang besteht, ist festgestellt. Keinenfalls kann die Tatsache, dass Granat an der Stelle von Pflanzenkelchen vorkommt, als Gegenbeweis dagegen dienen, dass kein Pflanzenkelch da ist. Ist die Pflanze einmal festgestellt, so sehe die Wissenschaft eben, wie sie die Tatsache erkl"aren kann.

Freilich bei uns f"angt man nicht mit dem Auge an, sondern mit der Rechnung. Statt zu sehen, zu betrachten, zu versuchen, macht man sich aus seinem beschr"ankten Wissen die Rechnung fertig. Wenn es nun aber doch so ist? Wenn eine unbefangene nur halbwegs sorgf"altige Beobachtung die Tatsache best"atigt, was dann?

Der scharfsinnige Arago (und es sind nicht einmal alle Leute, welche absprechende Behauptungen aufstellen, scharfsinnig) bewies vor den Augen und Ohren von ganz Frankreich, dass eine Eisenbahn unm"oglich seie. Es hat Jahrhunderte angestanden, bis gegen die Behauptungen aller Fachgelehrten die Versteinerungen als Tier-Reste anerkannt wurden.

Die feinen Linien im Granit mit schwarzen Punkten gezogen, kann man unm"oglich aus blossen Crystallen erkl"aren, sie k"onnen also nicht blos die Peripherie eines Minerals andeuten. Aber auch zuf"allige Linien sind sie nicht. Es herrscht ein Gesetz darin und dieses Gesetz ist die Form der Pflanze: nur die einzelnen Zellen haben die Form des Crystalls, sind aber keine vollkommenen Crystalle.

Erst am Schlusse der Arbeit, gestattete ich mir einen Blick auf die Pflanzenarten zu thun.

Ich fasse die "aussere Structur des Gesteins ins Auge, welche durch die Pflanze, aus welcher es sich zusammensetzt, bedingt ist.

Insbesondere zwei geschliffene Granite von Kullgren Uddevalla in Schweden ("`Granit of flera sorter"', wie er im Handel heisst) machen mich darauf aufmerksam, dass die Pflanzenform des Granits auch die von Flechten sein k"onnte. Es sind kurze Kelchzellen, welche senkrecht oder neben einander stehend, das Gestein bilden (Cladonia?).

Erst im Gneis findet sich horizontale Lagerung des Gesteins bedingt durch lange Pflanzen, welche nun mehr den See-Algen "ahnlich sind.

Ich ziehe aus der Gesteins-Structur der Granit-Pflanze den Schluss, dass die Erdoberfl"ache zur Granitzeit nicht unter Wasser, sondern erst von Dampf umgeben war. Die Pflanzen des Granits w"aren dann Landpflanzen. Im Wasser w"urden die Pflanzen-Schichten sofort in horizontalen Lagern sich angesetzt haben.

Die Flechten scheinen mir "uberhaupt die erste Pflanzenform zu sein, welche unmittelbar aus Mineralsubstanz entstehen.

Erst im Gneis-Alter der Erde finden sich die "achten Wasserpflanzen, in Wasser abgelagerte Pflanzen.

Mit der Structur der Pflanzen h"angt die Form der heutigen Gebirge (Pflanzenberge) zusammen.

Man sieht, wie schwer es ist, von angew"ohnten hergebrachten Vorstellungen sich loszumachen und mit unparteiischen Augen zu sehen.
\clearpage
\begin{figure}[b]
\caption{\swabfamily{Tafel 1}}
\includegraphics[width=\textwidth,height=\textheight,keepaspectratio]{Tab-1.png}
\centering
\end{figure}
\clearpage
\begin{figure}[b]
\caption{\swabfamily{Tafel 2}}
\includegraphics[width=\textwidth,height=\textheight,keepaspectratio]{Tab-2.png}
\centering
\end{figure}
\clearpage
\begin{figure}[b]
\caption{\swabfamily{Tafel 3}}
\includegraphics[width=\textwidth,height=\textheight,keepaspectratio]{Tab-3.png}
\centering
\end{figure}
\clearpage
\begin{figure}[b]
\caption{\swabfamily{Tafel 4}}
\includegraphics[width=\textwidth,height=\textheight,keepaspectratio]{Tab-4.png}
\centering
\end{figure}
\clearpage
\begin{figure}[b]
\caption{\swabfamily{Tafel 5}}
\includegraphics[width=\textwidth,height=\textheight,keepaspectratio]{Tab-5.png}
\centering
\end{figure}
\clearpage
\begin{figure}[b]
\caption{\swabfamily{Tafel 6}}
\includegraphics[width=\textwidth,height=\textheight,keepaspectratio]{Tab-6.png}
\centering
\end{figure}
\clearpage
\begin{figure}[b]
\caption{\swabfamily{Tafel 7}}
\includegraphics[width=\textwidth,height=\textheight,keepaspectratio]{Tab-7.png}
\centering
\end{figure}
\clearpage
\begin{figure}[b]
\caption{\swabfamily{Tafel 8}}
\includegraphics[width=\textwidth,height=\textheight,keepaspectratio]{Tab-8.png}
\centering
\end{figure}
\clearpage
\begin{figure}[b]
\caption{\swabfamily{Tafel 9}}
\includegraphics[width=\textwidth,height=\textheight,keepaspectratio]{Tab-9.png}
\centering
\end{figure}
\clearpage
\begin{figure}[b]
\caption{\swabfamily{Tafel 10}}
\includegraphics[width=\textwidth,height=\textheight,keepaspectratio]{Tab-10.png}
\centering
\end{figure}
\clearpage
\begin{figure}[b]
\caption{\swabfamily{Tafel 11}}
\includegraphics[width=\textwidth,height=\textheight,keepaspectratio]{Tab-11.png}
\centering
\end{figure}
\clearpage
\begin{figure}[b]
\caption{\swabfamily{Tafel 12}}
\includegraphics[width=\textwidth,height=\textheight,keepaspectratio]{Tab-12.png}
\centering
\end{figure}
\clearpage
\begin{figure}[b]
\caption{\swabfamily{Tafel 13}}
\includegraphics[width=\textwidth,height=\textheight,keepaspectratio]{Tab-13.png}
\centering
\end{figure}
\clearpage
\begin{figure}[b]
\caption{\swabfamily{Tafel 14}}
\includegraphics[width=\textwidth,height=\textheight,keepaspectratio]{Tab-14.png}
\centering
\end{figure}
\clearpage
\begin{figure}[b]
\caption{\swabfamily{Tafel 15}}
\includegraphics[width=\textwidth,height=\textheight,keepaspectratio]{Tab-15.png}
\centering
\end{figure}
\clearpage
\begin{figure}[b]
\caption{\swabfamily{Tafel 16}}
\includegraphics[width=\textwidth,height=\textheight,keepaspectratio]{Tab-16.png}
\centering
\end{figure}
\clearpage
\begin{figure}[b]
\caption{\swabfamily{Tafel 17}}
\includegraphics[width=\textwidth,height=\textheight,keepaspectratio]{Tab-17.png}
\centering
\end{figure}
\clearpage
\begin{figure}[b]
\caption{\swabfamily{Tafel 18}}
\includegraphics[width=\textwidth,height=\textheight,keepaspectratio]{Tab-18.png}
\centering
\end{figure}
\clearpage
\begin{figure}[b]
\caption{\swabfamily{Tafel 19}}
\includegraphics[width=\textwidth,height=\textheight,keepaspectratio]{Tab-19.png}
\centering
\end{figure}
\clearpage
\begin{figure}[b]
\caption{\swabfamily{Tafel 20}}
\includegraphics[width=\textwidth,height=\textheight,keepaspectratio]{Tab-20.png}
\centering
\end{figure}
\clearpage
\begin{figure}[b]
\caption{\swabfamily{Tafel 21}}
\includegraphics[width=\textwidth,height=\textheight,keepaspectratio]{Tab-21.png}
\centering
\end{figure}
\clearpage
\begin{figure}[b]
\caption{\swabfamily{Tafel 22}}
\includegraphics[width=\textwidth,height=\textheight,keepaspectratio]{Tab-22.png}
\centering
\end{figure}
\clearpage
\begin{figure}[b]
\caption{\swabfamily{Tafel 23}}
\includegraphics[width=\textwidth,height=\textheight,keepaspectratio]{Tab-23.png}
\centering
\end{figure}
\clearpage
\begin{figure}[b]
\caption{\swabfamily{Tafel 24}}
\includegraphics[width=\textwidth,height=\textheight,keepaspectratio]{Tab-24.png}
\centering
\end{figure}
\clearpage
\begin{figure}[b]
\caption{\swabfamily{Tafel 25}}
\includegraphics[width=\textwidth,height=\textheight,keepaspectratio]{Tab-25.png}
\centering
\end{figure}
\clearpage
\begin{figure}[b]
\caption{\swabfamily{Tafel 26}}
\includegraphics[width=\textwidth,height=\textheight,keepaspectratio]{Tab-26.png}
\centering
\end{figure}
\clearpage
\begin{figure}[b]
\caption{\swabfamily{Tafel 27}}
\includegraphics[width=\textwidth,height=\textheight,keepaspectratio]{Tab-27.png}
\centering
\end{figure}
\clearpage
\begin{figure}[b]
\caption{\swabfamily{Tafel 28}}
\includegraphics[width=\textwidth,height=\textheight,keepaspectratio]{Tab-28.png}
\centering
\end{figure}
\clearpage
\begin{figure}[b]
\caption{\swabfamily{Tafel 29}}
\includegraphics[width=\textwidth,height=\textheight,keepaspectratio]{Tab-29.png}
\centering
\end{figure}
\clearpage
\begin{figure}[b]
\caption{\swabfamily{Tafel 30}}
\includegraphics[width=\textwidth,height=\textheight,keepaspectratio]{Tab-30.png}
\centering
\end{figure}
\clearpage
\printindex
\clearpage
\end{document}
